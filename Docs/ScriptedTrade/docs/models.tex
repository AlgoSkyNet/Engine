\subsection{Pricing Engine Configuration}\label{pricingengine_config}

An example pricing engine configuration looks as follows.

\begin{minted}[fontsize=\footnotesize]{xml}
   <Product type="ScriptedTrade">
    <Model>Generic</Model>
    <ModelParameters>
      <!-- shared parameters -->
      <Parameter name="Model">BlackScholes</Parameter>
      <Parameter name="InfModelType">DK</Parameter>
      <Parameter name="BaseCcy">USD</Parameter>
      <Parameter name="EnforceBaseCcy">false</Parameter>
      <Parameter name="GridCoarsening">3M(1W),1Y(1M),5Y(3M),10Y(1Y),50Y(5Y)</Parameter>
      <Parameter name="IrReversion">0.0</Parameter> <!-- fallback for other ccys -->
      <Parameter name="IrReversion_EUR">0.0</Parameter>
      <Parameter name="IrReversion_GBP">0.0</Parameter>
      <Parameter name="FullDynamicIr">true</Parameter>
      <Parameter name="FullDynamicFx">true</Parameter>
      <Parameter name="ReferenceCalibrationGrid">400,3M</Parameter>
      <Parameter name="Calibration">Deal</Parameter>
      <!-- product specific parameters -->
      <Parameter name="Model_SingleAssetOption(EQ)">BlackScholes</Parameter>
      <Parameter name="Model_SingleAssetOption(FX)">BlackScholes</Parameter>
      <Parameter name="Model_SingleAssetOption(COMM)">BlackScholes</Parameter>
      <Parameter name="Model_SingleAssetOptionBwd(EQ)">BlackScholes</Parameter>
      <Parameter name="Model_SingleAssetOptionBwd(FX)">BlackScholes</Parameter>
      <Parameter name="Model_SingleAssetOptionBwd(COMM)">BlackScholes</Parameter>
      <Parameter name="Model_SingleUnderlyingIrOption">GaussianCam</Parameter>
      <Parameter name="Model_SingleUnderlyingIrOptionBwd">GaussianCam</Parameter>
      <Parameter name="Model_MultiUnderlyingIrOption">GaussianCam</Parameter>
      <Parameter name="Model_IrHybrid(EQ)">GaussianCam</Parameter>
      <Parameter name="Model_IrHybrid(FX)">GaussianCam</Parameter>
      <Parameter name="Model_IrHybrid(COMM)">GaussianCam</Parameter>
    </ModelParameters>
    <Engine>Generic</Engine>
    <EngineParameters>
      <!-- shared parameters -->
      <Parameter name="Engine">MC</Parameter>
      <Parameter name="Samples">10000</Parameter>
      <Parameter name="StateGridPoints">200</Parameter>
      <Parameter name="StateGridPoints_SingleUnderlyingIrOptionBwd">50</Parameter>
      <Parameter name="MesherEpsilon">1E-4</Parameter>
      <Parameter name="MesherScaling">1.5</Parameter>
      <Parameter name="MesherConcentration">0.1</Parameter>
      <Parameter name="MesherMaxConcentratingPoints">9999</Parameter>
      <Parameter name="MesherIsStatic">true</Parameter>
      <Parameter name="RegressionOrder">2</Parameter>
      <Parameter name="TimeStepsPerYear">24</Parameter>
      <Parameter name="Interactive">false</Parameter>
      <Parameter name="BootstrapTolerance">0.1</Parameter>
      <Parameter name="IncludePastCashflows">true</Parameter>
      <Parameter name="RegressionVarianceCutoff">1E-5</Parameter>
      <!-- product specific parameters -->
      <Parameter name="RegressionOrder_SingleAssetOption(EQ)">6</Parameter>
      <Parameter name="RegressionOrder_SingleAssetOption(FX)">6</Parameter>
      <Parameter name="RegressionOrder_SingleAssetOption(COMM)">6</Parameter>
      <Parameter name="Engine_SingleAssetOptionBwd(EQ)">FD</Parameter>
      <Parameter name="Engine_SingleAssetOptionBwd(FX)">FD</Parameter>
      <Parameter name="Engine_SingleAssetOptionBwd(COMM)">FD</Parameter>
      <Parameter name="Engine_SingleUnderlyingIrOption">FD</Parameter>
      <Parameter name="useAD_MultiAssetOptionAD(EQ)">true</Parameter>
      <Parameter name="useAD_MultiAssetOptionAD(FX)">true</Parameter>
      <Parameter name="useAD_MultiAssetOptionAD(COMM)">true</Parameter>
      <!-- MultiUnderlyingIrOption -->
      <!-- IrHybrid(EQ) -->
      <!-- IrHybrid(FX) -->
      <!-- IrHybrid(COMM) -->
    </EngineParameters>
  </Product>
\end{minted}

The model parameters have the following meaning:

\begin{itemize}
\item Model: The model to be used. Currently BlackScholes, LocalVolDupire, LocalVolAndreasenHuge, GaussianCam are
  supported
\item BaseCcy: The default base ccy for the model. See \ref{baseccy_determination} for the way the base ccy is
  determined for a model.
\item EnforceBaseCcy: Enforce the base ccy from the model settings to be used, see \ref{baseccy_determination} for more
  details.
\item FullDynamicFx: by default, dynamic fx processes are generated only for FX indices on which the script explicitly
  depends; if this parameter is true, dynamic fx processes are also generated for payment and equity / commodity ccys
  not equal to the base ccy of the model or any of the currencies covered by those FX indices; if the parameter is false
  on the other hand zero volatility paths are assumed for these additional ccys
\item FullDynamicIr: by default, dynamic ir processes are generated only for currencies of the IR indices on which the
  script depends; if the parameter is true, dynamic ir processes are also generated for currencies from fx indices, and
  payment and equity / commodity ccys not equal to the base ccy of the model; if the parameter is false zero volatility
  paths are assumed for these ccys
\item InfModelType: Optional, defaults to ``DK''. For the GaussianCam model two flavours of inflation models are
  avaiable, Dodgson-Kainth (``DK'') and Jarrow-Yildrim (``JY''). This parameter selects the flavour to use. Only applies
  to model = GaussianCam
\item ReferenceCalibrationGrid: If given, only one calibration point per defined interval will be used (to avoid
  oscillations in the calibrated model volatility function or also to improve the calibration speed), only applicable to
  model = GaussianCam
\item Calibration: Deal or ATM. For Deal the calibration spec of a script is used to calibrate a model. For ATM an ATM
  calibration is used. Optional, defaults to Deal.
\end{itemize}

The engine parameters have the following meaning:

\begin{itemize}
\item Engine: The engine to be used. Currently MC and FD is supported.
\item Samples: The number of MC samples used. Only relevant for Engine = MC.
\item StateGridPoints: The number of grid points in state direction. Only relevant for Engine = FD.
\item MesherEpsilon: The FD mesher epsilon, optional, defaults to 1E-4.
\item MesherScaling: The FD mesher scaling factor, optional, defaults tp 1.5.
\item MesherConcentration: The FD mesher concentration parameter, optional, defaults to 0.1
\item MesherMaxConcentrationPoints: The maximum number of mesher concentration points to use, optional, defaults to
  9999, in which all calibration strikes from a script are used. For 1 only the first strike from the list is used, for
  2 the first two strikes, etc.
\item MesherIsStatic: If true, the mesher is built only once and reused under scenario / sensitivity computations. If
  false, the mesher is rebuilt for each repricing. Optional, defaults to false. For sensitivity runs it should be set to
  true.
\item RegressionOrder: The order of the polynomial basis to compute conditional expectations via regression
  analysis. Applies to MC only.
\item SequenceType: The sequence type used for pricing. Defaults to SobolBrownianBridge. Possible values
  SobolBrownianBridge, Burley2020SobolBrownianBridge, MersenneTwister, MersenneTwisterAnithetic, Sobol,
  Burley2020Sobol. Applies to MC only.
\item PolynomType: The polynom type used for regression analysis. Defaults to Monomial. Possible values Monomial,
  Laguerre, Hermite, Hyperbolic, Legendre, Chebyshev, Chebychev2nd. Applies to MC only.
\item TrainingSamples: If given, pricing and training are separate phases and traning phase is using this number of
  samples. Notice that NPVMEM() should be used to reuse regression coefficients from the training phase in the pricing
  phase. Applies to MC only.
\item TrainingSeed: The seed used for rng in training phase. Only applies if TrainingSamples are given (and thus
  training and pricing phases are separate). Defaults to 43. Applies to MC only.
\item TrainingSequenceType: The sequence type used for training. Only applies if TrainingSamples are given (and thus
  training and pricing phases are separate). See SequenceType for possible values. Defaults to MersenneTwister. Applies
  to MC only.
\item SobolOrdering: Sobol sequence ordering for brownian bridge. Defaults to Steps. Possible values Steps, Factors,
  Diagonal. Applies to MC only.
\item SobolDirectionIntegers: Sobol direction integers. Defaults to JoeKuoD7. Possible values Unit, Jaeckel,
  SobolLevitan, SobolLevitanLemieux, JoeKuoD5, JoeKuoD6, JoeKuoD7, Kuo, Kuo2, Kuo3. Applies to MC only.
\item Seed: The seed for rng in pricing phase. Defaults to 42. Applies to MC only.
\item TimeStepsPerYear: The number of time steps used to discretise the process. For 0 only the relevant simulation
  times are used. Otherwise at least the given number of step are used in the discretisation grid per year.
\item CalibrationMoneyness: Moneyness of options used for smile calibration. Applies to the LocalVolAndreasenHuge model
  only. The moneyness is defined as a ``standardised moneyness'' $\ln(K/F) / \sigma\sqrt{t}$ with $K$ strike, $F$ ATMF
  forward, $\sigma$ ATMF market vol, $t$ option time to expiry
\item BootstrapTolerance: tolerance for calibration bootstrap, only applies to model = GaussianCam
\item IncludePastCashflows: if true, LOGPAY() will generate cashflow information for pay dates on or before the
  reference date. Optional, defaults to false.
\item RegressionVarianceCutoff: Optional. Only relevant for MC models. If given, a coordinate transform and (possibly) a
  factor reduction is applied to the regressors used for conditional expectation calculation, such that $1-\epsilon$ of
  the total variance of regressors is kept, where $\epsilon$ the given parameter. This helps dealing with collinearity
  and also reducing the dimnensionality of the regression model.
\item Interactive: If true an interactive session is started on script execution for debugging purposes; should be false
  except for debugging purposes
\item UseAD: If true and RunType in the global pricing engine parameters is SensitivityDelta, a first order pnl
  expansion using AD sensitivities is used to compute scenario NPVs.
\item UseCG: If true a computation graph is used to price trades instead of the runtime interpreter . If UseAD or
  UseExternalComputingDevice is true, this implies that UseCG is true irrespective of how it is configured.
\item UseExternalComputingDevice: If true and RunType is not NPV (generating additional results) and AD sensitivities
  are {\em not} used, an external compute device is used for the calculations.
\item UseDoublePrecisionForExternalCalculation: Use double precision for external computations. Defaults to false.
\item ExternalDeviceCompatibilityMode: Only applies if UseCG is enabled. Defaults to false. If enabled, random number
  generation for internal calculations using the CG is aligned as closely as possible with what is usually implemented
  for external calculations, i.e. if enabled, the MersenneTwister random number generation is done in
  step-dimension-path order. If disabled, the ``classic'' order path-step-dimension is used.
\item ExternalComputeDevice: The external compute device to use if UseExternalComputingDevice is effective.
\end{itemize}

\subsection{Product Tags und pricing engine configuration}\label{producttags_engineconfig}

All parameters in the pricing engine configuration can be differentiated by product tags, see \ref{scriptNode} for how
to add these to a script definition. If a script has a product tag \verb+TAG+ assigned, parameters with suffix
\verb+_TAG+ are relevant if given. If for a parameter no version with the product tag is given, the corresponding
parameter without a tag is used as a fallback.

Product tags may contain the asset class variable \verb+{AssetClass}+ which is replaced by \verb+EQ+, \verb+FX+ or
\verb+COMM+ depending on the underlying asset class. If more than one index type eq, fx, comm occurs in the trade, the
asset class variable is set to \verb+HYBRID+. Notice that interest rate indices do not affect the asset class variable:
If e.g. an equity trade contains an interest rate funding leg, the asset class shouuld still be EQ. Trades with exotic
interest rate elements on the other hand can be distinguished by the product tag itself, see below for an example. If no
eq, fx, comm index occurs in the trade, the asset class variable is left blank.

Table \ref{tab:producttag} gives a typical example for grouping scripts using the product tag. The suffix ``AD'' is
meant to mark scripts that are suitable for AAD sensitivity computations.

\begin{table}[!htbp]
  \scriptsize
  \begin{tabular}{l | l | l | l}
    Product Tag & Examples & Suitable Models & Suitable Engines \\ \hline
    \verb+SingleAssetOption({AssetClass})+ & FX TaRF & BS, LV, GCAM & MC \\
    \verb+SingleAssetOptionAD({AssetClass})+ & FX TaRF & BS, LV, GCAM & MC \\
    \verb+SingleAssetOptionBwd({AssetClass})+ & EQ American Option & BS, LV, GCAM & MC, FD \\
    \verb+MultiAssetOption({AssetClass})+ & EQ Autocallable & BS, LV, GCAM & MC \\
    \verb+MultiAssetOptionAD({AssetClass})+ & EQ Autocallable & BS, LV, GCAM & MC \\
    \verb+SingleUnderlyingIrOption+ & IR TaRN & LGM1F & MC \\
    \verb+SingleUnderlyingIrOptionBwd+ & Bermudan swaption & LGM1F & MC, FD, Conv \\
    \verb+MultiUnderlyingIrOption+ & Cross currency swaption & GCAM & MC \\
    \verb+IrHybrid({AssetClass})+ & IR-EQ basket option & GCAM & MC \\
  \end{tabular}
  \caption{Grouping trades by product tag (Example), models are abbreviated as\\ BS = BlackScholes, LV = LocalVol, GCAM =
    GaussianCrossAsset}
  \label{tab:producttag}
\end{table}

\subsection{BlackScholes model}\label{blackscholes}

Models black scholes processes for EQ, FX, COMM underlyings. The strike slice of the input volatility is chosen as ATMF
or as the first strike given in the list of calibration strikes for an index.

See \ref{pricingengine_config} for the impact of the model parameter FullDynamicFx on the model setup.

For MC TimeStepsPerYear are ignored if the correlation structure is trivial, because then the process can be discretised
exactly. Otherwise the given time steps per year are used to build a grid on which covariance matrices are computed
assuming a constant volatility between the grid points. The actual MC paths are evoloved using these covariance matrices
on the original (non-refined) time grid, i.e. taking large, exact steps again.

For FD TimeStepsPerYear, StateGridPoints, MesherEpsilon, MesherScaling, MesherConcentration,
MesehMaxConcentrationPoints, MesherIsStatic are used, see the description of these parameters for their detailled
interpretation.

\smallskip
Available Engine types: MC, FD

\subsection{LocalVolDupire, LocallVolAndreasenHuge models}

Models local volatility processes for EQ, FX, COMM underlyings using the full smile from input volatility term
structures. To construct the local volatility surface either the classic Dupire formula or the Andreasen-Huge method is
used, see \cite{andreasen_huge_localvol} for the latter. The parameter FullDynamicFx has an analoguous meaning as in the
case of the BlackScholes model.

Calibration strike specifications (see \ref{calibration}) are not relevant for this model.

See \ref{pricingengine_config} for the impact of the model parameter FullDynamicFx on the model setup.

\smallskip
Available Engine types: MC

\subsection{GaussianCam models}

Models black scholes processes for EQ, FX, COMM and LGM 1F processes for IR. For INF processes Dodgson-Kainth (``DK'')
and Jarrow-Yidlrim (``JY'') models are available. The strike slice of the input volatility is chosen as ATMF / fair
forward swap rate. Furthermore,

\begin{itemize}
\item For IR processes a strip of atm coterminal swaptions is used for the model calibration.
\item For INF processes a strip of atm CPI cap/floors is used for the model calibration. For DK the reversion is
  calibrated and the volatility is fixed at $0.00050$. For JY the index volatility is calibrated. The reversion of the
  real rate process is fixed at $0$ and the volatility at $0.0030$. The real rate process is assumed to be uncorrelated
  to the nominal rate and index process within the JY model, and also to all other IR, EQ, FX, COMM processes in the
  model.
\end{itemize}

Calibration strike specifications (see \ref{calibration}) are not yet supported by this model.

See \ref{pricingengine_config} for the impact of the model parameters FullDynamicFx, FullDynamicIr on the model setup.

The FD model variant is supported for a single underlying IR model only. A single calibration strike is supported that
can be specified for any IR model index appearing in the script. TimeStepsPerYear, StateGridPoints, MesherEpsilon are
used, see the description of these parameters for their detailled interpretation.

\smallskip
Available Engine types: MC, FD

\subsection{Base Currency Determination}\label{baseccy_determination}

The base currency of a model is determined using the following ruleset:

\begin{itemize}
\item If the model parameter EnforceBaseCcy is set to true, the base ccy is read from the model parameter BaseCcy
\item Otherwise base ccy candidates are collected as the
  \begin{itemize}
    \item target (domestic) currencies of underlying fx indices
    \item if no fx indices are present in the model, all pay currencies that can occur
  \end{itemize}
\item If the set of base ccy candidates contains exactly one element, the base ccy is chosed as this currency
\item Otherwise the base ccy from the model parameters is chosen to be the model base ccy
\end{itemize}

\subsection{Grid Coarsening}\label{grid_coarsening}

Date schedules that are eligible for grid coarsening are listed in the (optional) ScheduleCoarsening subnode of the
script (see \ref{scriptNode})

\begin{minted}[fontsize=\footnotesize]{xml}
    <Script>
      <Code>...</Code>
      <NPV>...</NPV>
      <Results>...</Results>
      <ProductTag>...</ProductTag>
      <ScheduleCoarsening>
        <EligibleSchedule>ObservationDates</EligibleSchedule>
        <EligibleSchedule>KnockOutDates</EligibleSchedule>
      </ScheduleCoarsening>
    </Script>
\end{minted}

If a date schedule is eligible for grid coarsening the original grid is coarsened using the model parameter
GridCoarsening if this is not empty (meaning no coarsening is applied). The parameter consists of a comma separated list
of pairs of periods like 3M(1W),1Y(1M),5Y(3M),10Y(1Y),50Y(5Y). The coarsening procedure then works as follows (for the
example rule):

\begin{itemize}
\item dates before or equal to the evaluation date are always kepts
\item out to 3M in each subperiod of length 1W starting at the evaluation date at most one date of the original grid is
  kept in the result grid
\item out to 1Y in each subperiod of length 1M starting at the last subperiod end date from the previous step at most
  one date of the original grid is kept
\item etc. ... until all subperiods out to the last period 50Y are covered, all dates in the original grid that lie
  beyond the last subperiod are not present in the result grid
\end{itemize}

\subsection{Calibration}\label{calibration}

The calibration approach for the model is specified on the engine parameter level (see \ref{pricingengine_config}). If
the model parameter \verb+Calibration+ is set to \verb+Deal+, information on the calibration instruments is extracted
from the CalibrationSpc subnode of the script node (see \ref{scriptNode}). If this subnode is not given, the calibration
falls back to ATMF, meaning that ATMF instruments (coterminals for IR) are used for calibration.

\begin{minted}[fontsize=\footnotesize]{xml}
   <Script>
     <Code>...</Code>
     <NPV>...</NPV>
     <Results>...</Results>
     <CalibrationSpec>
        <Calibration>
          <Index>Underlying</Index>
          <Strikes>
            <Strike>Strike</Strike>
            <Strike>KnockOutLevel</Strike>
          </Strikes>
        </Calibration>
      </CalibrationSpec>
    </Script>
\end{minted}

The node CalibrationSpec can have one subnode Calibration per Index occuring in the script. For each index a list of
calibration strikes can be specified. The list should start with the most important calibration strike and continue with
strikes of decreasing importance. In the example (which could be a typcial setup for barrier option) the most important
strike is given by the Strike variable and a secondary strike is given by the KnockOutLevel.

The usage of the calibration strikes is twofold:

\begin{itemize}
\item For the determination of calibration strikes. This is only relevant / supported by the BlackScholes and
  GaussianCam-FD model, which will use the first strike from the list to read the volatility from the market term
  structure (if Calibration is set to Deal in the model parameters).
\item To determine concentration points for an FD mesher if \verb+Engine+ is set to \verb+FD+. The first $n$ strikes are
  used as concentration points where $n$ is the minimum of specified strikes and the engine parameter
  \verb+MesherMaxConcentrationPoints+. This is supported by the BlackScholes model only.
\end{itemize}

\subsection{FX tags and correlation curves}\label{fxtags_correlationcurves}

FX indices contain a ``tag'', e.g. we can have two different EURUSD indices FX-ECB-EUR-USD and FX-CLOSE-EUR-USD with
different historical fixings. This is respected in the script engine, i.e. historical fixings are retrieved using the
exact FX index name including the tag.

For projection on the other hand we only have one set of market data per currency pair (FX Spot, forwarding curves and
volatilities). This also holds for implied correlations, therefore correlation curves should be set up using the tag
``GENERIC'' always, e.g. the correlation between the pairs EUR-USD and JPY-USD shold be set up as
FX-GENERIC-JPY-USD:FX-GENERIC-EUR-USD. The ordering of the indices in the pair follow the usual rules, i.e.

\begin{itemize}
\item COM $<$ EQ $<$ FX $<$ IBOR $<$ CMS
\item smaller period $<$ greater period for IBOR and CMS indices
\item alphabetical order of index names for COM, EQ indices resp. of CCY1 then CCY2 for FX indices
\end{itemize}
