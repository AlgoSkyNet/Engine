Indices are defined as

\begin{minted}[fontsize=\footnotesize]{xml}
      <Index>
        <Name>Underlying</Name>
        <Value>EQ-RIC:.SPX</Value>
      </Index>
\end{minted}

in case of a scalar and

\begin{minted}[fontsize=\footnotesize]{xml}
      <Index>
        <Name>Underlyings</Name>
        <Values>
          <Value>EQ-UND1</Value>
          <Value>EQ-UND2</Value>
          <Value>EQ-UND3</Value>
          <Value>EQ-UND4</Value>
          <Value>EQ-UND5</Value>
        </Values>
      </Index>
\end{minted}

in case of a vector. Currently, Equity, FX, Commodity, Interest Rate and Inflation indices as well as generic indices are
supported, the naming convention follows the standard ORE conventions, i.e.

\begin{itemize}
\item Equity: These are declared based on identifier type \begin{itemize}
    \item \verb+EQ-ISIN:{Name}:{Currency}:{Exchange}+ for ISIN equities,\\e.g.\ \verb+EQ-ISIN:NL0000852580:EUR:XAMS+,
    \item \verb+EQ-RIC:{Name}+ for RIC equities, e.g.\ \verb+EQ-RIC:.SPX+,
    \item \verb+EQ-FIGI:{Name}+ for FIGI equities, e.g.\ \verb+EQ-FIGI:BBG000BLNNV0+,
    \item \verb+EQ-BBG:{Name}+ for BBG equities, e.g.\ \verb+EQ-BBG:BARC LN Equity+.
  \end{itemize}
\item FX: \verb+FX-SOURCE-CCY1-CCY2+ with foreign currency \verb+CCY1+ and domestic currency \verb+CCY2+\ifdefined\STModuleDoc\else, as in Table \ref{tab:fxindex_data}\fi.
  The order of the currencies here is important as they will have different meanings. For a natural payoff, \verb+CCY2+ must correspond
  to the instrument's payment currency. Otherwise, we have a quanto payoff. See section \ref{sss:payccy_st} for more information.
\item Commodity: These are declared as \verb+COMM-{Name}+, with possible extensions as follows (see below also for
  more information on the meanings of the extensions to the commodity indices): \begin{itemize}
    \item \verb+COMM-NYMEX:CL+ (spot),
    \item \verb+COMM-NYMEX:CL-2020-09+ (future with expiry 01 Sep 2020),
    \item \verb+COMM-NYMEX:CL-2020-09-15+ (future with expiry 15 Sep 2020).
  \end{itemize}
\item Interest Rate: These are declared as \verb+CCY-INDEX-TENOR+, e.g.\ \verb+EUR-EURIBOR-6M+, \verb+EUR-CMS-10Y+.
  \ifdefined\STModuleDoc\else See Table \ref{tab:indices} for a list of allowable IR indices. \fi
\item Inflation: These are declared as \verb+{Name}+, with possible extensions as follows(see below also for more
  information on the meanings of the extensions to the inflation indices): \begin{itemize}
    \item \verb+EUHICPXT+ (constant/non-interpolated),
    \item \verb+EUHICPXT#F+ (flat interpolation),
    \item \verb+EUHICPXT#L+ (linear interpolation).
  \end{itemize}
\item Generic: These are declared as \verb+GENERIC-{Name}+.
\end{itemize}

Notice that generic indices only provide historical fixings.

\ifdefined\STModuleDoc
\else
% additional information on index name conventions
For additional information on the equity name conventions see \ref{tab:equity_credit_data} and \ref{ss:underlying}. In
general the equity name can be represented using the string representation {IdentifierType}:{Name}:{Currency}:{Exchange}
for the underlying. \\
For additional information on the fx index format see \ref{tab:fxindex_data}. \\
For additional information on the commodity names see \ref{tab:commodity_data}.
\fi

Within the scripting framework there are additional ways to reference commodity indices in a more dynamical fashion:
When evaluating a commodity index using the index evaluation operator (see \ref{index_evalop}) as in

\begin{minted}[fontsize=\footnotesize]{text}
 CommodityUnderlying(ObservationDate)
\end{minted}

for

\begin{itemize}
\item a variable of the form \verb+COMM-{Name}#N+, the index will be resolved to the $N+1$th future with expiry greater than
  ObservationDate for the given commodity underlying, $N\geq0$. The parameter $N$ corresponds to the field
  \verb+FutureMonthOffset+ commonly used in commodity trade xml schemas
\item the above form can take one or two additional parameters \verb+COMM-{Name}#N#D+ or \verb+COMM-{Name}#N#D#{Cal}+ where
  $D$ corresponds to \verb+DeliveryRollDays+ and \verb+Cal+ is the calendar used to roll the observation date forward
  before the next future expiry is looked up. If \verb+Cal+ is not given, it defaults to the null calendar (no
  holidays).
\item a variable of the form \verb+COMM-{Name}!N+, the index will be resolved to the $N$th future relative to the month and
  year of the ObservationDate.
\end{itemize}

In general, if a commodity future is referenced, the observation date should be be less or equal to the expiry date of
the future, since no historical fixing will in general be available after the future expiry date. In the future-looking
simulation the observation of an expired future is possible, in this case the (model) value of the future is kept
constant at its value at the expiry date for observation dates after the expiry date.

When evaluating an inflation index as in

\begin{minted}[fontsize=\footnotesize]{text}
 CPIIndex(FixingDate)
\end{minted}

the variable \verb+CPIIndex+ can be given in the following ways:

\begin{itemize}
\item a variable of the form \verb+EUHICPXT+ will evaluate the standard ORE inflation index, which is
  non-interpolated. This means the result will be constant for all FixingDates in an inflation period (i.e.\ usually
  constant for a calendar month)
\item a variable of the form \verb+EUHICPXT#F+, i.e.\ an ORE inflation index name followed by \verb+#F+ indicating a flat
  interpolation of the index, this form is equivalent to the previous form making the interpolation mode explicit as
  ``flat''
\item a variable of the form \verb+EUHICPXT#L+ where the suffix \verb+#L+ indicates linear interpolation of the index,
  i.e.\ the result will be the interpolated fixing between the usually monthly inflation index fixings
\end{itemize}

The use of the interpolation extensions is deprecated. 
Interpolation of fixings should be handled in the payoff script. The CPI indices are forecasting a flat fixing for given the inflation period.
The use of interpolation is still supported but will be eventually removed in a later release.