% no subsection numberings in product catalogue
\ifdefined\STModuleDoc
\newcommand{\stsubsection}{\subsection}
\else
\newcommand{\stsubsection}{\subsection*}
\fi

% ====================================================
\stsubsection{Whitespace}
% ====================================================

Whitespace (space, tab, return, newline) is ignored during the parsing. All variable identifiers and keywords are case
sensitive.

% ====================================================
\stsubsection{Keywords}\label{keywords}
% ====================================================

The language uses keywords and predefined function names as listed in table \ref{tab:keywords} which may may not be used as
variable identifiers.

\begin{table}[!htbp]
  \begin{tabular}{l | l}
    Keyword & Context\\ \hline
    \verb+IF+ & Control Flow \\
    \verb+THEN+ & \\
    \verb+ELSE+ & \\
    \verb+END+ & \\
    \verb+FOR+ & \\
    \verb+IN+ & \\
    \verb+DO+ & \\ \hline
    \verb+NUMBER+ & Type Identifiers \\ \hline
    \verb+OR+ & Logical Operators \\
    \verb+AND+ & \\ \hline
    \verb+abs+ & Functions\\
    \verb+exp+ & \\
    \verb+ln+ & \\
    \verb+sqrt+ & \\
    \verb+normalCdf+ & \\
    \verb+normalPdf+ & \\
    \verb+max+ & \\
    \verb+min+ & \\
    \verb+pow+ & \\
    \verb+black+ & \\
    \verb+dcf+ & \\
    \verb+days+ & \\ \hline
    \verb+PAY+ & Model dependent functions\\
    \verb+LOGPAY+ & \\
    \verb+NPV+ & \\
    \verb+NPVMEM+ & \\
    \verb+HISTFIXING+ & \\
    \verb+DISCOUNT+ & \\
    \verb+FWDCOMP+ & \\
    \verb+FWDAVG+ & \\
    \verb+ABOVEPROB+ & \\
    \verb+BELOWPORB+ & \\ \hline
    \verb+SIZE+ & Other Statements \\
    \verb+DATEINDEX+ & \\
    \verb+REQUIRE+ & \\
    \verb+SORT+ & \\
    \verb+PERMUTE+ & \\
  \end{tabular}
  \caption{Reserved keywords.}
  \label{tab:keywords}
\end{table}

In addition the following variable identifiers are automatically populated with special values when running the script
engine on a trade:

\begin{itemize}
\item \verb+TODAY+: the current evaluation date
\end{itemize}

% ====================================================
\stsubsection{Variables}
% ====================================================

Variables that can be used in the script are either

\begin{itemize}
\item externally defined variables, defined in the data node of the trade xml representation
\item variables local to the script, declared within the script
\end{itemize}

Externally defined variables are protected from being modified by the script. All variables used within the script must
be either externally defined or declared at the top of the script using

\begin{minted}[fontsize=\footnotesize]{text}
  NUMBER continuationValue, exerciseValue, x[10];
\end{minted}

which declares two scalars \verb+continuationValue+ and \verb+exerciseValue+ and an array \verb+x+ of size $10$ (see
\ref{arrays} for more details on arrays). The only exemption to this rule is the variable declared in the \verb+NPV+
node of the script, which is definied implicitly as a scalar number.

Notice that within the script only variables of type Number can be declared.\footnote{this restriction allows the static
  analysis of the script\ifdefined\STModuleDoc, see \ref{static_analysis}\fi
} All variables are initialised with $0$. The scope of a variable
declaration is always global to the script, multiple declarations of the same variable name are forbidden.

Variable identifiers are subject to the following restrictions

\begin{itemize}
\item must start with a character, then characters or numbers or underscores may follow
\item no other special characters, no keywords or predefined functions allowed (see \ref{keywords})
\item e.g. {\tt x} or {\tt x\_23}, {\tt aValue} are valid identifiers
\item identifiers starting with an underscore are technically allowed as well, but reserved for special use cases (e.g
  \verb+_AMC_SimDates+ and \verb+_AMC_NPV+ for AMC exposure generation)
\end{itemize}

% ====================================================
\stsubsection{Arrays, SIZE operator}\label{arrays}
% ====================================================

Arrays are declared by specifying the size of the array in sqaure brackets, e.g.

\begin{minted}[fontsize=\footnotesize]{text}
  NUMBER x[10], y[SIZE(ObservationDates)], z[5+3*v];
\end{minted}

declares arrays

\begin{itemize}
\item \verb+x+ of size $10$
\item \verb+y+ with the same size as the array \verb+ObservationDates+
\item \verb+z+ with size $5+3v$ where v is a number variable
\end{itemize}

Once an array is delcared its size can not be changed. The $i$th element of an array $a$ is accessed by \verb+a[i]+,
where $i$ is an expression evaluating to a number. Here $i=1,2,3,\ldots,n$, where $n$ is the fixed size of the array,
i.e. the subscripts start at $1$ (as opposed to $0$ as in some other languages).

The size of an array \verb+a+ can be evaluated by \verb+SIZE(a)+. Only one dimensional arrays are supported. The array
subscript must be {\em deterministic}, e.g.

\begin{minted}[fontsize=\footnotesize]{text}
  IF Underlying > Strike THEN
    i = 1;
  ELSE
    i = 2;
  END;
  Payoff = y[i];
\end{minted}
  
is illegal since $i$ in general will be path-dependent, but
  
\begin{minted}[fontsize=\footnotesize]{text}
  IF Underlying > Strike THEN
    Payoff = y[1];
  ELSE
    Payoff = y[2];
  END;
\end{minted}

is valid.\footnote{the background is the simplicity and performance of the engine implementation}

% ====================================================
\stsubsection{Sorting Arrays: SORT and PERMUTE instructions}\label{sorting_arrays}
% ====================================================

Given an array \verb+x+ of number type the statement

\begin{minted}[fontsize=\footnotesize]{text}
  SORT (x);
\end{minted}

will sort the array (pathwise) in ascending order. The statement

\begin{minted}[fontsize=\footnotesize]{text}
  SORT (x, y);
\end{minted}

will write a sorted version of \verb+x+ to \verb+y+ and leave \verb+x+ unchanged. The array \verb+y+ must be of number
type and have the same size as \verb+x+. The array \verb+y+ can also be equal to \verb+x+, the statement
\verb+SORT(x,x);+ is equivalent to \verb+SORT(x);+. Finally the statement

\begin{minted}[fontsize=\footnotesize]{text}
  SORT (x, y, p);
\end{minted}

will write a sorted version of \verb+x+ to \verb+y+ and populate another array \verb+p+ with indices $1,\ldots,\verb+SIZE+(x)$
such that $x[p[1]], \ldots, x[p[n]]$ is sorted. Here \verb+p+ must be an array with the same size as \verb+x+ and of
number type.

A permutation \verb+p+ generated as above (or set up otherwise) can be used to sort an unrelated array \verb+z+ using

\begin{minted}[fontsize=\footnotesize]{text}
  PERMUTE (z, p);
\end{minted}

which will reorder the values of \verb+z+ as $z[1] \rightarrow z[p[1]]$, $z[2] \rightarrow z[p[2]]$ ... etc. The statement

\begin{minted}[fontsize=\footnotesize]{text}
  PERMUTE (z, w, p);
\end{minted}

will do the same, but write the result to \verb+w+ and leave \verb+z+ untouched.

% ====================================================
\stsubsection{Function {\tt DATEINDEX}}\label{function_dateindex}
% ====================================================

Given an array \verb+a+ and a single date \verb+d+, the expression

\begin{minted}[fontsize=\footnotesize]{text}
  DATEINDEX(d, a, EQ)
\end{minted}

returns $0$ if the date $d$ is not found in the array $a$ and otherwise the (first) index $i$ for which \verb+a[i]+
equals \verb+d+. The variable $d$ is required to be of type event. The variable $a$ is only required to be an array, if
the type of its elements are not event, the return value will always be zero indicating that $d$ was not found in
$a$. Similarly,

\begin{minted}[fontsize=\footnotesize]{text}
  DATEINDEX(d, a, GEQ)
\end{minted}

returns the index of the earliest date in $a$ that is greater or equal than $d$, and

\begin{minted}[fontsize=\footnotesize]{text}
  DATEINDEX(d, a, GT)
\end{minted}

returns the index of the earliest date in $a$ that is greater than $d$. If no such dates exists for \verb+GEQ+ or
\verb+GT+, the size of $a$ plus 1 will be returned.

% ====================================================
\stsubsection{Instructions}
% ====================================================

A typical script comprises a sequence of instructions, each one terminated by {\tt ;}.

% ====================================================
\stsubsection{Index evaluation}\label{index_evalop}
% ====================================================

Given an variable \verb+index+ of type Index its historical or projected fixing at a date $d$ is evaluated using the
expression \verb+index(d)+. This is applicable to all index types. For example

\begin{minted}[fontsize=\footnotesize]{text}
 Underlying(ObservationDate)
\end{minted}

evaluates the index assigned to the variable Underlying at the date assigned to the variable ObservationDate. For FX,
EQ, IR and COMM Spot indices this corresponds to a a fixing at the observation date in the usual sense. For COMM Future
indices it is the observed future price at the observation date.

For INF indices the argument is the actual fixing date, which due to availability lags is observed at a later simulation
time in models with dynamical inflation simulation. For example in the GaussianCam model, this lag is defined as the
number of calendar days from the zero inflation term structure base date to its reference date (adjusted to the first
date of the inflation period to be consistent with the same adjustment applied to the base date). This means that when
observing an inflation index at a fixing date $d$, the model state at $d+\text{lag}$ is used to make this observation.

The extended syntax

\begin{minted}[fontsize=\footnotesize]{text}
 Underlying(ObservationDate, ForwardDate)
\end{minted}

evaluates the projected fixing for ForwardDate as seen from ObservationDate.

This is applicable to FX, EQ, IR, INF and COMM Spot indices, but not to COMM Future indices, since for the latter the
two concepts coincide (for ForwardDate < FutureExpiry). If a forward date is given for the observation of a COMM future
index, no error is thrown, but it will be ignored.

For inflation indices, the ForwardDate will be the actual fixing date again and the ObservationDate will be using a
lagged state as explained above.

The ForwardDate must be greater or equal than the ObservationDate. If the ForwardDate is strictly greater than the
ObservationDate the ObservationDate must not be a past date (for inflation indices it must not lie before the inflation
term structure's base date), since the computation of projected fixings for past dates would involve the knowledge of
past curves, i.e. past market data.

Notice also the further specifics of commodity and inflation indices in \ref{data_index}.

% ====================================================
\stsubsection{Comparisons {\tt==} and {\tt!=}}
% ====================================================

Compares two values, e.g. {\tt x==y} or {\tt x!=y}. This is applicable to all types. For a number the interpretation is
``numerically equal''.

% ====================================================
\stsubsection{Comparisons {\tt<}, {\tt<=}, {\tt>}, {\tt>=}}
% ====================================================

Compares two values {\tt x<y}, {\tt x<=y}, {\tt x>y}, {\tt x>=y}. Applicable to numbers and events, but not to
currencies or indices. For numbers the interpretation is ``less than, but not numerically equal'', ``less than or
numerically equal'', etc.

% ====================================================
\stsubsection{Operations {\tt+}, {\tt-}, {\tt*}, {\tt/}}
% ====================================================

Arithmetic operations {\tt x+y}, {\tt x-y}, {\tt x*y}, {\tt x/y}, applicable to numbers only.

% ====================================================
\stsubsection{Assignment {\tt =}}
% ====================================================

Assignment {\tt x = y}, only allowed for numbers within the script.

% ====================================================
\stsubsection{Logical Operators AND, OR, NOT}
% ====================================================

Connects results of comparisons or other logical expressions:

\begin{itemize}
\item {\tt x<y AND z!=0}
\item {\tt x<y OR z!=0}
\item {\tt NOT(x==y)}
\item AND has higher precedence than OR, e.g.
\item {\tt x<y AND y<z OR z!=0} same as {\tt \{x<y AND y<z\} OR z!=0}, but
\item {\tt x<y AND \{y<z OR z!=0\}} requires parenthesis
\item better always use parenthesis when mixing {\tt AND} / {\tt OR}
\end{itemize}

% ====================================================
\stsubsection{Conditionals: IF ... THEN ... ELSE ...}
% ====================================================

Conditional execution can be written as

\begin{minted}[fontsize=\footnotesize]{text}
  IF condition THEN
      ... if-body ...
  ELSE
      ... else-body ...
  END
\end{minted}

Examples:

\begin{minted}[fontsize=\footnotesize]{text}
  IF x == y THEN
    z = PAY(X,d,p,ccy);
    w = 1;
  END;
\end{minted}
  
\begin{minted}[fontsize=\footnotesize]{text}
  IF x == y THEN
    z = PAY(X,d,p,ccy);
  ELSE
    z = 0;
    w = 0;
  END;
\end{minted}

where the \verb+ELSE+ part is optional. The body can comprise one ore more instructions, each of which must be
terminated by {\tt ;}.

% ====================================================
\stsubsection{Loops: FOR ... IN ... DO}
% ====================================================

Loops are written as

\begin{minted}[fontsize=\footnotesize]{text}
  FOR i IN (a,b,s) DO
  ... body ...
  END
\end{minted}

where \verb+i+ is a number variable identifier, and \verb+a+, \verb+b+, \verb+s+ are expressions that yield a result of
type Number. The variable \verb+i+ must have been declared in the script before it can be used as a loop variable. The
code in the body is executed for the values $i=a, a+s, \ldots$ until $a+ks>b$ if $s>0$ or $a+ks<b$ if $s<0$ for some
integer $k>0$. All values $a,b,s$ must be integers and $s\neq 0$.

Example:

\begin{minted}[fontsize=\footnotesize]{text}
  NUMBER i,x;
  FOR i IN (1,100.1) DO x = x + i; END;
\end{minted}

Here {\tt a, b} must be deterministic, {\tt i} must not be modified in the loop body. If a or b are modified in the loop
body, still the initial values read at the start of the loop are used. The loop body can comprise one or more
instructions, each of which must be terminated by {\tt ;}.

% ====================================================
\stsubsection{Special variable: TODAY}\label{todayvar}
% ====================================================

A constant event variable, set to the current evaluation date. This can e.g. be used to restrict exercise decisions to
future dates, see \ref{function_npv} for an example.

% ====================================================
\stsubsection{Checks: REQUIRE}
% ====================================================

If the condition {\tt C} is not true, a runtime error is thrown. Examples:
 
\begin{itemize}
\item {\tt REQUIRE SIZE(ExerciseDates) == SIZE(SettlementDates);}
\item {\tt REQUIRE SIZE(Underlyings) == 2;}
\item {\tt REQUIRE Strike >= 0;}
\end{itemize}

% ====================================================
\stsubsection{Functions {\tt min}, {\tt max}, {\tt pow}}
% ====================================================

Binary functions {\tt min(x,y)}, {\tt max(x,y)}, {\tt pow(x,y)}, applicable to numbers only.

% ====================================================
\stsubsection{Functions {\tt -}, abs, exp, ln, sqrt}
% ====================================================

Unary functions {\tt -x}, {\tt abs(x)}, {\tt exp(x)}, {\tt ln(x)}, {\tt sqrt(x)}, applicable to numbers only.

% ====================================================
\stsubsection{Functions {\tt normalPdf}, {\tt normalCdf}}
% ====================================================

Returns the standard normal pdf $\phi(x)$ resp. cdf $\Phi(x)$, applicable to numbers only.

% ====================================================
\stsubsection{Function {\tt black}}
% ====================================================

Implements the black formula {\tt black(omega, obs, expiry, k, f, sigma)} with

\begin{eqnarray*}
  \text{black} &=&  \omega\cdot(f \Phi(\omega d_1) -k \Phi(\omega d_2)) \\
  d_{1,2} &=& \frac{\ln(f/k) \pm \frac{1}{2} \sigma^2t}{\sigma\sqrt{t}}
\end{eqnarray*}
    
where $t$ is the (model's) year fraction between obs and expiry date, i.e.:
\begin{itemize}
\item {\tt omega} is $1$ (call) or $-1$ (put)
\item {\tt obs, expiry} are the observation / expiry dates
\item {\tt k, f} are the strike and the forward
\item {\tt sigma} is the implied volatility
\item notice that no discounting is applied
\end{itemize}

% ====================================================
\stsubsection{Function {\tt dcf}}
% ====================================================

The expression \verb+dcf(dc, d1, d2)+ returns the day count fraction for a day count convention \verb+dc+ and a period
defined by dates \verb+d1+ and \verb+d2+.

% ====================================================
\stsubsection{Function {\tt days}}
% ====================================================

The expression \verb+days(dc, d1, d2)+ returns the number of days between \verb+d1+ and \verb+d2+ for a day count
convention \verb+dc+.

% ====================================================
\stsubsection{Function {\tt PAY}}\label{function_pay}
% ====================================================

The expression {\tt PAY(X, d, p, ccy)} calculates a discounted payoff for an amount $X$ observed at a date $d$, paid at a
date $p$ in currency ccy, i.e.
  
\begin{equation}
  \frac{X P_{ccy}(d,p) FX_{\text{ccy},\text{base}}(d)}{N(d)}
\end{equation}

where
\begin{itemize}
\item here $P_{ccy}$ is the discount factor in currency ccy, $FX$ is the FX spot from ccy to base and $N$ is the model
  numeraire
\item $d\leq p$ must hold
\item if $p$ lies on or before the evaluation date, the result is zero; $X$ is not evaluated in this case. 
Note that $X$ is evaluated in the LOGPAY function though, see \ref{function_logpay}.
\item avoids reading non-relevant past fixings from the index history
\item if $d$ lies before (but $p$ after) the evaluation date, it is set to the evaluation date, i.e. the result is
  computed as of the evaluation date
\end{itemize}

% ====================================================
\stsubsection{Function {\tt LOGPAY}}\label{function_logpay}
% ====================================================

The expression {\tt LOGPAY(X, d, p, ccy)} has the same meaning as {\tt PAY(X, d, p, ccy)} (see \ref{function_pay}) but
as a side effect populates an internal cashflow log that is used to generate expected flows. The generated flow is

\begin{equation}
  \frac{ N(0) E\left(\frac{X P_{ccy}(d,p) FX_{\text{ccy},\text{base}}(d)}{N(d)}\right) }{ FX_{\text{ccy},\text{base}}(0) P_{ccy}(0,p) }
\end{equation}

which ensures that the flows discounted on T0 curves and converted with T0 FX Spots reproduce the NPV generated from
\verb+LOGPAY+ expressions.

There is a second form {\tt LOGPAY(X, d, p, ccy, legNo, type)} taking in addition

\begin{itemize}
\item a leg number \verb+legNo+, which must evaluate to a determinisitc number
\item a cashflow type \verb+type+, which is an arbitrary string meeting the conventions for variable names
\end{itemize}

This additional information is used to populate the ORE cashflow report. If not given, \verb+legNo+ is set to $0$ and
type is set to \verb+Unspecified+. Notice that cashflows will equal pay dates, pay currencies, leg numbers and types are
aggregated to one number in the cashflow report.

A third form {\tt LOGPAY(X, d, p, ccy, legNo, type, slot)} takes an additional parameter \verb+slot+ which must evaluate
to a whole positive and deterministic number $1,2,3,\ldots$. If several cashflows are logged into the same slot,
previous results are overwritten. This is useful for scripts where tentative cashflows are generated that are later on
superseded by other cashflows (e.g. for an American option).

Examples for the three forms are given below:

\begin{minted}[fontsize=\footnotesize]{text}
  Payoff1 = LOGPAY( Notional * fixedRate, PayDate, PayDate, PayCcy);
  Payoff2 = LOGPAY( Notional * fixedRate, PayDate, PayDate, PayCcy, 2, Interest);
  Payoff3 = LOGPAY( Notional * fixedRate, PayDate, PayDate, PayCcy, 2, Interest, 3);
\end{minted}

Here, Payoff1 will appear under leg number $0$ and flow type ``Unspecified'' in the cashflow report. Payoff2 will appear
under leg number $2$ and flow Type ``Interest''. The same holds for Payoff3, but if any amounts were booked using the
slot parameter $3$ previously they will be overwritten with the current amount.

Note: Even if $p$ lies on or before the evaluation date, a cashflow entry will be generated. If one wants to explicitly
avoid that, e.g. because past fixings are not available to evaluate the past paid amounts, it can be done with an IF
construct comparing the pay date with TODAY.

% ====================================================
\stsubsection{Function {\tt NPV, NPVMEM}}\label{function_npv}
% ====================================================

The expression {\tt NPV(X, d, [C], [R1], [R2])} calculates a conditional NPV of an amount $X$ conditional on a date $d$, i.e.

\begin{equation}\label{condexp}
  E( X\: | \mathcal{F}_d \cap \mathcal{F}_C )
\end{equation}

where $\mathcal{F}_d$ is the sigma algebra representing the information generated by the model up to $d$ and
$\mathcal{F}_C$ represents the additional condition $C$ (if given). In an MC model \ref{condexp} is computed using a
regression against the model state at $d$. $C$ can be used to filter the training paths, e.g. on ITM paths only. $d$
must not lie before the evaluation date, but for convenience the scipt engines will treat $d$ as if it were equal to the
evaluation date in this case for the purpose of the NPV function evaluation.

The regressor can be enriched by (at most 2) additional variables $R_i$. A typical usage is the accumulated coupon in a
target redemption feature which heavily influences the future conditional NPV but is not captured in the model state.

A typical usage of the NPV function is to decide on early exercises in the Longstaff-Schwartz algorithm:

\begin{minted}[fontsize=\footnotesize]{text}
NUMBER Payoff, d;
FOR d IN (SIZE(ExerciseDates), 1) DO
    IF ExerciseDates[d] > TODAY THEN
      Payoff = PAY( PutCall * (Underlying(ExerciseDates[d]) - Strike),
                    ExerciseDates[d], ExerciseDates[d], PayCcy);
      IF Payoff > 0 AND Payoff > NPV( Option, ExerciseDates[d], Payoff > 0) THEN
        Option = Payoff;
      END;
    END;
END;
Option = LongShort * Quantity * Option;
\end{minted}

Here \verb+TODAY+ represents the evaluation date to ensure that only future exercise dates are evaluated, see
\ref{todayvar}.

{\em Note: It is the users responsibility to use NPV() correctly to a certain extend: An example would be that $X$ is
  composed from both past and future fixings w.r.t. the observation time $t$. In that case only the future fixings
  should be included in the argument of NPV(), whereas the past fixings are known and should just be added to the result
  of NPV().}

The variant {\tt NPVMEM(X, d, s, [C], [R1], [R2])} works exactly like {\tt NPV(X, d, [C], [R1], [R2])} except that it
takes an additional parameter \verb+s+ that must be an integer. If \verb+NPVMEM()+ is called more than once for the same
parameter \verb+s+ a regression model representing the conditional npv will only be trained once and after that the
trained model will be reused. The usual use case is for scripts used in combination with the AMC module where a
regression model will be trained on a relative large number of paths (specified in the pricing engine configuration) and
then reused in the global exposure simulation on a relatively small number of paths (specified in the xva simulation
setup).

% ====================================================
\stsubsection{Function {\tt HISTFIXING}}\label{function_histfixing}
% ====================================================

The expression {\tt HISTFIXING(Underlying, d)} returns $1$ if $d$ lies on or before the reference date {\em and} the
underlying has a historical fixing as of the date $d$ and $0$ otherwise.

% ====================================================
\stsubsection{Function {\tt DISCOUNT}}\label{function_discount}
% ====================================================

The expression {\tt DISCOUNT(d, p, ccy)} calculates a discount factor $P_{ccy}(d,p)$ as of $d$ for $p$ in currency
$ccy$. Here $d$ must not be a past date and $d\leq p$ must hold.

% ====================================================
\stsubsection{Functions {\tt FWDCOMP} and {\tt FWDAVG}}\label{function_fwdcomp}
% ====================================================

The {\tt FWDCOMP()} and {\tt FWDAVG()} functions are used to calculate a daily compounded or averaged rate over a
certain period based on an overnight index such as USD-SOFR, GBP-SONIA, EUR-ESTER etc..

The rate is estimated as seen from an observation date looking {\em forward} from that date, even if fixings relevant
for the rate lie in the past w.r.t. the observation date. In the latter case, an approximation to the true rate which is
then dependent on the path leading from TODAY to the current model state at the observation date is calculated. This
approximation is model-dependent. The only exception to this mechanics are historical fixings that are {\em known} as of
TODAY. Such fixings are always taken into consideration with their true value.

More specifically, the {\tt FWDCOMP()} and {\tt FWDAVG()} functions take the following parameters. The parameters must
be given in that order, and all parameters must be given in sequence up to the parameter ``end'' (last mandatory
parameter) or the end of an optional parameter group (i.e. an optional parameter group must be given as a
whole). Furthermore, all parameters must be deterministic.

\begin{itemize}
\item index [mandatory]: an overnight index \verb+index+, e.g. EUR-EONIA, USD-SOFR, ...
\item obs [mandatory]: an observation date obs $\leq$ start; if obs is $<$ TODAY it is set to TODAY, i.e. the result is
  as of TODAY in this case
\item start [mandatory]: the value start date, this might be modified by a non-zero lookback
\item end [mandatory]: the value end date, this might be modified by a non-zero lookback
\item spread [optional group 1]: a spread, defaults to $0$ if not given
\item gearing [optional group 1]: a gearing, defaults to $1$ if not given
\item lookback [optional group 2]: a lookback period given as number of days, defaults to $0$ if not given. This
  argument must be given as either a constant number or a plain variable, i.e. not as a more complex expression than
  either of these.
\item rateCutoff [optional group 2]: a rate cutoff given as number of days, defaults to $0$ if not given
\item fixingDays [optional group 2]: the fixing lag given as number of days, defaults to $0$ if not given. This argument
  must be given as either a constant number or a plain variable, i.e. not as a more complex expression than either of
  these.
\item includeSpread [optional group 2]: a flag indicating whether to include the spread in the compounding, a value equal to $1$ indicates 'true', $-1$ false, defaults to 'false' if not given
\item cap [optional group 3]: a cap value, defaults to $999999$ (no cap) if not given
\item floor [optional group 3]: a floor value, defaults to $-999999$ (no floor) if not given
\item nakedOption [optional group 3]: a flag indicating whether the embedded cap / floor should be estimated, a value equal to $-1$ indicates 'false' (capped / floored coupon rate is estimated), $1$ 'true' (embedded cap / floor rate is estimated), defaults to 'false' if not given
\item localCapFloor [optional group3]: a flag indicating whether the cap / floor is local, a value equal to $-1$ indicates 'false', $1$ 'true', defaults to 'false' if not given.
\end{itemize}

Based on these parameters a rate corresponding to that computed for a vanilla floating leg is estimated, see the
description in section ``Floating Leg Data, Spreads, Gearings, Caps and Floors'' for more details on this.

% ====================================================
\stsubsection{Functions {\tt ABOVEPROB, BELOWPROB}}\label{function_aboveprob}
% ====================================================

These functions are only available in Monte Carlo engines. The expression

\begin{minted}[fontsize=\footnotesize]{text}
  ABOVEPROB(underlying, d1, d2, U)
\end{minted}

returns the pathwise probability that the value of an index \verb+underlying+ lies at or above a number $U$ for at least
one time $t$ between dates $d1$ and $d2$ conditional on the underlying taking the simulated path values at $d1$ and
$d2$. The probability is by definition computed assuming a continuous monitoring. Similarly,

\begin{minted}[fontsize=\footnotesize]{text}
  BELOWPROB(underlying, d1, d2, D)
\end{minted}

returns the probability that the value of the underlying lies at or below $D$. Notice that $d1$ and $d2$ should be
adjacent simulation dates to ensure that the results computed in the script are meaningful. This means the script should
not evaluate the underlying at a date $d$ with $d1 < d < d2$.

We note that $U$ and $D$ are not required to be deterministic quantities, although the common use case will probably be
to have path-independent inputs.

Finally, if $d1>d2$ both functions return $0$.
