\subsubsection{Equity Option Position}
\label{ss:equity_option_position}

An equity option position represents a position in a single equity option - using a single \lstinline!Underlying! node,
or in a weighted basket of underlying equity options - using multiple \lstinline!Underlying! nodes.

An Equity Option Position can be used both as a stand alone trade type (TradeType: \emph{EquityOptionPosition}) or as a
trade component ({\tt EquityOptionPositionData}) used within the \emph{TotalReturnSwap} (Generic TRS) trade type, to set
up for example Equity Option Basket trades.

It is set up using an {\tt EquityOptionPositionData} block as shown in listing \ref{lst:equityoptionpositiondata}. The
meanings and allowable values of the elements in the block are as follows:

\begin{itemize}
\item Quantity: The number of options written on one underlying share resp. the number of units of the option basket
  held. \\
  Allowable values: Any positive real number
\item Underlying: One or more underlying descriptions, each comprising an \lstinline!Underlying! block, an \lstinline!Optiondata! block and a \lstinline!Strike! element, in that order:
  \begin{itemize}
  \item Underlying: an underlying description, see \ref{ss:underlying}, only equity underlying are allowed
  \item OptionData: the option description, see \ref{ss:option_data}, the relevant / allowed data is
    \begin{itemize}
    \item LongShort: the type of the position,\emph{long} and \emph{Short} positions are allowed. Note that negative weights are allowed. A \emph{long} position with a negative weight results in a \emph{short} position, and a \emph{short} position with a negative weight results in a \emph{long} position.
    \item OptionType: \emph{Call} or \emph{Put}
    \item Style: \emph{European} or \emph{American}
    \item Settlement: \emph{Cash} or \emph{Physical}
    \item ExerciseDates: exactly one exercise must be given representing the European exercise date or the last American
      exercise date
    \end{itemize}
  \item Strike: the strike of the option. Allowable values are non-negative real numbers.
  \end{itemize}
\end{itemize}

If a basket of equities is defined, the \verb+Weight+ field should be populated for each underlying. The weighted basket
price is then given by
$$\text{Basket-Price} = \text{Quantity} \times \sum_i \text{Weight}_i \times p_i \times \text{FX}_i$$
where
\begin{itemize}
\item $p_i$ is the price of the ith option in the basket, written on one underlying share
\item $FX_i$ is the FX Spot converting from the ith equity currency to the first equity currency which is by definition
  the currency in which the npv of the basket is expressed.
\end{itemize}

\begin{listing}[H]
\begin{minted}[fontsize=\footnotesize]{xml}
<Trade id="EquityOptionPositionTrade">
  <TradeType>EquityOptionPosition</TradeType>
  <EquityOptionPositionData>
    <!-- basket price = quantity x sum_i ( weight_i x equityOptionPrice_i x fx_i ) -->
    <Quantity>1000</Quantity>
    <!-- option #1 -->
    <Underlying>
      <Underlying>
        <Type>Equity</Type>
        <Name>.SPX</Name>
        <Weight>0.5</Weight>
        <IdentifierType>RIC</IdentifierType>
      </Underlying>
      <OptionData>
        <LongShort>Long</LongShort>
        <OptionType>Call</OptionType>
        <Style>European</Style>
        <Settlement>Cash</Settlement>
        <ExerciseDates>
          <ExerciseDate>2021-01-29</ExerciseDate>
        </ExerciseDates>
      </OptionData>
      <Strike>3300</Strike>
    </Underlying>
    <!-- option #2 -->
    <Underlying>
      <Underlying>
        <Type>Equity</Type>
        <Name>.SPX</Name>
        <Weight>0.5</Weight>
        <IdentifierType>RIC</IdentifierType>
      </Underlying>
      <OptionData>
        <LongShort>Long</LongShort>
        <OptionType>Call</OptionType>
        <Style>European</Style>
        <Settlement>Cash</Settlement>
        <ExerciseDates>
          <ExerciseDate>2021-01-29</ExerciseDate>
        </ExerciseDates>
      </OptionData>
      <Strike>3400</Strike>
    </Underlying>
    <!-- option #3 -->
    <!-- ... -->
  </EquityOptionPositionData>
</Trade>
\end{minted}
\caption{Equity Option position data}
\label{lst:equityoptionpositiondata}
\end{listing}