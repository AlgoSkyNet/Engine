\subsubsection{Generic Total Return Swap / Contract for Difference (CFD)}
\label{ss:GenericTRS}

A generic total return swap / CFD (Trade type: \emph{TotalReturnSwap} or \emph{ContractForDifference}) is set up using a
TotalReturnSwapData (or ContractForDifferenceData) block as shown in listing \ref{lst:trsdata} and
\ref{lst:trsdata_cfd}. Both trade types behave exactly the same.

Usually CFDs are traded without a funding component and captured with only two dates in the return schedule, namely the
start date on which the initial price is fixed and a fictitious closing date usually set to ``tomorrow'' or another
suitable future date. See listing \ref{lst:trsdata_cfd} for the setup of a CFD on STOXX50E with initial price 3399.20 on
2019-09-28.

The generic total return swap is priced using the {\em accrual method} as opposed to a {\em full discounting method} as
it is used for the {\em equity swap} trade type. The accrual method is common practice when daily unwind rights are
present in the trade terms or when the underlying valuation is too complex to allow for future projection.

The TotalReturnSwapData (ContractForDifferenceData) block is comprised of four sub-blocks, which are

\begin{itemize}
\item {\tt UnderlyingData} containing one or more {\tt Trade} subnodes describing the asset position of the TRS
\item {\tt ReturnData} describing the fixing and payment schedule of the return leg and specifying indices for FX conversion if applicable
\item {\tt FundingData} (optional) containing one or more funding legs of the TRS, whose notionals are based on either
  \begin{itemize}
    \item ``PeriodReset'': the underlying price on the last valuation date before or on the accrual start date of the relevant funding
      coupon, this price is converted to the funding currency using the FX rate on this same valuation date for compo /
      cross currency swaps (see below)
    \item ``DailyReset'': the underlying price on each day of the accrual period, again converted to the funding
      currency using the FX rate the the same date for compo / cross currency swaps. This notional type is only
      supported for fixed rate funding legs.
    \item ``Fixed'': a fixed notional given explicitly in the funding leg
  \end{itemize}

\item {\tt AdditionalCashflowData} (optional) a single leg of type Cashflow containing additional payments
\end{itemize}

The {\tt ReturnData} and {\tt FundingData} schedule periods often match, but this is not a strict requirement: In
general, the funding notional is determined as described above dependent on the notional types ``PeriodReset'',
``DailyReset'', ``Fixed''.

Notice that in every case, the {\tt UnderlyingData} schedule (if applicable to the underlying trade type as e.g. for a
bond) is completely independent from the funding / return schedules: The underlying schedule defines the underlying
flows to compute its NPV, and is not directly related to the return swap itself.

Generic TRS can be used to represent total return swaps on a wide range of underlying assets including e.g. single bonds
or equities, CFDs on an underlying basket of EquityPositions, proprietary indices on equity options and equity or bond
indices.

\begin{itemize}
\item The {\tt UnderlyingData} block specifies on or more underlyings, which can be a trades of one of the following
  types. See the trade type specific sections for details on the setup of these underlyings.
  \begin{itemize}
  \item Bond: See \ref{ss:bond}, the trade data is given in a BondData sub node.
  \item ForwardBond: See \ref{ss:BondForward_refdata}, the trade data is given in a ForwardBondData sub node.
  \item CBO: See \ref{ss:CBOData}, the trade data is given in a CBOData sub node.
  \item CommodityPosition: See \ref{ss:commodity_position}, the trade data is given in a CommodityPositionData sub node.
  \item ConvertibleBond: See \ref{ss:convertible_bond}, the trade data is given in a ConvertibleBondData sub
    node. When using reference data, a TRS on a convertible bond can also be captured as a TRS on a bond, i.e. there is
    no need to distinguish between a TRS on a Bond and a TRS on a convertible Bond in this case, the pricer will figure
    out which underlying to set up based on the type of reference data that is set up for the ISIN referenced in the
    security id field.
  \item EquityPosition: See \ref{ss:equity_position}, the trade data is given in a EquityPositionData sub
    node. Notice that the equities given in the basket must be available as quoted market data.
  \item EquityOptionPosition: See \ref{ss:equity_option_position}, the trade data is given in a EquityOptionPositionData
    sub node.
  \item BondPosition: See \ref{ss:bond_position}, the trade data is given in a BondBasketData sub node.
  \item Derivative: An arbitrary underlying derivative trade (of any type covered by ORE). The derivative subnode has
    exactly two subnodes
    \begin{itemize}
      \item Id: A unqiue identifier for the derivative position. Historical prices must be given under the fixing name
        ``GENERIC-<Id>''.
      \item Trade: The root node of a derivative trade.
    \end{itemize}

  \end{itemize}

  Each trade is specified by a \verb+TradeType+ and a trade type dependent data block as listed above. Listing
  \ref{lst:trsdata} shows an example for a convertible bond underlying. Listing \ref{lst:trsdata2} shows an example for
  an equity basket underlying. Listing \ref{lst:trsdata3} shows an example for a bond basket underlying. Listing
  \ref{lst:trsdata4} shows an example for a derivative underlying (a swaption in this case).

\item The {\tt ReturnData} block specifies the details of the return leg.
  \begin{itemize}
  \item Payer: Indicates whether the return leg is paid.
  
    Allowable values: \emph{true, false}
    
  \item Currency: The currency in which the return is expressed. This can be different from the underlying currency
    (``composite'' swap) and also from the funding leg currency (``cross currency'' swap). The ``composite'' and ``cross
    currency'' features can occur alone or in combination.
    
    Allowable values: A valid currency code, see \lstinline!Currency! in Table \ref{tab:allow_stand_data}, provided it is the same as on the funding leg.
    
  \item ScheduleData: The reference schedule for the return leg, where the valuation dates are derived from this schedule
    using the ObservationLag, ObservationConvention and ObservationCalendar fields. The payment dates are derived from
    this schedule using the PaymentLag, PaymentConvention and PaymentCalendar fields. The payment dates can also be
    given as an explicit list in the PaymentDates node.
    
    Allowable values: A \lstinline!ScheduleData! block as defined in section \ref{ss:schedule_data}
    
  \item ObservationLag [Optional]: The lag between the valuation date and the reference schedule period start date.
  
    Allowable values: Any valid period, i.e. a non-negative whole number, followed by \emph{D} (days), \emph{W} (weeks), \emph{M} (months), \emph{Y} (years). Defaults to \emph{0D} if left blank or omitted.
    
  \item ObservationConvention [Optional]: The roll convention to be used when applying the observation lag.
  
    Allowable values: A valid roll convention (\emph{F, MF, P, MP, U, NEAREST}), see Table \ref{tab:convention} Roll Convention. Defaults to \emph{U} if left blank or omitted.
    
  \item ObservationCalendar [Optional]: The calendar to be used when applying the observation lag.
  
      Allowable values: Any valid calendar, see Table \ref{tab:calendar} Calendar. Defaults to the \emph{NullCalendar} (no holidays) if left blank or omitted.
      
  \item PaymentLag [Optional]: The lag between the reference schedule period end date and the payment date.
  
    Allowable values: Any valid period, i.e.\ a non-negative whole number, optionally followed by \emph{D} (days), \emph{W} (weeks), \emph{M} (months),
  \emph{Y} (years). Defaults to \emph{0D} if left blank or omitted. If a whole number is given and no letter, it is assumed that it is a number of  \emph{D} (days).
    
  \item PaymentConvention [Optional]: The business day convention to be used when applying the payment lag.
  
    Allowable values: A valid roll convention (\emph{F, MF, P, MP, U, NEAREST}), see Table \ref{tab:convention} Roll Convention. Defaults to \emph{U} if left blank or omitted.
    
  \item PaymentCalendar [Optional]: The calendar to be used when applying the payment lag.
  
    Allowable values: Any valid calendar, see Table \ref{tab:calendar} Calendar. Defaults to the \emph{NullCalendar} (no holidays) if left blank or omitted.
    
  \item PaymentDates [Optional]: This node allows for the specification of a list of explicit payment dates, using
    \lstinline!PaymentDate! elements. The list must contain exactly $n-1$ dates where $n$ is the number of dates in the
    reference schedule given in the ScheduleData node. See Listing \ref{lst:paymentdatestrs} for an example with an
    assumed ScheduleData with 4 dates.
    
    \begin{listing}[H]
%\hrule\medskip
\begin{minted}[fontsize=\footnotesize]{xml}
                <PaymentDates>
                      <PaymentDate>2020-01-15</PaymentDate>
                      <PaymentDate>2021-01-15</PaymentDate>
                      <PaymentDate>2022-01-17</PaymentDate>
                </PaymentDates>
\end{minted}
\caption{Payment dates}
\label{lst:paymentdatestrs}
\end{listing}
    
  \item InitialPrice [Optional]: The equity (or bond) price of the underlying
    on the valuation date associated with the start date. Commonly contractually given. The price can be given in the
    underlying currency or the return currency as specified by the InitialPriceCurrency field and is given as
    \begin{itemize}
    \item a (dirty) price for Bond, ForwardBond and Convertible Bond underlyings, the format is dependent on the price quotation method of the referenced bond:
      \begin{itemize}
      \item Percentage of Par: the InitialPrice should be given as e.g. $1.02$ for $102\%$ relative dirty price
      \item Currency per Unit: the InitialPrice should be given as e.g. $0.51$ for a dirty amount of $51$ USD per unit
        of the bond worth (say) $50.0$ USD.
      \end{itemize}
      \item the weighted price of one unit of the bond underlying basket, notice that this is always a ``percentage of
        par'' price regardless of the quotation style of the the single bonds in the basket
      \item the (weighted) price of (one unit of) the equity underlying (basket)
      \item the (weighted) price of (one unit of) the equity option underlying (basket)
      \item an {\em absolute amount in the initial price ccy (``dollar amount'')} if more than one underlying is
        specified and if a derivative is specified
      \item absolute NPV if underlying is a CBO
    \end{itemize}
    Notice that for an equity basket underlying with several currencies involved, the initial price is assumed to be given in the
    return currency in case no InitialPriceCurrency is given.

    Allowable values: A real number. If omitted or left blank it defaults to the equity (or bond) price of the valuation
    date associated with the start date. When this valuation date is in the future there is no fixed price, 
    and in these cases the InitialPrice defaults to the forward price.
    
  \item InitialPriceCurrency [Optional]: Only relevant if InitialPrice is given. This specifies whether the initial
    price is given in the asset currency, the return currency or the funding currency.
    
    Allowable values: One of the currencies in ReturnData / Currency (return currency), FundingData/ LegData / currency
    (funding currency) or the currency of the underlying asset. Defaults to the return currency if omitted.

  \item FXTerms [Mandatory when underlying asset / return / additional cashflow / funding currencies differ]: If the
    underlying currency is different from the funding currency, a node for the conversion underlying / funding currency
    must be given. The same holds for the return and additional cashflow currencies: Whenever one of these currencies
    are different from the funding currency, an FXIndex node for the conversion to the funding currency must be given. If multiple currencies differ, 
    multiple FXIndex nodes must be given.

    \begin{itemize}
    \item FXIndex: The fx index to use for the conversion, this must contain the underlying asset / return / additional cashflow
      currency and the funding leg currency (in the order defined in table \ref{tab:fxindex_data}, i.e. it does not
      matter which one is the asset / return / additional cashflow currency and which is the funding currency)
        
        Allowable values: see \ref{tab:fxindex_data}
    \end{itemize}

    Notice that for an underlying of type EquityPosition or EquityOptionPosition additional \verb+FXIndex+
    entries are required if there is more than one equity position in a different currency: Eventually, for each equity currency there must be a \verb+FXIndex+ specifying the
    conversion from the equity currency to the funding currency (or for the return/cashflow vs funding currency conversion). In this case multiple \verb+FXIndex+ entries are used within a single \lstinline!FXTerms! node, see \ref{lst:fxterms}. 
    
    
        \begin{listing}[H]
\begin{minted}[fontsize=\footnotesize]{xml}
                <FXTerms>
                      <FXIndex>FX-TR20H-GBP-SEK</FXIndex>
                      <FXIndex>FX-TR20H-GBP-EUR</FXIndex>
                      <FXIndex>FX-TR20H-GBP-USD</FXIndex>
                </FXTerms>
\end{minted}
\caption{FXTerms with multiple FXIndex}
\label{lst:fxterms}
\end{listing}

  \item PayUnderlyingCashFlowsImmediately [Optional]: If true, underlying cashflows like coupon or amortisation payments
    from bonds or dividend payments from equities, are paid when they occur. If false, these cashflows are paid together
    with the next return payment. If omitted, the default value is false for trade type TotalReturnSwap and true for
    trade type ContractForDifference.

  Allowable values: true (immediate payment of underlying cashflwos) or false (underlying cashflows are paid on the next
                    return payment date)

\end{itemize}

\item The {\tt FundingData} block specifies the details of the funding leg(s). The block is optional and can be omitted
  if no funding legs are present in the swap (e.g. for CFDs). It contains one or more LegData nodes, see
  \ref{ss:leg_data}. Allowed leg types are
  \begin{itemize}
  \item Fixed
  \item Floating
  \item CMS
  \item CMB
  \end{itemize}
  The number of coupons defined by the legs often match the number of periods of the return schedule, but this is not a
  strict requirement. All funding legs must share the same payment currency.

  There are several ways to determine the notional of each funding leg, which is determined by additional, optional
  NotionalType tags. If given, there must be exactly one NotionalType tag for each LegData nodes. The types have the
  following meanings:

  \begin{itemize}
    \item ``PeriodReset'': the notional of a funding period is determined by the underlying price on the last valuation
      date before or on the accrual start date of the relevant funding coupon, this price is converted to the funding
      currency using the FX rate on this same valuation date for compo / cross currency swaps.
    \item ``DailyReset'': the notional of a funding period is determined by the underlying price on each day of the
      accrual period, again converted to the funding currency using the FX rate the the same date for compo / cross
      currency swaps. This notional type is only supported for fixed rate funding legs.
    \item ``Fixed'': The notional is explicitly given in the leg data.
  \end{itemize}

  If the NotionalType tags are not given, they default to ``PeriodReset'' in case no explicit notional is given on the
  leg and ``Fixed'' in case an explicit notional is given on the leg. See listing \ref{lst:trsdata} for and example with
  two funding legs, one with a notional of type DailyReset and one with a notional of type PeriodReset.

  If a FundingResetGracePeriod is given, a lag of the given number of calendar days is applied when determining the
  relevant return valuation date that determines the funding notional. For example if FundingResetGracePeriod is set to
  2, a valuation date that lies at most 2 calendar days after the funding accrual start date will be still considered
  eligible for this period.

\item The {\tt AdditionalCashflowData} block is optional and specifies unpaid amounts to be included in the NPV. The
  type of this leg must be Cashflow. The currency of the leg must be either the asset currency or the funding currency
  or the return currency.
\end{itemize}

\begin{listing}[H]
%\hrule\medskip
\begin{minted}[fontsize=\footnotesize]{xml}
<TotalReturnSwapData>
  <UnderlyingData>
    <Trade>
      <TradeType>Bond</TradeType>
      <BondData>
        <SecurityId>ISIN:XY1000000000</SecurityId>
        <BondNotional>1000000.00</BondNotional>
      </BondData>
    </Trade>
  </UnderlyingData>
  <ReturnData>
    <Payer>false</Payer>
    <Currency>EUR</Currency>
    <ScheduleData>...</ScheduleData>
    <ObservationLag>0D</ObservationLag>
    <ObservationConvention>P</ObservationConvention>
    <ObservationCalendar>USD</ObservationCalendar>
    <PaymentLag>2D</PaymentLag>
    <PaymentConvention>F</PaymentConvention>
    <PaymentCalendar>TARGET</PaymentCalendar>
    <!-- <PaymentDates> -->
    <!--   <PaymentDate> ... </PaymentDate> -->
    <!--   <PaymentDate> ... </PaymentDate> -->
    <!-- </PaymentDates> -->
    <InitialPrice>1.05</InitialPrice>
    <InitialPriceCurrency>EUR</InitialPriceCurrency>
    <FXTerms>
      <FXIndex>FX-ECB-EUR-USD</FXIndex>
      <FXIndex>FX-ECB-GBP-USD</FXIndex>
    </FXTerms>
    <PayUnderlyingCashFlowsImmediately>false</PayUnderlyingCashFlowsImmediately>
  </ReturnData>
  <FundingData>
    <FundingResetGracePeriod>2</FundingResetGracePeriod>
    <NotionalType>DailyReset</NotionalType>
    <LegData>
      <Payer>true</Payer>
      <LegType>Fixed</LegType>
      ...
    </LegData>
    <NotionalType>PeriodReset</NotionalType>
    <LegData>
      <Payer>true</Payer>
      <LegType>Floating</LegType>
      ...
    </LegData>
  </FundingData>
  <AdditionalCashflowData>
    <LegData>
      <Payer>false</Payer>
      <LegType>Cashflow</LegType>
      ...
    </LegData>
  </AdditionalCashflowData>
</TotalReturnSwapData>
\end{minted}
\caption{Generic Total Return Swap with Convertible Bond underlying}
\label{lst:trsdata}
\end{listing}

\begin{listing}[H]
%\hrule\medskip
\begin{minted}[fontsize=\footnotesize]{xml}
    <TotalReturnSwapData>
      <UnderlyingData>
        <Trade>
          <TradeType>EquityPosition</TradeType>
          <EquityPositionData>
            <!-- basket price = quantity x sum_i ( weight_i x equityPrice_i x fx_i ) -->
            <Quantity>1000</Quantity>
            <Underlying>
              <Type>Equity</Type>
              <Name>BE0003565737</Name>
              <Weight>0.5</Weight>
              <IdentifierType>ISIN</IdentifierType>
              <Currency>EUR</Currency>
              <Exchange>XFRA</Exchange>
            </Underlying>
            <Underlying>
              <Type>Equity</Type>
              <Name>GB00BH4HKS39</Name>
              <Weight>0.5</Weight>
              <IdentifierType>ISIN</IdentifierType>
              <Currency>GBP</Currency>
              <Exchange>XLON</Exchange>
            </Underlying>
          </EquityPositionData>
        </Trade>
      </UnderlyingData>
      <ReturnData>
        ...
        <InitialPrice>112.0</InitialPrice>
        <InitialPriceCurrency>USD</InitialPriceCurrency>
        <FXTerms>
          <FXIndex>FX-ECB-EUR-USD</FXIndex>
          <FXIndex>FX-TR20H-GBP-USD</FXIndex>
        </FXTerms>
      </ReturnData>
      <FundingData>
        <LegData>
          <Payer>true</Payer>
          <LegType>Floating</LegType>
          <Currency>USD</Currency>
          ...
        </LegData>
      </FundingData>
      <AdditionalCashflowData>
        <LegData>
          <Payer>false</Payer>
          <LegType>Cashflow</LegType>
          ...
        </LegData>
      </AdditionalCashflowData>
    </TotalReturnSwapData>
  </Trade>
\end{minted}
\caption{Generic Total Return Swap with equity basket underlying}
\label{lst:trsdata2}
\end{listing}

\begin{listing}[H]
%\hrule\medskip
\begin{minted}[fontsize=\footnotesize]{xml}
    <TotalReturnSwapData>
      <UnderlyingData>
        <Trade>
          <TradeType>BondPosition</TradeType>
          <BondBasketData>
            <Quantity>100000000</Quantity>
            <Identifier>ISIN:GB00B4KT9Q30</Identifier>
          </BondBasketData>
        </Trade>
      </UnderlyingData>
      <!-- omitting ReturnData, FundingData, AdditionalCashflowData -->
    </TotalReturnSwapData>
  </Trade>
\end{minted}
\caption{Generic Total Return Swap with bond basket underlying}
\label{lst:trsdata3}
\end{listing}

\begin{listing}[H]
%\hrule\medskip
\begin{minted}[fontsize=\footnotesize]{xml}
    <TotalReturnSwapData>
      <UnderlyingData>
        <Derivative>
          <Id>DERIV:XR3JF32BFD</Id>
          <Trade>
            <TradeType>Swaption</TradeType>
              <SwaptionData> ... </SwaptionData>
          </Trade>
        </Derivative>
      </UnderlyingData>
      <!-- omitting ReturnData, FundingData, AdditionalCashflowData -->
    </TotalReturnSwapData>
  </Trade>
\end{minted}
\caption{Generic Total Return Swap on a derivative underlying}
\label{lst:trsdata4}
\end{listing}

\begin{listing}[H]
%\hrule\medskip
\begin{minted}[fontsize=\footnotesize]{xml}
    <TotalReturnSwapData>
      <UnderlyingData>
        <Trade>
          <TradeType>CommodityPosition</TradeType>
          <CommodityPositionData>
            <!-- basket price = quantity x sum_i ( weight_i x equityPrice_i x fx_i ) -->
            <Quantity>1000</Quantity>
            <Underlying>
              <Type>Commodity</Type>
              <Name>RIC:.BCOM</Name>
              <Weight>1.0</Weight>
              <PriceType>Spot</PriceType>
            </Underlying>
          </CommodityPositionData>
        </Trade>
      </UnderlyingData>
      <!-- omitting ReturnData, FundingData, AdditionalCashflowData -->
    </TotalReturnSwapData>
  </Trade>
\end{minted}
\caption{Generic Total Return Swap on a commodity index underlying}
\label{lst:trsdata5}
\end{listing}

\begin{listing}[H]
%\hrule\medskip
\begin{minted}[fontsize=\footnotesize]{xml}
    <ContractForDifferenceData>
      <UnderlyingData>
	<Trade>
	  <TradeType>EquityPosition</TradeType>
	  <EquityPositionData>
	    <Quantity>1000</Quantity>
	    <Underlying>
	      <Type>Equity</Type>
	      <Name>.STOXX50E</Name>
	      <Weight>1.0</Weight>
	      <IdentifierType>RIC</IdentifierType>
	    </Underlying>
	  </EquityPositionData>
	</Trade>
      </UnderlyingData>
      <ReturnData>
	<Payer>false</Payer>
	<Currency>EUR</Currency>
	<ScheduleData>
	  <Dates>
	    <Dates>
	      <!-- the start date of the CFD on which the initial price was set -->
	      <Date>2018-09-28</Date>
	      <!-- fictitious closing date, e.g. set to "tomorrow" -->
	      <Date>2019-01-04</Date>
	    </Dates>
	  </Dates>
	</ScheduleData>
	<InitialPrice>3399.20</InitialPrice>
	<InitialPriceCurrency>EUR</InitialPriceCurrency>
      </ReturnData>
      </ContractForDifferenceData>
\end{minted}
\caption{CFD on STOXX50E with initial price 3399.20 EUR}
\label{lst:trsdata_cfd}
\end{listing}
