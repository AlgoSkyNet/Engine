\subsubsection{Balance Guaranteed Swap (BGS)}

The \lstinline!BalanceGuaranteedSwapData! node is the trade data container for trade type \emph{BalanceGuaranteedSwap}.  A BGS must have two legs, one fixed and one floating. Each leg typically has an amortising notional and is represented by a \lstinline!LegData! trade component sub-node, described in section \ref{ss:leg_data}.
The \lstinline!BalanceGuaranteedSwapData! node also contains a \lstinline!ReferenceSecurity! sub-node specifying the Asset Backed Security to which the notional schedule of the BGS is linked. 
%and a \lstinline!MaxCPR!  sub-node indicating the maximum allowable prepayment. 
An example structure of a  \lstinline!BalanceGuaranteedSwapData! node is shown in Listing \ref{lst:bgs_data}.
\begin{listing}[H]
%\hrule\medskip
\begin{minted}[fontsize=\footnotesize]{xml}
<BalanceGuaranteedSwapData>
  <ReferenceSecurity>ISIN:XS0983610930</ReferenceSecurity>
  <Tranches>
    <Tranche>
      <Description>Class A</Description>
      <SecurityId>ISIN:XS0983610930</SecurityId>
      <Seniority>1</Seniority>
      <Notionals>
      ...
      </Notionals>
    </Tranche>
    <Tranche>
      <Description>Class B</Description>
      <SecurityId>ISIN:XS0983610931</SecurityId>
      <Seniority>2</Seniority>
      <Notionals>
      ...
      </Notionals>
    </Tranche>
    <ScheduleData>
    ...
    </ScheduleData>
  </Tranches>
  <LegData>
	<LegType>Fixed</LegType>
	 ...
  </LegData>
  <LegData>
	<LegType>Floating</LegType>
	 ...
  </LegData>
<BalanceGuaranteedSwapData>
\end{minted}
\caption{Balance Guaranteed Swap data}
\label{lst:bgs_data}
\end{listing}

The meanings and allowable values of the elements in the \lstinline!BalanceGuaranteedSwapData!  node follow below.

\begin{itemize}

\item ReferenceSecurity: The ISIN of the Asset Backed Security tranche to which the BGS is linked.

Allowable values:  The prefix {\tt ISIN:} followed by an ISIN code for the Reference Security.

\item Tranches: A description of the Asset Backed Security tranche notionals. Each Tranche is identified by a {\tt
    SecurityId} and an optional {\tt Description}. Each Tranche has a {\tt Seniority} given as a positive integer value
  where lower values mean higher seniority, i.e. $1$ is the most senior tranche (e.g. ``class A'') followed by $2$
  (e.g. ``class B'') etc.  The notionals are given in a sub-node {\tt Notionals} as described in section
  \ref{ss:leg_data} w.r.t. a schedule given in {\tt ScheduleData} which is shared across all tranches. There must be
  exactly one tranche with a security id matching the reference security.

\item LegData: This is a trade component sub-node outlined in section \ref{ss:leg_data}. A BGS must have two  \lstinline!LegData! nodes and the LegType element must be set to \emph{Floating} on one leg and \emph{Fixed} on the other. The two legs must have the same \lstinline!Currency!.

\end{itemize}

The notionals of the swap and the referenced tranche must be consistent. Furthermore, notionals for periods with a start
date in the past must be given with their actual value, i.e. including actual prepayments that were made in the previous
periods. Notionals for periods with a start dats in the future on the other hand must be given assuming a zero
conditional prepayment rate. For the latter periods a prepayment model is used to generate suitable notional schedules
when pricing the swap. The prepayment model assumes that tranches with higher seniority are amortised first, i.e. in the
example here the class A tranche is amortised before the class B tranche.

