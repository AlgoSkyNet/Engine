\subsubsection{Forward Bond}
\label{ss:forwardbond}

A Forward Bond (or Bond Forward) is a contract that establishes an agreement to buy or sell (determined by
\lstinline!LongInForward!) an underlying bond at a future point in time (the {\tt ForwardMaturityDate}) at an agreed
price (the settlement {\tt Amount}).

A T-Lock is a Forward Bond with a US Treasury Bond as underlying, whereas a J-Lock is a Forward Bond with a Japanese
Government Bond as underlying. T-Locks can be specified in terms of a lock-in yield rather then a settlement
amount. The cash settlement amount is given by (bond yield at maturity - lock rate) x DV01 in this case.

Listing \ref{lst:forward_bond} shows an example for a physically settled forward bond. Listing
\ref{lst:forward_bond_tlock} shows an example for a cash settled T-Lock transaction specified by a lock-in yield.

A Forward Bond is set up using a {\tt ForwardBondData} block as shown below and the trade type is
\emph{ForwardBond}. The specific elements are

\begin{itemize}
   \item BondData: A {\tt BondData} block specifying the underlying bond as described in section~\ref{ss:bond}. A long
     position must be taken in the bond, i.e.~({\tt Payer}) flag must be set to ({\tt true}). The bond data block
     contains an additional field for forward bonds
     \begin{itemize}
     \item IncomeCurveId: The benchmark curve to be used for compounding, this must match a name of a curve in the yield
       curves or index curve block in {\tt todaysmarket.xml}. It is optional to provide this curve. If left out the
       market reference yield curve from {\tt todaysmarket.xml} is used for compounding.
     \end{itemize}
   \item SettlementData: The entity defining the terms of settlement:
   \begin{itemize}
       \item ForwardMaturityDate: The date of maturity of the forward contract. \\
         Allowable values: See \lstinline!Date! in Table \ref{tab:allow_stand_data}.
       \item Settlement [Optional]: Cash or Physical. Option, defaults to Physcial, except in case the settlement is
         defined by LockRate, in which case it defaults to Cash. \\
         Allowable values: Cash, Physical
       \item Amount [Optional]: The settlement amount (also called strike) transferred at forward maturity in return for
         the bond (physical delivery) or a cash amount equal to the dirty price of the bond (cash settlement). This is
         transferred from the party that is long to the party that is short (determined by \lstinline!LongInForward!)
         and cannot be a negative amount. It is assumed to be in the same currency as the underlying bond. Exactly one
         of the fields Amount, LockRate must be given. \\
         Allowable values: Any non-negative real number.
       \item LockRate [Optional]: The payoff is given by (yield at forward maturity - LockRate) x DV01 (LongInForward =
         true). Exactly one of the fields Amount, LockRate must be given. In case the LockRate is given, the Settlement
         must be set to Cash. If Settlement is not given, it defaults to Cash in this case. \\
         Allowable values: Any non-negative real number.
       \item LockRateDayCounter [Optional]: The day counter w.r.t. which the lock rate is expressed. Optional, defaults to A360. \\
         Allowable values: see table \ref{tab:daycount}
       \item SettlementDirty [Optional]: A flag that determines whether the settlement amount {({\tt Amount})} reflects
         a clean (\emph{false}) or dirty (\emph{true}) price. In either case, the dirty amount is actually paid on the
         forward maturity date, i.e. if SettlementDirty = \emph{false}, the (forward) accruals are computed internally
         and added to the given amount to get the actual settlement amount. Optional, defaults to true. \\
         Allowable values: \emph{true}, \emph{false}
   \end{itemize}
   \item PremiumData: The entity defining the terms of a potential premium payment. This node is optional. If left out it is assumed that no premium is paid.
   \begin{itemize}
       \item Date: The date when a premium is paid. \\
       Allowable values: See \lstinline!Date! in Table \ref{tab:allow_stand_data}.
       \item Amount: The amount transferred as a premium. This is transferred from the party that is long to the party
         that is short (determined by \lstinline!LongInForward!) and cannot be a negative amount. It is assumed to be in
         the same currency as the underlying bond.\\
       Allowable values: Any non-negative real number.
   \end{itemize}
   \item LongInForward: A flag that determines whether the forward contract is entered in long (\emph{true}) or short
     (\emph{false}) position. \\
       Allowable values: \emph{true}, \emph{false}
 \end{itemize}

\begin{listing}[H]
  \begin{minted}[fontsize=\small]{xml}
   <ForwardBondData>
     <BondData>
      ...
      <IncomeCurveId>BENCHMARKINCOME-EUR<IncomeCurveId>
     </BondData>
     <SettlementData>
       <ForwardMaturityDate>20160808</ForwardMaturityDate>
       <Settlement>Physcial</Settlement>
       <ForwardSettlementDate>20160810</ForwardSettlementDate>
       <Amount>1000000.00</Amount>
       <SettlementDirty>true</SettlementDirty>
     </SettlementData>
     <PremiumData>
       <Amount>1000.00</Amount>
       <Date>20160808</Date>
     </PremiumData>
     <LongInForward>true</LongInForward>
   </ForwardBondData>
  \end{minted}
\caption{Forward Bond Data}
\label{lst:forward_bond}
\end{listing}

\begin{listing}[H]
   \begin{minted}[fontsize=\small]{xml}
   <ForwardBondData>
     <BondData>
      ...
     </BondData>
     <SettlementData>
       <ForwardMaturityDate>20160808</ForwardMaturityDate>
       <ForwardSettlementDate>20160810</ForwardSettlementDate>
       <LockRate>0.02365</LockRate>
     </SettlementData>
     <LongInForward>true</LongInForward>
   </ForwardBondData>
\end{minted}
\caption{Forward Bond Date (T-Lock)}
\label{lst:forward_bond_tlock}
\end{listing}

As for the ordinary bond the forward bond pricing requires a recovery rate that can be specified in ORE per SecurityId.

\subsubsection*{Forward Bond - Pricing Engine configuration}

The configuration for the pricing engine of the forward bond is identical to the ordinary bond.%, cf.~Section~\ref{BondEngineConfig}.
The pricing engine called by forward bond products is the {\tt DiscountingForwardBondEngine}, see below for a configuration example.

%\hrule\medskip
   \begin{minted}[fontsize=\small]{xml}
   <Product type="ForwardBond">
   <Model>DiscountedCashflows</Model>
   <ModelParameters></ModelParameters>
   <Engine>DiscountingForwardBondEngine</Engine>
   <EngineParameters>
    <Parameter name="TimestepPeriod">3M</Parameter>
   </EngineParameters>
   </Product>
   \end{minted}
%\caption{Bond Data}
