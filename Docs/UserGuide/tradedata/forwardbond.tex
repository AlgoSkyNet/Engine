\subsubsection{Forward Bond}
\label{ss:forwardbond}

A Forward Bond is set up using a {\tt ForwardBondData} block as shown below. The specific elements
are

\begin{itemize}
   \item BondData: A {\tt BondData} block specifying the underlying bond as described in section~\ref{ss:bond}. A long
     position must be taken in the bond, i.e.~({\tt Payer}) flag must be set to ({\tt true}). The bond data block
     contains an additional field for forward bonds
     \begin{itemize}
     \item IncomeCurveId: The benchmark curve to be used for compounding, this must match a name of a curve in the yield
       curves or index curve block in {\tt todaysmarket.xml}. It is optional to provide this curve. If left out the
       market reference yield curve from {\tt todaysmarket.xml} is used for compounding.
     \end{itemize}
   \item SettlementData: The entity defining the terms of settlement:
   \begin{itemize}
       \item ForwardMaturityDate: The date of maturity of the forward contract.
       \item Amount: The settlement amount transferred at forward maturity in return for the bond.
       \item SettlementDirty: A flag that determines whether  the settlement amount {({\tt Amount})} reflects a clean ({\tt false}) or dirty ({\tt true}) price.
   \end{itemize}
   % \item LGD (optional): If given, this LGD is used for pricing, overriding the default LGD of the default curve
   \item PremiumData: The entity defining the terms of a potential premium payment. This node is optional. If left out it is assumed that no premium is paid.
   \begin{itemize}
       \item Date: The date when a premium is paid.
       \item Amount: The amount transferred as a premium.
   \end{itemize}
   \item LongInForward: A flag that determines whether the forward contract is entered in long ({\tt true}) or short ({\tt false}) position.
 \end{itemize}

   \begin{minted}[fontsize=\small]{xml}
   <ForwardBondData>
     <BondData>
      ...
      <IncomeCurveId>BENCHMARKINCOME-EUR<IncomeCurveId>
     </BondData>
     <SettlementData>
       <Amount>1000000.00</Amount>
       <ForwardMaturityDate>20160808</ForwardMaturityDate>
       <SettlementDirty>true</SettlementDirty>
     </SettlementData>
     <PremiumData>
       <Amount>1000.00</Amount>
       <Date>20160808</Date>
     </PremiumData>
     <LongInForward>true</LongInForward>
   </ForwardBondData>
   \end{minted}

%incomeCurveId_ = XMLUtils::getChildValue(fwdBondNode, "IncomeCurveId", false);

As for the ordinary bond the forward bond pricing requires a recovery rate that can be specified in ORE per SecurityId.

\subsubsection*{Forward Bond - Pricing Engine configuration}

The configuration for the pricing engine of the forward bond is identical to the ordinary bond.%, cf.~Section~\ref{BondEngineConfig}.
The pricing engine called by forward bond products is the {\tt DiscountingForwardBondEngine}, see below for a configuration example.

%\hrule\medskip
   \begin{minted}[fontsize=\small]{xml}
   <Product type="ForwardBond">
   <Model>DiscountedCashflows</Model>
   <ModelParameters></ModelParameters>
   <Engine>DiscountingForwardBondEngine</Engine>
   <EngineParameters>
    <Parameter name="TimestepPeriod">3M</Parameter>
   </EngineParameters>
   </Product>
   \end{minted}
%\caption{Bond Data}
