\subsubsection{Bond Total Return Swap}
\label{ss:TRS}

A vanilla Bond Total Return Swap (Trade type: \emph{BondTRS}) is set up using a {\tt BondTRSData} block as shown in listing \ref{lst:bondtrsdata}. The block is comprised of three sub-blocks, which
are {\tt BondData}, {\tt TotalReturnData} and {\tt FundingData}.

\begin{itemize}
\item The {\tt BondData} block specifies the underlying bond, usually by specifiying the security id and the quantity /
  bond notional and relying on reference data:

  \begin{itemize}
  \item SecurityId: The underlying security identifier\\
      Allowable values: Typically the ISIN of the underlying bond, with the ISIN: prefix.
  \item BondNotional: The quantity or number of bonds that is relevant for the TRS, with the convention that 1 bond always corresponds to a face value of 1 unit of bond currency.\\
      Allowable values: Any positive real number.
  \item CreditRisk [Optional] Boolean flag indicating whether to show Credit Risk on the Bond product. If set to \emph{true}, the product class will be set to \emph{Credit} instead of \emph{RatesFX}, and there will be credit sensitivities. Note that if the underlying bond reference is set up without a CreditCurveId - typically for some highly rated government bonds -  the CreditRisk flag will have no impact on the product class and no credit sensitivities will be shown even if CreditRisk is set to \emph{true}. \\
      Allowable Values: \emph{true} or \emph{false} Defaults to \emph{true} if left blank or omitted.          
  \end{itemize}

  Alternatively, the BondData block can be specified fully explicit, as outlined in \ref{ss:bond}

\item The {\tt TotalReturnData} block specifies
  \begin{itemize}
  \item Payer: Indicates whether the total return leg is paid.\\
    Allowable values: \emph{true} or \emph{false}

  \item InitialPrice [Optional]: Should be filled if the bond price on the first date of the total return schedule is
    contractually given, in which case the price must correspond to the price type of the total return leg, i.e. if the
    price type is \emph{Dirty} then the InitialPrice must also be a dirty price (as it is usually given in the term
    sheet in this case). The price must given in percent, e.g. $101.20$.\footnote{as opposed to the bond price in the
    fixing history, where it must be given as $1.0120$ and is always a clean quotation} If not given, the bond price for
    the first date of the total return schedule is read from the price history.  Notice that if a bond is quoted in
    Currency per Unit the initial price should be given in this format too: If e.g. one unit is $50.0$ USD an initial
    price of $51.0$ would correspond a dirty amount of $51.0$ USD for one unit of the bond.\\

    Allowable values: Any positive real number.

  \item PriceType: The price type on which these payments are based\\
    Allowable values: \emph{Dirty} or \emph{Clean}

  \item ObservationLag [Optional]: The lag between the valuation date and the reference schedule period start date.
  
    Allowable values: Any valid period, i.e. a non-negative whole number, followed by \emph{D} (days), \emph{W} (weeks), \emph{M} (months), \emph{Y} (years). Defaults to \emph{0D} if left blank or omitted.
    
  \item ObservationConvention [Optional]: The roll convention to be used when applying the observation lag.
  
    Allowable values: A valid roll convention (\emph{F, MF, P, MP, U, NEAREST}), see Table \ref{tab:convention} Roll Convention. Defaults to \emph{U} if left blank or omitted.
    
  \item ObservationCalendar [Optional]: The calendar to be used when applying the observation lag.
  
      Allowable values: Any valid calendar, see Table \ref{tab:calendar} Calendar. Defaults to the \emph{NullCalendar} (no holidays) if left blank or omitted.
      
  \item PaymentLag [Optional]: The lag between the reference schedule period end date and the payment date.
  
    Allowable values: Any valid period, i.e.\ a non-negative whole number, optionally followed by \emph{D} (days), \emph{W} (weeks), \emph{M} (months),
  \emph{Y} (years). Defaults to \emph{0D} if left blank or omitted. If a whole number is given and no letter, it is assumed that it is a number of  \emph{D} (days).
    
  \item PaymentConvention [Optional]: The business day convention to be used when applying the payment lag.
  
    Allowable values: A valid roll convention (\emph{F, MF, P, MP, U, NEAREST}), see Table \ref{tab:convention} Roll Convention. Defaults to \emph{U} if left blank or omitted.
    
  \item PaymentCalendar [Optional]: The calendar to be used when applying the payment lag.
  
    Allowable values: Any valid calendar, see Table \ref{tab:calendar} Calendar. Defaults to the \emph{NullCalendar} (no holidays) if left blank or omitted.

  \item PaymentDates [Optional]: This node allows for the specification of a list of explicit payment dates, using \lstinline!PaymentDate! elements.
    The list must contain exactly $n-1$ dates where $n$ is the number of dates in the reference schedule given in the ScheduleData node.
    See Listing \ref{lst:paymentdatesbondtrs} for an example with an assumed ScheduleData with 4 dates.

  \begin{listing}[H]
    %\hrule\medskip
    \begin{minted}[fontsize=\footnotesize]{xml}
                    <PaymentDates>
                          <PaymentDate>2020-01-15</PaymentDate>
                          <PaymentDate>2021-01-15</PaymentDate>
                          <PaymentDate>2022-01-17</PaymentDate>
                    </PaymentDates>
    \end{minted}
    \caption{Payment dates}
    \label{lst:paymentdatesbondtrs}
  \end{listing}


  \item FXTerms [Mandatory when underlying bond and BondTRS currencies differ]: Required if the bond currency is different from the return currency, which is always
    assumed to be equal to the funding leg currency. This kind of trade is also known as a ``composite trs''. The
    subnode for the FXTerms node is:
    \begin{itemize}
      \item FXIndex: The fx index to use for the conversion, this must contain the bond currency and the funding leg
        currency (in the order defined in table \ref{tab:fxindex_data}, i.e. it does not matter which one is the bond currency and which is the funding currency)
        
        Allowable values: See Table \ref{tab:fxindex_data}
     % \item FXIndexFixingDays [Optional]: The fixing days of the fx index
       
     %   Allowable values: Non-negative whole numbers. Defaults to \emph{0} if left blank or omitted.
   %   \item FXIndexCalendar [Optional]: The fixing calendar for the fx index
        
    %    Allowable values: Any valid calendar, see Table \ref{tab:calendar} Calendar. Defaults to the \emph{NullCalendar} (no holidays) if left blank or omitted.
    \end{itemize}

  \item ScheduleData: The reference schedule for the return leg, where the valuation dates are derived from this schedule
    using the ObservationLag, ObservationConvention and ObservationCalendar fields. The payment dates are derived from
    this schedule using the PaymentLag, PaymentConvention and PaymentCalendar fields. The payment dates can also be
    given as an explicit list in the PaymentDates node.
    Allowable values: A \lstinline!ScheduleData! block as defined in section \ref{ss:schedule_data}
    
  \item PayBondCashFlowsImmediately [Optional]: If true, bond cashflows like coupon or amortisation payments
    are paid when they occur. If false, these cashflows are paid together with the next return payment.
    If omitted, the default value is false.

    Allowable values: \emph{true} (immediate payment of bond cashflows) or \emph{false} (bond cashflows are paid on the next
                      return payment date)
  \end{itemize}

\item The {\tt FundingData} block specifies the funding leg, which can be of any leg type. The {\tt FundingData}
  contains exactly one {\tt Leg}. The currency of this leg also defines the currency in which the return is paid.
  Usually the funding leg's notional will be aligned with the return leg's notional. To achieve this, indexings on the
  floating leg can be used, see \ref{ss:indexings}. In the context of bond total return swaps, the indexings can be
  defined in a simplified way by adding an Indexings node with a subnode FromAssetLeg set to true to the funding leg's
  LegData node. The notionals node is not required either in the funding leg's LegData in this case. An example for this
  setup is shown in \ref{lst:bondtrsdata}.

\end{itemize}

\begin{listing}[H]
%\hrule\medskip
\begin{minted}[fontsize=\footnotesize]{xml}
<BondTRSData>
  <BondData>
    <SecurityId>ISIN:NZIIBDT005C5</SecurityId>
    <BondNotional>100000</BondNotional>
  </BondData>
  <TotalReturnData>
    <Payer>false</Payer>
    <InitialPrice>102.0</InitialPrice>
    <PriceType>Clean</PriceType>
    <ObservationLag>0D</ObservationLag>
    <ObservationConvention>P</ObservationConvention>
    <ObservationCalendar>USD</ObservationCalendar>
    <PaymentLag>2D</PaymentLag>
    <PaymentConvention>F</PaymentConvention>
    <PaymentCalendar>TARGET</PaymentCalendar>
    <!-- <PaymentDates> -->
    <!--   <PaymentDate> ... </PaymentDate> -->
    <!--   <PaymentDate> ... </PaymentDate> -->
    <!-- </PaymentDates> -->
    <FXTerms>
      <FXIndex>FX-TR20H-NZD-USD</FXIndex>
    </FXTerms>
    <ScheduleData>
    ...
    </ScheduleData>
    <PayBondCashFlowsImmediately>false</PayBondCashFlowsImmediately>
  </TotalReturnData>
  <FundingData>
    <LegData>
      <Payer>true</Payer>
      <LegType>Floating</LegType>
      <Currency>USD</Currency>
      ...
      <!-- Notionals node is not required, set to 1 internally -->
      ...
      <Indexings>
      <!-- derive the indexing information (bond price, FX) from the total return leg -->
      <FromAssetLeg>true</FromAssetLeg>
      </Indexings>
      ...
    </LegData>
  </FundingData>
</BondTRSData>
\end{minted}
\caption{Bond Total Return Swap Data with indexed funding leg}
\label{lst:bondtrsdata}
\end{listing}
