\subsubsection{Cap/Floor}
\label{ss:capfloor}

The \lstinline!CapFloorData! node is the trade data container for the \emph{CapFloor} trade type.  It's a cap, floor or collar
(i.e. a portfolio of a long cap and a short floor for a long position in the collar) on a series of Ibor or CMS rates. The
\lstinline!CapFloorData! node contains a \lstinline!LongShort! sub-node which indicates whether the cap (floor, collar)
is long or short, and a \lstinline!LegData!  sub-node where the
LegType can be set to \emph{Floating}, \emph{CMS}, \emph{DurationAdjustedCMS}, \emph{CPI}
or \emph{YY}, plus elements for the Cap and Floor rates. An example structure with Cap rates is shown in in Listing
\ref{lst:capfloor_data}. A \lstinline!CapFloorData! node must have either \lstinline!Caps! or \lstinline!Floors!
elements, or both. In the case of both (I.e. a collar with long cap and short floor) the sequence is that  \lstinline!Caps! elements must be above the \lstinline!Floors! elements. Note that 
the \lstinline!Caps! and \lstinline!Floors! elements must be outside the \lstinline!LegData! sub-node, i.e. a \emph{CapFloor} 
can't have a capped or floored  \emph{Floating}  or \emph{CMS} leg.
THe \emph{Payer} flag in the LegData subnode should be set to false, because the position type (long, short) is already
defined by the \emph{LongShort} flag. If the \emph{Payer} flag is set to true, this will effectively flip the position
type once again. Notice that the signs in the definition of a collar (long cap, short floor) for the CapFloor
instruments is exactly opposite to \ref{ss:floatingleg_data}.

\begin{listing}[H]
%\hrule\medskip
\begin{minted}[fontsize=\footnotesize]{xml}
<CapFloorData>
  <LongShort>Long</LongShort>
  <LegData>
    <Payer>false</Payer>
    <LegType>Floating</LegType>
     ...
  </LegData>
  <Caps>
    <Cap>0.05</Cap>
  </Caps>
  <PremiumAmount>1000</PremiumAmount>
  <PremiumCurrency>EUR</PremiumCurrency>
  <PremiumPayDate>2021-01-27</PremiumPayDate>
</CapFloorData>
\end{minted}
\caption{Cap/Floor data}
\label{lst:capfloor_data}
\end{listing}

The meanings and allowable values of the elements in the \lstinline!CapFloorData!  node follow below.

\begin{itemize}

\item LongShort: This node defines the position in the cap (floor, collar) and can take values \emph{Long} or \emph{Short}.

\item LegData: This is a trade component sub-node outlined in section \ref{ss:leg_data}. Exactly
  one \lstinline!LegData! node is allowed, and the LegType element must
  be set to \emph{Floating}, \emph{CMS}, \emph{CPI} or \emph{YY}.

\item Caps: This node has child elements of type \lstinline!Cap!
  capping the floating leg (after applying spread if any). The first rate value corresponds to the
  first coupon, the second rate value corresponds to the second
  coupon, etc. If the number of coupons exceeds the number of rate
  values, the rate will be kept flat at the value of last entered rate
  for the remaining coupons. For a fixed cap rate over all coupons,
  one single rate value is sufficient. The number of entered rate
  values cannot exceed the number of coupons.  

  Allowable values for each \lstinline!Cap! element: Any real number. The rate is expressed in decimal form, eg 0.05 is
  a rate of 5\%

\item Floors: This node has child elements of type
  \lstinline!Floor! flooring the floating leg (after applying spread if any).  The first rate value
  corresponds to the first coupon, the second rate value corresponds
  to the second coupon, etc. If the number of coupons exceeds the
  number of rate values, the rate will be kept flat at the value of
  last entered rate for the remaining coupons. For a fixed floor rate
  over all coupons, one single rate value is sufficient. The number of
  entered rate values cannot exceed the number of coupons.

  Allowable values for each \lstinline!Floor! element: Any real number. The rate is expressed in decimal form, eg 0.05 is
  a rate of 5\%

\item PremiumAmount [Optional]: Option premium amount paid by the option buyer to the option seller.

Allowable values:  Any positive real number.

\item PremiumCurrency [Optional]: Currency of the option premium.

Allowable values:  See \lstinline!Currency! in Table \ref{tab:allow_stand_data}.

\item PremiumPayDate [Optional]: Date of the option premium payment.

Allowable values:  See \lstinline!Date! in Table \ref{tab:allow_stand_data}.

\end{itemize}
