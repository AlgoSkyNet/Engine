\subsubsection{Synthetic CDO}

A Synthetic CDO is set up using a {\tt  CdoData} block as shown in listing 
\ref{lst:cdodata}.

\begin{listing}[H]
%\hrule\medskip
\begin{minted}[fontsize=\footnotesize]{xml}
    <CdoData>
      <Qualifier> ItraxxEuropeS9V1 </Qualifier>
      <ProtectionStart> 20140425 </ProtectionStart>
      <UpfrontDate/>
      <UpfrontFee/>
      <AttachmentPoint>0.12</AttachmentPoint>
      <DetachmentPoint>0.22</DetachmentPoint>
      <SettlesAccrual>Y</SettlesAccrual>
      <ProtectionPaymentTime>atDefault</ProtectionPaymentTime>
      <!-- Premium leg -->
      <LegData>
          ...
      </LegData>
      <!-- Basket -->
      <BasketData>
        ...
      </BasketData>
    </CdoData>
\end{minted}
\caption{CDO Data}
\label{lst:cdodata}
\end{listing}

The meanings of the elements of the {\tt CdoData}  node follow below:

\begin{itemize}
\item Qualifier: Used to reference the relevant base correlation curve
\item ProtectionStart: The first date where a default event will
  trigger the contract
\item UpfrontDate[Optional]: Settlement date for the upfront payment.
\item UpfrontFee[Optional]: The upfront payment, expressed as a rate, to be multiplied by notional amount.
\item LegData: Premium leg description as in an Index CDS (see section
  \ref{ss:indexcds}) with notional correspondig to the initial tranche
  notional
\item BasketData: Underlying basket description as in an Index CDS  (see section
  \ref{ss:indexcds})
\item AttachmentPoint: Losses where protection starts, expressed as a
  fraction of the basket notional
\item DetachmentPoint: Losses where protection end, expressed as a
  fraction of the basket notional
\end{itemize}