\subsubsection{Worst Of Basket Swaps}

The FxWorstOfBasketSwap, EquityWorstOfBasketSwap, and CommodityWorstOfBasketSwap trade types
have trade data containers (respectively):
\begin{itemize}
  \item \lstinline!FxWorstOfBasketSwapData!
  \item \lstinline!EquityWorstOfBasketSwapData!
  \item \lstinline!CommodityWorstOfBasketSwapData!
\end{itemize}
Listing \ref{lst:eqworstofbasketswap_data} shows the structure of example trades for an Equity underlying.

\begin{listing}[H]
\begin{minted}[fontsize=\footnotesize]{xml}
<Trade id="EQ_WorstOfBasketSwap">
  <TradeType>EquityWorstOfBasketSwap</TradeType>
  <Envelope>
    ......
  </Envelope>
  <EquityWorstOfBasketSwapData>
    <LongShort>Long</LongShort>
    <Currency>EUR</Currency>
    <Quantity>1955000</Quantity>
    <Underlyings>
      <Underlying>
        .......
      </Underlying>
    </Underlyings>
    <InitialPrices>
      <InitialPrice>...</InitialPrice>
    </InitialPrices>
    <FloatingPeriodSchedule>
      <Dates>
        ...
      </Dates>
    </FloatingPeriodSchedule>
    <FixedDeterminationSchedule>
            ............
    </FixedDeterminationSchedule>
    <KnockOutDeterminationSchedule>
             .............
    </KnockOutDeterminationSchedule>
    <FloatingPayDates>
      <Dates>
        ....
      </Dates>
    </FloatingPayDates>
    <KnockInDeterminationSchedule>
            ............
    </KnockInDeterminationSchedule>
    <FixedPayDates>
      ............
    </FixedPayDates>
    <KnockOutLevels>
      <KnockOutLevel>...</KnockOutLevel>
    </KnockOutLevels>
    <FixedTriggerLevels>
      <FixedTriggerLevel>...</FixedTriggerLevel>
    </FixedTriggerLevels>
    <KnockInLevel>0.75</KnockInLevel>
    <FixedRate>0.04</FixedRate>
    <FixedAccrualSchedule>
      <Rules>
        ...
      </Rules>
    </FixedAccrualSchedule>
    <FloatingIndex>EUR-EURIBOR-3M</FloatingIndex>
    <FloatingDayCountFraction>Actual/360</FloatingDayCountFraction>
    <FloatingFixingSchedule>
          ............
    </FloatingFixingSchedule>
  </EquityWorstOfBasketSwapData>
</Trade>
\end{minted}
\caption{EquityWorstOfBasketSwap data}
\label{lst:eqworstofbasketswap_data}
\end{listing}

The net cashflows under a worst of basket swap from the long perspective are:
\begin{enumerate}
  \item Receive an initial fixed amount: \lstinline!InitialFixedRate! $*$ \lstinline!Quantity!,
  \item At each determination date (if a knock-out has not occurred for any previous determination date): \begin{itemize}
      \item Pay a floating amount: \lstinline!Quantity! $* floatingRate * accrualFraction$,
      \item If a coupon trigger event occurs for this date, receive a fixed coupon amount (as well as any other fixed coupon amounts not previously received because a coupon trigger event had not occurred): \lstinline!Quantity! $*$ \lstinline!CouponRate!.

      If \lstinline!AccumulatingFixedCoupons! is \emph{True}, the coupon amounts are guaranteed to be paid out,
      with the amounts scaled relative to the fraction of days during the accrual period where a coupon trigger event occurs:
      \lstinline!Quantity! $*$ \lstinline!CouponRate! $* triggerAccrualFraction$
    \end{itemize}
  \item On the final determination date, if $worstPerformance < \min(Strike, KnockInLevel)$, pay \lstinline!Quantity! $* ($\lstinline!Strike!$ - worstPerformance)$, where $worstPerformance$ is the performance, i.e.\ $S_T/S_0$, of the worst-performing asset as of the final determination date $T$.
\end{enumerate}

The meanings and allowable values of elements in the \lstinline!EquityWorstOfBasketSwapData! node follow below.

\begin{itemize}
  \item \lstinline!Currency!: The payout currency for all cashflows. For FX, this should
  be \lstinline!CCY2! as defined (see Table \ref{tab:fxindex_data}) in the \lstinline!Name!
  field in the \lstinline!Underlying! node. Note that for (non-quanto) FxWorstOfBasketSwaps,
  this should be the domestic (\lstinline!CCY2!) currency, as defined in the Name field in the
  \lstinline!Underlying! node. For non-quanto Equity- and CommodityWorstOfBasketSwaps, this
  should be currency the equity or commodity is quoted in. Notice Section \ref{sss:payccy_st}. \\
    Allowable values: See Table \ref{tab:currency} \lstinline!Currency!.
  \item \lstinline!LongShort!: Own party position. \\
    Allowable values: \emph{Long}, \emph{Short}
  \item \lstinline!Quantity!: The equity notional amount, or the quantity multiplier. \\
    Allowable values: Any number.
  \item \lstinline!InitialFixedRate! [Optional]: The coupon rate for the initial fixed
  payment, to be paid to the \emph{Long} party. \\
    Allowable values: Any number, as a percentage expressed in decimal form. If left blank
    or omitted, defaults to zero.
  \item \lstinline!InitialFixedPayDate! [Optional]: Settlement date for the initial fixed
  payment. \\
    Allowable values: See \lstinline!Date! in Table \ref{tab:allow_stand_data}. If left blank
    or omitted, this defaults to the first date in the \lstinline!FloatingPayDates! schedule.
  \item \lstinline!Underlyings!: The basket of underlyings. \\
    Allowable values: For each underlying, see \ref{ss:underlying}.
  \item \lstinline!InitialPrices!: The observed price of each underlying in \lstinline!Underlyings!. \\
    Allowable values: For each underlying, an \lstinline!InitialPrice! sub-node with
    any positive number.
  \item \lstinline!BermudanKnockIn! [Optional]: Whether the knock-in event is observed
  on a regular basis (\emph{True}), as defined by the \lstinline!KnockInDeterminationSchedule!,
  or only at the final date in the \lstinline!FloatingPeriodSchedule! (\emph{False}). \\
    Allowable values: Boolean node, allowing \emph{Y}, \emph{N}, \emph{1}, \emph{0},
    \emph{true}, \emph{false}, etc. The full set of allowable values is given in Table
    \ref{tab:boolean_allowable}. If left blank or omitted, defaults to \emph{False}.
  \item \lstinline!KnockInLevel!: The barrier level used to determine whether
  a knock-in event has occurred for the given knock-in observation date/s, upon which the
  equity amount is payable at the \lstinline!KnockInPayDate!, subject to no knock-out event
  having occurred over the life of the contract. \\
  Allowable values: Any number, as a percentage of the initial prices expressed in decimal
  form.
  \item \lstinline!FloatingPeriodSchedule!: Period dates used for calculating floating coupon
  accrual fractions, and for observing underlying prices for calculating performances. \\
    Allowable values: See section \ref{ss:schedule_data} Schedule Data and Dates/Rules, or
    DerivedSchedule for scripted trades (see the Event data structure \ref{app:scriptedtrade}).
  \item \lstinline!FloatingFixingSchedule! [Optional]: Fixing dates for floating coupons linked
    to Ibor indices. The $d^\text{th}$ date in this schedule is the fixing date for the $(d+1)^\text{th}$
    date in the \lstinline!FloatingPayDates!. For floating coupons linked to overnight indices
    this scheulde is not relevant, the fixing dates are derived from the FloatingPeriodScheudle
    and FloatingLookback. \\
    Allowable values: See section \ref{ss:schedule_data} Schedule Data and Dates/Rules, or
    DerivedSchedule for scripted trades (see the Event data structure \ref{app:scriptedtrade}).
    If left blank or omitted, defaults to a \lstinline!DerivedSchedule! with the
    \lstinline!FloatingPeriodSchedule! set as the base schedule.
  \item \lstinline!FixedDeterminationSchedule! [Optional]: If \lstinline!AccruingFixedCoupons!
  is \emph{False}, the observation dates on which a fixed trigger event is evaluated, upon
  which a fixed coupon payout is made. Otherwise, there is no payout. If
  \lstinline!AccruingFixedCoupons! is \emph{True}, these will be the dates on which fixed
  coupon amounts are calculated. The coupon amount is scaled by the number of fixed trigger
  days relative to the total number of days in the period. \\
    Allowable values: See section \ref{ss:schedule_data} Schedule Data and Dates/Rules, or
    DerivedSchedule for scripted trades (see the Event data structure \ref{app:scriptedtrade}).
    If left blank or omitted, defaults to a \lstinline!DerivedSchedule! with the
    \lstinline!FloatingPeriodSchedule! set as the base schedule.
  \item \lstinline!KnockOutDeterminationSchedule! [Optional]: Dates on which a knock-out
  event is evaluated, after which the instrument is considered to be terminated and no
  payout is made. \\
    Allowable values: See section \ref{ss:schedule_data} Schedule Data and Dates/Rules, or
    DerivedSchedule for scripted trades (see the Event data structure \ref{app:scriptedtrade}).
    If left blank or omitted, defaults to a \lstinline!DerivedSchedule! with the
    \lstinline!FloatingPeriodSchedule! set as the base schedule.
  \item \lstinline!KnockInDeterminationSchedule! [Optional]: Observation dates for
  determining if a knock-in event has occurred. If \lstinline!BermudanKnockIn!
  is \emph{True}, this node is required. \\
    Allowable values: See section \ref{ss:schedule_data} Schedule Data and Dates/Rules, or
    DerivedSchedule for scripted trades (see the Event data structure \ref{app:scriptedtrade}).
  \item \lstinline!FixedAccrualSchedule! [Optional]: The fixed coupon paid out for each determination date
  in the \lstinline!FixedDeterminationSchedule! is scaled according to the number of days
  in the \lstinline!FixedAccrualSchedule! for which a fixed trigger event occurred, as a fraction of
  the total number of days in the period (defined by the
  \lstinline!FixedDeterminationSchedule!). If \lstinline!AccruingFixedCoupons! is
  \emph{True}, this node is required. \\
    Allowable values: See section \ref{ss:schedule_data} Schedule Data and Dates/Rules, or
    DerivedSchedule for scripted trades (see the Event data structure \ref{app:scriptedtrade}).
  \item \lstinline!FloatingPayDates!: Floating coupon payment dates. \\
    Allowable values: See section \ref{ss:schedule_data} Schedule Data and Dates, Rules, or
    DerivedSchedule (see \ref{app:scriptedtrade}).
  \item \lstinline!FixedPayDates! [Optional]: Fixed coupon payment dates. \\
    Allowable values: See section \ref{ss:schedule_data} Schedule Data and Dates/Rules, or
    DerivedSchedule for scripted trades (see the Event data structure \ref{app:scriptedtrade}).
    If left blank or omitted, defaults to the \lstinline!FloatingPayDates!.
  \item \lstinline!KnockInPayDate! [Optional]: Settlement date for the knock-in
  payment, subject to a knock-in event having been triggered. \\
    Allowable values: See \lstinline!Date! in Table \ref{tab:allow_stand_data}. If left blank
    or omitted, this defaults to the last date in the \lstinline!FloatingPayDates! schedule.
  \item \lstinline!KnockOutLevels!: For each determination date in
  \lstinline!KnockOutDeterminationSchedule! except for the first date, the barrier level used
  to determine whether a knock-out trigger event has occurred, resulting in a knock-out (i.e.\
  contract termination). \\
    Allowable values: For each knock-out determination date, a \lstinline!KnockOutLevel!
    sub-node with any positive number, as a percentage of the initial prices, expressed in
    decimal form.
  \item \lstinline!FixedTriggerLevels!: For each fixed trigger determination date, the
  barrier level used to determine whether a coupon trigger event has occurred, upon which a
  fixed coupon is paid out at the corresponding payment date. \\
    Allowable values: For each fixed trigger evaluation date, a \lstinline!FixedTriggerLevel!
    sub-node with any positive number, as a percentage of the initial prices expressed in
    decimal form.
  \item \lstinline!FixedRate!: Rate of the fixed coupon leg. \\
    Allowable values: Any number, as a percentage expressed in decimal form.
  \item \lstinline!AccumulatingFixedCoupons! [Optional]: Whether the fixed coupons are accumulating or not. \\
    Allowable values: Boolean node, allowing \emph{Y}, \emph{N}, \emph{1}, \emph{0},
    \emph{true}, \emph{false}, etc. The full set of allowable values is given in Table
    \ref{tab:boolean_allowable}. If left blank or omitted, defaults to \emph{False}.
  \item \lstinline!AccruingFixedCoupons! [Optional]: Whether the fixed coupons are paid
  out on an accrual basis (\emph{True}), or subject to a single fixed trigger event
  (given by the \lstinline!FixedDeterminationSchedule!). \\
    Allowable values: Boolean node, allowing \emph{Y}, \emph{N}, \emph{1}, \emph{0},
    \emph{true}, \emph{false}, etc. The full set of allowable values is given in Table
    \ref{tab:boolean_allowable}. If left blank or omitted, defaults to \emph{False}.
  \item \lstinline!FloatingIndex!: Underlying index of the floating leg. \\
    Allowable values: See \ref{app:scriptedtrade} (section Data Node / Index). This must
    be an InterestRate underlying. Overnight indices are not supported.
  \item \lstinline!FloatingSpread! [Optional]: Spread on the floating rate. \\
    Allowable values: Any number. If left blank or omitted, defaults to zero.
  \item \lstinline!FloatingDayCountFraction!: The day count fraction for the floating leg
  accrual calculations. \\
    Allowable values: See \lstinline!DayCount Convention! in Table \ref{tab:daycount}.
  \item \lstinline!Strike! [Optional]: Strike price for for each underlying in the basket.
  When calculating the equity amount, this is the price used to calculate the number of
  shares (for Equity underlyings) owing of the worst-performing asset. \\
    Allowable values: Any number, as a percentage of the initial prices, expressed in decimal
    form. If left blank or omitted, defaults to one, i.e.\ the initial price of the
    underlying.
  \item \lstinline!FloatingLookback! [Optional]: Only applicable when the floating leg reference
  rate is an overnight index. The shift applied to each value date in \lstinline!FloatingFixingSchedule!
  to obtain the fixing reference date. \\
    Allowable values: Any period definition in unit ``Days'' (e.g.\ \emph{2D}, \emph{30D}, but not \emph{1M}).
    If omitted or left blank, defaults to \emph{0D}.
  \item \lstinline!FloatingRateCutoff! [Optional]: Only applicable when the floating leg reference
  rate is an overnight index. The number of fixing dates at the end of the fixing period for which
  the fixing value is held constant and set to the previous value. \\
    Allowable values: Any non-negative whole number. If left blank or omitted, defaults to zero.
  \item \lstinline!IsAveraged! [Optional]: Only applicable when the floating leg
  reference rate is an overnight index (where there are multiple fixings over a period). If \emph{True},
  the average of the fixings is used to calculate the coupon. If \emph{False},
  the coupon is calculated by compounding the fixings. \\
    Allowable values: Boolean node, allowing \emph{Y}, \emph{N}, \emph{1}, \emph{0}, \emph{true},
    \emph{false}, etc. The full set of allowable values is given in Table \ref{tab:boolean_allowable}.
    Defaults to \emph{false} If left blank or omiited.
  \item \lstinline!IncludeSpread! [Optional]: Only applicable when the floating leg
  reference rate is an overnight index and if IsAveraged is false, i.e.\ the coupon rate is compounded.
  If \emph{true} the spread is included in the compounding, otherwise it is excluded. \\
     Allowable values:  Boolean node, allowing \emph{Y}, \emph{N}, \emph{1}, \emph{0}, \emph{true}, \emph{false}, etc.
  The full set of allowable values is given in Table \ref{tab:boolean_allowable}.
\end{itemize}

Worst Of Basket Swaps can alternatively be represented as {\em scripted trades}, refer to Section \ref{app:scriptedtrade} for an introduction.

\begin{minted}[fontsize=\footnotesize]{xml}
  <TradeType>ScriptedTrade</TradeType>
  <Envelope>
     ....
  </Envelope>
  <WorstOfBasketSwapData>
    <LongShort type="longShort">Long</LongShort>
    <Quantity type="number">1955000</Quantity>
    <InitialFixedRate type="number">0.015</InitialFixedRate>
    <Underlyings type="index">
      <Value>EQ-RIC:.STOXX50E</Value>
      <Value>EQ-RIC:.SPX</Value>
    </Underlyings>
    <InitialPrices type="number">
      <Value>3481.44</Value>
      <Value>3714.24</Value>
    </InitialPrices>
    <DeterminationDates type="event">
      <ScheduleData>
          ....
      </ScheduleData>
    </DeterminationDates>
    <SettlementDates type="event">
      <ScheduleData>
        <Dates>
          ....
        </Dates>
      </ScheduleData>
    </SettlementDates>
    <KnockOutLevels type="number">
      <Value>...</Value>
    </KnockOutLevels>
    <CouponTriggerLevels type="number">
      <Value>...</Value>
    </CouponTriggerLevels>
    <KnockInLevel type="number">0.75</KnockInLevel>
    <CouponRate type="number">0.04</CouponRate>
    <AccumulatingCoupons type="bool">true</AccumulatingCoupons>
    <FloatingIndex type="index">EUR-EURIBOR-3M</FloatingIndex>
    <FloatingSpread type="number">0.002</FloatingSpread>
    <FloatingDayCountFraction type="dayCounter">Actual/360</FloatingDayCountFraction>
    <FixingSchedule type="event">
      <DerivedSchedule>
          ....
      </DerivedSchedule>
    </FixingSchedule>
    <Strike type="number">1.0</Strike>
    <PayCcy type="currency">EUR</PayCcy>
  </WorstOfBasketSwapData>
\end{minted}

The WorstOfBasketSwap script referenced in the trade above is shown in Listing \ref{lst:worst_of_basket_swap}.

\begin{listing}[hbt] 
\begin{minted}[fontsize=\footnotesize]{Basic} 
REQUIRE SIZE(Underlyings) == SIZE(InitialPrices);
REQUIRE SIZE(SettlementDates) == SIZE(DeterminationDates);
REQUIRE SIZE(KnockOutLevels) == SIZE(DeterminationDates) - 1;
REQUIRE SIZE(CouponTriggerLevels) == SIZE(DeterminationDates) - 1;

NUMBER alive, couponAccumulation, numOfKnockedAssets, fixing, n, accrualFraction, indexInitial;
NUMBER numOfTriggeredAssets, indexFinal, performance, worstPerformance, d, payoff, u;

Option = Option + LOGPAY(LongShort * Quantity * InitialFixedRate,
                         SettlementDates[1], SettlementDates[1], PayCcy, 0, InitialFixedAmount);

alive = 1;
couponAccumulation = 1;
n = SIZE(DeterminationDates);

FOR d IN (2, n, 1) DO
  fixing = FloatingIndex(FixingSchedule[d-1]) + FloatingSpread;
  accrualFraction = dcf(FloatingDayCountFraction, DeterminationDates[d-1], DeterminationDates[d]);
  Option = Option - LOGPAY(LongShort * Quantity * alive * fixing * accrualFraction,
                           FixingSchedule[d-1], SettlementDates[d], PayCcy, 1, FloatingLeg);
  
  numOfTriggeredAssets = 0;
  FOR u IN (1, SIZE(Underlyings), 1) DO
    IF Underlyings[u](DeterminationDates[d]) >= CouponTriggerLevels[d-1] * InitialPrices[u] THEN
      numOfTriggeredAssets = numOfTriggeredAssets + 1;
    END;
  END;
  IF numOfTriggeredAssets == SIZE(Underlyings) THEN
    Option = Option + LOGPAY(LongShort * Quantity * alive * CouponRate * couponAccumulation,
                             SettlementDates[d], SettlementDates[d], PayCcy, 2, FixedCouponLeg);
    couponAccumulation = 1;
  ELSE
    IF AccumulatingCoupons == 1 THEN
      couponAccumulation = couponAccumulation + 1;
    END;
  END;

  IF d == n THEN
    FOR u IN (1, SIZE(Underlyings), 1) DO
      indexInitial = InitialPrices[u];
      indexFinal = Underlyings[u](DeterminationDates[n]);
      performance = indexFinal / indexInitial;

      IF {u == 1} OR {performance < worstPerformance} THEN
        worstPerformance = performance;
      END;
    END;       

    IF worstPerformance < min(Strike, KnockInLevel) THEN
      payoff = worstPerformance - Strike;
      Option = Option + LOGPAY(LongShort * Quantity * alive * payoff, DeterminationDates[n],
                               SettlementDates[n], PayCcy, 3, EquityAmountPayoff);
    END;
  END;

  IF d != n THEN
    numOfKnockedAssets = 0;
    FOR u IN (1, SIZE(Underlyings), 1) DO
      IF Underlyings[u](DeterminationDates[d]) >= KnockOutLevels[d-1] * InitialPrices[u] THEN
        numOfKnockedAssets = numOfKnockedAssets + 1;
      END;
    END;
    IF numOfKnockedAssets == SIZE(Underlyings) THEN
      alive = 0;
    END;
  END;
END;
\end{minted} 
\caption{Payoff script for a Worst Of Basket Swap.} 
\label{lst:worst_of_basket_swap}
\end{listing} 
 
The net cashflows under a (scripted) worst of basket swap from the long perspective are:
\begin{enumerate}
  \item Receive an initial fixed amount: \lstinline!InitialFixedRate! $*$ \lstinline!Quantity!,
  \item At each determination date (if a knock-out has not occurred for any previous determination date): \begin{itemize}
      \item Pay a floating amount: \lstinline!Quantity! $* floatingRate * accrualFraction$,
      \item If a coupon trigger event occurs for this date, receive a fixed coupon amount (as well as any other fixed coupon amounts not previously received because a coupon trigger event had not occurred): \lstinline!Quantity! $*$ \lstinline!CouponRate!
    \end{itemize}
  \item On the final determination date, if $worstPerformance < \min(Strike, KnockInLevel)$, pay \lstinline!Quantity! $* ($\lstinline!Strike!$ - worstPerformance)$, where $worstPerformance$ is the performance, i.e.\ $S_T/S_0$, of the worst-performing asset as of the final determination date $T$.
\end{enumerate}

The meanings and allowable values of the elements in the \lstinline!WorstOfBasketSwap! node below.

\begin{itemize}
  \item{}[longShort] \lstinline!LongShort!: Own party position. \\
  Allowable values: \emph{Long, Short}
  \item{}[number] \lstinline!Quantity!: Quantity multiplier, or the equity notional amount. \\
  Allowable values: Any number.
  \item{}[number] \lstinline!InitialFixedRate!: Rate of the initial fixed payment. \\
  Allowable values: Any number, as a percentage expressed in decimal form.
  \item{}[index] \lstinline!Underlyings!: The basket of underlyings. \\
  Allowable values: See Section \ref{data_index} for allowable values.
  \item{}[number] \lstinline!InitialPrices!: The agreed initial price for each underlying in the basket. \\
  Allowable values: Any positive number. The number of \lstinline!Value! sub-nodes should match the number of underlyings.
  This must have the same number of entries as \lstinline!Underlyings!.
  \item{}[event] \lstinline!DeterminationDates!: Floating leg period dates, knock-out determination dates, and fixed trigger
  determination dates. The first date is only used for calculating the floating leg accrual fraction, and corresponds
  to the contract effective date. \\
  Allowable values: See Section \ref{ss:schedule_data} ScheduleData, or DerivedSchedule (see \ref{app:scriptedtrade}) if
  derived from \lstinline!SettlementDates! or \lstinline!FixingSchedule!. The number of determination dates should
  match the number of settlement and fixing dates.
  \item{}[event] \lstinline!SettlementDates!: The set of settlement dates corresponding to each determination date. While the first
  settlement date is unused in the calculations, this format simplifies the input, e.g.\ the settlement schedule can be
  a derived schedule, with the \emph{DeterminationDates} as the base schedule. \\
  Allowable values: See section \ref{ss:schedule_data} ScheduleData, or DerivedSchedule (see \ref{app:scriptedtrade}) if
  derived from \lstinline!DeterminationDates! or \lstinline!FixingSchedule!. The number of settlement dates should
  match the number of determination and fixing dates.
  \item{}[number] \lstinline!KnockOutLevels!: For each determination date, the barrier level used to determine whether a knock-out
  trigger event has occurred, resulting in a knock-out (i.e.\ contract termination) at the corresponding settlement date. \\
  Allowable values: Any number, as a percentage of the initial prices expressed in decimal form. This should have the same number
  of entries as the number of dates in \lstinline!DeterminationDates! minus one.
  \item{}[number] \lstinline!CouponTriggerLevels!: For each determination date, the barrier level used to determine whether a coupon
  trigger event has occurred, upon which a fixed coupon is paid out at the corresponding settlement date. \\
  Allowable values: Any number, as a percentage of the initial prices expressed in decimal form. This should have the same number
  of entries as the number of dates in \lstinline!DeterminationDates! minus one.
  \item{}[number] \lstinline!KnockInLevel!: For the final determination date, the barrier level used to determine whether
  a knock-in event has occurred, upon which the equity amount is payable at the final settlement date, subject to no
  knock-out event having occurred in the life of the contract. \\
  Allowable values: Any number, as a percentage of the initial price expressed in decimal form.
  \item{}[number] \lstinline!CouponRate!: Rate of the fixed coupon leg. \\
  Allowable values: Any number, as a percentage expressed in decimal form.
  \item{}[bool] \lstinline!AccumulatingCoupons!: Whether the fixed coupons are accumulating or not. \\
  Allowable values: Boolean node, allowing \emph{Y}, \emph{N}, \emph{1}, \emph{0}, \emph{true}, \emph{false}, etc. The
  full set of allowable values is given in Table \ref{tab:boolean_allowable}.
  \item{}[index] \lstinline!FloatingIndex!: Underlying index of the floating leg. \\
  Allowable values: See Section \ref{data_index} for allowable values. Overnight indices are not supported.
  \item{}[number] \lstinline!FloatingSpread!: Spread on the floating rate. \\
  Allowable values: Any number.
  \item{}[dayCounter] \lstinline!FloatingDayCountFraction!: The day count fraction for the floating leg accrual
  calculations. \\
  Allowable values: See \lstinline!DayCount Convention! in Table \ref{tab:daycount}.
  \item{}[event] \lstinline!FixingSchedule!: The fixing schedule of the floating leg payments. \\
  Allowable values: See Section \ref{ss:schedule_data} ScheduleData, or DerivedSchedule (see \ref{app:scriptedtrade}) if
  derived from \lstinline!DeterminationDates! or \lstinline!SettlementDates!. The number of fixing dates should
  match the number of determination and settlement dates.
  \item{}[number] \lstinline!Strike!: Strike price for each underlying in the basket. When calculating the equity amount,
  this is the price used to calculate the number of shares owing of the worst-performing asset. \\
  Allowable values: Any number, as a percentage of the initial prices, expressed in decimal form.
  \item{}[currency] \lstinline!PayCcy!: Settlement currency. Notice section \ref{sss:payccy_st}. \\
  Allowable values: See Table \ref{tab:currency} \lstinline!Currency!.
\end{itemize}