\subsubsection{Autocallable Type 01}

The \verb+Autocallable_01+ node is the trade data container for the Autocallable\_01 trade type, listing
\ref{lst:autocallable01_data} shows the structure of an example.

\begin{listing}[H]
\begin{minted}[fontsize=\footnotesize]{xml}
    <Autocallable01Data>
      <NotionalAmount>12000000</NotionalAmount>
      <DeterminationLevel>11.0</DeterminationLevel>
      <TriggerLevel>9.8</TriggerLevel>
      <Underlying>
        <Type>FX</Type>
        <Name>ECB-EUR-NOK</Name>
      </Underlying>
      <Position>Long</Position>
      <PayCcy>EUR</PayCcy>
      <FixingDates>
        <ScheduleData>
          <Dates>
            <Dates>
              <Date>2018-09-27</Date>
              <Date>2019-09-27</Date>
              <Date>2020-09-27</Date>
              <Date>2021-09-29</Date>
              <Date>2022-09-28</Date>
            </Dates>
          </Dates>
        </ScheduleData>
      </FixingDates>
      <SettlementDates>
        <ScheduleData>
          <Dates>
            <Dates>
              <Date>2018-10-07</Date>
              <Date>2019-10-09</Date>
              <Date>2020-10-07</Date>
              <Date>2021-10-07</Date>
              <Date>2022-10-07</Date>
            </Dates>
          </Dates>
        </ScheduleData>
      </SettlementDates>
      <AccumulationFactors>
        <Factor>0.344</Factor>
        <Factor>0.733</Factor>
        <Factor>0.911</Factor>
        <Factor>1.123</Factor>
        <Factor>1.544</Factor>
      </AccumulationFactors>
      <Cap>1.0</Cap>
    </Autocallable01Data>
\end{minted}
\caption{Autocallable Type 01 data}
\label{lst:autocallable01_data}
\end{listing}

If a trigger event occurs on the $i$-th fixing date, the option holder receives the following:

\begin{equation*}
  Payout = \text{\lstinline!NotionalAmount!} * AccumulationFactor_i.
\end{equation*}

If a trigger event never occurs and the underlying spot at the fixing date $f_n$ is above the \lstinline!DeterminationLevel!,
the option holder pays the following:

\begin{equation*}
  Payout = \min (\text{\lstinline!Cap!}, \text{\lstinline!Underlying!}(f_n) - \text{\lstinline!DeterminationLevel!}).
\end{equation*}

The meanings and allowable values of the elements in the \verb+Autocallable01Data+ node follow below.

\begin{itemize}
\item NotionalAmount: The notional amount of the option. Allowable values are non-negative numbers.
\item DeterminationLevel: The determination level. For an FX underlying FX-SOURCE-CCY1-CCY2 this is the number of units
  of CCY2 per units of CCY1. For an EQ underlying this is the equity price expressed in the equity ccy.  Allowable
  values are non-negative numbers.
\item TriggerLevel: The trigger level. For an FX underlying FX-SOURCE-CCY1-CCY2 this is the number of units of CCY2 per
  units of CCY1. For an EQ underlying this is the equity price expressed in the equity ccy.  Allowable values are
  non-negative numbers.
\item Underlying: The option underlying, see \ref{ss:underlying}.
\item Position: The option position type. Allowable values: {\em Long} or {\em Short}.
\item PayCcy: The pay currency of the option. See the appendix for allowable currency codes.
\item FixingDates: The fixing date schedule given as a \verb+ScheduleData+ subnode, see \ref{ss:schedule_data}
\item SettlementDates: The settlement date schedule given as a \verb+ScheduleData+ subnode, see \ref{ss:schedule_data}
\item AccumulationFactors: The accumulation factors given as a list of values corresponding to the fixing
  dates. Allowable values are non-negative numbers.
\item Cap: The maximum amount, per unit of notional, payable by the option holder if a trigger event never occurred
(i.e.\ underlying value was never below the \verb+TriggerLevel+ on any of the fixing dates) and if the underlying value
is greater than the \verb+DeterminationLevel+ at the last fixing date. Allowable values are non-negative numbers.
\end{itemize}
