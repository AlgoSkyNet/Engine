\subsubsection{Credit Default Swap}

A credit default swap is set up using a {\tt CreditDefaultSwapData} block as shown in listing \ref{lst:cdsdata}. The CDS specific elements are

\begin{itemize}
\item IssuerId: An identifier for the issuer of the reference entity of the CDS. For informational purposes and not used for pricing.
\item CreditCurveId: The ISIN of the reference entity defining the default curve used for pricing.
\item SettlesAccrual: Whether or not the accrued coupon is due in the event of a default.
\item PaysAtDefaultTime: If set to true, any payments triggered by a default event are due at default time. If set to false, they are due at the end of the accrual period.
\item ProtectionStart: The first date where a default event will trigger the contract.
\item UpfrontDate [Optional]: Settlement date for the upfront payment.
\item UpfrontFee [Optional]: The upfront payment, expressed as a rate, to be multiplied by notional amount.
\end{itemize}

A LegData block then defines the cashflow structure of the credit default swap, this must be be of type \emph{Fixed}.

\begin{listing}[H]
%\hrule\medskip
\begin{minted}[fontsize=\footnotesize]{xml}
    <CreditDefaultSwapData>
      <IssuerId>CPTY_A</IssuerId>
      <CreditCurveId>ISIN:XS0982710740</CreditCurveId>
      <SettlesAccrual>Y</SettlesAccrual>
      <PaysAtDefaultTime>Y</PaysAtDefaultTime>
      <ProtectionStart>20160206</ProtectionStart>
      <UpfrontDate>20160208</UpfrontDate>
      <UpfrontFee>0.0</UpfrontFee>
      <LegData>
            <LegType>Fixed</LegType>
            <Payer>false</Payer>
            ...
      </LegData>
    </CreditDefaultSwapData>
\end{minted}
\caption{CreditDefaultSwap Data}
\label{lst:cdsdata}
\end{listing}