\subsubsection{Equity Swap}
\label{ss:equity_swap}

An Equity Swap uses its own trade type  \emph{EquitySwap}, and is set up using a {\tt EquitySwapData} node with one leg of type  \emph{Equity} and one more leg that can be either \emph{Fixed} or  \emph{Floating}. Listing \ref{lst:equityswap} shows an example. The
Equity leg contains an additional {\tt EquityLegData} block. See \ref{ss:equitylegdata} for details on the Equity leg specification.

\begin{listing}[H]
%\hrule\medskip
\begin{minted}[fontsize=\footnotesize]{xml}
    <EquitySwapData>
      <LegData>
        <LegType>Floating</LegType>
        <Payer>true</Payer>
        ...
      </LegData>
      <LegData>
        <LegType>Equity</LegType>
        <Payer>false</Payer>
        ...
        <EquityLegData>
        ...
        </EquityLegData>
      </LegData>
    </EquitySwapData>
\end{minted}
\caption{Equity Swap Data}
\label{lst:equityswap}
\end{listing}

If the equity swap has a resetting notional, typically the funding leg's notional will be aligned with the equity leg's
notional. To achieve this, indexings on the floating leg can be used, see \ref{ss:indexings}. In the context of equity
swaps the indexings can be defined in a simplified way:

\begin{itemize}
\item the Index node can be omitted in the Indexing definition for the equity price scaling on the funding leg, in which
  case the equity index identifier is copied to the funding leg from the equity leg
\item the Notionals node on the funding leg can be omitted, in which case it is filled with a single notional of 1
\item the InitialFixing in the equity Indexing defintion is set to the equity leg InitialPrice if the latter is given
  and the former is not given
\item the valuation schedule for all indexings on the funding leg can be omitted, in which case the relevant fixing
  dates from the equity leg are copied to the funding leg and the fields FixingDays, FixingCalendar, FixingConvention
  and IsInArrears are reset to the default values to ensure that the equity leg's fixing dates are preserved
\end{itemize}

Listing \ref{lst:equityswap_reset} shows an example for an equity swap with notional reset, equity ccy = EUR and funding
ccy = USD. The equity indexing definition on the funding leg only requires the Quantity, the FX indexing definition the
FX index and index fixing days and calendar.


\begin{listing}[H]
\begin{minted}[fontsize=\footnotesize]{xml}
  <SwapData>
    <LegData>
      <LegType>Floating</LegType>
      <Currency>USD</Currency>
      ...
      <Indexings>
        <Indexing>
          <Quantity>1000</Quantity>
        </Indexing>
        <Indexing>
          <Index>FX-ECB-EUR-USD</Index>
          <IndexFixingDays>2</IndexFixingDays>
          <IndexFixingCalendar>EUR,USD</IndexFixingCalendar>
        </Indexing>
      </Indexings>
    </LegData>
    <LegData>
      <LegType>Equity</LegType>
      <Currency>USD</Currency>
      ...
      <EquityLegData>
        <Quantity>1000</Quantity>
        <Name>RIC:.STOXX50E</Name>
        <InitialPrice>2937.36</InitialPrice>
        <NotionalReset>true</NotionalReset>
        <FXTerms>
          <EquityCurrency>EUR</EquityCurrency>
          <FXIndex>FX-ECB-EUR-USD</FXIndex>
          <FXIndexFixingDays>2</FXIndexFixingDays>
          <FXIndexCalendar>EUR,USD</FXIndexCalendar>
        </FXTerms>
      </EquityLegData>
      ...
    </LegData>
  </SwapData>
\end{minted}
\caption{Equity Swap Data with notional reset and FX indexing}
\label{lst:equityswap_reset}
\end{listing}
