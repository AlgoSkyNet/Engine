\subsubsection{Bond}
\label{ss:bond}

A vanilla Bond is set up using a {\tt BondData} block as shown in listing \ref{lst:bonddata_refdata}. The details of the
bond are read from the reference data in this case using the SecurityId as a key. The bond trade is fully specified by

\begin{itemize}
\item SecurityId: The id identifying the bond\\
  Allowable Values: each valid bond identifier, usually this is of the form \verb+ISIN:XXNNNNNNNNNN+
\item BondNotional: The bond notional, i.e. the number of bonds held\\
  Allowable Values: a real number
\end{itemize}

in this case.

\begin{listing}[H]
%\hrule\medskip
\begin{minted}[fontsize=\footnotesize]{xml}
    <BondData>
      <SecurityId>ISIN:XS0982710740</SecurityId>
      <BondNotional>100000000.0</BondNotional>
    </BondData>
\end{minted}
\caption{Bond Data}
\label{lst:bonddata_refdata}
\end{listing}

The bond details can also be inlined in the trade as shown in listing \ref{lst:bonddata}. The bond specific elements are

\begin{itemize}
\item IssuerId: A unique identifier for the issuer of the bond.
\item CreditCurveId: The identifier of the default curve used for pricing, which must
match a default curve id in the market data configuration.
%Allowable values: 
%See \lstinline!Name! for credit trades in Table \ref{tab:equity_credit_data}. \\
%via the default curves block in {\tt  todaysmarket.xml}
% \item LGD (optional): If given, this LGD is used for pricing, overriding the default LGD of the default curve
\item SecurityId: A unique identifier for the security, this defines the security specific spread to be used for pricing.
\item ProxySecurityId [Optional]: Only applicable to exotic bonds, which have the BondData block embedded as one of the
  components typically. An identifier of a proxy security. If given, the security curve configuration, i.e. the security
  spread and recovery rate of the proxy security will be used for the pricing of the exotic bond. Typically the ISIN of
  a liquid vanilla bond of the same issuer and with comparable maturity as the convertible bond. The proxy security must
  be a vanilla bond.
\item ReferenceCurveId: The benchmark curve to be used for pricing, this must match a name of a yield curve or an index curve in the market data configuration.
\item SettlementDays: The settlement delay applicable to the security.
\item Calendar: The calendar associated to the settlement lag.
\item IssueDate: The issue date of the security.
\end{itemize}

A LegData block then defines the cashflow structure of the bond, this can be of type fixed, floating etc.

\begin{listing}[H]
%\hrule\medskip
\begin{minted}[fontsize=\footnotesize]{xml}
    <BondData>
        <IssuerId>CPTY_C</IssuerId>
        <CreditCurveId>ISIN:XS0982710740</CreditCurveId>
        <SecurityId>ISIN:XS0982710740</SecurityId>
        <ProxySecurityId>ISIN:XS2000000000</ProxySecurityId>
        <ReferenceCurveId>EUR-EURIBOR-6M</ReferenceCurveId>
        <SettlementDays>2</SettlementDays>
        <Calendar>TARGET</Calendar>
        <IssueDate>20160203</IssueDate>
        <LegData>
            <LegType>Fixed</LegType>
            <Payer>false</Payer>
            ...
        </LegData>
    </BondData>
\end{minted}
\caption{Bond Data}
\label{lst:bonddata}
\end{listing}

The bond pricing requires a recovery rate that can be specified per SecurityId in the market data configuration.