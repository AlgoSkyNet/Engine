\subsubsection{Bond}
\label{ss:bond}

A vanilla Bond is set up using a {\tt BondData} block as shown in listing \ref{lst:bonddata}. The bond specific elements
are

\begin{itemize}
\item IssuerId: A unique identifier for the issuer of the bond.
\item CreditCurveId: The identifier of the default curve used for pricing, which must
match a default curve id in the market data configuration.
%Allowable values: 
%See \lstinline!Name! for credit trades in Table \ref{tab:equity_credit_data}. \\
%via the default curves block in {\tt  todaysmarket.xml}
% \item LGD (optional): If given, this LGD is used for pricing, overriding the default LGD of the default curve
\item SecurityId: A unique identifier for the security, this defines the security specific spread to be used for pricing.
\item ReferenceCurveId: The benchmark curve to be used for pricing, this must match a name of a yield curve in the market data configuration.
\item SettlementDays: The settlement delay applicable to the security.
\item Calendar: The calendar associated to the settlement lag.
\item IssueDate: The issue date of the security.
\end{itemize}

A LegData block then defines the cashflow structure of the bond, this can be of type fixed, floating etc.

\begin{listing}[H]
%\hrule\medskip
\begin{minted}[fontsize=\footnotesize]{xml}
    <BondData>
        <IssuerId>CPTY_C</IssuerId>
        <CreditCurveId>ISIN:XS0982710740</CreditCurveId>
        <SecurityId>SECURITY_1</SecurityId>
        <ReferenceCurveId>EUR-EURIBOR-6M</ReferenceCurveId>
        <SettlementDays>2</SettlementDays>
        <Calendar>TARGET</Calendar>
        <IssueDate>20160203</IssueDate>
        <LegData>
            <LegType>Fixed</LegType>
            <Payer>false</Payer>
            ...
        </LegData>
    </BondData>
\end{minted}
\caption{Bond Data}
\label{lst:bonddata}
\end{listing}

The bond pricing requires a recovery rate that can be specified per SecurityId in the market data configuration.