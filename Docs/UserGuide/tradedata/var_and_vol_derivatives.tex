\subsubsection{Exotic Variance and Volatility Derivatives}
\label{SubSectionExoticVarianceSwap}

Exotic variance/volatility swaps and options are represented as {\em scripted trades}, refer to Section
\ref{app:scriptedtrade} for an introduction. Each of the supported variations and their payoff script
name are shown in Table \ref{tab:exotic_variance_swap}. All swaps have an optional cap/floor feature.

\begin{table}[hbt]
\begin{center}
\begin{tabular}{|l|l|}
\hline
Variance/Volatility Product Variation & Payoff Script Name \\
\hline
\hline
Variance/Volatility Option & \lstinline!VarianceOption! \\
\hline
Variance/Volatility Swap with KI/KO Barrier & \lstinline!KIKOVarianceSwap! \\
\hline
Corridor Volatility/Variance Swap & \lstinline!CorridorVarianceSwap! \\
\hline
Corridor Variance Swap with KI/KO Barrier & \lstinline!KIKOCorridorVarianceSwap! \\
\hline
Conditional Variance/Volatility Swap &  \begin{tabular}{@{}l@{}} \lstinline!ConditionalVarianceSwap01!\\\lstinline!ConditionalVarianceSwap02!
                                        \end{tabular} \\
\hline
Pairwise Variance Swap & \lstinline!PairwiseVarianceSwap! \\
\hline
Variance Dispersion Swap & \lstinline!VarianceDispersionSwap! \\
\hline
Corridor Variance Dispersion Swap & \lstinline!CorridorVarianceDispersionSwap! \\
\hline
Corridor Variance Dispersion Swap with KO Barrier & \lstinline!KOCorridorVarianceDispersionSwap! \\
\hline
Gamma Swap & \lstinline!GammaSwap! \\
\hline
Basket Variance Swap & \lstinline!BasketVarianceSwap! \\
\hline
\end{tabular}
\end{center}
\caption{Exotic variance/volatility product types and associated script names.}
\label{tab:exotic_variance_swap}
\end{table}

Trade input and the associated payoff script are described in the following for all supported variations.

\subsubsection*{Variance Option}

The traditional trade representation is as follows, using an EQ underlying in this example:

\begin{minted}[fontsize=\footnotesize]{xml}
<Trade id="EQ_VarianceOption">
  <TradeType>ScriptedTrade</TradeType>
  <Envelope>
    .....
  </Envelope>
  <VarianceOptionData>
    <LongShort type="longShort">Long</LongShort>
    <PutCall type="optionType">Call</PutCall>
    <PremiumAmount type="number">0</PremiumAmount>
    <PremiumDate type="event">2020-11-26</PremiumDate>
    <Notional type="number">138000</Notional>
    <VarianceReference type="number">0.19</VarianceReference>
    <Strike type="number">0.19</Strike>
    <Underlying type="index">EQ-RIC:.SPX</Underlying>
    <ValuationSchedule type="event">
      <ScheduleData>
        <Rules>
          <StartDate>2020-11-26</StartDate>
          <EndDate>2021-09-18</EndDate>
          <Tenor>1D</Tenor>
          <Convention>Following</Convention>
          <TermConvention>Following</TermConvention>
          <Calendar>USA</Calendar>
          <Rule>Forward</Rule>
        </Rules>
      </ScheduleData>
    </ValuationSchedule>
    <SquaredPayoff type="bool">true</SquaredPayoff>
    <SettlementDate type="event">2021-09-22</SettlementDate>
    <PayCcy type="currency">USD</PayCcy>
  </VarianceOptionData>
</Trade>
\end{minted}

The VarianceOption script referenced in the trade above is shown in listing
\ref{lst:variance_option}

\begin{listing}[hbt]
\begin{minted}[fontsize=\footnotesize]{Basic}
REQUIRE {Notional >= 0} AND {Strike >= 0};

NUMBER expectedN, realisedVariance, currPrice, currentNotional;
NUMBER prevPrice, payoff, realisedVariation, strike, premium, d;

FOR d IN (2, SIZE(ValuationSchedule), 1) DO
  currPrice = Underlying(ValuationSchedule[d]);
  prevPrice = Underlying(ValuationSchedule[d-1]);
  realisedVariance = realisedVariance + pow(ln(currPrice/prevPrice), 2);
END;

expectedN = SIZE(ValuationSchedule) - 1;
realisedVariance = (252/expectedN) * realisedVariance;

IF SquaredPayoff == 1 THEN
  realisedVariation = realisedVariance;
  currentNotional = pow(100, 2) * Notional / (2 * 100 * VarianceReference);
  strike = pow(Strike, 2);
ELSE
  realisedVariation = sqrt(realisedVariance);
  currentNotional = 100 * Notional;
  strike = Strike;
END;

payoff = currentNotional * max(PutCall * (realisedVariation - strike), 0);

NUMBER ExerciseProbability;
IF payoff > 0 THEN
  ExerciseProbability = 1;
END;

premium = PAY(PremiumAmount, PremiumDate, SettlementDate, PayCcy);
payoff = PAY(payoff, ValuationSchedule[SIZE(ValuationSchedule)],
             SettlementDate, PayCcy);
Option = LongShort * (payoff - premium);
\end{minted}
\caption{Payoff script for a VarianceOption.}
\label{lst:variance_option}
\end{listing}

The payout formulas for a variance put and call option are:

\begin{align*}
  Put Payoff &= 100^2 * \frac{\text{\lstinline!Notional!}}{2 * 100 * \text{\lstinline!VarianceReference!}} * \max[\text{\lstinline!Strike!} - \text{\lstinline!RealisedVariance!}, 0], \\
  Call Payoff &= 100^2 * \frac{\text{\lstinline!Notional!}}{2 * 100 * \text{\lstinline!VarianceReference!}} * \max[\text{\lstinline!Strike!} - \text{\lstinline!RealisedVariance!}, 0].
\end{align*}

The meanings and allowable values for the \lstinline!VarianceOptionData! node below.

\begin{itemize}
  \item{}[longShort] \lstinline!LongShort!: Own party position in the option. \emph{Long} corresponds to paying out on the
  fixed/strike variance (volatility) and receiving on the floating/realised variance (volatility). In other words,
  a long position has positive value if the realised variance (volatility) exceeds the variance (volatility)
  strike. \\
  Allowable values: \emph{Long, Short}.
  \item{}[optionType] \lstinline!PutCall!: Option type. For FX, this should be \emph{Call} if we buy
  \lstinline!CCY1! and sell \lstinline!CCY2!,
  \emph{Put} if we buy \lstinline!CCY2! and sell \lstinline!CCY1! (where the \lstinline!Underlying! is in the
  form \lstinline!FX-SOURCE-CCY1-CCY2!). \\
  Allowable values: \emph{Put, Call}
  \item{}[number] \lstinline!PremiumAmount!: The total option premium amount.
  Allowable values: Any non-negativenumber.
  \item{}[event] \lstinline!PremiumDate!: The option premium payment date.
  Allowable values: See \lstinline!Date! in Table \ref{tab:allow_stand_data}.
  \item{}[number] \lstinline!Notional!: The vega notional amount. If the option was struck in terms of a variance notional
  $N_{var}$, the corresponding vega notional is given by $N_{vol} = N_{var} \cdot 2 \cdot 100 \cdot K_{vol}$ (where
  $K_{vol}$ is in absolute terms). \\
  Allowable values: Any non-negative number.
  \item{}[number] \lstinline!Strike!: The volatility strike $K_{vol}$ quoted in absolute terms. If
  the option was struck in terms of variance, the square root of that variance should be used here. \\
  Allowable values: Any non-negative number, as a percentage expressed in decimal form.
  \item{}[number] \lstinline!VarianceReference!: The parameter used to convert the vega notional amount into the corresponding
  variance amount. Similar to the \lstinline!Strike!, this should be quoted in absolute terms, e.g.\ if the
  (volatility) strike price is 20\% and the variance reference is 32.4\%, then the \lstinline!Strike! is \emph{0.20}
  and the \lstinline!VarianceReference! should be \emph{0.324}. \\
  Allowable values: Any non-negative number, as a percentage expressed in decimal form.
  \item{}[index] \lstinline!Underlying!: Underlying index. \\
  Allowable values: See Section \ref{data_index} for allowable values.
  \item{}[event] \lstinline!ValuationSchedule!: The schedule defining the (daily) observation period for the variance accrual. \\
  Allowable values: See Section \ref{ss:schedule_data}.
  \item{}[bool] \lstinline!SquaredPayoff!: Flag indicating whether the trade is a variance option (\emph{True}) or a volatility
  option (\emph{False}). \\
  Allowable values: Boolean node, allowing \emph{Y}, \emph{N}, \emph{1}, \emph{0}, \emph{true}, \emph{false}, etc.
  The full set of allowable values is given in Table \ref{tab:boolean_allowable}.
  \item{}[event] \lstinline!SettlementDate!: The date on which the option payoff is settled. \\
  Allowable values: See \lstinline!Date! in Table \ref{tab:allow_stand_data}.
  \item{}[currency] \lstinline!PayCcy!: The payment currency. For FX, where the underlying is provided
      in the form \lstinline!FX-SOURCE-CCY1-CCY2! (see Table \ref{tab:fxindex_data}) this should
      be \lstinline!CCY2!. If \lstinline!CCY1! or the currency of the underlying (for EQ and
      COMM underlyings), this will result in a quanto payoff. Notice section \ref{sss:payccy_st}. \\
        Allowable values: See Table \ref{tab:currency}  for allowable currency codes.
\end{itemize}

\subsubsection*{Variance Swap with KI/KO Barrier}

The traditional trade representation is as follows, using an FX underlying in this example:

\begin{minted}[fontsize=\footnotesize]{xml}
<Trade id="FX_VarianceSwap_KIKO">
  <TradeType>ScriptedTrade</TradeType>
  <Envelope>
    .....
  </Envelope>
  <KIKOVarianceSwapData>    
    <LongShort type="longShort">Long</LongShort>
    <Strike type="number">0.02</Strike>
    <Notional type="number">50000</Notional>
    <Underlying type="index">FX-ECB-EUR-USD</Underlying>
    <ValuationSchedule type="event">
      <ScheduleData>
        <Rules>
          <StartDate>2018-12-31</StartDate>
          <EndDate>2019-05-05</EndDate>
          <Tenor>1D</Tenor>
          <Convention>F</Convention>
          <TermConvention>F</TermConvention>
          <Calendar>US</Calendar>
          <Rule>Forward</Rule>
        </Rules>
      </ScheduleData>
    </ValuationSchedule>
    <SquaredPayoff type="bool">true</SquaredPayoff>
    <BarrierType type="barrierType">UpIn</BarrierType>
    <BarrierLevel type="number">0</BarrierLevel>
    <Cap type="number">2.5</Cap>
    <Floor type="number">0</Floor>
    <SettlementDate type="event">2019-05-06</SettlementDate>
    <PayCcy type="currency">USD</PayCcy>
  </KIKOVarianceSwapData>
</Trade>
\end{minted}

The KIKOVarianceSwap script referenced in the trade above is shown in listing
\ref{lst:kiko_variance_swap}

\begin{listing}[hbt]
\begin{minted}[fontsize=\footnotesize]{Basic}
REQUIRE {Notional >= 0} AND {Strike >= 0};
REQUIRE {Cap >= 0} AND {Floor >= 0};

alive = 1;
FOR d IN (2, SIZE(ValuationSchedule), 1) DO
  IF alive == 1 THEN
    currPrice = Underlying(ValuationSchedule[d]);
    prevPrice = Underlying(ValuationSchedule[d-1]);
    realisedVariance = realisedVariance + pow(ln(currPrice/prevPrice), 2);

    IF BarrierType == 3 OR BarrierType == 4 THEN
      daysBeforeKO = daysBeforeKO + 1;
    END;

    IF {BarrierType == 3 AND currPrice <= BarrierLevel} OR
       {BarrierType == 4 AND currPrice >= BarrierLevel} THEN
      alive = 0;
    END;

    IF knockedIn == 0 THEN
      IF {BarrierType == 1 AND currPrice <= BarrierLevel} OR
         {BarrierType == 2 AND currPrice >= BarrierLevel} THEN
        knockedIn = 1;
      END;
    END;
  END;
END;

expectedN = SIZE(ValuationSchedule) - 1;
realisedVariance = (252/expectedN) * realisedVariance;

IF SquaredPayoff == 1 THEN
  realisedVariation = realisedVariance;
  currentNotional = pow(100, 2) * Notional / (2 * 100 * Strike);
  strike = pow(Strike, 2);
ELSE
  realisedVariation = sqrt(realisedVariance);
  currentNotional = 100 * Notional;
  strike = Strike;
END;

IF Floor > 0 THEN
  IF SquaredPayoff == 1 THEN
    floor = pow(Floor, 2);
  ELSE
    floor = Floor;
  END;
  realisedVariation = max(floor * strike, realisedVariation);
END;
IF Cap > 0 THEN
  IF SquaredPayoff == 1 THEN
    cap = pow(Cap, 2);
  ELSE
    cap = Cap;
  END;
  realisedVariation = min(cap * strike, realisedVariation);
END;

payoff = LongShort * knockedIn * (daysBeforeKO / expectedN) *
         currentNotional * (realisedVariation - strike);

Swap = PAY(payoff, ValuationSchedule[SIZE(ValuationSchedule)],
           SettlementDate, PayCcy);
\end{minted}
\caption{Payoff script for a KIKOVarianceSwap.}
\label{lst:kiko_variance_swap}
\end{listing}

The payout formula when no knock-out has occurred is:

\begin{equation*}
  Payout = 100^2 * varianceAmount * \Big[ \min \big( \max (realisedVariance^2, \text{\lstinline!Floor!}), \text{\lstinline!Cap!}\big) - \text{\lstinline!Strike!}^2 \Big].
\end{equation*}

If a knock-out has occurred, the above amount is scaled by the number of days between the first variance
accrual date and the date of knock-out, as a fraction of the total number of variance accrual dates.

The meanings and allowable values for the \lstinline!KIKOVarianceSwapData! node below.

\begin{itemize}
  \item{}[longShort] \lstinline!LongShort!: Own party position in the swap. \emph{Long} corresponds to paying out on the
  fixed/strike variance (volatility) and receiving on the floating/realised variance (volatility). In other words,
  a long position has positive value if the realised variance (volatility) exceeds the variance (volatility)
  strike. \\
  Allowable values: \emph{Long, Short}.
  \item{}[number] \lstinline!Strike!: The volatility strike $K_{vol}$ of the variance swap quoted in absolute terms.
  If the swap was struck in terms of variance, the square root of that variance should be used here. \\
  Allowable values: Any non-negative number, as a percentage expressed in decimal form.
  \item{}[number] \lstinline!Notional!: The vega notional amount. If the swap was struck in terms of a variance notional
  $N_{var}$, the corresponding vega notional is given by $N_{vol} = N_{var} \cdot 2 \cdot 100 \cdot K_{vol}$ (where
  $K_{vol}$ is in absolute terms). \\
  Allowable values: Any non-negative number.
  \item{}[index] \lstinline!Underlying!: Underlying index. \\
  Allowable values: See Section \ref{data_index} for allowable values.
  \item{}[event] \lstinline!ValuationSchedule!: The schedule defining the (daily) observation period for the variance accrual. \\
  Allowable values: \lstinline!See! Section \ref{ss:schedule_data}.
  \item{}[bool] SquaredPayoff: Flag indicating whether the trade is a variance swap (\emph{True}) or a volatility
  swap (\emph{False}). \\
  Allowable values: Boolean node, allowing \emph{Y}, \emph{N}, \emph{1}, \emph{0}, \emph{true}, \emph{false}, etc.
  The full set of allowable values is given in Table \ref{tab:boolean_allowable}.
  \item{}[barrierType] \lstinline!BarrierType!: Whether the barrier is a knock-in (\emph{DownIn} or \emph{UpIn}) or a knock-out
  (\emph{DownOut} or \emph{UpOut}) barrier. For trades with no barrier, which is equivalent to a capped/floored
  variance swap, set \lstinline!BarrierType! to \emph{UpIn} and \lstinline!BarrierLevel! to zero, as in the sample trade
  representation above. \\
  Allowable values: \emph{DownIn, UpIn, DownOut, UpOut}.
  \item{}[number] \lstinline!BarrierLevel!: The agreed knock-in/knock-out barrier price level. \\
  Allowable values: Any non-negative real number.
  \item{}[number] \lstinline!Cap!: The cap on the realised variance (or volatility), as a factor of the \lstinline!Strike!. For example,
  if \lstinline!Cap! is 2.5, then the cap level will be $2.5^2 \times \text{\lstinline!Strike!}^2$ for variance swaps, and
  $2.5 \times \text{\lstinline!Strike!}$ for volatility swaps. For trades with no cap, set \lstinline!Cap! to zero. \\
  Allowable values: Any non-negative number.
  \item{}[number] \lstinline!Floor!: The floor on the realised variance (or volatility), as a factor of the \lstinline!Strike!. For example,
  if \lstinline!Floor! is 0.1, then the floor level will be $0.1^2 \times \text{\lstinline!Strike!}$ for variance swaps, and
  $0.1 \times \text{\lstinline!Strike!}$ for volatility swaps. For trades with no floor, set
  \lstinline!Floor! to zero. \\
  Allowable values: Any non-negative number.
  \item{}[event] \lstinline!SettlementDate!: The date on which the swap payoff is settled. \\
  Allowable values: See \lstinline!Date! in Table \ref{tab:allow_stand_data}.
  \item{}[currency] \lstinline!PayCcy!: The payment currency. For FX, where the underlying is provided
      in the form \lstinline!FX-SOURCE-CCY1-CCY2! (see Table \ref{tab:fxindex_data}) this should
      be \lstinline!CCY2!. If \lstinline!CCY1! or the currency of the underlying (for EQ and
      COMM underlyings), this will result in a quanto payoff. Notice section \ref{sss:payccy_st}. \\
        Allowable values: See Table \ref{tab:currency} for allowable currency codes.
\end{itemize}

\subsubsection*{Corridor Variance Swap}

The traditional trade representation is as follows, using an FX underlying in this example:

\begin{minted}[fontsize=\footnotesize]{xml}
<Trade id="FX_VarianceSwap_Corridor">
  <TradeType>ScriptedTrade</TradeType>
  <Envelope>
    .....
  </Envelope>
  <CorridorVarianceSwapData>
    <LongShort type="longShort">Long</LongShort>
    <Strike type="number">0.02</Strike>
    <Notional type="number">50000</Notional>
    <Underlying type="index">FX-ECB-EUR-USD</Underlying>
    <ValuationSchedule type="event">
      <ScheduleData>
        <Rules>
          <StartDate>2018-12-31</StartDate>
          <EndDate>2019-05-05</EndDate>
          <Tenor>1D</Tenor>
          <Convention>F</Convention>
          <TermConvention>F</TermConvention>
          <Calendar>US</Calendar>
          <Rule>Forward</Rule>
        </Rules>
      </ScheduleData>
    </ValuationSchedule>
    <SquaredPayoff type="bool">true</SquaredPayoff>
    <UpperBarrierLevel type="number">1000000000</UpperBarrierLevel>
    <LowerBarrierLevel type="number">0</LowerBarrierLevel>
    <CountBothObservations type="bool">true</CountBothObservations>
    <AccrualAdjustment type="bool">false</AccrualAdjustment>
    <Cap type="number">2.5</Cap>
    <Floor type="number">0</Floor>
    <SettlementDate type="event">2019-05-06</SettlementDate>
    <PayCcy type="currency">USD</PayCcy>
  </CorridorVarianceSwapData>
</Trade>
\end{minted}

The CorridorVarianceSwap script referenced in the trade above is shown in listing
\ref{lst:corridor_variance_swap}

\begin{listing}[hbt]
\begin{minted}[fontsize=\footnotesize]{Basic}
REQUIRE {Notional >= 0} AND {Strike >= 0};
REQUIRE UpperBarrierLevel >= LowerBarrierLevel;

FOR d IN (2, SIZE(ValuationSchedule), 1) DO
  currPrice = Underlying(ValuationSchedule[d]);
  prevPrice = Underlying(ValuationSchedule[d-1]);

  IF {CountBothObservations == 1 AND 
      currPrice >= LowerBarrierLevel AND currPrice <= UpperBarrierLevel AND
      prevPrice >= LowerBarrierLevel AND prevPrice <= UpperBarrierLevel} OR
     {CountBothObservations == -1 AND
      prevPrice >= LowerBarrierLevel AND prevPrice <= UpperBarrierLevel} THEN
        realisedVariance = realisedVariance + pow(ln(currPrice/prevPrice), 2);
        accruedDays = accruedDays + 1;
  END;
END;

expectedN = SIZE(ValuationSchedule) - 1;
realisedVariance = (252/expectedN) * realisedVariance;

IF AccrualAdjustment == -1 THEN
  accruedDays = expectedN;
END;

IF SquaredPayoff == 1 THEN
  realisedVariation = realisedVariance;
  currentNotional = pow(100, 2) * Notional / (2 * 100 * Strike);
  strike = pow(Strike, 2);
  adjustedStrike = (accruedDays / expectedN) * strike;
ELSE
  realisedVariation = sqrt(realisedVariance);
  currentNotional = 100 * Notional;
  strike = Strike;
  adjustedStrike = sqrt(accruedDays/expectedN) * strike;
END;

IF Floor > 0 THEN
  IF SquaredPayoff == 1 THEN
    floor = pow(Floor, 2);
  ELSE
    floor = Floor;
  END;
  realisedVariation = max(floor * adjustedStrike, realisedVariation);
END;
IF Cap > 0 THEN
  IF SquaredPayoff == 1 THEN
    cap = pow(Cap, 2);
  ELSE
    cap = Cap;
  END;
  realisedVariation = min(cap * adjustedStrike, realisedVariation);
END;

payoff = LongShort * currentNotional * (realisedVariation - adjustedStrike);

Swap = PAY(payoff, ValuationSchedule[SIZE(ValuationSchedule)],
           SettlementDate, PayCcy);
\end{minted}
\caption{Payoff script for a CorridorVarianceSwap.}
\label{lst:corridor_variance_swap}
\end{listing}

The payout formula is:

\begin{equation*}
  Payout = 100^2 * varianceAmount * \Big[ \min \big( \max (realisedVariance^2, \text{\lstinline!Floor!}), \text{\lstinline!Cap!}\big) - \text{\lstinline!Strike!}^2 \Big],
\end{equation*}

where the $realisedVariance$ only consists of variance accrual contributions for when the underlying price/level
was trading within the window defined by the corridor.

The meanings and allowable values for the \lstinline!CorridorVarianceSwapData! node below.

\begin{itemize}
  \item{}[longShort] \lstinline!LongShort!: Own party position in the swap. \emph{Long} corresponds to paying out on the
  fixed/strike variance (volatility) and receiving on the floating/realised variance (volatility). In other words,
  a long position has positive value if the realised variance (volatility) exceeds the variance (volatility)
  strike. \\
  Allowable values: \emph{Long, Short}.
  \item{}[number] \lstinline!Strike!: The volatility strike $K_{vol}$ of the variance swap quoted in absolute terms.
  If the swap was struck in terms of variance, the square root of that variance should be used here. \\
  Allowable values: Any non-negative number, as a percentage expressed in decimal form.
  \item{}[number] \lstinline!Notional!: The vega notional amount. If the swap was struck in terms of a variance notional
  $N_{var}$, the corresponding vega notional is given by $N_{vol} = N_{var} \cdot 2 \cdot 100 \cdot K_{vol}$ (where
  $K_{vol}$ is in absolute terms). \\
  Allowable values: Any non-negative number.
  \item{}[index] \lstinline!Underlying!: Underlying index. \\
  Allowable values: See Section \ref{data_index} for allowable values.
  \item{}[event] \lstinline!ValuationSchedule!: The schedule defining the (daily) observation period for the variance accrual. \\
  Allowable values: See Section \ref{ss:schedule_data}.
  \item{}[bool] \lstinline!SquaredPayoff!: Flag indicating whether the trade is a variance swap (\emph{True}) or a volatility
  swap (\emph{False}). \\
  Allowable values: Boolean node, allowing \emph{Y}, \emph{N}, \emph{1}, \emph{0}, \emph{true}, \emph{false}, etc.
  The full set of allowable values is given in Table \ref{tab:boolean_allowable}.
  \item{}[number] \lstinline!UpperBarrierLevel!: The agreed upper barrier price level. \\
  Allowable values: Any non-negative real number.
  \item{}[number] \lstinline!LowerBarrierLevel!: The agreed lower barrier price level. \\
  Allowable values: Any non-negative real number.
  \item{}[bool] \lstinline!CountBothObservations!: Whether the variance/volatility is accrued based on both the current and
  previous underlying prices falling within the corridor (\emph{True}) or only on the previous underlying price
  (\emph{False}). \\
  Allowable values: Boolean node, allowing \emph{Y}, \emph{N}, \emph{1}, \emph{0}, \emph{true}, \emph{false}, etc.
  The full set of allowable values is given in Table \ref{tab:boolean_allowable}.
  \item{}[bool] \lstinline!AccrualAdjustment!: Whether the strike will be scaled relative to the number of days that the
  underlying traded within the corridor. See $\widetilde{K}_{var}$ (for variance swaps) and $\widetilde{K}_{vol}$
  (for volatility swaps) in section Capped/Floored Corridor Variance Swap of the Product Description for
  Exotic Variance and Volatility Swaps.
  Allowable values: Boolean node, allowing \emph{Y}, \emph{N}, \emph{1}, \emph{0}, \emph{true}, \emph{false}, etc.
  \item{}[number] \lstinline!Cap!: The cap on the realised variance (or volatility), as a factor of the \lstinline!Strike! after
  the accrual adjustment is applied (only when \lstinline!AccrualAdjustment! is \emph{True}). For example,
  if \lstinline!Cap! is 2.5, then the cap level will be $2.5^2 \times \widetilde{K}_{var}$ for variance swaps, and
  $2.5 \times \widetilde{K}_{vol}$ for volatility swaps. For trades with no cap, set \lstinline!Cap! to zero. \\
  Allowable values: Any non-negative number.
  \item{}[number] \lstinline!Floor!: The floor on the realised variance (or volatility), as a factor of the \lstinline!Strike! after
  the accrual adjustment is applied (only when \lstinline!AccrualAdjustment! is \emph{True}). For example,
  if \lstinline!Floor! is 0.1, then the floor level will be $0.1^2 \times \widetilde{K}_{var}$ for variance swaps, and
  $0.1 \times \widetilde{K}_{var}$ for volatility swaps. For trades with no floor, set
  \lstinline!Floor! to zero. \\
  Allowable values: Any non-negative number.
  The full set of allowable values is given in Table \ref{tab:boolean_allowable}.
  \item{}[event] \lstinline!SettlementDate!: The date on which the swap payoff is settled. \\
  Allowable values: See \lstinline!Date! in Table \ref{tab:allow_stand_data}.
  \item{}[currency] \lstinline!PayCcy!: The payment currency. For FX, where the underlying is provided
      in the form \lstinline!FX-SOURCE-CCY1-CCY2! (see Table \ref{tab:fxindex_data}) this should
      be \lstinline!CCY2!. If \lstinline!CCY1! or the currency of the underlying (for EQ and
      COMM underlyings), this will result in a quanto payoff. Notice section \ref{sss:payccy_st}. \\
        Allowable values: See Table \ref{tab:currency}  for allowable currency codes.
\end{itemize}

\subsubsection*{Corridor Variance Swap with KI/KO Barrier}

The traditional trade representation is as follows, using an EQ underlying in this example:

\begin{minted}[fontsize=\footnotesize]{xml}
<Trade id="EQ_VarianceSwap_KIKO_Corridor">
  <TradeType>ScriptedTrade</TradeType>
  <Envelope>
     .....
  </Envelope>
  <KIKOCorridorVarianceSwapData>
    <LongShort type="longShort">Long</LongShort>
    <Strike type="number">0.19</Strike>
    <Notional type="number">1380</Notional>
    <Underlying type="index">EQ-RIC:.SPX</Underlying>
    <ValuationSchedule type="event">
      <ScheduleData>
        <Rules>
          <StartDate>2020-11-26</StartDate>
          <EndDate>2021-09-18</EndDate>
          <Tenor>1D</Tenor>
          <Convention>Following</Convention>
          <TermConvention>Following</TermConvention>
          <Calendar>USA</Calendar>
          <Rule>Forward</Rule>
        </Rules>
      </ScheduleData>
    </ValuationSchedule>
    <CorridorUpperBarrierLevel type="number">3469.18</CorridorUpperBarrierLevel>
    <CorridorLowerBarrierLevel type="number">2207.66</CorridorLowerBarrierLevel>
    <KIKOBarrierType type="barrierType">UpOut</KIKOBarrierType>
    <KIKOBarrierLevel type="number">3469.18</KIKOBarrierLevel>
    <CountBothObservations type="bool">true</CountBothObservations>
    <AccrualAdjustment type="bool">true</AccrualAdjustment>
    <Cap type="number">0</Cap>
    <Floor type="number">0</Floor>
    <SettlementSchedule type="event">
      <DerivedSchedule>
        <BaseSchedule>ValuationSchedule</BaseSchedule>
        <Shift>2D</Shift>
        <Calendar>USA</Calendar>
        <Convention>Following</Convention>
      </DerivedSchedule>
    </SettlementSchedule>
    <PayCcy type="currency">USD</PayCcy>
  </KIKOCorridorVarianceSwapData>
</Trade>
\end{minted}

The KIKOCorridorVarianceSwap script referenced in the trade above is shown in listing
\ref{lst:kiko_variance_swap}

\begin{listing}[hbt]
\begin{minted}[fontsize=\footnotesize]{Basic}
REQUIRE {Notional >= 0} AND {Strike >= 0} AND {KIKOBarrierLevel > 0};
REQUIRE CorridorUpperBarrierLevel >= CorridorLowerBarrierLevel;

n = SIZE(ValuationSchedule);

alive = 1;
FOR d IN (2, n, 1) DO
  currPrice = Underlying(ValuationSchedule[d]);
  prevPrice = Underlying(ValuationSchedule[d-1]);

  IF alive == 1 THEN
    IF {CountBothObservations == 1 AND 
        currPrice >= CorridorLowerBarrierLevel AND currPrice <= CorridorUpperBarrierLevel AND
        prevPrice >= CorridorLowerBarrierLevel AND prevPrice <= CorridorUpperBarrierLevel} OR
       {CountBothObservations == -1 AND
        prevPrice >= CorridorLowerBarrierLevel AND prevPrice <= CorridorUpperBarrierLevel} THEN
          realisedVariance = realisedVariance + pow(ln(currPrice/prevPrice), 2);
          accruedDays = accruedDays + 1;
    END;

    IF {KIKOBarrierType == 3 AND currPrice <= KIKOBarrierLevel} OR
       {KIKOBarrierType == 4 AND currPrice >= KIKOBarrierLevel} THEN
      alive = 0;
    END;
  END;

  IF knockedIn == 0 THEN
    IF {KIKOBarrierType == 1 AND currPrice <= KIKOBarrierLevel} OR
       {KIKOBarrierType == 2 AND currPrice >= KIKOBarrierLevel} THEN
      knockedIn = 1;
    END;
  END;

  IF {alive == 0 OR d == n} AND {calculated == 0} THEN
    calculated = 1;
    expectedN = n - 1;
    realisedVariance = (252/expectedN) * realisedVariance;

    IF AccrualAdjustment == -1 THEN
      accruedDays = expectedN;
    END;

    currentNotional = pow(100, 2) * Notional / (2 * 100 * Strike);
    strike = pow(Strike, 2);
    adjustedStrike = (accruedDays / expectedN) * strike;

    IF Floor > 0 THEN
      floor = pow(Floor, 2);
      realisedVariance = max(floor * adjustedStrike, realisedVariance);
    END;
    IF Cap > 0 THEN
      cap = pow(Cap, 2);
      realisedVariance = min(cap * adjustedStrike, realisedVariance);
    END;

    IF {{KIKOBarrierType == 1 OR KIKOBarrierType == 2} AND {knockedIn == 1}} OR
       {{KIKOBarrierType == 3 OR KIKOBarrierType == 4} AND {alive == 0}} THEN
      TriggerProbability = 1;
    END;

    IF KIKOBarrierType == 3 OR KIKOBarrierType == 4 THEN
      knockedIn = 1;
    END;

    payoff = LongShort * knockedIn * currentNotional
              * (realisedVariance - adjustedStrike);

    Swap = LOGPAY(payoff, ValuationSchedule[d], SettlementSchedule[d], PayCcy);
  END;
END;
\end{minted}
\caption{Payoff script for a KIKOCorridorVarianceSwap.}
\label{lst:kiko_corridor_variance_swap}
\end{listing}

The meanings and allowable values for the \lstinline!KIKOCorridorVarianceSwapData! node below.

\begin{itemize}
  \item{}[longShort] \lstinline!LongShort!: Own party position in the swap. \emph{Long} corresponds to paying out on the
  fixed/strike variance and receiving on the floating/realised variance. In other words, a long position has
  positive value if the realised variance exceeds the variance strike. \\
  Allowable values: \emph{Long, Short}.
  \item{}[number] \lstinline!Strike!: The volatility strike $K_{vol}$ of the variance swap quoted in absolute terms.
  If the swap was struck in terms of variance, the square root of that variance should be used here. \\
  Allowable values: Any non-negative number, as a percentage expressed in decimal form.
  \item{}[number] \lstinline!Notional!: The vega notional amount. If the swap was struck in terms of a variance notional
  $N_{var}$, the corresponding vega notional is given by $N_{vol} = N_{var} \cdot 2 \cdot 100 \cdot K_{vol}$ (where
  $K_{vol}$ is in absolute terms). \\
  Allowable values: Any non-negative number.
  \item{}[index] \lstinline!Underlying!: Underlying index. Currently only EQ, FX and COMM underlyings are supported. \\
  Allowable values: See Section \ref{data_index} for allowable values.
  \item{}[event] \lstinline!ValuationSchedule!: The schedule defining the (daily) observation period for the
  variance accrual. \\
  Allowable values: See Section \ref{ss:schedule_data}.
  \item{}[number] \lstinline!CorridorUpperBarrierLevel!: The agreed upper barrier price level for the corridor. \\
  Allowable values: Any non-negative real number.
  \item{}[number] \lstinline!CorridorLowerBarrierLevel!: The agreed lower barrier price level for the corridor. \\
  Allowable values: Any non-negative real number.
  \item{}[barrierType] \lstinline!KIKOBarrierType!: Whether the barrier is a knock-in (\emph{DownIn}
  or \emph{UpIn}) or a knock-out (\emph{DownOut} or \emph{UpOut}) barrier. For trades with no barrier, which is
  equivalent to a capped/floored corridor variance swap, set \lstinline!KIKOBarrierType! to \emph{UpIn} and
  \lstinline!KIKOBarrierLevel! to zero, as in the sample trade representation above. \\
  Allowable values: \emph{DownIn, UpIn, DownOut, UpOut}.
  \item{}[number] \lstinline!KIKOBarrierLevel!: The agreed knock-in/knock-out barrier price level. \\
  Allowable values: Any non-negative real number.
  \item{}[bool] \lstinline!CountBothObservations!: Whether the variance is accrued based on both the current and
  previous underlying prices falling within the corridor (\emph{True}) or only on the previous underlying price
  (\emph{False}). \\
  Allowable values: Boolean node, allowing \emph{Y}, \emph{N}, \emph{1}, \emph{0}, \emph{true}, \emph{false}, etc.
  The full set of allowable values is given in Table \ref{tab:boolean_allowable}.
  \item{}[bool] \lstinline!AccrualAdjustment!: Whether the strike will be scaled relative to the number of days that the
  underlying traded within the corridor. See $\widetilde{K}_{var}$ in section Capped/Floored Corridor Variance
  Swap of the Product Description for Exotic Variance and Volatility Swaps.
  Allowable values: Boolean node, allowing \emph{Y}, \emph{N}, \emph{1}, \emph{0}, \emph{true}, \emph{false}, etc.
  \item{}[number] \lstinline!Cap!: The cap on the realised variance, as a factor of the \lstinline!Strike!. For example,
  if \lstinline!Cap! is 2.5, then the cap level will be $2.5^2 \times \text{\lstinline!Strike!}^2$. For trades
  with no cap, set \lstinline!Cap! to zero. \\
  Allowable values: Any non-negative number.
  \item{}[number] \lstinline!Floor!: The floor on the realised variance, as a factor of the \lstinline!Strike!. For example,
  if \lstinline!Floor! is 0.1, then the floor level will be $0.1^2 \times \text{\lstinline!Strike!}$. For trades
  with no floor, set \lstinline!Floor! to zero. \\
  Allowable values: Any non-negative number.
  \item{}[event] \lstinline!SettlementSchedule!: The settlement dates derived from the \lstinline!ValuationSchedule! using a
  settlement lag to reflect settlement after a knock-out event, or after the final variance observation date if
  no knock-out occurs. If the barrier is knock-in, this represents the settlement date, which is the final
  valuation date plus the settlement lag. \\
  Allowable values: See section \ref{ss:schedule_data} Schedule Data and Dates, or DerivedSchedule (see \ref{app:scriptedtrade}).
  \item{}[currency] \lstinline!PayCcy!: The payment currency. For FX, where the underlying is provided
      in the form \lstinline!FX-SOURCE-CCY1-CCY2! (see Table \ref{tab:fxindex_data}) this should
      be \lstinline!CCY2!. If \lstinline!CCY1! or the currency of the underlying (for EQ and
      COMM underlyings), this will result in a quanto payoff. Notice section \ref{sss:payccy_st}. \\
        Allowable values: See Table \ref{tab:currency}  for allowable currency codes.
\end{itemize}

\subsubsection*{Conditional Variance Swap 01}

The traditional trade representation is as follows, using an FX underlying in this example:

\begin{minted}[fontsize=\footnotesize]{xml}
<Trade id="FX_VarianceSwap_Conditional_01">
  <TradeType>ScriptedTrade</TradeType>
  <Envelope>
    .....
  </Envelope>
  <ConditionalVarianceSwap01Data>
    <LongShort type="longShort">Long</LongShort>
    <Strike type="number">0.02</Strike>
    <Notional type="number">50000</Notional>
    <Underlying type="index">FX-ECB-EUR-USD</Underlying>
    <ValuationSchedule type="event">
      <ScheduleData>
        <Rules>
          <StartDate>2018-12-31</StartDate>
          <EndDate>2019-05-05</EndDate>
          <Tenor>1D</Tenor>
          <Convention>F</Convention>
          <TermConvention>F</TermConvention>
          <Calendar>US</Calendar>
          <Rule>Forward</Rule>
        </Rules>
      </ScheduleData>
    </ValuationSchedule>
    <SquaredPayoff type="bool">true</SquaredPayoff>
    <BarrierType type="barrierType">UpIn</BarrierType>
    <BarrierLevel type="number">0</BarrierLevel>
    <CountBothObservations type="bool">true</CountBothObservations>
    <Cap type="number">2.5</Cap>
    <Floor type="number">0</Floor>
    <AccrualAdjustment type="bool">false</AccrualAdjustment>
    <SettlementDate type="event">2019-05-06</SettlementDate>
    <PayCcy type="currency">USD</PayCcy>
  </ConditionalVarianceSwap01Data>
</Trade>
\end{minted}

The ConditionalVarianceSwap01 script referenced in the trade above is shown in listing
\ref{lst:conditional_variance_swap_01}

\begin{listing}[hbt]
\begin{minted}[fontsize=\footnotesize]{Basic}
REQUIRE {Notional >= 0} AND {Strike > 0};

FOR d IN (2, SIZE(ValuationSchedule), 1) DO
  currPrice = Underlying(ValuationSchedule[d]);
  prevPrice = Underlying(ValuationSchedule[d-1]);

  IF {CountBothObservations == 1 AND {
      {{BarrierType == 1 OR BarrierType == 4} AND
        currPrice <= BarrierLevel AND prevPrice <= BarrierLevel} OR
      {{BarrierType == 2 OR BarrierType == 3} AND
        currPrice >= BarrierLevel AND prevPrice >= BarrierLevel}}}
  OR {CountBothObservations == -1 AND {
      {{BarrierType == 1 OR BarrierType == 4} AND
        prevPrice <= BarrierLevel} OR 
      {{BarrierType == 2 OR BarrierType == 3} AND
        prevPrice <= BarrierLevel} }}
  THEN
    realisedVariance = realisedVariance + pow(ln(currPrice/prevPrice), 2);
    accruedDays = accruedDays + 1;
  END;
END;

expectedN = SIZE(ValuationSchedule) - 1;
realisedVariance = (252/expectedN) * realisedVariance;

IF AccrualAdjustment == -1 THEN
  accruedDays = expectedN;
END;

IF SquaredPayoff == 1 THEN
  realisedVariation = realisedVariance;
  currentNotional = pow(100, 2) * Notional / (2 * 100 * Strike);
  strike = pow(Strike, 2);
  adjustedStrike = (accruedDays / expectedN) * strike;
ELSE
  realisedVariation = sqrt(realisedVariance);
  currentNotional = 100 * Notional;
  strike = Strike;
  adjustedStrike = sqrt(accruedDays/expectedN) * strike;
END;

IF Floor > 0 THEN
  IF SquaredPayoff == 1 THEN
    floor = pow(Floor, 2);
  ELSE
    floor = Floor;
  END;
  realisedVariation = max(floor * adjustedStrike, realisedVariation);
END;
IF Cap > 0 THEN
  IF SquaredPayoff == 1 THEN
    cap = pow(Cap, 2);
  ELSE
    cap = Cap;
  END;
  realisedVariation = min(cap * adjustedStrike, realisedVariation);
END;

payoff = LongShort * currentNotional * (realisedVariation - adjustedStrike);

Swap = PAY(payoff, ValuationSchedule[SIZE(ValuationSchedule)],
           SettlementDate, PayCcy);
\end{minted}
\caption{Payoff script for a ConditionalVarianceSwap01.}
\label{lst:conditional_variance_swap_01}
\end{listing}

The payout formula is:
\begin{equation*}
  Payout = 100^2 * varianceAmount * \Big[ \min \big( \max (realisedVariance^2, \text{\lstinline!Floor!}), \text{\lstinline!Cap!}\big) - \text{\lstinline!Strike!}^2 \Big],
\end{equation*}
where the $realisedVariance$ only consists of variance accrual contributions for when the underlying price/level
was trading within the pre-defined window.

The meanings and allowable values for the \lstinline!ConditionalVarianceSwapData01! node below.

\begin{itemize}
  \item{}[longShort] \lstinline!LongShort!: Own party position in the swap. \emph{Long} corresponds to paying out on the
  fixed/strike variance (volatility) and receiving on the floating/realised variance (volatility). In other words,
  a long position has positive value if the realised variance (volatility) exceeds the variance (volatility)
  strike. \\
  Allowable values: \emph{Long, Short}.
  \item{}[number] \lstinline!Strike!: The volatility strike $K_{vol}$ of the variance swap quoted in absolute terms.
  If the swap was struck in terms of variance, the square root of that variance should be used here. \\
  Allowable values: Any non-negative number, as a percentage expressed in decimal form.
  \item{}[number] \lstinline!Notional!: The vega notional amount. If the swap was struck in terms of a variance notional
  $N_{var}$, the corresponding vega notional is given by $N_{vol} = N_{var} \cdot 2 \cdot 100 \cdot K_{vol}$ (where
  $K_{vol}$ is in absolute terms). \\
  Allowable values: Any non-negative number.
  \item{}[index] \lstinline!Underlying!: Underlying index. \\
  Allowable values: See Section \ref{data_index} for allowable values.
  \item{}[event] \lstinline!ValuationSchedule!: The schedule defining the (daily) observation period for the variance accrual. \\
  Allowable values: See Section \ref{ss:schedule_data}.
  \item{}[bool] \lstinline!SquaredPayoff!: Flag indicating whether the trade is a variance swap (\emph{True}) or a volatility
  swap (\emph{False}). \\
  Allowable values: Boolean node, allowing \emph{Y}, \emph{N}, \emph{1}, \emph{0}, \emph{true}, \emph{false}, etc.
  The full set of allowable values is given in Table \ref{tab:boolean_allowable}.
  \item{}[barrierType] \lstinline!BarrierType!: Whether the instrument is an up-variance swap (\emph{UpIn} or \emph{DownOut}) or 
  a down-variance swap (\emph{DownIn} or \emph{UpOut}). \\
  Allowable values: \emph{DownIn, UpIn, DownOut, UpOut}.
  \item{}[number] \lstinline!BarrierLevel!: The agreed barrier price level. \\
  Allowable values: Any non-negative real number.
  \item{}[bool] \lstinline!CountBothObservations!: Whether the variance/volatility is accrued based on both the current and
  previous underlying prices falling within the range (\emph{True}) or only on the previous underlying price
  (\emph{False}). \\
  Allowable values: Boolean node, allowing \emph{Y}, \emph{N}, \emph{1}, \emph{0}, \emph{true}, \emph{false}, etc.
  The full set of allowable values is given in Table \ref{tab:boolean_allowable}.
  \item{}[bool] \lstinline!AccrualAdjustment!: Whether the strike will be scaled relative to the number of days that the
  underlying traded within the range. See $\widetilde{K}_{var}$ (for variance swaps) and $\widetilde{K}_{vol}$
  (for volatility swaps) in section Capped/Floored Conditional Variance Swap of the Product Description for
  Exotic Variance and Volatility Swaps.
  Allowable values: Boolean node, allowing \emph{Y}, \emph{N}, \emph{1}, \emph{0}, \emph{true}, \emph{false}, etc.
  \item{}[number] \lstinline!Cap!: The cap on the realised variance (or volatility), as a factor of the \lstinline!Strike! after
  the accrual adjustment is applied (only when \lstinline!AccrualAdjustment! is \emph{True}). For example,
  if \lstinline!Cap! is 2.5, then the cap level will be $2.5^2 \times \widetilde{K}_{var}$ for variance swaps, and
  $2.5 \times \widetilde{K}_{vol}$ for volatility swaps. For trades with no cap, set \lstinline!Cap! to zero. \\
  Allowable values: Any non-negative number.
  \item{}[number] \lstinline!Floor!: The floor on the realised variance (or volatility), as a factor of the \lstinline!Strike! after
  the accrual adjustment is applied (only when \lstinline!AccrualAdjustment! is \emph{True}). For example,
  if \lstinline!Floor! is 0.1, then the floor level will be $0.1^2 \times \widetilde{K}_{var}$ for variance swaps, and
  $0.1 \times \widetilde{K}_{var}$ for volatility swaps. For trades with no floor, set
  \lstinline!Floor! to zero. \\
  Allowable values: Any non-negative number.
  The full set of allowable values is given in Table \ref{tab:boolean_allowable}.
  \item{}[event] \lstinline!SettlementDate!: The date on which the swap payoff is settled. \\
  Allowable values: See \lstinline!Date! in Table \ref{tab:allow_stand_data}.
  \item{}[currency] \lstinline!PayCcy!: The payment currency. For FX, where the underlying is provided
      in the form \lstinline!FX-SOURCE-CCY1-CCY2! (see Table \ref{tab:fxindex_data}) this should
      be \lstinline!CCY2!. If \lstinline!CCY1! or the currency of the underlying (for EQ and
      COMM underlyings), this will result in a quanto payoff. Notice section \ref{sss:payccy_st}. \\
        Allowable values: See Table \ref{tab:currency} for allowable currency codes.
\end{itemize}

\subsubsection*{Conditional Variance Swap 02}

The traditional trade representation is as follows, using an FX underlying in this example:

\begin{minted}[fontsize=\footnotesize]{xml}
<Trade id="FX_VarianceSwap_Conditional_02">
  <TradeType>ScriptedTrade</TradeType>
  <Envelope>
    .....
  </Envelope>
  <ConditionalVarianceSwap02Data>
    <LongShort type="longShort">Long</LongShort>
    <Strike type="number">0.02</Strike>
    <Notional type="number">50000</Notional>
    <VarianceReference type="number">0.03</VarianceReference>
    <Underlying type="index">FX-ECB-EUR-USD</Underlying>
    <ValuationSchedule type="event">
      <ScheduleData>
        <Rules>
          <StartDate>2018-12-31</StartDate>
          <EndDate>2019-05-05</EndDate>
          <Tenor>1D</Tenor>
          <Convention>F</Convention>
          <TermConvention>F</TermConvention>
          <Calendar>US</Calendar>
          <Rule>Forward</Rule>
        </Rules>
      </ScheduleData>
    </ValuationSchedule>
    <SquaredPayoff type="bool">true</SquaredPayoff>
    <BarrierType type="barrierType">UpIn</BarrierType>
    <BarrierLevel type="number">0</BarrierLevel>
    <CountBothObservations type="bool">true</CountBothObservations>
    <Cap type="number">2.5</Cap>
    <Floor type="number">0</Floor>
    <AccrualAdjustment type="bool">false</AccrualAdjustment>
    <SettlementDate type="event">2019-05-06</SettlementDate>
    <PayCcy type="currency">USD</PayCcy>
  </ConditionalVarianceSwap02Data>
</Trade>
\end{minted}

The ConditionalVarianceSwap02 script referenced in the trade above is shown in listing
\ref{lst:conditional_variance_swap_02}

\begin{listing}[hbt]
\begin{minted}[fontsize=\footnotesize]{Basic}
REQUIRE {Notional >= 0} AND {Strike > 0} AND {VarianceReference > 0};

FOR d IN (2, SIZE(ValuationSchedule), 1) DO
  currPrice = Underlying(ValuationSchedule[d]);
  prevPrice = Underlying(ValuationSchedule[d-1]);

  IF {CountBothObservations == 1 AND {
      {{BarrierType == 1 OR BarrierType == 4} AND
        currPrice <= BarrierLevel AND prevPrice <= BarrierLevel} OR
      {{BarrierType == 2 OR BarrierType == 3} AND
        currPrice >= BarrierLevel AND prevPrice >= BarrierLevel}}}
  OR {CountBothObservations == -1 AND {
      {{BarrierType == 1 OR BarrierType == 4} AND
        prevPrice <= BarrierLevel} OR 
      {{BarrierType == 2 OR BarrierType == 3} AND
        prevPrice <= BarrierLevel} }}
  THEN
    realisedVariance = realisedVariance + pow(ln(currPrice/prevPrice), 2);
    accruedDays = accruedDays + 1;
  END;
END;

expectedN = SIZE(ValuationSchedule) - 1;
realisedVariance = (252/expectedN) * realisedVariance;

IF AccrualAdjustment == -1 THEN
  accruedDays = expectedN;
END;

IF SquaredPayoff == 1 THEN
  realisedVariation = realisedVariance;
  currentNotional = pow(100, 2) * Notional / (2 * 100 * VarianceReference);
  strike = pow(Strike, 2);
  adjustedStrike = (accruedDays / expectedN) * strike;
ELSE
  realisedVariation = sqrt(realisedVariance);
  currentNotional = 100 * Notional;
  strike = Strike;
  adjustedStrike = sqrt(accruedDays/expectedN) * strike;
END;

IF Floor > 0 THEN
  IF SquaredPayoff == 1 THEN
    floor = pow(Floor, 2);
  ELSE
    floor = Floor;
  END;
  realisedVariation = max(floor * adjustedStrike, realisedVariation);
END;
IF Cap > 0 THEN
  IF SquaredPayoff == 1 THEN
    cap = pow(Cap, 2);
  ELSE
    cap = Cap;
  END;
  realisedVariation = min(cap * adjustedStrike, realisedVariation);
END;

payoff = LongShort * currentNotional * (realisedVariation - adjustedStrike);

Swap = PAY(payoff, ValuationSchedule[SIZE(ValuationSchedule)],
           SettlementDate, PayCcy);
\end{minted}
\caption{Payoff script for a ConditionalVarianceSwap02.}
\label{lst:conditional_variance_swap_02}
\end{listing}

The payout formula is:
\begin{equation*}
  Payout = 100^2 * varianceAmount * \Big[ \min \big( \max (realisedVariance^2, \text{\lstinline!Floor!}), \text{\lstinline!Cap!}\big) - \text{\lstinline!Strike!}^2 \Big],
\end{equation*}
where the $realisedVariance$ only consists of variance accrual contributions for when the underlying price/level
was trading within the pre-defined window, and the ``variance strike'' used to obtain the varianceAmount is 
given by the variance reference strike (instead of the actual strike).

The meanings and allowable values for the \lstinline!ConditionalVarianceSwapData02! node below.

\begin{itemize}
  \item{}[longShort] \lstinline!LongShort!: Own party position in the swap. \emph{Long} corresponds to paying out on the
  fixed/strike variance (volatility) and receiving on the floating/realised variance (volatility). In other words,
  a long position has positive value if the realised variance (volatility) exceeds the variance (volatility)
  strike. \\
  Allowable values: \emph{Long, Short}.
  \item{}[number] \lstinline!Strike!: The volatility strike $K_{vol}$ of the variance swap quoted in absolute terms.
  If the swap was struck in terms of variance, the square root of that variance should be used here. \\
  Allowable values: Any non-negative number, as a percentage expressed in decimal form.
  \item{}[number] \lstinline!Notional!: The vega notional amount. If the swap was struck in terms of a variance notional
  $N_{var}$, the corresponding vega notional is given by $N_{vol} = N_{var} \cdot 2 \cdot 100 \cdot K_{vol}$ (where
  $K_{vol}$ is in absolute terms). \\
  Allowable values: Any non-negative number.
  \item{}[number] \lstinline!VarianceReference!: The (volatility) strike value used (in place of the actual strike) to scale down the
  vega notional in order to obtain the variance amount. If the swap was struck in terms of variance, the square root of that
  variance should be used here. \\
  Allowable values: Any non-negative number.
  \item{}[index] \lstinline!Underlying!: Underlying index. \\
  Allowable values: See Section \ref{data_index} for allowable values.
  \item{}[event] \lstinline!ValuationSchedule!: The schedule defining the (daily) observation period for the variance accrual. \\
  Allowable values: See Section \ref{ss:schedule_data}.
  \item{}[bool] \lstinline!SquaredPayoff!: Flag indicating whether the trade is a variance swap (\emph{True}) or a volatility
  swap (\emph{False}). \\
  Allowable values: Boolean node, allowing \emph{Y}, \emph{N}, \emph{1}, \emph{0}, \emph{true}, \emph{false}, etc.
  The full set of allowable values is given in Table \ref{tab:boolean_allowable}.
  \item{}[barrierType] \lstinline!BarrierType!: Whether the instrument is an up-variance swap (\emph{UpIn} or \emph{DownOut}) or 
  a down-variance swap (\emph{DownIn} or \emph{UpOut}). \\
  Allowable values: \emph{DownIn, UpIn, DownOut, UpOut}.
  \item{}[number] \lstinline!BarrierLevel!: The agreed barrier price level. \\
  Allowable values: Any non-negative real number.
  \item{}[bool] \lstinline!CountBothObservations!: Whether the variance/volatility is accrued based on both the current and
  previous underlying prices falling within the range (\emph{True}) or only on the previous underlying price
  (\emph{False}). \\
  Allowable values: Boolean node, allowing \emph{Y}, \emph{N}, \emph{1}, \emph{0}, \emph{true}, \emph{false}, etc.
  The full set of allowable values is given in Table \ref{tab:boolean_allowable}.
  \item{}[bool] \lstinline!AccrualAdjustment!: Whether the strike will be scaled relative to the number of days that the
  underlying traded within the range. See $\widetilde{K}_{var}$ (for variance swaps) and $\widetilde{K}_{vol}$
  (for volatility swaps) in section Capped/Floored Conditional Variance Swap of the Product Description for
  Exotic Variance and Volatility Swaps.
  Allowable values: Boolean node, allowing \emph{Y}, \emph{N}, \emph{1}, \emph{0}, \emph{true}, \emph{false}, etc.
  \item{}[number] \lstinline!Cap!: The cap on the realised variance (or volatility), as a factor of the \lstinline!Strike! after
  the accrual adjustment is applied (only when \lstinline!AccrualAdjustment! is \emph{True}). For example,
  if \lstinline!Cap! is 2.5, then the cap level will be $2.5^2 \times \widetilde{K}_{var}$ for variance swaps, and
  $2.5 \times \widetilde{K}_{vol}$ for volatility swaps. For trades with no cap, set \lstinline!Cap! to zero. \\
  Allowable values: Any non-negative number.
  \item{}[number] \lstinline!Floor!: The floor on the realised variance (or volatility), as a factor of the \lstinline!Strike! after
  the accrual adjustment is applied (only when \lstinline!AccrualAdjustment! is \emph{True}). For example,
  if \lstinline!Floor! is 0.1, then the floor level will be $0.1^2 \times \widetilde{K}_{var}$ for variance swaps, and
  $0.1 \times \widetilde{K}_{var}$ for volatility swaps. For trades with no floor, set
  \lstinline!Floor! to zero. \\
  Allowable values: Any non-negative number.
  The full set of allowable values is given in Table \ref{tab:boolean_allowable}.
  \item{}[event] \lstinline!SettlementDate!: The date on which the swap payoff is settled. \\
  Allowable values: See \lstinline!Date! in Table \ref{tab:allow_stand_data}.
  \item{}[currency] \lstinline!PayCcy!: The payment currency. For FX, where the underlying is provided
      in the form \lstinline!FX-SOURCE-CCY1-CCY2! (see Table \ref{tab:fxindex_data}) this should
      be \lstinline!CCY2!. If \lstinline!CCY1! or the currency of the underlying (for EQ and
      COMM underlyings), this will result in a quanto payoff. Notice section \ref{sss:payccy_st}. \\
        Allowable values: See Table \ref{tab:currency} for allowable currency codes.
\end{itemize}

\subsubsection*{Pairwise Variance Swap}

The \lstinline!FxPairwiseVarianceSwap! and \lstinline!EquityPairwiseVarianceSwap! trade types
have trade data containers (respectively):
\begin{itemize}
  \item \lstinline!FxPairwiseVarianceSwapData!
  \item \lstinline!EquityPairwiseVarianceSwapData!
\end{itemize}
Listing \ref{lst:eqpairwisevarianceswap_data} shows the structure of example trades for an Equity underlying.

\begin{listing}[H]
\begin{minted}[fontsize=\footnotesize]{xml}
<Trade id="EQ_VarianceSwap_Pairwise">
  <TradeType>EquityPairwiseVarianceSwap</TradeType>
  <Envelope>
    .......
  </Envelope>
  <EquityPairwiseVarianceSwapData>
    <LongShort>Long</LongShort>
    <Underlyings>
      <Value>EQ-RIC:.STOXX50E</Value>
      <Value>EQ-RIC:.SPX</Value>
    </Underlyings>
    <UnderlyingStrikes>
      <Value>0.33859119894055129</Value>
      <Value>0.39039467209479178</Value>
    </UnderlyingStrikes>
    <UnderlyingNotionals>
      <Value>577.75847822419905</Value>
      <Value>603.54235516510619</Value>
    </UnderlyingNotionals>
    <BasketNotional>1339.2898822637151</BasketNotional>
    <BasketStrike>0.31612972020991642</BasketStrike>
    <ValuationSchedule>
      <Rules>
      .......
      </Rules>
    </ValuationSchedule>
    <LaggedValuationSchedule>
      <Derived>
       .......
      </Derived>
    </LaggedValuationSchedule>
    <AccrualLag>3</AccrualLag>
    <PayoffLimit>5</PayoffLimit>
    <Cap>2.5</Cap>
    <Floor>0</Floor>
    <SettlementDate>2022-06-29</SettlementDate>
    <PayCcy>USD</PayCcy>
  </EquityPairwiseVarianceSwapData>
</Trade>
\end{minted}
\caption{EquityPairwiseVarianceSwap data}
\label{lst:eqpairwisevarianceswap_data}
\end{listing}

The payout formula is:

\begin{equation*}
  Payout = \min \Big( \max(equityAmount1 + equityAmount2 + equityAmountBasket, LowerLimit), UpperLimit \Big)
\end{equation*}

where

\begin{align*}
  UpperLimit &= &&\text{\lstinline!PayoffLimit!} * \big( |notional1| + |notional2| \big) \\
  LowerLimit &= - &&\text{\lstinline!PayoffLimit!} * \big( |notional1| + |notional2| \big)
\end{align*}

The meanings and allowable values of elements in the \lstinline!EquityPairwiseVarianceSwapData! node follow below.

\begin{itemize}
  \item \lstinline!LongShort!: Own party position. \\
    Allowable values: \emph{Long}, \emph{Short}
  \item \lstinline!Underlyings!: The basket of underlyings. \\
    Allowable values: The format follows that of a scripted trade underlying. See \ref{app:scriptedtrade}
    (section Data Node / Index) for allowable values.
  \item \lstinline!UnderlyingStrikes!: The volatility strikes $K^{1,2}_{vol}$ of the underlyings quoted in absolute terms.
  If the swap was struck in terms of variance, the square roots of the variances should be used here. \\
    Allowable values: Any non-negative number, as a percentage expressed in decimal form.
  \item \lstinline!UnderlyingNotionals!: The vega notional amount. If the swap was struck in terms of variance notionals
  $N_{var}$, the corresponding vega notionals are given by $N^{1,2}_{vol} = N^{1,2}_{var} \cdot 2 \cdot 100 \cdot K^{1,2}_{vol}$
  (where $K^{1,2}_{vol}$ is in absolute terms). \\
    Allowable values: Any non-negative number. 
  \item{} \lstinline!BasketNotional!: The basket vega notional amount. \\
    Allowable values: Any non-negative number.
  \item{} \lstinline!ValuationSchedule!: The start dates of the variance accrual schedule. This node can
  also be used to derive the variance accrual schedule (or vice versa) using a DerivedSchedule (e.g.\ with a two-day lag, \emph{-2D}).
  For the standard accrual period lengths of one day, set \lstinline!Shift! to \emph{-1D}. \\
    Allowable values: See Section \ref{ss:schedule_data}. If this uses a DerivedSchedule, the \lstinline!Shift! must be less
  than or equal to \emph{-1D}.
  \item{} \lstinline!LaggedValuationSchedule!: The end dates of the variance accrual schedule. This node can be derived from
  the \lstinline!ValuationSchedule! (or vice versa) using a DerivedSchedule (e.g.\ with a two-day lag, \emph{2D}). For the standard
  accrual period lengths of one day, set \lstinline!Shift! to \emph{1D}. \\
    Allowable values: See section \ref{ss:schedule_data}. If this uses a DerivedSchedule, the \lstinline!Shift! must be greater
  than or equal to \emph{1D}. If left blank or omitted, defaults to a derived schedule, with \lstinline!Shift! equal to \emph{1D}.
  \item{} \lstinline!AccrualLag!: The length (in days) of each variance accrual period. For classic variance swaps,
  set this to \emph{1}. \\
    Allowable values: Any integer greater than or equal to \emph{1}. If left blank or omitted, defaults to \emph{1}.
  \item{} \lstinline!PayoffLimit!: The factor used to determine the maximum/minimum payoff under the swap. This corresponds to $C$
  in the final equation in the Product Description for Pairwise Variance Swap. \\
    Allowable values: Any non-negative number. If left blank or omitted, defaults to zero, i.e.\ no payoff limit.
  \item{} \lstinline!Cap!: The cap on the realised variances, as a factor of the corresponding strikes. For example,
  if \lstinline!Cap! is 2.5, then the cap level will be $2.5^2 \times \hat{K}_{var}$ for the basket realised variance.
  For trades with no cap, set \lstinline!Cap! to zero. \\
    Allowable values: Any non-negative number.
  \item{} \lstinline!Floor!:  The floor on the realised variances, as a factor of the corresponding strikes. For example,
  if \lstinline!Floor! is 0.1, then the floor level will be $0.1^2 \times \hat{K}_{var}$ for the basket realised variance.
  For trades with no floor, set \lstinline!Floor! to zero. \\
    Allowable values: Any non-negative number.
  \item{} \lstinline!SettlementDate!: The date on which the swap payoff is settled. \\
    Allowable values: See \lstinline!Date! in Table \ref{tab:allow_stand_data}.
  \item{} \lstinline!PayCcy!: The settlement currency. \\
    Allowable values: See Table \ref{tab:currency} \lstinline!Currency!.
\end{itemize}

Pairwise variance swaps can alternatively be represented as {\em scripted trades}, refer to Section \ref{app:scriptedtrade} for an introduction.

\begin{minted}[fontsize=\footnotesize]{xml}
<Trade id="EQ_VarianceSwap_Pairwise">
  <TradeType>ScriptedTrade</TradeType>
  <Envelope>
    .....
  </Envelope>
  <PairwiseVarianceSwapData>
    <LongShort type="longShort">Long</LongShort>
    <Underlyings type="index">
      <Value>EQ-RIC:.STOXX50E</Value>
      <Value>EQ-RIC:.SPX</Value>
    </Underlyings>
    <UnderlyingStrikes type="number">
      <Value>0.3385911989405513</Value>
      <Value>0.3903946720947918</Value>
    </UnderlyingStrikes>
    <UnderlyingNotionals type="number">
      <Value>577.7584782241991</Value>
      <Value>603.5423551651062</Value>
    </UnderlyingNotionals>
    <BasketNotional type="number">1339.289882263715</BasketNotional>
    <BasketStrike type="number">0.3161297202099164</BasketStrike>
    <ValuationSchedule type="event">
      <ScheduleData>
        <Rules>
          <StartDate>2021-06-25</StartDate>
          <EndDate>2022-06-25</EndDate>
          <Tenor>3D</Tenor>
          <Convention>F</Convention>
          <TermConvention>F</TermConvention>
          <Calendar>US</Calendar>
          <Rule>Forward</Rule>
        </Rules>
      </ScheduleData>
    </ValuationSchedule>
    <LaggedValuationSchedule type="event">
      <DerivedSchedule>
        <BaseSchedule>ValuationSchedule</BaseSchedule>
        <Shift>2D</Shift>
        <Calendar>US</Calendar>
        <Convention>F</Convention>
      </DerivedSchedule>
    </LaggedValuationSchedule>
    <AccrualLag type="number">3</AccrualLag>
    <PayoffLimit type="number">5</PayoffLimit>
    <Cap type="number">2.5</Cap>
    <Floor type="number">0</Floor>
    <SettlementDate type="event">2022-06-29</SettlementDate>
    <PayCcy type="currency">USD</PayCcy>
  </PairwiseVarianceSwapData>
</Trade>
\end{minted}

The PairwiseVarianceSwap script referenced in the trade above is shown in listing
\ref{lst:pairwise_variance_swap}

\begin{listing}[hbt]
\begin{minted}[fontsize=\footnotesize]{Basic}
REQUIRE {SIZE(Underlyings) == 2} AND {SIZE(UnderlyingStrikes) == 2}
REQUIRE {SIZE(UnderlyingNotionals) == 2} AND {UnderlyingStrikes[1] >= 0};
REQUIRE {UnderlyingStrikes[2] >= 0} AND {BasketStrike >= 0};
REQUIRE {UnderlyingNotionals[1] >= 0} AND {UnderlyingNotionals[2] >= 0};
REQUIRE {BasketNotional >= 0} AND {PayoffLimit > 0};
REQUIRE {SIZE(ValuationSchedule) == SIZE(LaggedValuationSchedule)};

FOR d IN (1, SIZE(ValuationSchedule)-1, 1) DO
  performance1 = ln(Underlyings[1](LaggedValuationSchedule[d]) /
                    Underlyings[1](ValuationSchedule[d]));
  performance2 = ln(Underlyings[2](LaggedValuationSchedule[d]) /
                    Underlyings[2](ValuationSchedule[d]));
  basketPerformance = (performance1 + performance2) / 2;
  
  realisedVariance1 = realisedVariance1 + pow(performance1, 2);
  realisedVariance2 = realisedVariance2 + pow(performance2, 2);
  realisedVarianceBasket = realisedVarianceBasket + pow(basketPerformance, 2);
END;

expectedN = SIZE(ValuationSchedule) - 1;
realisedVariance1 = 252 / (expectedN * AccrualLag) * realisedVariance1;
realisedVariance2 = 252 / (expectedN * AccrualLag) * realisedVariance2;
realisedVarianceBasket = 252 / (expectedN * AccrualLag) * realisedVarianceBasket;

currentNotional1 = pow(100, 2) * UnderlyingNotionals[1] /
                                  (2 * 100 * UnderlyingStrikes[1]);
currentNotional2 = pow(100, 2) * UnderlyingNotionals[2] /
                                  (2 * 100 * UnderlyingStrikes[2]);
currentNotionalBasket = pow(100, 2) * BasketNotional / (2 * 100 * BasketStrike);
strike1 = pow(UnderlyingStrikes[1], 2);
strike2 = pow(UnderlyingStrikes[2], 2);
strikeBasket = pow(BasketStrike, 2);

IF Floor > 0 THEN
  floor = pow(Floor, 2);
  realisedVariance1 = max(floor * strike1, realisedVariance1);
  realisedVariance2 = max(floor * strike2, realisedVariance2);
  realisedVarianceBasket = max(floor * strikeBasket, realisedVarianceBasket);
END;
IF Cap > 0 THEN
  cap = pow(Cap, 2);
  realisedVariance1 = min(cap * strike1, realisedVariance1);
  realisedVariance2 = min(cap * strike2, realisedVariance2);
  realisedVarianceBasket = min(cap * strikeBasket, realisedVarianceBasket);
END;

equityAmount1 = currentNotional1 * (realisedVariance1 - strike1);
equityAmount2 = currentNotional2 * (realisedVariance2 - strike2);
equityAmountBasket = currentNotionalBasket * (realisedVarianceBasket - strikeBasket);
pairEquityAmount = equityAmount1 + equityAmount2 + equityAmountBasket;

maxPairEquityAmount = PayoffLimit * (abs(UnderlyingNotionals[1]) +
                                     abs(UnderlyingNotionals[2]));
minPairEquityAmount = -maxPairEquityAmount;

pairEquityAmount = max(minPairEquityAmount, pairEquityAmount);
pairEquityAmount = min(maxPairEquityAmount, pairEquityAmount);

Swap = PAY(LongShort * pairEquityAmount, ValuationSchedule[SIZE(ValuationSchedule)],
           SettlementDate, PayCcy);
\end{minted}
\caption{Payoff script for a PairwiseVarianceSwap.}
\label{lst:pairwise_variance_swap}
\end{listing}

The meanings and allowable values for the \lstinline!PairwiseVarianceSwapData! node below.

\begin{itemize}
  \item{}[longShort] \lstinline!LongShort!: Own party position in the swap. \emph{Long} corresponds to paying out on the
  fixed/strike variance (volatility) and receiving on the floating/realised variance (volatility). In other words,
  a long position has positive value if the realised variance (volatility) exceeds the variance (volatility)
  strike. \\
  Allowable values: \emph{Long, Short}.
  \item{}[number] \lstinline!UnderlyingStrikes!: The volatility strikes $K^{1,2}_{vol}$ of the underlyings quoted in
  absolute terms. If the swap was struck in terms of variance, the square roots of the variances should be used here. \\
  Allowable values: Any non-negative number, as a percentage expressed in decimal form.
  \item{}[number] \lstinline!UnderlyingNotionals!: The vega notional amount. If the swap was struck in terms of variance notionals
  $N_{var}$, the corresponding vega notionals are given by $N^{1,2}_{vol} = N^{1,2}_{var} \cdot 2 \cdot 100 \cdot K^{1,2}_{vol}$
  (where $K^{1,2}_{vol}$ is in absolute terms). \\
  Allowable values: Any non-negative number.
  \item{}[index] \lstinline!Underlyings!: Underlying index. \\
  Allowable values: See Section \ref{data_index} for allowable values. 
  \item{}[number] \lstinline!BasketStrike!: The basket volatility strike $\hat{K}_{vol}$ quoted in absolute terms. \\
  Allowable values: Any non-negative number, as a percentage expressed in decimal form.
  \item{}[number] \lstinline!BasketNotional!: The basket vega notional amount. \\
  Allowable values: Any non-negative number.
  \item{}[event] \lstinline!ValuationSchedule!: The base schedule which defines the start dates of the variance accrual schedule. This node is
  also used to derive the variance accrual schedule. \\
  Allowable values: See Section \ref{ss:schedule_data}.
  \item{}[event] \lstinline!LaggedValuationSchedule!: The end dates of the variance accrual schedule. This can be derived from the \lstinline!Schedule!
  using a DerivedSchedule (e.g.\ with a two-day lag, \emph{2D}). For the standard accrual period lengths of one day,
  set \lstinline!Shift! to \emph{1D}. \\
  Allowable values: See section \ref{ss:schedule_data} Schedule Data and Dates, or DerivedSchedule (see \ref{app:scriptedtrade}).
  \item{}[number] \lstinline!AccrualLag!: The length (in days) of each variance accrual period. For classic variance swaps with no lag, set this to 1. \\
  Allowable values: Any integer greater than or equal to 1.
  \item{}[number] \lstinline!PayoffLimit!: The factor used to determine the maximum/minimum payoff under the swap. This corresponds to $C$
  in the final equation in the Product Description for Pairwise Variance Swap. \\
  Allowable values: Any non-negative number.
  \item{}[number] \lstinline!Cap!: The cap on the realised variances, as a factor of the corresponding strikes. For example,
  if \lstinline!Cap! is 2.5, then the cap level will be $2.5^2 \times \hat{K}_{var}$ for the basket realised variance.
  For trades with no cap, set \lstinline!Cap! to zero. \\
  Allowable values: Any non-negative number.
  \item{}[number] \lstinline!Floor!: The floor on the realised variances, as a factor of the corresponding strikes. For example,
  if \lstinline!Floor! is 0.1, then the floor level will be $0.1^2 \times \hat{K}_{var}$ for the basket realised variance.
  For trades with no floor, set \lstinline!Floor! to zero. \\
  Allowable values: Any non-negative number.
  \item{}[event] \lstinline!SettlementDate!: The date on which the swap payoff is settled. \\
  Allowable values: See \lstinline!Date! in Table \ref{tab:allow_stand_data}.
  \item{}[currency] \lstinline!PayCcy!: The payment currency. For FX, where the underlying is provided
      in the form \lstinline!FX-SOURCE-CCY1-CCY2! (see Table \ref{tab:fxindex_data}) this should
      be \lstinline!CCY2!. If \lstinline!CCY1! or the currency of the underlying (for EQ and
      COMM underlyings), this will result in a quanto payoff. Notice section \ref{sss:payccy_st}. \\
        Allowable values: See Table \ref{tab:currency} for allowable currency codes.
\end{itemize}

\subsubsection*{Variance Dispersion Swap}

The traditional trade representation is as follows, using EQ underlyings in this example:

\begin{minted}[fontsize=\footnotesize]{xml}
<Trade id="EQ_VarianceDispersionSwap">
  <TradeType>ScriptedTrade</TradeType>
  <Envelope>
    .....
  </Envelope>
  <VarianceDispersionSwapData>
    <LongShort type="longShort">Long</LongShort>
    <Underlyings1 type="index">
      <Value>EQ-RIC:HSBA.L</Value>
      <Value>EQ-RIC:MSFT.OQ</Value>
      <Value>EQ-RIC:AMZN.O</Value>
    </Underlyings1>
    <Weights1 type="number">
        .....
    </Weights1>
    <Strikes1 type="number">
        .....
    </Strikes1>
    <Spreads1 type="number">
        .....
    </Spreads1>
    <Notionals1 type="number">
        .....
    </Notionals1>
    <Caps1 type="number">
        .....
    </Caps1>
    <Floors1 type="number">
        .....
    </Floors1>
    <Underlyings2 type="index">
      <Value>EQ-RIC:.STOXX50E</Value>
      <Value>EQ-RIC:.SPX</Value>
    </Underlyings2>
    <Weights2 type="number">
        .....
    </Weights2>
    <Strikes2 type="number">
        .....
    </Strikes2>
    <Spreads2 type="number">
        .....
    </Spreads2>
    <Notionals2 type="number">
        .....
    </Notionals2>
    <Caps2 type="number">
        .....
    </Caps2>
    <Floors2 type="number">
        .....
    </Floors2>
    <ValuationSchedule type="event">
      <ScheduleData>
        .....
      </ScheduleData>
    </ValuationSchedule>
    <DividendAdjustment type="bool">false</DividendAdjustment>
    <SettlementDate type="event">2021-01-10</SettlementDate>
    <PayCcy type="currency">USD</PayCcy>
  </VarianceDispersionSwapData>
</Trade>
\end{minted}

The VarianceDispersionSwap script referenced in the trade above is shown in listing
\ref{lst:variance_dispersion_swap}

\begin{listing}[hbt]
\begin{minted}[fontsize=\footnotesize]{Basic}
D = SIZE(ValuationSchedule);
expectedN = D - 1;

FOR u IN (1, n1, 1) DO
  FOR d IN (2, D, 1) DO
    currPrice = Underlyings1[u](ValuationSchedule[d]);
    prevPrice = Underlyings1[u](ValuationSchedule[d-1]);
    realisedVariance = realisedVariance + pow(ln(currPrice/prevPrice), 2);
  END;

  realisedVariance = (252/expectedN) * realisedVariance + Spreads1[u];
  currentNotional = pow(100, 2) * Notionals1[u] / (2 * 100 * Strikes1[u]);
  strike = pow(Strikes1[u], 2);

  IF Floors1[u] > 0 THEN
    floor = pow(Floors1[u], 2);
    realisedVariance = max(floor * strike, realisedVariance);
  END;
  IF Caps1[u] > 0 THEN
    cap = pow(Caps1[u], 2);
    realisedVariance = min(cap * strike, realisedVariance);
  END;
  payoff1 = payoff1 + currentNotional * (realisedVariance - strike) * Weights1[u];
END;

FOR u IN (1, n2, 1) DO
  FOR d IN (2, D, 1) DO
    currPrice = Underlyings2[u](ValuationSchedule[d]);
    prevPrice = Underlyings2[u](ValuationSchedule[d-1]);
    realisedVariance = realisedVariance + pow(ln(currPrice/prevPrice), 2);
  END;

  realisedVariance = (252/expectedN) * realisedVariance + Spreads2[u];
  currentNotional = pow(100, 2) * Notionals2[u] / (2 * 100 * Strikes2[u]);
  strike = pow(Strikes2[u], 2);

  IF Floors2[u] > 0 THEN
    floor = pow(Floors2[u], 2);
    realisedVariance = max(floor * strike, realisedVariance);
  END;
  IF Caps2[u] > 0 THEN
    cap = pow(Caps2[u], 2);
    realisedVariance = min(cap * strike, realisedVariance);
  END;
  payoff2 = payoff2 + currentNotional * (realisedVariance - strike) * Weights2[u];
END;

payoff = LongShort * (payoff1 - payoff2);

Swap = PAY(payoff, ValuationSchedule[D], SettlementDate, PayCcy);

NUMBER currentNotional1, currentNotional2;

FOR u IN (1, n1, 1) DO
  currentNotional1 = currentNotional1 + (pow(100, 2) * Notionals1[u]
                                         / (2 * 100 * Strikes1[u]));
END;
FOR u IN (1, n2, 1) DO
  currentNotional2 = currentNotional2 + (pow(100, 2) * Notionals2[u]
                                         / (2 * 100 * Strikes2[u]));
END;
\end{minted}
\caption{Payoff script for a VarianceDispersionSwap.}
\label{lst:variance_dispersion_swap}
\end{listing}

The meanings and allowable values for the \lstinline!VarianceDispersionSwapData! node below.

\begin{itemize}
  \item{}[longShort] \lstinline!LongShort!: Own party position in the swap. \emph{Long} corresponds to selling volatility on
  the \lstinline!Underlyings1! basket and buying volatility on \lstinline!Underlyings2!. \\
  Allowable values: \emph{Long, Short}.
  \item{}[index] \lstinline!Underlyings1!: The basket of underlyings whose volatility is bought in the \emph{Long} position. \\
  Allowable values: For each underlying, see \ref{ss:underlying}.
  \item{}[number] \lstinline!Weights1!: List of weights applied to the final realised volatility of each underlying
  in the \lstinline!Underlyings1! basket. \\
  Allowable values: Any positive number.
  \item{}[number] \lstinline!Strikes1!: The volatility strike $K_{1,u,vol}$ of the variance swap for each underlying $u$
  in the \lstinline!Underlyings1! basket, quoted in absolute terms. If the swap was
  struck in terms of variance, the square root of that variance should be used here.\\
  Allowable values: Any positive number, as a percentage expressed in decimal form.
  \item{}[number] \lstinline!Spreads1!: Additional spread to the realised variance, for each underlying in the
  \lstinline!Underlyings1! basket. \\
  Allowable values: Any real number.
  \item{}[number] \lstinline!Notionals1!: For each underlying $u$ in the \lstinline!Underlyings1! basket, the vega notional amount.
  If the swap was struck in terms of variance notionals $N_{1,u,var}$, the corresponding vega notionals are given by
  $N_{1,u,vol} = N_{1,u,var} \cdot 2 \cdot 100 \cdot K_{1,u,vol}$ (where $K_{1,u,vol}$ is in
  absolute terms). \\
  Allowable values: Any real number.
  \item{}[number] \lstinline!Caps1!: For each underlying $u$ in the \lstinline!Underlyings1! basket, the cap on the realised variance
  as a factor of the strike. For example, if the value in \lstinline!Caps1! is 2.5, then the cap level will be
  $2.5^2 \times K_{1,u,\text{var}}$. For underlyings with no cap, set the value in \lstinline!Caps1! to zero. \\
  Allowable values: Any non-negative number.
  \item{}[number] \lstinline!Floors1!: For each underlying $u$ in the \lstinline!Underlyings1! basket, the floor on the realised
  variance as a factor of the strike. For example, if the value in \lstinline!Floors1! is 0.1, then the floor level will
  be $0.1^2 \times K_{1,u,\text{var}}$. For underlyings with no floor, set the value in \lstinline!Floors1! to zero. \\
  Allowable values: Any non-negative number.
  \item{}[index] \lstinline!Underlyings2!: The basket of underlyings whose volatility is sold in the \emph{Long} position.\\
  Allowable values: For each underlying, see \ref{ss:underlying}.
  \item{}[number] \lstinline!Weights2!: List of weights applied to the final realised volatility of each underlying
  in the \lstinline!Underlyings2! basket. \\
  Allowable values: Any positive number.
  \item{}[number] \lstinline!Strikes2!: The volatility strike $K_{2,u,vol}$ of the variance swap for each underlying $u$
  in the \lstinline!Underlyings2! basket, quoted in absolute terms. If the swap was
  struck in terms of variance, the square root of that variance should be used here.\\
  Allowable values: Any positive number, as a percentage expressed in decimal form.
  \item{}[number] \lstinline!Spreads2!: Additional spread to the realised variance, for each underlying in the
  \lstinline!Underlyings2! basket. \\
  Allowable values: Any real number.
  \item{}[number] \lstinline!Notionals2!: For each underlying $u$ in the \lstinline!Underlyings2! basket, the vega notional amount.
  If the swap was struck in terms of variance notionals $N_{2,u,var}$, the corresponding vega notionals are given by
  $N_{2,u,vol} = N_{2,u,var} \cdot 2 \cdot 100 \cdot K_{2,u,vol}$ (where $K_{2,u,vol}$ is in
  absolute terms). \\
  Allowable values: Any real number.
  \item{}[number] \lstinline!Caps2!: For each underlying $u$ in the \lstinline!Underlyings2! basket, the cap on the realised variance
  as a factor of the strike. For example, if the value in  \lstinline!Caps2! is 2.5, then the cap level will be
  $2.5^2 \times K_{2,u,\text{var}}$. For underlyings with no cap, set the value in \lstinline!Caps2! to zero. \\
  Allowable values: Any non-negative number.
  \item{}[number] \lstinline!Floors2!: For each underlying $u$ in the \lstinline!Underlyings2! basket, the floor on the realised
  variance as a factor of the strike. For example, if the value in \lstinline!Floors2! is 0.1, then the floor level will
  be $0.1^2 \times K_{2,u,\text{var}}$. For underlyings with no floor, set the value in \lstinline!Floors2! to zero. \\
  Allowable values: Any non-negative number.
  \item{}[bool] \lstinline!DividendAdjustment!: Whether or not the underlying price is adjusted for dividend payments. This feature is
  not yet supported.
  Allowable values: Boolean node, allowing \emph{Y}, \emph{N}, \emph{1}, \emph{0}, \emph{true}, \emph{false}, etc.
  The full set of allowable values is given in Table \ref{tab:boolean_allowable}.
  \item{}[event] \lstinline!ValuationSchedule!: The schedule defining the (daily) observation period for the variance accrual. \\
  Allowable values: See Section \ref{ss:schedule_data}.
  \item{}[event] \lstinline!SettlementDate!: The date on which the swap payoff is settled. \\
  Allowable values: See \lstinline!Date! in Table \ref{tab:allow_stand_data}.
  \item{}[currency] \lstinline!PayCcy!: The payment currency. For FX, where the underlying is provided
      in the form \lstinline!FX-SOURCE-CCY1-CCY2! (see Table \ref{tab:fxindex_data}) this should
      be \lstinline!CCY2!. If \lstinline!CCY1! or the currency of the underlying (for EQ and
      COMM underlyings), this will result in a quanto payoff. Notice section \ref{sss:payccy_st}. \\
        Allowable values: See Table \ref{tab:currency} for allowable currency codes.
\end{itemize}

\subsubsection*{Corridor Variance Dispersion Swap}

This instrument is a variance dispersion swap with a corridor feature, where variance is accrued only when the
the price of the first underlying of each pair of underlyings (between the first and second basket) falls within
the `corridor'. This means that the number of underlyings in each basket should be the same, and each pair is
determined based on the order that they are provided in the trade XML, as in the example below.

The traditional trade representation is as follows, using EQ underlyings in this example:

\begin{minted}[fontsize=\footnotesize]{xml}
<Trade id="EQ_VarianceDispersionSwap_Corridor">
  <TradeType>ScriptedTrade</TradeType>
  <Envelope>
    .....
  </Envelope>
  <CorridorVarianceDispersionSwapData>
    <LongShort type="longShort">Long</LongShort>
    <Weights type="number">
        .....
    </Weights>
    <Underlyings1 type="index">
      <Value>EQ-RIC:HSBA.L</Value>
      <Value>EQ-RIC:MSFT.OQ</Value>
      <Value>EQ-RIC:AMZN.O</Value>
    </Underlyings1>
    <Strikes1 type="number">
        .....
    </Strikes1>
    <Spreads1 type="number">
        .....
    </Spreads1>
    <Notionals1 type="number">
        .....
    </Notionals1>
    <Caps1 type="number">
        .....
    </Caps1>
    <Floors1 type="number">
        .....
    </Floors1>
    <Underlyings2 type="index">
      <Value>EQ-RIC:.STOXX50E</Value>
      <Value>EQ-RIC:.SPX</Value>
      <Value>EQ-RIC:.SPX</Value>
    </Underlyings2>
    <Strikes2 type="number">
        .....
    </Strikes2>
    <Spreads2 type="number">
        .....
    </Spreads2>
    <Notionals2 type="number">
        .....
    </Notionals2>
    <Caps2 type="number">
        .....
    </Caps2>
    <Floors2 type="number">
        .....
    </Floors2>
    <UpperBarrierLevels type="number">
        .....
    </UpperBarrierLevels>
    <LowerBarrierLevels type="number">
        .....
    </LowerBarrierLevels>
    <CountBothObservations type="bool">true</CountBothObservations>
    <AccrualAdjustment type="bool">true</AccrualAdjustment>
    <DividendAdjustment type="bool">false</DividendAdjustment>
    <ValuationSchedule type="event">
      <ScheduleData>
        .....
      </ScheduleData>
    </ValuationSchedule>
    <SettlementDate type="event">2021-01-10</SettlementDate>
    <PayCcy type="currency">USD</PayCcy>
  </CorridorVarianceDispersionSwapData>
</Trade>
\end{minted}

The CorridorVarianceDispersionSwap script referenced in the trade above is shown in listing
\ref{lst:corridor_variance_dispersion_swap}

\begin{listing}[hbt]
\begin{minted}[fontsize=\footnotesize]{Basic}
N = SIZE(Underlyings1);
D = SIZE(ValuationSchedule);
expectedN = D - 1;

FOR u IN (1, N, 1) DO
  FOR d IN (2, D, 1) DO
    currPrice1 = Underlyings1[u](ValuationSchedule[d]);
    prevPrice1 = Underlyings1[u](ValuationSchedule[d-1]);

    IF {CountBothObservations == 1 AND
        currPrice1 >= LowerBarrierLevels[u] AND currPrice1 <= UpperBarrierLevels[u] AND
        prevPrice1 >= LowerBarrierLevels[u] AND prevPrice1 <= UpperBarrierLevels[u]} OR
       {CountBothObservations == -1 AND
        prevPrice >= LowerBarrierLevels[u] AND prevPrice <= UpperBarrierLevels[u]} THEN
          currPrice2 = Underlyings2[u](ValuationSchedule[d]);
          prevPrice2 = Underlyings2[u](ValuationSchedule[d-1]);
          realisedVariance1 = realisedVariance1 + pow(ln(currPrice1/prevPrice1), 2);
          realisedVariance2 = realisedVariance2 + pow(ln(currPrice2/prevPrice2), 2);
          accruedDays = accruedDays + 1;
    END;
  END;

  IF AccrualAdjustment == -1 THEN
    accruedDays = expectedN;
  END;

  realisedVariance1 = (252/expectedN) * realisedVariance1 + Spreads1[u];
  realisedVariance2 = (252/expectedN) * realisedVariance2 + Spreads2[u];

  currentNotional1 = pow(100, 2) * Notionals1[u] / (2 * 100 * Strikes1[u]);
  currentNotional2 = pow(100, 2) * Notionals2[u] / (2 * 100 * Strikes2[u]);

  strike1 = pow(Strikes1[u], 2);
  strike2 = pow(Strikes2[u], 2);

  adjustedStrike1 = (accruedDays / expectedN) * strike1;
  adjustedStrike2 = (accruedDays / expectedN) * strike2;

  IF Floors1[u] > 0 THEN
    floor1 = pow(Floors1[u], 2);
    realisedVariance1 = max(floor1 * adjustedStrike1, realisedVariance1);
  END;
  IF Floors2[u] > 0 THEN
    floor2 = pow(Floors2[u], 2);
    realisedVariance2 = max(floor2 * adjustedStrike2, realisedVariance2);
  END;
  IF Caps1[u] > 0 THEN
    cap1 = pow(Caps1[u], 2);
    realisedVariance1 = max(cap1 * adjustedStrike1, realisedVariance1);
  END;
  IF Caps2[u] > 0 THEN
    cap2 = pow(Caps2[u], 2);
    realisedVariance2 = max(cap2 * adjustedStrike2, realisedVariance2);
  END;

  payoff1 = currentNotional1 * (realisedVariance1 - adjustedStrike1);
  payoff2 = currentNotional2 * (realisedVariance2 - adjustedStrike2);
  payoff = payoff + LongShort * Weights[u] * (payoff1 - payoff2);
END;

Swap = PAY(payoff, ValuationSchedule[D], SettlementDate, PayCcy);
\end{minted}
\caption{Payoff script for a CorridorVarianceDispersionSwap.}
\label{lst:corridor_variance_dispersion_swap}
\end{listing}

The meanings and allowable values for the \lstinline!CorridorVarianceDispersionSwapData! node below.

\begin{itemize}
  \item{}[longShort] \lstinline!LongShort!: Own party position in the swap. \emph{Long} corresponds to selling volatility on
  the \lstinline!Underlyings1! basket and buying volatility on \lstinline!Underlyings2!. \\
  Allowable values: \emph{Long, Short}.
  \item{}[number] \lstinline!Weights!: List of weights applied to the final realised volatility of each underlying
  in the \lstinline!Underlyings1! and \lstinline!Underlyings2! baskets. \\
  Allowable values: Any positive number.
  \item{}[index] \lstinline!Underlyings1!: The basket of underlyings whose volatility is bought in the \emph{Long} position. \\
  Allowable values: For each underlying, see \ref{ss:underlying}.
  \item{}[number] \lstinline!Strikes1!: The volatility strike $K_{1,u,vol}$ of the variance swap for each underlying $u$
  in the \lstinline!Underlyings1! basket, quoted in absolute terms. If the swap was
  struck in terms of variance, the square root of that variance should be used here.\\
  Allowable values: Any positive number, as a percentage expressed in decimal form.
  \item{}[number] \lstinline!Spreads1!: Additional spread to the realised variance, for each underlying in the
  \lstinline!Underlyings1! basket. \\
  Allowable values: Any real number.
  \item{}[number] \lstinline!Notionals1!: For each underlying $u$ in the \lstinline!Underlyings1! basket, the vega notional amount.
  If the swap was struck in terms of variance notionals $N_{1,u,var}$, the corresponding vega notionals are given by
  $N_{1,u,vol} = N_{1,u,var} \cdot 2 \cdot 100 \cdot K_{1,u,vol}$ (where $K_{1,u,vol}$ is in
  absolute terms). \\
  Allowable values: Any real number.
  \item{}[number] \lstinline!Caps1!: For each underlying $u$ in the \lstinline!Underlyings1! basket, the cap on the realised variance
  as a factor of the strike. For example, if the value in \lstinline!Caps1! is 2.5, then the cap level will be
  $2.5^2 \times K_{1,u,\text{var}}$. For underlyings with no cap, set the value in \lstinline!Caps1! to zero. \\
  Allowable values: Any non-negative number.
  \item{}[number] \lstinline!Floors1!: For each underlying $u$ in the \lstinline!Underlyings1! basket, the floor on the realised
  variance as a factor of the strike. For example, if the value in \lstinline!Floors1! is 0.1, then the floor level will
  be $0.1^2 \times K_{1,u,\text{var}}$. For underlyings with no floor, set the value in \lstinline!Floors1! to zero. \\
  Allowable values: Any non-negative number.
  \item{}[index] \lstinline!Underlyings2!: The basket of underlyings whose volatility is sold in the \emph{Long} position.\\
  Allowable values: For each underlying, see \ref{ss:underlying}.
  \item{}[number] \lstinline!Strikes2!: The volatility strike $K_{2,u,vol}$ of the variance swap for each underlying $u$
  in the \lstinline!Underlyings2! basket, quoted in absolute terms. If the swap was
  struck in terms of variance, the square root of that variance should be used here.\\
  Allowable values: Any positive number, as a percentage expressed in decimal form.
  \item{}[number] \lstinline!Spreads2!: Additional spread to the realised variance, for each underlying in the
  \lstinline!Underlyings2! basket. \\
  Allowable values: Any real number.
  \item{}[number] \lstinline!Notionals2!: For each underlying $u$ in the \lstinline!Underlyings2! basket, the vega notional amount.
  If the swap was struck in terms of variance notionals $N_{2,u,var}$, the corresponding vega notionals are given by
  $N_{2,u,vol} = N_{2,u,var} \cdot 2 \cdot 100 \cdot K_{2,u,vol}$ (where $K_{2,u,vol}$ is in
  absolute terms). \\
  Allowable values: Any real number.
  \item{}[number] \lstinline!Caps2!: For each underlying $u$ in the \lstinline!Underlyings2! basket, the cap on the realised variance
  as a factor of the strike. For example, if the value in  \lstinline!Caps2! is 2.5, then the cap level will be
  $2.5^2 \times K_{2,u,\text{var}}$. For underlyings with no cap, set the value in \lstinline!Caps2! to zero. \\
  Allowable values: Any non-negative number.
  \item{}[number] \lstinline!Floors2!: For each underlying $u$ in the \lstinline!Underlyings2! basket, the floor on the realised
  variance as a factor of the strike. For example, if the value in \lstinline!Floors2! is 0.1, then the floor level will
  be $0.1^2 \times K_{2,u,\text{var}}$. For underlyings with no floor, set the value in \lstinline!Floors2! to zero. \\
  Allowable values: Any non-negative number.
  \item{}[number] \lstinline!UpperBarrierLevels!: The agreed upper barrier price levels for each underlying in \lstinline!Underlyings1!. \\
  Allowable values: Any real number.
  \item{}[number] \lstinline!LowerBarrierLevels!: The agreed lower barrier price levels for each underlying in \lstinline!Underlyings1!. \\
  Allowable values: Any real number.
  \item{}[bool] \lstinline!CountBothObservations!: Whether the variance is accrued based on both the current and
  previous underlying prices (of each underlying in \lstinline!Underlyings1!) falling within the corridor (\emph{True})
  or only on the previous underlying price (\emph{False}). \\
  Allowable values: Boolean node, allowing \emph{Y}, \emph{N}, \emph{1}, \emph{0}, \emph{true}, \emph{false}, etc.
  The full set of allowable values is given in Table \ref{tab:boolean_allowable}.
  \item{}[bool] \lstinline!AccrualAdjustment!: Whether the strike will be scaled relative to the number of days that the
  underlying traded within the corridor. See, for example, $\widetilde{K}_{var}$ in section Corridor Variance Swap
  of the Product Description for Exotic Variance and Volatility Swaps.
  \item{}[bool] \lstinline!DividendAdjustment!: Whether or not the underlying price is adjusted for dividend payments. This feature is
  not yet supported.
  Allowable values: Boolean node, allowing \emph{Y}, \emph{N}, \emph{1}, \emph{0}, \emph{true}, \emph{false}, etc.
  The full set of allowable values is given in Table \ref{tab:boolean_allowable}.
  \item{}[event] \lstinline!ValuationSchedule!: The schedule defining the (daily) observation period for the variance accrual. \\
  Allowable values: See Section \ref{ss:schedule_data}.
  \item{}[event] \lstinline!SettlementDate!: The date on which the swap payoff is settled. \\
  Allowable values: See \lstinline!Date! in Table \ref{tab:allow_stand_data}.
  \item{}[currency] \lstinline!PayCcy!: The payment currency. For FX, where the underlying is provided
      in the form \lstinline!FX-SOURCE-CCY1-CCY2! (see Table \ref{tab:fxindex_data}) this should
      be \lstinline!CCY2!. If \lstinline!CCY1! or the currency of the underlying (for EQ and
      COMM underlyings), this will result in a quanto payoff. Notice section \ref{sss:payccy_st}. \\
        Allowable values: See Table \ref{tab:currency} for allowable currency codes.
\end{itemize}

\subsubsection*{KO Corridor Variance Dispersion Swap}

This instrument is a variance dispersion swap with a double-barrier knock-out and a corridor feature.

With the corridor feature, variance is accrued only when the the price of the first underlying of each pair of
underlyings (between the first and second basket) falls within the `corridor'. This means that the number of
underlyings in each basket should be the same, and each pair is determined based on the order that they are
provided in the trade XML, as in the example below.

The traditional trade representation is as follows, using EQ underlyings in this example:

\begin{minted}[fontsize=\footnotesize]{xml}
<Trade id="EQ_VarianceDispersionSwap_Corridor_KO">
  <TradeType>ScriptedTrade</TradeType>
  <Envelope>
    .....
  </Envelope>
  <KOCorridorVarianceDispersionSwapData>
    <LongShort type="longShort">Long</LongShort>
    <Weights type="number">
        .....
    </Weights>
    <Underlyings1 type="index">
      <Value>EQ-RIC:HSBA.L</Value>
      <Value>EQ-RIC:MSFT.OQ</Value>
      <Value>EQ-RIC:AMZN.O</Value>
    </Underlyings1>
    <Strikes1 type="number">
        .....
    </Strikes1>
    <Spreads1 type="number">
        .....
    </Spreads1>
    <Notionals1 type="number">
        .....
    </Notionals1>
    <Caps1 type="number">
        .....
    </Caps1>
    <Floors1 type="number">
        .....
    </Floors1>
    <Underlyings2 type="index">
      <Value>EQ-RIC:.STOXX50E</Value>
      <Value>EQ-RIC:.SPX</Value>
      <Value>EQ-RIC:.SPX</Value>
    </Underlyings2>
    <Strikes2 type="number">
        .....
    </Strikes2>
    <Spreads2 type="number">
        .....
    </Spreads2>
    <Notionals2 type="number">
        .....
    </Notionals2>
    <Caps2 type="number">
        .....
    </Caps2>
    <Floors2 type="number">
        .....
    </Floors2>
    <CorridorUpperBarrierLevels type="number">
        .....
    </CorridorUpperBarrierLevels>
    <CorridorLowerBarrierLevels type="number">
        .....
    </CorridorLowerBarrierLevels>
    <KOUpperBarrierLevels type="number">
        .....
    <KOUpperBarrierLevels/>
    <KOLowerBarrierLevels type="number">
        .....
    </KOLowerBarrierLevels>
    <CountBothObservations type="bool">true</CountBothObservations>
    <AccrualAdjustment type="bool">true</AccrualAdjustment>
    <DividendAdjustment type="bool">false</DividendAdjustment>
    <KnockOutSchedule type="event">
      <ScheduleData>
        .....
      </ScheduleData>
    </KnockOutSchedule>
    <VarianceAccrualStartDate type="event">2020-12-02</VarianceAccrualStartDate>
    <SettlementSchedule type="event">
      <DerivedSchedule>
        <BaseSchedule>KnockOutSchedule</BaseSchedule>
          .....
      </DerivedSchedule>
    </SettlementSchedule>
    <PayCcy type="currency">USD</PayCcy>
  </KOCorridorVarianceDispersionSwapData>
</Trade>
\end{minted}

The KOCorridorVarianceDispersionSwap script referenced in the trade above is shown in listing
\ref{lst:ko_corridor_variance_dispersion_swap}

\begin{listing}[hbt]
\begin{minted}[fontsize=\footnotesize]{Basic}
N = SIZE(Underlyings1);
Dk = SIZE(KnockOutSchedule);
expectedN = D - 1;

FOR d IN (1, Dk, 1) DO
  IF KnockOutSchedule[d] > VarianceAccrualStartDate THEN
    expectedN = expectedN + 1;
  END;
END;

alive = 1;
FOR d IN (1, Dk, 1) DO
  IF alive == 1 THEN
    IF KnockOutSchedule[d] > VarianceAccrualStartDate THEN
      FOR u IN (1, N, 1) DO
        currPrice1 = Underlyings1[u](KnockOutSchedule[d]);
        prevPrice1 = Underlyings1[u](KnockOutSchedule[d-1]);

        IF {CountBothObservations == 1 AND
            currPrice1 >= CorridorLowerBarrierLevels[u] AND currPrice1 <= CorridorUpperBarrierLevels[u] AND
            prevPrice1 >= CorridorLowerBarrierLevels[u] AND prevPrice1 <= CorridorUpperBarrierLevels[u]} OR
           {CountBothObservations == -1 AND
            prevPrice >= CorridorLowerBarrierLevels[u] AND prevPrice <= CorridorUpperBarrierLevels[u]} THEN
              currPrice2 = Underlyings2[u](KnockOutSchedule[d]);
              prevPrice2 = Underlyings2[u](KnockOutSchedule[d-1]);
              realisedVariance1 = realisedVariance1 + pow(ln(currPrice1/prevPrice1), 2);
              realisedVariance2 = realisedVariance2 + pow(ln(currPrice2/prevPrice2), 2);
              accruedDays = accruedDays + 1;
        END;
      END;
    END;

    FOR u IN (1, N, 1) DO
      currPrice1 = Underlyings1[u](KnockOutSchedule[d]);
      IF currPrice1 <= KOLowerBarrierLevels[u] OR currPrice1 >= KOUpperBarrierLevels[u] THEN
        alive = 0;
      END;
    END;
  END;

  IF {alive == 0 OR d == Dk} AND {calculated == 0} THEN
    calculated = 1;

    realisedVariance1 = (252/expectedN) * realisedVariance1 + Spreads1[u];
    realisedVariance2 = (252/expectedN) * realisedVariance2 + Spreads2[u];

    IF AccrualAdjustment == -1 THEN
      accruedDays = expectedN;
    END;

    currentNotional1 = pow(100, 2) * Notionals1[u] / (2 * 100 * Strikes1[u]);
    currentNotional2 = pow(100, 2) * Notionals2[u] / (2 * 100 * Strikes2[u]);

    strike1 = pow(Strikes1[u], 2);
    strike2 = pow(Strikes2[u], 2);

    adjustedStrike1 = (accruedDays / expectedN) * strike1;
    adjustedStrike2 = (accruedDays / expectedN) * strike2;

    IF Floors1[u] > 0 THEN
      floor1 = pow(Floors1[u], 2);
      realisedVariance1 = max(floor1 * adjustedStrike1, realisedVariance1);
    END;
    IF Floors2[u] > 0 THEN
      floor2 = pow(Floors2[u], 2);
      realisedVariance2 = max(floor2 * adjustedStrike2, realisedVariance2);
    END;
    IF Caps1[u] > 0 THEN
      cap1 = pow(Caps1[u], 2);
      realisedVariance1 = max(cap1 * adjustedStrike1, realisedVariance1);
    END;
    IF Caps2[u] > 0 THEN
      cap2 = pow(Caps2[u], 2);
      realisedVariance2 = max(cap2 * adjustedStrike2, realisedVariance2);
    END;

    payoff1 = currentNotional1 * (realisedVariance1 - adjustedStrike1);
    payoff2 = currentNotional2 * (realisedVariance2 - adjustedStrike2);
    payoff = payoff + LongShort * Weights[u] * (payoff1 - payoff2);

    Swap = PAY(payoff, KnockOutSchedule[d], SettlementSchedule[d], PayCcy);
  END;
END;
\end{minted}
\caption{Payoff script for a KOCorridorVarianceDispersionSwap.}
\label{lst:ko_corridor_variance_dispersion_swap}
\end{listing}

The meanings and allowable values for the \lstinline!KOCorridorVarianceDispersionSwapData! node below.

\begin{itemize}
  \item{}[longShort] \lstinline!LongShort!: Own party position in the swap. \emph{Long} corresponds to selling volatility on
  the \lstinline!Underlyings1! basket and buying volatility on \lstinline!Underlyings2!. \\
  Allowable values: \emph{Long, Short}.
  \item{}[number] \lstinline!Weights!: List of weights applied to the final realised volatility of each underlying
  in the \lstinline!Underlyings1! and \lstinline!Underlyings2! baskets. \\
  Allowable values: Any positive number.
  \item{}[index] \lstinline!Underlyings1!: The basket of underlyings whose volatility is bought in the \emph{Long} position. \\
  Allowable values: For each underlying, see \ref{ss:underlying}.
  \item{}[number] \lstinline!Strikes1!: The volatility strike $K_{1,u,vol}$ of the variance swap for each underlying $u$
  in the \lstinline!Underlyings1! basket, quoted in absolute terms. If the swap was
  struck in terms of variance, the square root of that variance should be used here.\\
  Allowable values: Any positive number, as a percentage expressed in decimal form.
  \item{}[number] \lstinline!Spreads1!: Additional spread to the realised variance, for each underlying in the
  \lstinline!Underlyings1! basket. \\
  Allowable values: Any real number.
  \item{}[number] \lstinline!Notionals1!: For each underlying $u$ in the \lstinline!Underlyings1! basket, the vega notional amount.
  If the swap was struck in terms of variance notionals $N_{1,u,var}$, the corresponding vega notionals are given by
  $N_{1,u,vol} = N_{1,u,var} \cdot 2 \cdot 100 \cdot K_{1,u,vol}$ (where $K_{1,u,vol}$ is in
  absolute terms). \\
  Allowable values: Any real number.
  \item{}[number] \lstinline!Caps1!: For each underlying $u$ in the \lstinline!Underlyings1! basket, the cap on the realised variance
  as a factor of the strike. For example, if the value in \lstinline!Caps1! is 2.5, then the cap level will be
  $2.5^2 \times K_{1,u,\text{var}}$. For underlyings with no cap, set the value in \lstinline!Caps1! to zero. \\
  Allowable values: Any non-negative number.
  \item{}[number] \lstinline!Floors1!: For each underlying $u$ in the \lstinline!Underlyings1! basket, the floor on the realised
  variance as a factor of the strike. For example, if the value in \lstinline!Floors1! is 0.1, then the floor level will
  be $0.1^2 \times K_{1,u,\text{var}}$. For underlyings with no floor, set the value in \lstinline!Floors1! to zero. \\
  Allowable values: Any non-negative number.
  \item{}[index] \lstinline!Underlyings2!: The basket of underlyings whose volatility is sold in the \emph{Long} position.\\
  Allowable values: For each underlying, see \ref{ss:underlying}.
  \item{}[number] \lstinline!Strikes2!: The volatility strike $K_{2,u,vol}$ of the variance swap for each underlying $u$
  in the \lstinline!Underlyings2! basket, quoted in absolute terms. If the swap was
  struck in terms of variance, the square root of that variance should be used here.\\
  Allowable values: Any positive number, as a percentage expressed in decimal form.
  \item{}[number] \lstinline!Spreads2!: Additional spread to the realised variance, for each underlying in the
  \lstinline!Underlyings2! basket. \\
  Allowable values: Any real number.
  \item{}[number] \lstinline!Notionals2!: For each underlying $u$ in the \lstinline!Underlyings2! basket, the vega notional amount.
  If the swap was struck in terms of variance notionals $N_{2,u,var}$, the corresponding vega notionals are given by
  $N_{2,u,vol} = N_{2,u,var} \cdot 2 \cdot 100 \cdot K_{2,u,vol}$ (where $K_{2,u,vol}$ is in
  absolute terms). \\
  Allowable values: Any real number.
  \item{}[number] \lstinline!Caps2!: For each underlying $u$ in the \lstinline!Underlyings2! basket, the cap on the realised variance
  as a factor of the strike. For example, if the value in  \lstinline!Caps2! is 2.5, then the cap level will be
  $2.5^2 \times K_{2,u,\text{var}}$. For underlyings with no cap, set the value in \lstinline!Caps2! to zero. \\
  Allowable values: Any non-negative number.
  \item{}[number] \lstinline!Floors2!: For each underlying $u$ in the \lstinline!Underlyings2! basket, the floor on the realised
  variance as a factor of the strike. For example, if the value in \lstinline!Floors2! is 0.1, then the floor level will
  be $0.1^2 \times K_{2,u,\text{var}}$. For underlyings with no floor, set the value in \lstinline!Floors2! to zero. \\
  Allowable values: Any non-negative number.
  \item{}[number] \lstinline!CorridorUpperBarrierLevels!: The agreed corridor upper barrier price levels for each underlying
  in \lstinline!Underlyings1!. \\
  Allowable values: Any real number.
  \item{}[number] \lstinline!CorridorLowerBarrierLevels!: The agreed corridor lower barrier price levels for each underlying
  in \lstinline!Underlyings1!. \\
  Allowable values: Any real number.
  \item{}[number] \lstinline!KOUpperBarrierLevels!: The agreed knock-out upper barrier price levels for each underlying in
  \lstinline!Underlyings1!. \\
  Allowable values: Any real number.
  \item{}[number] \lstinline!KOLowerBarrierLevels!: The agreed knock-out lower barrier price levels for each underlying in
  \lstinline!Underlyings1!. \\
  Allowable values: Any real number.
  \item{}[bool] \lstinline!CountBothObservations!: Whether the variance is accrued based on both the current and
  previous underlying prices (of each underlying in \lstinline!Underlyings1!) falling within the corridor (\emph{True})
  or only on the previous underlying price (\emph{False}). \\
  Allowable values: Boolean node, allowing \emph{Y}, \emph{N}, \emph{1}, \emph{0}, \emph{true}, \emph{false}, etc.
  The full set of allowable values is given in Table \ref{tab:boolean_allowable}.
  \item{}[bool] \lstinline!AccrualAdjustment!: Whether the strike will be scaled relative to the number of days that the
  underlying traded within the corridor. See, for example, $\widetilde{K}_{var}$ in section Corridor Variance Swap
  of the Product Description for Exotic Variance and Volatility Swaps.
  \item{}[bool] \lstinline!DividendAdjustment!: Whether or not the underlying price is adjusted for dividend payments. This feature is
  not yet supported.
  Allowable values: Boolean node, allowing \emph{Y}, \emph{N}, \emph{1}, \emph{0}, \emph{true}, \emph{false}, etc.
  The full set of allowable values is given in Table \ref{tab:boolean_allowable}.
  \item{}[event] \lstinline!KnockOutSchedule!: The schedule defining the (daily) knock-out observation period. \\
  Allowable values: See Section \ref{ss:schedule_data}.
  \item{}[event] \lstinline!VarianceAccrualStartDate!: The date on which variance observation/accrual starts. The last variance
  observation date is the same as the last date in the \lstinline!KnockOutSchedule!.  \\
  Allowable values: See \lstinline!Date! in Table \ref{tab:allow_stand_data}.
  \item{}[event] \lstinline!SettlementSchedule!: The settlement dates derived from the \lstinline!KnockOutSchedule! using a
  settlement lag to reflect settlement after a knock-out event, or after the final variance observation date if
  no knock-out occurs.. \\
  Allowable values: See section \ref{ss:schedule_data} Schedule Data and Dates, or DerivedSchedule (see \ref{app:scriptedtrade}).
  \item{}[currency] \lstinline!PayCcy!: The payment currency. For FX, where the underlying is provided
      in the form \lstinline!FX-SOURCE-CCY1-CCY2! (see Table \ref{tab:fxindex_data}) this should
      be \lstinline!CCY2!. If \lstinline!CCY1! or the currency of the underlying (for EQ and
      COMM underlyings), this will result in a quanto payoff. Notice section \ref{sss:payccy_st}. \\
        Allowable values: See Table \ref{tab:currency} for allowable currency codes.
\end{itemize}

\subsubsection*{Gamma Swap}

The traditional trade representation is as follows, using EQ underlyings in this example:

\begin{minted}[fontsize=\footnotesize]{xml}
<Trade id="EQ_GammaSwap">
  <TradeType>ScriptedTrade</TradeType>
  <Envelope>
   .....
  </Envelope>
    <GammaSwapData>
  <LongShort type="longShort">Long</LongShort>
  <Strike type="number">0.19</Strike>
  <Notional type="number">1380</Notional>
  <SettlementDate type="event">2021-09-25</SettlementDate>
  <Underlying type="index">EQ-RIC:.SPX</Underlying>
  <ValuationSchedule type="event">
    <ScheduleData>
      <Rules>
    <StartDate>2020-11-26</StartDate>
    <EndDate>2021-09-18</EndDate>
    <Tenor>1D</Tenor>
    <Convention>Following</Convention>
    <TermConvention>Following</TermConvention>
    <Calendar>USA</Calendar>
    <Rule>Forward</Rule>
       </Rules>
     </ScheduleData>
  </ValuationSchedule>
  <PayCcy type="currency">USD</PayCcy>
  </GammaSwapData>
</Trade>
\end{minted}

The GammaSwap script referenced in the trade above is shown in listing
\ref{lst:gamma_swap}

\begin{listing}[hbt]
\begin{minted}[fontsize=\footnotesize]{Basic}
        REQUIRE {Notional >= 0} AND {Strike >= 0};

        NUMBER d, expectedN, realisedVariance, currPrice, prevPrice, presPrice, currentNotional;
        NUMBER payoff, realisedVariation, strike;

        presPrice = Underlying(ValuationSchedule[1]);

        FOR d IN (2, SIZE(ValuationSchedule), 1) DO
          currPrice = Underlying(ValuationSchedule[d]);
          prevPrice = Underlying(ValuationSchedule[d-1]);
          realisedVariance = realisedVariance + (currPrice/presPrice) * pow(ln(currPrice/prevPrice), 2);
        END;

        expectedN = SIZE(ValuationSchedule) - 1;
        realisedVariance = (252/expectedN) * realisedVariance;

        realisedVariation = realisedVariance;
        currentNotional = pow(100, 2) * Notional / (2 * 100 * Strike);
        strike = pow(Strike, 2);

        payoff = LongShort * currentNotional * (realisedVariation - 
                    strike);

        Swap = PAY(payoff, ValuationSchedule[SIZE(ValuationSchedule)],
                   SettlementDate, PayCcy);
\end{minted}
\caption{Payoff script for a GammaSwap.}
\label{lst:gamma_swap}
\end{listing}

The meanings and allowable values for the \lstinline!GammaSwapData! node below.

\begin{itemize}
  \item{}[longShort] \lstinline!LongShort!: Own party position in the swap. \emph{Long} corresponds to paying out on the
  fixed/strike variance (volatility) and receiving on the floating/realised variance (volatility). In other words,
  a long position has positive value if the realised variance (volatility) exceeds the variance (volatility)
  strike. \\
  Allowable values: \emph{Long, Short}.
  \item{}[number] \lstinline!Strike!: The volatility strike $K$ of the underlying quoted in absolute terms.
  If the swap was struck in terms of variance, the square roots of the variances should be used here. \\
  Allowable values: Any non-negative number, as a percentage expressed in decimal form.
  \item{}[number] \lstinline!Notional!: The vega notional amount. If the swap was struck in terms of a variance notional
  $N_{var}$, the corresponding vega notional is given by $N_{vol} = N_{var} \cdot 2 \cdot 100 \cdot K$
  (where $K$ is in absolute terms). \\
  Allowable values: Any non-negative number.
  \item{}[event] \lstinline!SettlementDate!: The date on which the swap payoff is settled. \\
  Allowable values: See \lstinline!Date! in Table \ref{tab:allow_stand_data}.
  \item{}[index] \lstinline!Underlying!: Underlying index. \\
  Allowable values: See Section \ref{data_index} for allowable values.
  \item{}[event] \lstinline!ValuationSchedule!: The base schedule which defines the start dates of the variance accrual schedule. This node is
  also used to derive the variance accrual schedule. Note that the end date (or last date) of this schedule must be the final
  period end date of the instrument. \\
  Allowable values: See Section \ref{ss:schedule_data}.
  \item{}[currency] \lstinline!PayCcy!: The payment currency. For FX, where the underlying is provided
      in the form \lstinline!FX-SOURCE-CCY1-CCY2! (see Table \ref{tab:fxindex_data}) this should
      be \lstinline!CCY2!. If \lstinline!CCY1! or the currency of the underlying (for EQ and
      COMM underlyings), this will result in a quanto payoff. Notice section \ref{sss:payccy_st}. \\
        Allowable values: See Table \ref{tab:currency} for allowable currency codes.
\end{itemize}

\subsubsection*{Basket Variance Swap}

The \lstinline!FxBasketVarianceSwap! and \lstinline!EquityBasketVarianceSwap! \lstinline!CommodityBasketVarianceSwap! trade types
have trade data containers (respectively):
\begin{itemize}
  \item \lstinline!FxBasketVarianceSwapData!
  \item \lstinline!EquityBasketVarianceSwapData!
  \item \lstinline!CommodityBasketVarianceSwapData!
\end{itemize}
Listing \ref{lst:eqbasketvarianceswap_data} shows the structure of example trades for an Equity underlying.

\begin{listing}[H]
\begin{minted}[fontsize=\footnotesize]{xml}
<Trade id="Equity_VarianceSwap_Basket">
  <TradeType>EquityBasketVarianceSwap</TradeType>
  <Envelope>
    .......
  </Envelope>
  <EquityBasketVarianceSwapData>
    <LongShort>Long</LongShort>
    <Strike>0.31613</Strike>
    <Notional>1340.0</Notional>
    <Underlyings>
      <Underlying>
        <Type>Equity</Type>
        <Name>RIC:.SPX</Name>
        <Weight>0.75</Weight>
      </Underlying>
      <Underlying>
        <Type>Equity</Type>
        <Name>RIC:.STOXX50E</Name>
        <Weight>0.25</Weight>
      </Underlying>
    </Underlyings>
    <SquaredPayoff>true</SquaredPayoff>
    <ValuationSchedule>
        <Rules>
        .......
        </Rules>
    </ValuationSchedule>
    <Cap>2.5</Cap>
    <SettlementDate>2022-06-29</SettlementDate>
    <Currency>USD</Currency>
  </EquityBasketVarianceSwapData>
</Trade>
\end{minted}
\caption{EquityBasketVarianceSwap data}
\label{lst:eqbasketvarianceswap_data}
\end{listing}

The payout formula is the same as for a standard capped/floored variance swap, with the
realised variance calculated as follows:

\begin{equation*}
  Realised Variance = \frac{252}{D} \sum_{i=1}^D \left[\sum_{n=1}^N w_n \text{ln}\left(\frac{X_{i,n}}{X_{i-1,n}}\right) \right]^2,
\end{equation*}

for $D$ variance accrual/valuation dates over the life of the trade, where

\begin{itemize}
  \item $X_{i,n}$ denotes the fixing for the $n$-th underlying at $t_i$.
  \item $w_n$ is the basket weight for the $n$-th underlying.
\end{itemize}

The meanings and allowable values of elements in the \lstinline!EquityBasketVarianceSwapData! node follow below.

\begin{itemize}
  \item{} \lstinline!LongShort!: Own party position. \\
    Allowable values: \emph{Long}, \emph{Short}
  \item{} \lstinline!Strike!: The volatility strike $K_{vol}$ of the variance swap quoted in absolute terms.
  If the swap was struck in terms of variance, the square root of that variance should be used here. \\
  Allowable values: Any non-negative number, as a percentage expressed in decimal form.
  \item{} \lstinline!Notional!: The vega notional amount. If the swap was struck in terms of a variance notional
  $N_{var}$, the corresponding vega notional is given by $N_{vol} = N_{var} \cdot 2 \cdot 100 \cdot K_{vol}$ (where
  $K_{vol}$ is in absolute terms). \\
  Allowable values: Any non-negative number.
  \item \lstinline!Underlyings!: The basket of underlyings. \\
    For \lstinline!FxBasketVarianceSwap!, \lstinline!Type! is set to \emph{FX}
    and \lstinline!Name! is a string of the form \emph{SOURCE-CCY1-CCY2} where \emph{CCY1} is
    the foreign currency, \emph{CCY2} is the domestic currency, and \emph{SOURCE} is the fixing
    source, see Table \ref{tab:fxindex_data}.
    (section Data Node / Index) for allowable values.

    For \lstinline!EquityBasketVarianceSwap!, \lstinline!Type! is set to \emph{Equity} and
    \lstinline!Name! and other fields are as outlined in \ref{ss:underlying}.

    For \lstinline!CommodityBasketVarianceSwap!, \lstinline!Type! is set to \emph{Commodity} and
    \lstinline!Name! is an identifier of the commodity as outlined in \ref{ss:underlying} and
    in Table \ref{tab:commodity_data}

    Allowable values: For each underlying, an \lstinline!Underlying! node as outlined in
    \ref{ss:underlying}. All underlyings must be from the same asset class.
  \item{} \lstinline!SquaredPayoff! [Optional]: Flag indicating whether the trade is a variance swap (\emph{True})
  or a volatility swap (\emph{False}). \\
  Allowable values: Boolean node, allowing \emph{Y}, \emph{N}, \emph{1}, \emph{0}, \emph{true}, \emph{false}, etc.
  Defaults to \emph{False} if left blank or omitted. The full set of allowable values is given in Table \ref{tab:boolean_allowable}.
\item{} \lstinline!ValuationSchedule!: The schedule defining the (daily) observation period for the variance accrual. \\
    Allowable values: See Section \ref{ss:schedule_data}.
  \item{} \lstinline!Cap! [Optional]: The cap on the realised variance, as a factor of the corresponding strike. For example,
  if \lstinline!Cap! is 2.5, then the cap level will be $2.5^2 \times \hat{K}_{var}$ for the basket realised variance.
  For trades with no cap, set \lstinline!Cap! to zero. \\
    Allowable values: Any non-negative number. Defaults to zero if left blank or omitted.
  \item{} \lstinline!Floor! [Optional]:  The floor on the realised variance, as a factor of the corresponding strike. For example,
  if \lstinline!Floor! is 0.1, then the floor level will be $0.1^2 \times \hat{K}_{var}$ for the basket realised variance.
  For trades with no floor, set \lstinline!Floor! to zero. \\
    Allowable values: Any non-negative number. Defaults to zero if left blank or omitted.
  \item{} \lstinline!SettlementDate!: The date on which the swap payoff is settled. \\
    Allowable values: See \lstinline!Date! in Table \ref{tab:allow_stand_data}.
  \item{} \lstinline!Currency!: The settlement currency. \\
    Allowable values: See Table \ref{tab:currency} \lstinline!Currency!.
\end{itemize}

Basket variance swaps can alternatively be represented as {\em scripted trades}, refer to Section \ref{app:scriptedtrade} for an introduction.

\begin{minted}[fontsize=\footnotesize]{xml}
<Trade id="EQ_VarianceSwap_Basket">
  <TradeType>ScriptedTrade</TradeType>
  <Envelope>
    ......
  </Envelope>
  <BasketVarianceSwapData>
    <LongShort type="longShort">Long</LongShort>
    <Underlyings type="index">
      <Value>EQ-RIC:.STOXX50E</Value>
      <Value>EQ-RIC:.SPX</Value>
    </Underlyings>
    <Weights type="number">
      <Value>0.25</Value>
      <Value>0.75</Value>
    </Weights>
    <Strike type="number">0.31613</Strike>
    <Notional type="number">1339.0</Notional>
    <SquaredPayoff type="bool">true</SquaredPayoff>
    <ValuationSchedule type="event">
      <ScheduleData>
        <Rules>
         ......
        </Rules>
      </ScheduleData>
    </ValuationSchedule>
    <Cap type="number">2.5</Cap>
    <Floor type="number">0</Floor>
    <SettlementDate type="event">2022-06-29</SettlementDate>
    <PayCcy type="currency">USD</PayCcy>
  </BasketVarianceSwapData>
</Trade>
\end{minted}

The PairwiseVarianceSwap script referenced in the trade above is shown in listing
\ref{lst:basket_variance_swap}

\begin{listing}[hbt]
\begin{minted}[fontsize=\footnotesize]{Basic}
        NUMBER sumOfWeights;
        FOR i IN (1, n, 1) DO
          sumOfWeights = sumOfWeights + Weights[i];
        END;
        REQUIRE sumOfWeights == 1;

        NUMBER d, expectedN, currPrice[n], prevPrice[n];
        NUMBER realisedVariance, basketVariation, realisedVariation;
        NUMBER strike, cap, floor, currentNotional, payoff;

        FOR d IN (2, SIZE(ValuationSchedule), 1) DO
          basketVariation = 0;
          FOR i IN (1, n, 1) DO
            currPrice[i] = Underlyings[i](ValuationSchedule[d]);
            prevPrice[i] = Underlyings[i](ValuationSchedule[d-1]);
            basketVariation = basketVariation + Weights[i] * ln(currPrice[i]/prevPrice[i]);
          END;
          realisedVariance = realisedVariance + pow(basketVariation, 2);
        END;

        expectedN = SIZE(ValuationSchedule) - 1;
        realisedVariance = (252/expectedN) * realisedVariance;

        IF SquaredPayoff == 1 THEN
          realisedVariation = realisedVariance;
          currentNotional = pow(100, 2) * Notional / (2 * 100 * Strike);
          strike = pow(Strike, 2);
        ELSE
          realisedVariation = sqrt(realisedVariance);
          currentNotional = 100 * Notional;
          strike = Strike;
        END;

        IF Floor > 0 THEN
          IF SquaredPayoff == 1 THEN
            floor = pow(Floor, 2);
          ELSE
            floor = Floor;
          END;
          realisedVariation = max(floor * strike, realisedVariation);
        END;
        IF Cap > 0 THEN
          IF SquaredPayoff == 1 THEN
            cap = pow(Cap, 2);
          ELSE
            cap = Cap;
          END;
          realisedVariation = min(cap * strike, realisedVariation);
        END;

        payoff = LongShort * currentNotional * (realisedVariation - strike);

        Swap = PAY(payoff, ValuationSchedule[SIZE(ValuationSchedule)],
                   SettlementDate, PayCcy);
\end{minted}
\caption{Payoff script for a BasketVarianceSwap.}
\label{lst:basket_variance_swap}
\end{listing}

The meanings and allowable values for the \lstinline!BasketVarianceSwapData! node below.

\begin{itemize}
  \item{}[longShort] \lstinline!LongShort!: Own party position in the swap. \emph{Long} corresponds to paying out on the
  fixed/strike variance (volatility) and receiving on the floating/realised variance (volatility). In other words,
  a long position has positive value if the realised variance (volatility) exceeds the variance (volatility)
  strike. \\
  Allowable values: \emph{Long, Short}.
  \item{}[index] \lstinline!Underlyings!: List of underlying indices. \\
  Allowable values: See Section \ref{data_index} for allowable values. 
  \item{}[number] \lstinline!Strike!: The volatility strike $K_{vol}$ of the variance swap quoted in absolute terms.
  If the swap was struck in terms of variance, the square root of that variance should be used here. \\
  Allowable values: Any non-negative number, as a percentage expressed in decimal form.
  \item{}[number] \lstinline!Notional!: The vega notional amount. If the swap was struck in terms of a variance notional
  $N_{var}$, the corresponding vega notional is given by $N_{vol} = N_{var} \cdot 2 \cdot 100 \cdot K_{vol}$ (where
  $K_{vol}$ is in absolute terms). \\
  Allowable values: Any non-negative number.
  \item{} \lstinline!SquaredPayoff!: Flag indicating whether the trade is a variance swap (\emph{True})
  or a volatility swap (\emph{False}). \\
  Allowable values: Boolean node, allowing \emph{Y}, \emph{N}, \emph{1}, \emph{0}, \emph{true}, \emph{false}, etc.
  Defaults to \emph{False} if left blank or omitted. The full set of allowable values is given in Table \ref{tab:boolean_allowable}.
  \item{}[event] \lstinline!ValuationSchedule!: The schedule defining the (daily) observation period for the variance accrual. \\
  Allowable values: See Section \ref{ss:schedule_data}.
  \item{}[number] \lstinline!Cap!: The cap on the realised variance, as a factor of the corresponding strike. For example,
  if \lstinline!Cap! is 2.5, then the cap level will be $2.5^2 \times \hat{K}_{var}$ for the basket realised variance.
  For trades with no cap, set \lstinline!Cap! to zero. \\
  Allowable values: Any non-negative number.
  \item{}[number] \lstinline!Floor!: The floor on the realised variance, as a factor of the corresponding strike. For example,
  if \lstinline!Floor! is 0.1, then the floor level will be $0.1^2 \times \hat{K}_{var}$ for the basket realised variance.
  For trades with no floor, set \lstinline!Floor! to zero. \\
  Allowable values: Any non-negative number.
  \item{}[event] \lstinline!SettlementDate!: The date on which the swap payoff is settled. \\
  Allowable values: See \lstinline!Date! in Table \ref{tab:allow_stand_data}.
  \item{}[currency] \lstinline!PayCcy!: The settlement currency. \\
  Allowable values: See Table \ref{tab:currency} for allowable currency codes.
\end{itemize}