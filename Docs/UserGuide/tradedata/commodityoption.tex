
\subsubsection{Commodity Option}
\label{ss:input_commodity_option}

The \lstinline!CommodityOptionData! node is the trade data container for the \emph{CommodityOption} trade type.  Vanilla commodity 
options are supported. The exercise style may be \emph{European} or \emph{American}. The \lstinline!CommodityOptionData! node includes exactly 
one \lstinline!OptionData! trade component sub-node plus elements specific to the commodity option. The structure of 
a \lstinline!CommodityOptionData! node for a commodity option is shown in Listing \ref{lst:comoption_data}.

\begin{listing}[H]
%\hrule\medskip
\begin{minted}[fontsize=\footnotesize]{xml}
<CommodityOptionData>
  <OptionData>
   <LongShort>Short</LongShort>
   <OptionType>Put</OptionType>
   <Style>European</Style>
   <Settlement>Cash</Settlement>
   <PayOffAtExpiry>false</PayOffAtExpiry>
    <ExerciseDates>
      <ExerciseDate>2029-04-28</ExerciseDate>
     </ExerciseDates>
  </OptionData>
  <Name>NYMEX:CL</Name>
  <Currency>USD</Currency>
  <Strike>100</Strike>
  <Quantity>500000</Quantity>
  <IsFuturePrice>true<IsFuturePrice>
  <FutureExpiryDate>2029-04-28<FutureExpiryDate>
</CommodityOptionData>
\end{minted}
\caption{Commodity Option data}
\label{lst:comoption_data}
\end{listing}

The meanings and allowable values of the elements in the \lstinline!CommodityOptionData!  node follow below.

\begin{itemize}

\item The \lstinline!CommodityOptionData! node contains an \lstinline!OptionData! node described in \ref{ss:option_data}. The relevant fields in the \lstinline!OptionData! node for a CommodityOption are:

\begin{itemize}

\item \lstinline!LongShort!: The allowable values are \emph{Long} or \emph{Short}.

\item \lstinline!OptionType!: The allowable values are \emph{Call} or \emph{Put}.

\item  \lstinline!Style!: The exercise style of the CommodityOption. The allowable values are \emph{European} or \emph{American}.

\item  \lstinline!PayOffAtExpiry!:  This must be set to \emph{false} as payoff at expiry is not currently supported.

\item An \lstinline!ExerciseDates! node where exactly one \lstinline!ExerciseDate! date element must be given for. Allowable values: See Date in Table \ref{tab:allow_stand_data}.

\item \lstinline!Premiums! [Optional]: Option premium node with amounts paid by the option buyer to the option seller.
Allowable values:  See section \ref{ss:premiums}

\end{itemize}





\item Name: The name of the underlying commodity. \\
Allowable values: See \lstinline!Name! for commodity trades in Table \ref{tab:commodity_data}.
\item Currency: The currency of the commodity option. \\
Allowable values: See \lstinline!Currency! in Table \ref{tab:allow_stand_data}.
\item Strike: The option strike price. It uses the price quotation outlined in the underlying contract specs for the commodity name in question.  \\
Allowable values: Any positive real number.
\item Quantity: The number of units of the underlying commodity covered by the transaction. The unit type is defined in the underlying contract specs for the commodity name in question. For avoidance of doubt, the Quantity is the number of units of the underlying commodity, not the number of contracts. \\
Allowable values: Any positive real number.

\item IsFuturePrice [Optional]: A boolean indicating if the underlying is a future contract settlement price, \lstinline!true!, or a spot price, \lstinline!false!.

Allowable values: A boolean value given in Table \ref{tab:boolean_allowable}. If not provided, the default value is \lstinline!true!.

\item FutureExpiryDate [Optional]: If \lstinline!IsFuturePrice! is \lstinline!true! and the underlying is a future contract settlement price, this node allows the user to specify the expiry date of the underlying future contract.

Allowable values: This should be a valid date as outlined in Table \ref{tab:allow_stand_data}. If not provided, it is assumed that the future contract's expiry date is equal to the option expiry date provided in the \lstinline!OptionData! node.
\end{itemize}

%\subsubsection{Precious Metal Option}
%A Precious Metal Option should be represented as a Commodity Option as above.
%Note that a Precious Metal Option should be represented as an FX
%Option using the appropriate commodity ``currency'' (XAU, XAG, XPT, XPD).

\subsubsection{Commodity Digital Option}
\label{ss:input_commodity_digital_option}

A commodity digital option is represented with trade type  \emph{CommodityDigitalOption} and a corresponding
\lstinline!CommodityDigitalOptionData! node.
The latter differs from the \lstinline!CommodityOptionData! node in section \ref{ss:input_commodity_option} by replacing tag \emph{Quantity}
with tag \emph{Payoff} which is the cash amount paid in the Currency of the option from the party that is short to the party that is long, when the underlying price exceeds the strike at expiry in case of a Call (or falls below the strike in case of a Put). The digital option is priced in ORE as a spread of vanilla Commodity options at two slightly different strikes. For option type \emph{Call}
and \emph{Put}, respectively, the digital call/put is constructed as
\begin{align*}
\mbox{Digital Call} =  \frac{\mbox{Payoff}}{\Delta}  \times  \left( \mbox{Call}(K- \Delta/2) - \mbox{Call}(K+ \Delta/2) \right) \\
\mbox{Digital Put} = \frac{\mbox{Payoff}}{\Delta}  \times \left( \mbox{Put}(K+ \Delta/2) - \mbox{Put}(K- \Delta/2)  \right)
\end{align*}
so that the long digital option has positive value in both cases. The strike spread $\Delta$ used here is set to 1\% of strike $K$.
