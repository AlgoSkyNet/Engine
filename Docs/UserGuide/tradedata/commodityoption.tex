\subsubsection{Commodity Option}

The \lstinline!CommodityOptionData!  node is the trade data container for the \emph{CommodityOption} trade type.  Vanilla commodity 
options are supported, the exercise style must be \emph{European}. The \lstinline!CommodityOptionData!  node includes one and 
only one \lstinline!OptionData! trade component sub-node plus elements specific to the commodity option. The structure of 
an example \lstinline!CommodityOptionData! node for a commodity option is shown in Listing
\ref{lst:comoption_data}.

\begin{listing}[H]
%\hrule\medskip
\begin{minted}[fontsize=\footnotesize]{xml}
<CommodityOptionData>
    <OptionData>
      ...
    </OptionData>
    <Name>NYMEX:CL</Name>
    <Currency>USD</Currency>
    <Strike>70</Strike>
    <Quantity>500000</Quantity>
</CommodityOptionData>
\end{minted}
\caption{Commodity Option data}
\label{lst:comoption_data}
\end{listing}

The meanings and allowable values of the elements in the \lstinline!CommodityOptionData!  node follow below.

\begin{itemize}
	\item OptionData: This is a trade component sub-node outlined in section \ref{ss:option_data} Option Data. Note 
	that the commodity option type allows for \emph{European} option style only.	
	\item Name: The name of the underlying commodity. \\
	Allowable values:  See \lstinline!Name! for commodity trades in Table \ref{tab:commodity_data}. \\
	\item Currency: The currency of the commodity option. \\
	Allowable values:  See \lstinline!Currency! in Table \ref{tab:allow_stand_data}.	
	\item Strike: The option strike price.\\
	Allowable values:  Any positive real number.	
	\item Quantity: The number of units of the underlying commodity covered by the transaction. \\
	Allowable values:  Any positive real number.
\end{itemize}

\subsubsection{Precious Metal Option}
A Precious Metal Option should be represented as a Commodity Option as above.
