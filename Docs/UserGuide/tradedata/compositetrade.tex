\subsubsection{Composite Trade}

The \lstinline!CompositeTradeData!  node is the trade data container for the  \emph{CompositeTrade} trade type.   A composite trade is a hybrid position consisting of multiple component trades. The structure of an example \lstinline!CompositeTradeData! node for a commodity option is shown in 
Listing \ref{lst:compositetrade_data}.

\begin{listing}[H]
	%\hrule\medskip
	\begin{minted}[fontsize=\footnotesize]{xml}
		<CompositeTradeData>
		  <Currency>USD</Currency>
		  <NotionalCalculation>Sum</NotionalCalculation>
		  <Components>
		    <Trade id="">
		      <!-- A valid trade xml -->
		    </Trade>
		    <Trade id="">
		      <!-- A valid trade xml -->
		    </Trade>
		  </Components>
		</CompositeTradeData>
	\end{minted}
	\caption{Composite trade data}
\end{listing}

\begin{listing}[H]
	%\hrule\medskip
	\begin{minted}[fontsize=\footnotesize]{xml}
		<CompositeTradeData>
		<Currency>USD</Currency>
		<NotionalCalculation>Sum</NotionalCalculation>
		<PortfolioBasket>true</PortfolioBasket>
		<BasketName>NAME</BasketName>
		</CompositeTradeData>
	\end{minted}
	\caption{Composite trade data with Reference Data}
	\label{lst:compositetrade_data_refdata}
\end{listing}

The meanings and allowable values of the elements in the \lstinline!CompositeTradeData!  node follow below.

\begin{itemize}
	\item Currency: Defines the currency the NPV of the composite trade will be represented in. \\
	  Allowable values:  See Table \ref{tab:currency} \lstinline!Currency!.
	\item NotionalCalculation [Optional]: The method by which the notional of the composite trade will be calculated. \\
	  Allowable values:
	  \begin{itemize}
	  	\item[] \emph{Sum}: The notional will be calculated as the sum of the notionals of the constituent trades. This is the default behaviour if the field is omitted (unless an override is provided).
	  	\item[] \emph{Mean} or \emph{Average}: The notional will be calculated as the mean of the notionals of the constituent trades.
	  	\item[] \emph{First}: The notional of the first constituent trade will be used.
	  	\item[] \emph{Last}: The notional of the first constituent trade will be used.
	  	\item[] \emph{Min}: The notional will be calculated as the minimum of the notionals of the constituent trades.
	  	\item[] \emph{Max}: The notional will be calculated as the minimum of the notionals of the constituent trades.
	  	\item[] \emph{Override}: the notional will be read directly from the notional override field.
	  \end{itemize}
	\item NotionalOverride [Optional]: The notional which will be used for the trade, overriding any calculation method specified. \\
      Allowable values: Any non-negative real number.
	\item Components: The portfolio of trades that make up the composite trade. \\
      Allowable values: These trades should be valid xmls that could otherwise be entered into the portfolio, with the exception that they can have empty ids.
    \item PortfolioBasket [Optional]: Indicate if the Component represent a portfolio basket. \\
    Allowable values: Boolean true or false.
    \item BasketName [Optional]: The portfolio Id. \\
    Allowable values: Any string. Note that if PortfolioBasket is True then there must be a BasketName. We look up the Basket within the reference data.
\end{itemize}