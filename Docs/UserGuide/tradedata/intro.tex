%========================================================
\section{Trade Data}\label{sec:portfolio_data}
%========================================================

The trades that make up the portfolio are specified in an XML file where the portfolio data is specified in a hierarchy
of nodes and sub-nodes.  The nodes containing individual trade data are referred to as elements or XML elements. These
are generally the lowest level nodes.

\medskip

The top level portfolio node is delimited by an opening {\tt <Portfolio>} and a closing {\tt </Portfolio>} tag. Within
the portfolio node, each trade is defined by a starting {\tt <Trade id="[Tradeid]">} and a closing {\tt </Trade>} tag.
Further, the trade type is set by the TradeType XML element. Each trade has an Envelope node that includes the same XML
elements for all trade types (Id, Type, Counterparty, Rating, NettingSetId) plus the Additional fields node, and after
that, a node containing trade specific data.

\vspace{1em}
An example of a {\tt portfolio.xml} file with one Swap trade including the full envelope node is shown in Listing \ref{lst:portfolio}.

\begin{listing}[H]
%\hrule\medskip
\begin{minted}[fontsize=\footnotesize]{xml}
<Portfolio>
  <Trade id="Swap#1">
    <TradeType> Swap </TradeType>
    <Envelope>
      <CounterParty> Counterparty#1 </CounterParty>
      <NettingSetId> NettingSet#2 </NettingSetId>
      <PortfolioIds>
          <PortoliodId> PF#1 </PortfolioId>
          <PortoliodId> PF#2 </PortfolioId>
      </PortfolioIds>
      <AdditionalFields>
        <Sector> SectorA </Sector>
        <Book> BookB </Book>
        <Rating> A1 </Rating>
      </AdditionalFields>
    </Envelope>
    <SwapData>
        ...
        [Trade specific data for a Swap]
        ...
    </SwapData>
  </Trade>
</Portfolio>
\end{minted}
\caption{Portfolio}
\label{lst:portfolio}
\end{listing}

A description of all portfolio data, i.e. of each node and XML element in the portfolio file, with examples and
allowable values follows below. There is only one XML elements directly under the top level {\tt Portfolio} node:

\begin{itemize}
%\item {\tt Trade id}: The first element of each trade is the {\tt Trade id} and it is used to identify trades within a
%  portfolio. Trade ids should be unique within a portfolio.  The {\tt
%   Trade id} element is entered as an attribute to the XML element {\tt <Trade>}, such as {\tt <Trade id="ExampleTrade">}
%  in the beginning closed by {\tt </Trade>} at the end of the trade data.
%
%Allowable values:  Any alphanumeric string. The underscore (\_) sign may be used as well. 
\item {\tt TradeType}: %The Trade type is set with the {\tt TradeType} element, as well as with the {\tt Type} element in  the envelope. The two Trade type entries should be identical. 
ORE currently supports 14 trade types.

%Allowable values:  \emph{Swap, CapFloor, Swaption, FxForward, FxOption, Bond, CreditDefaultSwap}
Allowable values: \emph{ForwardRateAgreement, Swap, CapFloor, Swaption, FxForward, FxSwap, FxOption,
EquityForward, EquityOption, VarianceSwap, CommodityForward, CommodityOption, CreditDefaultSwap, Bond}

\end{itemize}
