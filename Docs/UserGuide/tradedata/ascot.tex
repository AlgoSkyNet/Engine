\subsubsection{Ascot}
\label{ss:ascot}

An Ascot is set up using an {\tt AscotData} block as shown in listing
\ref{lst:ascotdata}. The bond details are read from reference data in this case.

An Ascot or a Convertible Bond Option is an American style  option to buy back a convertible bond. The buyer of a Call Ascot can exercise the deal and get the underlying bond in
exchange for paying the strike.

The payout formula for a Call Ascot is:

$$
Payout = \max(0, convertiblePrice - Strike)
$$

And for a Put Ascot:

$$
Payout = \max(0, Strike - convertiblePrice)
$$

where: 
$$
Strike = bondQuantity \cdot (upfrontPayment + assetLeg - redemptionLeg) - fundingLeg
$$


\begin{listing}[H]
\begin{minted}[fontsize=\footnotesize]{xml}
  <Trade id="Ascot">
    <TradeType>Ascot</TradeType>
    <Envelope>...</Envelope>
    <AscotData>
      <ConvertibleBondData>
        <BondData>
          <SecurityId>ISIN:XY1000000000</SecurityId>
          <BondNotional>1000000.00</BondNotional>
        </BondData>
      </ConvertibleBondData>
      <OptionData>
       <LongShort>Long</LongShort>
       <OptionType>Call</OptionType>
       <Style>American</Style>
       <Settlement>Physical</Settlement>
       <ExerciseDates>
         <ExerciseDate>2029-02-03</ExerciseDate>
       </ExerciseDates>  
      </OptionData>
      <ReferenceSwapData>
        <LegData>
          <LegType>Floating</LegType>
          <Payer>false</Payer>
          ...
        </LegData>
      </ReferenceSwapData>
    <AscotData>
  </Trade>
\end{minted}
\caption{Ascot set up using reference data}
\label{lst:ascotdata}
\end{listing}

The meanings and allowable values of the elements in the block are as follows:

\begin{itemize}
  \item ConvertibleBondData: This describes the underlying convertible bond, see \ref{ss:convertible_bond}. 
  \item OptionData: This is a trade component sub-node outlined in section \ref{ss:option_data} Option Data. The relevant fields in the \lstinline!OptionData! node for an Ascot are: 

\begin{itemize}

\item \lstinline!LongShort! The allowable values are \emph{Long} or \emph{Short}. The LongShort flag multiplies the option price with +1 / -1. Call and Put payout formulas above are from the long perspective 

\item \lstinline!OptionType! The allowable values are \emph{Call} or \emph{Put}. See payout formulas above.

\item  \lstinline!Style! The Ascot type allows for \emph{American} option exercise style only.

\item  \lstinline!Settlement! The allowable values are \emph{Cash} or \emph{Physical}.

\item An \lstinline!ExerciseDates! node where exactly one \lstinline!ExerciseDate! date element must be given.

\item \lstinline!Premiums! [Optional]: Option premium amounts paid by the option buyer to the option seller.

Allowable values:  See section \ref{ss:premiums}

\end{itemize}	
	
	
	
  \item ReferenceSwapData: Contains a single  \lstinline!LegData! node that describes the trade's reference swap funding leg. The asset leg is implied from the bond data. Payer should always be \emph{false}  i.e. the swap is entered from the viewpoint of the asset swap buyer.
\end{itemize}

