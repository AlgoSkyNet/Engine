\subsubsection{Swaption}
\label{ss:swaption} 

The \lstinline!SwaptionData!  node is the trade data container for the \emph{Swaption} trade type. The \lstinline!SwaptionData!
node has one and exactly one \lstinline!OptionData! trade component sub-node, and at least one \lstinline!LegData! trade
component sub-node.  These trade components are outlined in section \ref{ss:option_data} and section
\ref{ss:leg_data}.\\
\vspace{5mm}
Supported swaption exercise styles are \emph{European} and \emph{Bermudan}. Swaptions of both exercise styles can have an arbitrary number of legs, with
each leg represented by a \lstinline!LegData! sub-node.  Cross currency swaptions are not supported for either exercise style, i.e. the Currency element must
have the same value for all \lstinline!LegData! sub-nodes of a swaption. There must be at least one full coupon period after the exercise date for European 
Swaptions, and after the last exercise date for Bermudan Swaptions. See Table \ref{tab:swaption_requirements} for further details on requirements for
 swaptions.\\
\vspace{5mm}
The structure of an example \lstinline!SwaptionData!  node of a European swaption is shown in Listing
\ref{lst:swaption_data}.

\begin{listing}[H]
%\hrule\medskip
\begin{minted}[fontsize=\footnotesize]{xml}
<SwaptionData>
    <OptionData>
        <LongShort>Long</LongShort>
        <Style>European</Style>
        <Settlement>Physical</Settlement>
        <ExerciseDates>
          <ExerciseDate>2027-03-02</ExerciseDate>
        </ExerciseDates>
        ...
        <Premiums>
          <Premium>
            <Amount>807000</Amount>
            <Currency>GBP</Currency>
            <PayDate>2021-06-15</PayDate>
          </Premium>
        </Premiums>
    </OptionData>
    <LegData>
        <LegType>Fixed</LegType>
        <Payer>false</Payer>    
        <Currency>GBP</Currency>
        ...
    </LegData>
    <LegData>
        <LegType>Floating</LegType>
        <Payer>true</Payer>     
        <Currency>GBP</Currency>
        ...
    </LegData>
</SwaptionData>
\end{minted}
\caption{Swaption data}
\label{lst:swaption_data}
\end{listing}

\begin{table}[H]
\centering
\begin{tabular} {|l|p{10cm}|}
    \hline
        & \bfseries{A  Swaption requires:} \\  \hline
    \lstinline!OptionData! & One \lstinline!OptionData! sub-node  \\  \hline
    \lstinline!Style! &  \emph{Bermudan} or \emph{European}\\ \hline
    \lstinline!ExerciseDates! & \emph{European} swaptions can only have one \lstinline!ExerciseDate! child element. \\ \hline
    \lstinline!LegData! &  At least one \lstinline!LegData! sub-node \\ \hline
    \lstinline!Currency! & The same currency for all \lstinline!LegData! sub-nodes.\\ \hline
    \lstinline!LegType! & Allowed types are \emph{Cashflow}, \emph{Fixed} or \emph{Floating}. Floating coupons can be (capped / floored) Ibor, (capped / floored) compounded or averaged OIS, or BMA/SIFMA. Standalone options (nakedOption = true) are not allowed, neither are local OIS cap/floors. \\ \hline
  \end{tabular}
  \caption{Requirements for Swaptions}
  \label{tab:swaption_requirements}
\end{table}

The \lstinline!OptionData! trade component sub-node is outlined in section \ref{ss:option_data}. 
The relevant fields in the \lstinline!OptionData! node for a Swaption are:

\begin{itemize}
\item \lstinline!LongShort!: The allowable values are \emph{Long} or \emph{Short}. Note that the payer and receiver legs in the underlying swap are always from the perspective of the party that is \emph{Long}. E.g. for a \emph{Short} swaption with a fixed leg where the Payer flag is set to \emph{false}, it means that the counterparty receives the fixed flows.  

\begin{table}[H]
\centering
\begin{tabular} {| c | c | c | c |}    \hline
        \lstinline!LongShort! & \makecell{\lstinline!Payer! for Fixed leg \\ on underlying Swap} & \makecell{\lstinline!Payer! for Floating leg \\ on underlying Swap} & Resulting Set Up and Flows \\  \hline
   \emph{Long} & \emph{true} & \emph{false} & \makecell[l]{The Party to the trade buys an \\ option to enter a swap where the \\ Party pays fixed and receives \\ floating}  \\  \hline
    \emph{Short} & \emph{true} & \emph{false} & \makecell[l]{The Party to the trade sells an \\ option to the Counterparty to enter \\ a swap where the Counterparty \\ pays fixed and receives floating}  \\  \hline
    \emph{Long} & \emph{false} & \emph{true} & \makecell[l]{The Party to the trade buys an \\ option to enter a swap where the \\ Party receives fixed and pays \\ floating}  \\  \hline
    \emph{Short} & \emph{false} & \emph{true} & \makecell[l]{The Party to the trade sells an \\ option to the Counterparty to enter \\ a swap where the Counterparty \\ receives fixed and pays floating}  \\  \hline        
  \end{tabular}
  \caption{Swaption set up and resulting flows}
  \label{tab:swaption_setup}
\end{table}




\item \lstinline!OptionType![Optional]: This flag is optional for swaptions, and even if set, has no impact. Whether a swaption is a payer or receiver swaption is determined by the Payer flags on the legs of the underlying swap.

\item  \lstinline!Style!: The exercise style of the Swaption. The allowable values are \emph{European},  \emph{Bermudan} or  \emph{American}.  Note that  \emph{American} exercise style isn't supported, and if set to  \emph{American}, it will behave as  \emph{Bermudan} exercise style.

\item \lstinline!NoticePeriod![Optional]: The notice period defining the date (relative to the exercise date) on which the exercise
  decision has to be taken. If not given the notice period defaults to \emph{0D}, i.e. the notice date is identical to the
  exercise date. Allowable values: A number followed by \emph{D, W, M, or Y}

\item \lstinline!NoticeCalendar![Optional]: The calendar used to compute the notice date from the exercise date. If not given
  defaults to the \emph{NullCalendar} (no holidays, weekends are no holidays either). Allowable values: See Table \ref{tab:calendar} \lstinline!Calendar!.

\item \lstinline!NoticeConvention![Optional]: The roll convention used to compute the notice date from the exercise date. Defaults to
  \emph{Unadjusted} if not given. Allowable values: See Table \ref{tab:convention} Roll Convention.

\item  \lstinline!Settlement!: Delivery Type. The allowable values are \emph{Cash} or \emph{Physical}.

\item \lstinline!SettlementMethod![Optional]: Specifies the method to calculate the settlement amount for Swaptions and CallableSwaps. Allowable values: \emph{PhysicalOTC}, \emph{PhysicalCleared}, \emph{CollateralizedCashPrice}, \emph{ParYieldCurve}. Defaults to \emph{ParYieldCurve} if Settlement is \emph{Cash} and defaults to \emph{PhysicalOTC} if Settlement is \emph{Physical}.

\emph{PhysicalOTC} = OTC traded swaptions with physical settlement\\
\emph{PhysicalCleared} = Cleared swaptions with physical settlement\\
\emph{CollateralizedCashPrice} = Cash settled swaptions with settlement price calculation using zero coupon curve discounting \\
\emph{ParYieldCurve}  = Cash settled swaptions with settlement price calculation using par yield discounting \footnote{https://www.isda.org/book/2006-isda-definitions/} \footnote{https://www.isda.org/a/TlAEE/Supplement-No-58-to-ISDA-2006-Definitions.pdf} \\

\item \lstinline!ExerciseFees![Optional]: This node contains child elements of type \lstinline!ExerciseFee!. Similar to a list of notionals
  (see \ref{ss:leg_data}) the fees can be given either

  \begin{itemize}
  \item as a list where each entry corresponds to an exercise date and the last entry is used for all remaining exercise
    dates if there are more exercise dates than exercise fee entries, or
  \item using the \verb+startDate+ attribute to specify a change in a fee from a certain day on (w.r.t. the exercise
    date schedule)
  \end{itemize}

  Fees can either be given as an absolute amount or relative to the current notional of the period immediately following
  the exercise date using the \verb+type+ attribute together with specifiers \verb+Absolute+ resp. \verb+Percentage+. If
  not given, the type defaults to \verb+Absolute+. \verb+Percentage+ fees are expressed in decimal form, e.g. 0.05 is a fee of 5\% of notional.

  If a fee is given as a positive number the option holder has to pay a corresponding amount if they exercise the
  option. If the fee is negative on the other hand, the option holder receives an amount on the option exercise.

  Only supported for Swaptions and Callable Swaps currently.

\item \lstinline!ExerciseFeeSettlementPeriod![Optional]: The settlement lag for exercise fee payments. Defaults to 0D if not
  given. This lag is relative to the exercise date (as opposed to the notice date). Allowable values: A number followed by \emph{D, W, M, or Y}

\item \lstinline!ExerciseFeeSettlementCalendar![Optional]: The calendar used to compute the exercise fee settlement date from the
  exercise date. If not given defaults to the \emph{NullCalendar} (no holidays, weekends are no holidays either). Allowable values: See Table \ref{tab:calendar} Calendar.

\item \lstinline!ExerciseFeeSettlementConvention![Optional]: The roll convention used to compute the exercise fee settlement date from
  the exercise date. Defaults to \emph{Unadjusted} if not given. Allowable values: See Table \ref{tab:convention} Roll Convention.

\item An \lstinline!ExerciseDates! node where exactly one \lstinline!ExerciseDate! date element must be given for \emph{European} style swaptions, and for \emph{Bermudan} style swaptions  at least two \lstinline!ExerciseDate! date elements must be given. \\

\item \lstinline!Premiums! [Optional]: Option premium node with amounts paid by the option buyer to the option seller.

Allowable values:  See section \ref{ss:premiums}

\item \lstinline!ExerciseData! [Optional]: Marks the swaption as exercised. The effective date is the next exercise date
  greater or equal the given date in ExerciseData. For a cash-settled swaption, the price given in ExerciseData
  represents the cash settlement amount. It is paid according to the PaymentData: If an explicit list of dates is given,
  the payment takes place on the next date following the effective exercise date. If the payment data is rules-based,
  the payment date is derived from the effective exercise date using the given calendar, lag and convention (only
  ``Exercise'' is allowed in the RelativeTo field).

\end{itemize}



