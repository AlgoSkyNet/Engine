\subsubsection{Knock Out Swap}

A Knock Out Swap is represented using the TradeType \emph{KnockOutSwap} and a KnockOutSwapData block as shown in listing
\ref{lst:knock_out_swap}. It must have one \lstinline!BarrierData! node, and two legs, one fixed and one floating, each represented by a \lstinline!LegData! trade component. 

A Knock Out Swap is a Swap with one Fixed and one Floating leg, where the Swap is terminated if the Floating leg Index hits a barrier. The barrier is monitored on all floating leg fixing dates after the BarrierStartDate.

The meanings and allowable values in this block are as follows:

\begin{itemize}
\item BarrierData: This node specifies the barrier Type and Level. The Type must be \emph{UpAndOut} or
  \emph{DownAndOut}. Exactly one Level must be specified. All other values of the barrier data block are not relevant. \\
  Allowable values: See \ref{ss:barrier_data}.

\item BarrierStartDate: The barrier is monitored on all floating leg fixing dates that are on or after the barrier start date. \\
  Allowable values: See \lstinline!Date! in Table \ref{tab:allow_stand_data}.

\item LegData: This specifies the swap terms. Exactly two LegData nodes must be given, one of type Fixed and one of type Floating. \\
    Allowable values: See \ref{ss:leg_data}.
\end{itemize}

\begin{listing}[hbt]
\begin{minted}[fontsize=\footnotesize]{xml}
<Trade id="194837232">
  <TradeType>KnockOutSwap</TradeType>
  <Envelope>...</Envelope>
  <KnockOutSwapData>
    <!-- BarrierData and BarrierStartDate specify the knock out terms -->
    <BarrierData>
      <Type>UpAndOut</Type>
      <Levels>
        <Level>0.05</Level>
      </Levels>
    </BarrierData>
    <BarrierStartDate>2024-10-01</BarrierStartDate>
    <!-- we require exactly one Floating and one Fixed Leg -->
    <LegData>
      <LegType>Floating</LegType>
      ...
    </LegData>
    <LegData>
      <LegType>Fixed</LegType>
      ...
    </LegData>
  </KnockOutSwapData>
</Trade>
\end{minted}
\caption{Knock Out Swap}
\label{lst:knock_out_swap}
\end{listing}