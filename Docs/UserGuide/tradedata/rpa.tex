\subsubsection{Risk Participation Agreement (RPA)}
\label{input:cr_rpa}

A risk participation agreement is set up using the trade type {\tt RiskParticipationAgreement} and a {\tt
  RiskParticipationAgreementData} block as shown in listing \ref{lst:rpadata}. The block contains a {\tt ProtectionFee}
block that can include one or more legs representing the fees paid by the protection buyer and an {\tt Underlying} block
containing either the legs of the underlying swap or the Treasury-Lock data that the contract references.

If the underlying reference entity defaults, the protection buyer receives the PV of the underlying if this is
positive. Here, the underlying PV is computed using the payer / receiver flags as set up for the legs under the
underlying node. Whether the trade represents a protection buyer or seller position is indicated by the payer flag in
the protection fee leg data: If true protection is bought (and the protection fee is paid), if false the protection is
sold (and the protection fee is received).

\begin{listing}[H]
\begin{minted}[fontsize=\footnotesize]{xml}
  <RiskParticipationAgreementData>
    <ParticipationRate>0.8</ParticipationRate>
    <ProtectionStart>2018-10-01</ProtectionStart>
    <ProtectionEnd>2038-10-01</ProtectionEnd>
    <CreditCurveId>RED:008CA0|SNRFOR|USD|MR14</CreditCurveId>
    <IssuerId>CompanyXZY</IssuerId>
    <SettlesAccrual>true</SettlesAccrual>
    <FixedRecoveryRate>0.6</FixedRecoveryRate>
    <ProtectionFee>
      <LegData>
        <LegType>Cashflow</LegType>
        <Payer>true</Payer>
        <Currency>EUR</Currency>
        <CashflowData>
          <Cashflow>
            <Amount date="2018-10-03">91171.72</Amount>
          </Cashflow>
        </CashflowData>
      </LegData>
    </ProtectionFee>
    <Underlying>
      <!-- Alternatives:
           - Sequence of LegData, possibly with OptionData to represent callability
           - A single block of TreasuryLockData -->
      <OptionData> ... </OptionData>
      <NakedOption> ... </NakedOption>
      <LegData>
        <LegType>Floating</LegType>
        ...
      </LegData>
      <LegData>
        <LegType>Fixed</LegType>
        <Payer>false</Payer>
        ...
      </LegData>
    </Underlying>
  </RiskParticipationAgreementData>
\end{minted}
\caption{Risk Participation Agreement Data}
\label{lst:rpadata}
\end{listing}

\begin{itemize}
\item ParticipationRate: The rate reflecting the participation amount relative to the swap volume.

Allowable values: Any number between $0$ and $1$.

\item ProtectionStart: The date on which the protection starts (inclusive).

Allowable values: Any valid date, see See \lstinline!Date! in Table \ref{tab:allow_stand_data}.

\item ProtectionEnd: The date on which the protection ends (exclusive).

Allowable values. Any valid date greater than the protection start date.

\item CreditCurveId: Typically the RED-code of the underlying swap reference entity defining the default curve used for
  pricing. Other identifiers may be used as well, provided they are supported in the market data configuration.

Allowable values: Any valid credit curve identifier.

\item IssuerId [Optional]: An identifier for the underlying swap reference entity. For informational purposes and not
  used for pricing. Defaults to an empty string.

Allowable values: Any string.

\item SettlesAccrual [Optional]: Whether or not the accrued coupon of the protection fee is due in the event of a
  default. This defaults to \lstinline!true! if not provided. Only applies to coupon legs (i.e. not simple cashflows)
  within the protection fee block, otherwise it is ignored.

Allowable values: {\tt true} or {\tt false}

\item FixedRecoveryRate [Optional]: This node holds the fixed recovery rate if the RPA assumes a fixed recovery to
  calculate the settlement amount in case of a default event. If the field is omitted the recovery rate associated to
  the credit curve is used instead.

Allowable values: Any number between $0$ and $1$.

\item ProtectionFee: The fees that are paid (if protection is bought) or received (if protection is sold). The fees are
  given by one or more legs as described under \ref{ss:leg_data} with identical Payer flags, typically this will be a
  single {\tt Cashflow} leg holding zero or more fixed fee amounts or a {\tt Fixed} leg representing a series of periodic
  fee payments. Fees are paid up to (but excluding) the default event. If the fees are given as coupons the accrued
  amount between the accrual start date and the default date is paid if and only if {\tt SettlesAccrual} is set to {\tt
    true}. The protection fees can be given in any arbitrary currency.

\item Underlying: The reference underlying. There are several subtypes to distinguish, all of which have separate
  pricing engines attached. There is no need to specify the subtype in the trade xml, this is deduced automatically
  during the trade building:
  \begin{itemize}
  \item Vanilla Swap: This is a vanilla swap given by two legs in the same currency, one receiver, one payer and one
    Fixed (or Cashflow), one Floating. For the floating part only Ibor coupons (no averaging) or (compounded, averaging)
    OIS coupons are allowed. Spreads and gearings are allowed, but no embedded caps/floors, no in arrears fixings for
    Ibor coupons. This type allows an analytic Black engine where the RPA Options are found via a representative
    swaption matching.
  \item Structured Swap: As vanilla, but an arbitrary number of legs of type Fixed, Floating, Cashflow is
    allowed. Embedded caps/floors/collars and in arrears fixing are allowed. For floating legs, Ibor (no averaging) and
    OIS (compounded, averaging) coupons are allowed. All legs must be in the same currency. Standalone caps, floors,
    collars are allowed as an underlying of the RPA, if specified by a floating leg with NakedOption set to true. See
    \ref{ss:floatingleg_data} for details on the floating leg specification, amd likewise \ref{ss:fixedleg_data} for the
    fixed leg and \ref{ss:leg_data} for the cashflow leg. This type requires a numeric grid engine.
  \item Callable Swap / Swaption: As structured swap, but an additional OptionData block allows to specify callability
    of the swap. The relevant fields in OptionData are the same as for callable swaps, see \ref{ss:callable_swap}. This
    type requires a numeric grid engine as the structured swap. If NakedOption is set to true, an option to exercise
    into the underlying swap is represented, i.e. a swaption.
  \item Cross Currency Swap: Underlying legs as in structured swap, but the legs can be in two different currencies. No
    optionality is allowed though. At most two different currencies are allowed. This type can be priced using an
    analytic Black engine which models the FX Risk and assumes deterministic interest rates.
  \item T-Lock. The underlying is a T-Lock, represented as shown in listing \ref{lst:tlock_data} and explained in more
    detail below. This type requires a numeric grid engine.
  \end{itemize}
\end{itemize}

\underline{Treasury Lock Underlying Specification}

Listing \ref{lst:tlock_data} shows the specification of a T-Lock underlying. The fields have the following meaning:

\begin{itemize}
\item Payer: Boolean, true if the fixed reference rate is paid, false otherwise. I.e. if the payer flag is true and the
  yield is lower than the reference rate, then the underlying T-Lock trade pays the amount $(r-y) \cdot d$ where $r$ is
  the reference rate, $y$ is the yield, both expressed in basis points, and $d>0$ is the (absolute) price change of the
  treasury bond when the yield moves by $1$ basis point. Likewise, if the yield is higher than the reference rate, the
  underlying T-Lock trade receives $(y-r) \cdot d$. \\
  Allowable values: \emph{true} or \emph{false}

\item BondData: Reference to the underlying security, given in in the BondData sub node, minimum required data are notional and security ID \\
  Allowable values: See \ref{ss:bond}

\item ReferenceRate: Fixed rate paid or received on the T-Lock underlying \\
  Allowable values: Any real number. The rate is expressed in decimal form, eg 0.05 is a rate of 5\%

\item DayCounter [Optional]: Reference rate day counter. Optional, defaults to the coupon day counter of the underlying bond. \\
  Allowable values: See Table \ref{tab:daycount}

\item TerminationDate: Date for the cash settlement amount calculation \\
  Allowable values: See \lstinline!Date! in Table \ref{tab:allow_stand_data}.

\item PaymentGap [Optional]: Business day gap between termination and payment date. Optional, defaults to zero. \\
  Allowable values: Any non-negative integer

\item PaymentCalendar: Calendar to determine the payment date. \\
  Allowable values: See Table \ref{tab:calendar}.
\end{itemize}

\begin{listing}[H]
  \begin{minted}[fontsize=\footnotesize]{xml}
    <Underlying>
      <TreasuryLockData>
      <Payer>true</Payer>
      <BondData>
      ...
      </BondData>
      <ReferenceRate>0.05</ReferenceRate>
      <DayCounter>A360</DayCounter>
      <TerminationDate>2022-01-05</TerminationDate>
      <PaymentGap>5</PaymentGap>
      <PaymentCalendar>US</PaymentCalendar>
      </TreasuryLockData>
    </Underlying>
\end{minted}
\caption{Treasury-Lock Data}
\label{lst:tlock_data}
\end{listing}
