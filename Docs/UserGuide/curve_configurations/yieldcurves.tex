\subsubsection{Yield Curves}

The top level XML elements for each \lstinline!YieldCurve! node are shown in Listing \ref{lst:top_level_yc}.

\begin{listing}[H]
%\hrule\medskip
\begin{minted}[fontsize=\footnotesize]{xml}
<YieldCurve>
  <CurveId> </CurveId>
  <CurveDescription> </CurveDescription>
  <Currency> </Currency>
  <DiscountCurve> </DiscountCurve>
  <Segments> </Segments>
  <InterpolationVariable> </InterpolationVariable>
  <InterpolationMethod> </InterpolationMethod>
  <ZeroDayCounter> </ZeroDayCounter>
  <Tolerance> </Tolerance>
  <Extrapolation> </Extrapolation>
  <BootstrapConfig>
    ...
  </BootstrapConfig>
</YieldCurve>
\end{minted}
\caption{Top level yield curve node}
\label{lst:top_level_yc}
\end{listing}

The meaning of each of the top level elements in Listing \ref{lst:top_level_yc} is given below. If an element is labelled 
as 'Optional', then it may be excluded or included and left blank.
\begin{itemize}
\item CurveId: Unique identifier for the yield curve.
\item CurveDescription: A description of the yield curve. This field may be left blank.
\item Currency: The yield curve currency.
\item DiscountCurve: If the yield curve is being bootstrapped from market instruments, this gives the CurveId of the
yield curve used to discount cash flows during the bootstrap procedure. If this field is left blank or set equal to the
current CurveId, then this yield curve itself is used to discount cash flows during the bootstrap procedure.
\item Segments: This element contains child elements and is described in the following subsection.
\item InterpolationVariable [Optional]: The variable on which the interpolation is performed. The allowable values are
given in Table \ref{tab:allow_interp_variables}. If the element is omitted or left blank, then it defaults to
\emph{Discount}.
\item InterpolationMethod [Optional]: The interpolation method to use. The allowable values are given in Table
\ref{tab:allow_interp_methods}. If the element is omitted or left blank, then it defaults to \emph{LogLinear}.
\item ZeroDayCounter [Optional]: The day count basis used internally by the yield curve to calculate the time between
dates. In particular, if the curve is queried for a zero rate without specifying the day count basis, the zero rate that
is returned has this basis. If the element is omitted or left blank, then it defaults to \emph{A365}.

\item \lstinline!Tolerance! [Optional]: The tolerance used by the root finding procedure in the bootstrapping algorithm. If the
element is omitted or left blank, then it defaults to \num[scientific-notation=true]{1.0e-12}. It is preferable to use the 
\lstinline!Accuracy! node in the \lstinline!BootstrapConfig! node below for specifying this value. However, if this node is 
explicitly supplied, it takes precedence for backwards compatibility purposes.

\item Extrapolation [Optional]: Set to \emph{True} or \emph{False} to enable or disable extrapolation respectively. If
the element is omitted or left blank, then it defaults to \emph{True}.

\item \lstinline!BootstrapConfig! [Optional]: this node holds configuration details for the iterative bootstrap 
that are described in section \ref{sec:bootstrap_config}. If omitted, this node's default values described 
in section \ref{sec:bootstrap_config} are used.

\end{itemize}

\begin{table}[h]
\centering
  \begin{tabu} to 0.9\linewidth {| X[-1.5,l,m] | X[-5,l,m] |}
    \hline
    \bfseries{Variable} & \bfseries{Description} \\
    \hline
    Zero & The continuously compounded zero rate \\ \hline
    Discount & The discount factor \\ \hline
    Forward & The instantaneous forward rate \\ \hline
  \end{tabu}
  \caption{Allowable interpolation variables.}
  \label{tab:allow_interp_variables}
\end{table}

\begin{table}[h]
\centering
  \begin{tabu} to 0.9\linewidth {| X[-1.5,l,m] | X[-5,l,m] |}
    \hline
    \bfseries{Method} & \bfseries{Description} \\
    \hline
    Linear & Linear interpolation \\ \hline
    LogLinear & Linear interpolation on the natural log of the interpolation variable \\ \hline
    NaturalCubic & Monotonic Kruger cubic interpolation with second derivative at left and right \\ \hline
    FinancialCubic & Monotonic Kruger cubic interpolation with second derivative at left and 
                     first derivative at right \\ \hline
    ConvexMonotone & Convex Monotone Interpolation (Hagan, West) \\ \hline
    ExponentialSplines & Exponential Spline curve fitting, for Fitted Bond Curves only \\ \hline
    NelsonSiegel & Nelson-Siegel curve fitting, for Fitted Bond Curves only \\ \hline
    Svensson & Svensson curve fitting, for Fitted Bond Curves only \\ \hline
  \end{tabu}
  \caption{Allowable interpolation methods.}
  \label{tab:allow_interp_methods}
\end{table}
%- - - - - - - - - - - - - - - - - - - - - - - - - - - - - - - - - - - - - - - -
\subsubsection*{Segments Node} \label{ss:segments_node}
%- - - - - - - - - - - - - - - - - - - - - - - - - - - - - - - - - - - - - - - -
The \lstinline!Segments! node gives the zero rates, discount factors and instruments that comprise the yield curve. This
node consists of a number of child nodes where the node name depends on the segment being described. Each node has a
\lstinline!Type! that determines its structure. The following sections describe the type of child nodes that are
available. Note that for all segment types below, with the exception of \lstinline!DiscountRatio! and \lstinline!AverageOIS!, the 
\lstinline!Quote! elements within the \lstinline!Quotes! node may have an \lstinline!optional! attribute indicating whether or
not the quote is optional. Example:
%\hrule\medskip
\begin{minted}[fontsize=\footnotesize]{xml}
<Quotes>
  <Quote optional="true"></Quote>
</Quotes>
\end{minted}
%\hrule

\subsubsection*{Direct Segment}
When the node name is \lstinline!Direct!, the \lstinline!Type! node has the value \emph{Zero} or \emph{Discount} and the
node has the structure shown in Listing \ref{lst:direct_segment}. We refer to this segment here as a direct segment
because the discount factors, or equivalently the zero rates, are given explicitly and do not need to be
bootstrapped. The \lstinline!Quotes! node contains a list of \lstinline!Quote! elements. Each \lstinline!Quote! element
contains an ID pointing to a line in the {\tt market.txt} file, i.e.\ in this case, pointing to a particular zero rate
or discount factor. The \lstinline!Conventions! node contains the ID of a node in the {\tt conventions.xml} file
described in section \ref{sec:conventions}. The \lstinline!Conventions! node associates conventions with the quotes.

\begin{listing}[H]
%\hrule\medskip
\begin{minted}[fontsize=\footnotesize]{xml}
<Direct>
  <Type> </Type>
  <Quotes>
    <Quote> </Quote>
    <Quote> </Quote>
     <!--...-->
  </Quotes>
  <Conventions> </Conventions>
</Direct>
\end{minted}
\caption{Direct yield curve segment}
\label{lst:direct_segment}
\end{listing}


\subsubsection*{Simple Segment}
When the node name is \lstinline!Simple!, the \lstinline!Type! node has the value \emph{Deposit}, \emph{FRA},
\emph{Future}, \emph{OIS} or \emph{Swap} and the node has the structure shown in Listing \ref{lst:simple_segment}. This
segment holds quotes for a set of deposit, FRA, Future, OIS or swap instruments corresponding to the value in the
\lstinline!Type! node. These quotes will be used by the bootstrap algorithm to imply a discount factor, or equivalently
a zero rate, curve. The only difference between this segment and the direct segment is that there is a
\lstinline!ProjectionCurve! node. This node allows us to specify the CurveId of another curve to project floating rates
on the instruments underlying the quotes listed in the \lstinline!Quote! nodes during the bootstrap procedure. This is
an optional node. If it is left blank or omitted, then the projection curve is assumed to equal the curve being
bootstrapped i.e.\ the current CurveId.

\begin{listing}[H]
%\hrule\medskip
\begin{minted}[fontsize=\footnotesize]{xml}
<Simple>
  <Type> </Type>
  <Quotes>
    <Quote> </Quote>
    <Quote> </Quote>
    <!--...-->
  </Quotes>
  <Conventions> </Conventions>
  <ProjectionCurve> </ProjectionCurve>
</Simple>
\end{minted}
\caption{Simple yield curve segment}
\label{lst:simple_segment}
\end{listing}

\subsubsection*{Average OIS Segment}
When the node name is \lstinline!AverageOIS!, the \lstinline!Type! node has the value \emph{Average OIS} and the node
has the structure shown in Listing \ref{lst:average_ois_segment}. This segment is used to hold quotes for Average OIS
swap instruments. The \lstinline!Quotes! node has the structure shown in Listing \ref{lst:average_ois_quotes}. Each
quote for an Average OIS instrument (a typical example in a USD Overnight Index Swap) consists of two quotes, a vanilla
IRS quote and an OIS-LIBOR basis swap spread quote.  The IDs of these two quotes are stored in the
\lstinline!CompositeQuote! node. The \lstinline!RateQuote! node holds the ID of the vanilla IRS quote and the
\lstinline!SpreadQuote! node holds the ID of the OIS-LIBOR basis swap spread quote.

\begin{listing}[H]
%\hrule\medskip
\begin{minted}[fontsize=\footnotesize]{xml}
<AverageOIS>
  <Type> </Type>
  <Quotes>
    <CompositeQuote> </CompositeQuote>
    <CompositeQuote> </CompositeQuote>
    <!--...-->
  </Quotes>
  <Conventions> </Conventions>
  <ProjectionCurve> </ProjectionCurve>
</AverageOIS>
\end{minted}
\caption{Average OIS yield curve segment}
\label{lst:average_ois_segment}
\end{listing}

\begin{listing}[H]
%\hrule\medskip
\begin{minted}[fontsize=\footnotesize]{xml}
<Quotes>
  <CompositeQuote>
    <SpreadQuote> </SpreadQuote>
    <RateQuote> </RateQuote>
  </CompositeQuote>
  <!--...-->
</Quotes>
\end{minted}
\caption{Average OIS segment's quotes section}
\label{lst:average_ois_quotes}
\end{listing}

\subsubsection*{Tenor Basis Segment}
When the node name is \lstinline!TenorBasis!, the \lstinline!Type! node has the value \emph{Tenor Basis Swap} or
\emph{Tenor Basis Two Swaps} and the node has the structure shown in Listing \ref{lst:tenor_basis_segment}. This segment
is used to hold quotes for tenor basis swap instruments. The quotes may be for a conventional tenor basis swap where
Ibor of one tenor is swapped for Ibor of another tenor plus a spread. In this case, the \lstinline!Type! node has the
value \emph{Tenor Basis Swap}. The quotes may also be for the difference in fixed rates on two fair swaps where one swap
is against Ibor of one tenor and the other swap is against Ibor of another tenor. In this case, the \lstinline!Type!
node has the value \emph{Tenor Basis Two Swaps}. Again, the structure is similar to the simple segment in Listing
\ref{lst:simple_segment} except that there are two projection curve nodes. There is a \lstinline!ProjectionCurveShort!
node for the index with the shorter tenor. This node holds the CurveId of a curve for projecting the floating rates on
the short tenor index. Similarly, there is a \lstinline!ProjectionCurveLong! node for the index with the longer
tenor. This node holds the CurveId of a curve for projecting the floating rates on the long tenor index. These are
optional nodes. If they are left blank or omitted, then the projection curve is assumed to equal the curve being
bootstrapped i.e.\ the current CurveId. However, at least one of the nodes needs to be populated to allow the bootstrap
to proceed.

\begin{listing}[H]
%\hrule\medskip
\begin{minted}[fontsize=\footnotesize]{xml}
<TenorBasis>
  <Type> </Type>
  <Quotes>
    <Quote> </Quote>
    <Quote> </Quote>
    <!--...-->
  </Quotes>
  <Conventions> </Conventions>
  <ProjectionCurveLong> </ProjectionCurveLong>
  <ProjectionCurveShort> </ProjectionCurveShort>
</TenorBasis>
\end{minted}
\caption{Tenor basis yield curve segment}
\label{lst:tenor_basis_segment}
\end{listing}

\subsubsection*{Cross Currency Segment}
When the node name is \lstinline!CrossCurrency!, the \lstinline!Type! node has the value \emph{FX Forward} or
\emph{Cross Currency Basis Swap}. When the \lstinline!Type! node has the value \emph{FX Forward}, the node has the
structure shown in Listing \ref{lst:fx_forward_segment}. This segment is used to hold quotes for FX forward
instruments. The \lstinline!DiscountCurve! node holds the CurveId of a curve used to discount cash flows in the other
currency i.e.\ the currency in the currency pair that is not equal to the currency in Listing
\ref{lst:top_level_yc}. The \lstinline!SpotRate! node holds the ID of a spot FX quote for the currency pair that is
looked up in the {\tt market.txt} file.

\begin{listing}[H]
%\hrule\medskip
\begin{minted}[fontsize=\footnotesize]{xml}
<CrossCurrency>
  <Type> </Type>
  <Quotes>
    <Quote> </Quote>
    <Quote> </Quote>
          ...
  </Quotes>
  <Conventions> </Conventions>
  <DiscountCurve> </DiscountCurve>
  <SpotRate> </SpotRate>
</CrossCurrency>
\end{minted}
\caption{FX forward yield curve segment}
\label{lst:fx_forward_segment}
\end{listing}

When the \lstinline!Type! node has the value \emph{Cross Currency Basis Swap} then the node has the structure shown in
Listing \ref{lst:xccy_basis_segment}. This segment is used to hold quotes for cross currency basis swap instruments. The
\lstinline!DiscountCurve! node holds the CurveId of a curve used to discount cash flows in the other currency i.e.\ the
currency in the currency pair that is not equal to the currency in Listing \ref{lst:top_level_yc}. The
\lstinline!SpotRate! node holds the ID of a spot FX quote for the currency pair that is looked up in the {\tt
  market.txt} file. The \lstinline!ProjectionCurveDomestic! node holds the CurveId of a curve for projecting the
floating rates on the index in this currency i.e.\ the currency in the currency pair that is equal to the currency in
Listing \ref{lst:top_level_yc}. It is an optional node and if it is left blank or omitted, then the projection curve is
assumed to equal the curve being bootstrapped i.e.\ the current CurveId. Similarly, the
\lstinline!ProjectionCurveForeign! node holds the CurveId of a curve for projecting the floating rates on the index in
the other currency. If it is left blank or omitted, then it is assumed to equal the CurveId provided in the
\lstinline!DiscountCurve! node in this segment.

\begin{listing}[H]
%\hrule\medskip
\begin{minted}[fontsize=\footnotesize]{xml}
<CrossCurrency>
  <Type> </Type>
  <Quotes>
    <Quote> </Quote>
    <Quote> </Quote>
          ...
  </Quotes>
  <Conventions> </Conventions>
  <DiscountCurve> </DiscountCurve>
  <SpotRate> </SpotRate>
  <ProjectionCurveDomestic> </ProjectionCurveDomestic>
  <ProjectionCurveForeign> </ProjectionCurveForeign>
</CrossCurrency>
\end{minted}
\caption{Cross currency basis yield curve segment}
\label{lst:xccy_basis_segment}
\end{listing}

\subsubsection*{Zero Spread Segment}

When the node name is \lstinline!ZeroSpread!, the \lstinline!Type!
node has the only allowable value \emph{Zero Spread},  and the node has the structure shown in 
Listing \ref{lst:zero_spread_segment}. This segment is used to build yield
curves which are expressed as a spread over some reference yield curve.

\begin{listing}[H]
%\hrule\medskip
\begin{minted}[fontsize=\footnotesize]{xml}
    <ZeroSpread>
          <Type>Zero Spread</Type>
          <Quotes>
            <Quote>ZERO/YIELD_SPREAD/EUR/BANK_EUR_LEND/A365/2Y</Quote>
            <Quote>ZERO/YIELD_SPREAD/EUR/BANK_EUR_LEND/A365/5Y</Quote>
            <Quote>ZERO/YIELD_SPREAD/EUR/BANK_EUR_LEND/A365/10Y</Quote>
            <Quote>ZERO/YIELD_SPREAD/EUR/BANK_EUR_LEND/A365/20Y</Quote>
          </Quotes>
          <Conventions>EUR-ZERO-CONVENTIONS-TENOR-BASED</Conventions>
          <ReferenceCurve>EUR1D</ReferenceCurve>
    </ZeroSpread>
\end{minted}
\caption{Zero spread yield curve segment}
\label{lst:zero_spread_segment}
\end{listing}


\subsubsection*{Fitted Bond Segment}
\label{sec:fitted_bond_segment}

When the node name is \lstinline!FittedBond!, the \lstinline!Type! node has the only allowable value \emph{FittedBond},
and the node has the structure shown in Listing \ref{lst:fitted_bond_segment}. This segment is used to build yield
curves which are fitted to liquid bond prices. The segment has the following elements:

\begin{itemize}
\item Quotes: a list of bond price quotes, for each security in the list, reference data must be available
\item IborIndexCurves: for each Ibor index that is required by one of the bonds to which the curve is fitted, a mapping
  to an estimation curve for that index must be provided
\item ExtrapolateFlat: if true, the parametric curve is extrapolated flat in the instantaneous forward rate before the
  first and after the last maturity of the bonds in the calibration basket. This avoids unrealistic rates at the short
  end or for long maturities in the resulting curve.
\end{itemize}

The \lstinline!BootstrapConfig! has the following interpretation for a fitted bond curve:

\begin{itemize}
\item Accuracy [Optional, defaults to 1E-12]: the desired accuracy expressed as a weighted rmse in the implied quote,
  where 0.01 = 1 bp. Once this accuracy is reached in a calibration trial, the fit is accepted, no further calibration
  trials re run. In general, this parameter should be set to a higher than the default value for fitted bond curves.
\item GlobalAccuracy [Optional]: the acceptable accuracy. If the Accuracy is not reached in any calibration trial, but
  the GloablAccuracy is met, the best fit among the calibration trials is selected as a result of the calibration. If
  not given, the best calibration trial is compared to the Accuracy parameter instead.
\item DontThrow [Optional, defaults to false]: If true, the best calibration is always accepted as a result, i.e. no
  error is thrown even if the GlobalAccuracy is breached.
\item MaxAttempts [Optional, defaults to 5]: The maximum number of calibration trials. Each calibration trial is run with a random calibratio
  seed. Random calibration seeds are currently only supported for the NelsonSiegel interpolation method.
\end{itemize}

\begin{listing}[H]
%\hrule\medskip
\begin{minted}[fontsize=\footnotesize]{xml}
    <YieldCurve>
      ...
      <Segments>
        <FittedBond>
          <Type>FittedBond</Type>
          <Quotes>
            <Quote>BOND/PRICE/SECURITY_1</Quote>
            <Quote>BOND/PRICE/SECURITY_2</Quote>
            <Quote>BOND/PRICE/SECURITY_3</Quote>
            <Quote>BOND/PRICE/SECURITY_4</Quote>
            <Quote>BOND/PRICE/SECURITY_5</Quote>
          </Quotes>
          <!-- mapping of Ibor curves used in the bonds from which the curve is built -->
          <IborIndexCurves>
            <IborIndexCurve iborIndex="EUR-EURIBOR-6M">EUR-EURIBOR-6M</IborIndexCurve>
          </IborIndexCurves>
          <!-- flat extrapolation before first and after last bond maturity -->
          <ExtrapolateFlat>true</ExtrapolateFlat>
        </FittedBond>
      </Segments>
      <!-- NelsonSiegel, Svensson, ExponentialSplines -->
      <InterpolationMethod>NelsonSiegel</InterpolationMethod>
      <YieldCurveDayCounter>A365</YieldCurveDayCounter>
      <Extrapolation>true</Extrapolation>
      <BootstrapConfig>
        <!-- desired accuracy (in implied quote) -->
        <Accuracy>0.1</Accuracy>
        <!-- tolerable accuracy -->
        <GlobalAccuracy>0.5</GlobalAccuracy>
        <!-- do not throw even if tolerable accuracy is breached -->
        <DontThrow>false</DontThrow>
        <!-- max calibration trials to reach desired accuracy -->
        <MaxAttempts>20</MaxAttempts>
      </BootstrapConfig>
    </YieldCurve>
\end{minted}
\caption{Fitted bond yield curve segment}
\label{lst:fitted_bond_segment}
\end{listing}

\subsubsection*{Yield plus Default Segment}
\label{sec:yield_plus_default}

When the node name is \lstinline!YieldPlusDefault!, the \lstinline!Type! node has the only allowable value \emph{Yield
 Plus Default}, and the node has the structure shown in Listing \ref{lst:yield_plus_default_segment}. This segment is
used to build all-in discounting yield curves from a benchmark curve and (a weighted sum of) default curves. The
construction is in some sense inverse to the benchmark default curve construction, see \ref{ss:benchmark_default_curve}.

\begin{itemize}
\item ReferenceCurve: the benchmark yield curve serving as the basis of the resulting yield curve
\item DefaultCurves: a list of default curves whose weighted sum is added to the benchmark yield curve
\item Weights: a list of weights for the default curves, the number of weights must match the number of default curves
\end{itemize}

Notice that it is explicitly allowed to use default curves in different currencies than the benchmark yield curve. In
the construction, the hazard rate is reinterpreted as an instantaneous forward rate, and the sum of the curves is being
built in the instantaneous forward rate.

The definition takes into account the recovery rates associated to each default curve. The resulting discount factor is
computed as

\begin{equation}
P(0,t) = \prod_i  S_i(t)^{(1-R)w_i}
\end{equation}

where $S_i$ and $R_i$ are the survival probabilities and recovery rates of the source default curves, and $w_i$ are the
weights.

\begin{listing}[H]
%\hrule\medskip
\begin{minted}[fontsize=\footnotesize]{xml}
  <YieldCurve>
    <CurveId>BenchmarkPlusDefault</CurveId>
    <CurveDescription>USD Libor 3M + 0.5 x CDX.NA.HY + 0.5 x EUR.10BP</CurveDescription>
    <Currency>USD</Currency>
    <DiscountCurve/>
    <Segments>
      <YieldPlusDefault>
        <Type>Yield Plus Default</Type>
        <ReferenceCurve>USD3M</ReferenceCurve>
        <DefaultCurves>
          <DefaultCurve>Default/USD/CDX.NA.HY</DefaultCurve>
          <DefaultCurve>Default/EUR/EUR.10BP</DefaultCurve>
        </DefaultCurves>
        <Weights>
          <Weight>0.5</Weight>
          <Weight>0.5</Weight>
        </Weights>
      </YieldPlusDefault>
    </Segments>
  </YieldCurve>
</YieldCurves>
\end{minted}
\caption{Yield plus default curve segment}
\label{lst:yield_plus_default_segment}
\end{listing}

\subsubsection*{Weighted Average Segment}
\label{sec:weigthed_average}

When the node name is \lstinline!WeightedAverage!, the \lstinline!Type! node has the only allowable value
\emph{WeightedAverage}, and the node has the structure shown in Listing \ref{lst:weighted_average_segment}. This segment
is used to build a curve with instantaneous forward rates that are the weighted sum of instantaneous forward rates of
reference curves. This way a projection curve for non-standard Ibor curves can be build, e.g. to project a Euribor2M
index using the curves for 1M and 3M.

\begin{itemize}
\item ReferenceCurve1: the first source curve
\item ReferenceCurve2: the second source curve
\item Weight1: the weight of the first curve
\item Weights: the weight of the second curve
\end{itemize}

If $P_1(0,t)$ and $P_2(0,t)$ denote the discount factors of the two reference curves, the discount factor $P(0,t)$ of
the resulting curve is defined as

\begin{equation}
P(0,t) = P_1(0,t)^{w_1}P_2(0,t)^{w_2}
\end{equation}

\begin{listing}[H]
%\hrule\medskip
\begin{minted}[fontsize=\footnotesize]{xml}
<YieldCurve>
  <CurveId>EUR2M</CurveId>
  <CurveDescription>Euribor2M forwarding curve, interpolated from 1M and 3M</CurveDescription>
  <Currency>EUR</Currency>
  <DiscountCurve>EUR1D</DiscountCurve>
  <Segments>
    <WeightedAverage>
      <Type>Weighted Average</Type>
      <ReferenceCurve1>EUR1M</ReferenceCurve1>
      <ReferenceCurve2>EUR3M</ReferenceCurve2>
      <Weight1>0.5</Weight1>
      <Weight2>0.5</Weight2>
    </WeightedAverage>
  </Segments>
</YieldCurve>
\end{minted}
\caption{Weighted Average yield curve segment}
\label{lst:weighted_average_segment}
\end{listing}
