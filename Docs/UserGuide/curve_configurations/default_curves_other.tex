\subsubsection{Benchmark Default Curve}\label{ss:benchmark_default_curve}

Default curves can be set up as a difference curve of two yield curves as shown in listing
\ref{lst:defaultcurve_benchmark}. A typical use case is to back out a default curve from an all-in discounting curve
fitted to a series of liquid bond prices (the ``source curve'') and a benchmark curve representing a benchmark funding
level. The default curve can then be used in models consuming a benchmark curve and a default curve.

If $P_B(0,t)$ and $P_S(0,t)$ denote the discount factors of the given benchmark
and source curve respectively the resulting default term structures has survival probabilities

\begin{equation}
S(t) = \left( P_S(0,t) / P_B(0,t) \right) ^ { 1/(1-R) }
\end{equation}

on the given pillar times. Her, $R$ is the specified recovery rate. If the recovery rate is zero, which is the usual
case, the formula simplifies to

\begin{equation}
  S(0,t) = P_S(0,t) / P_B(0,t)
\end{equation}

The interpolation is backward flat in the hazard rate. The meaning of each node is as follows:

\begin{itemize}
\item CurveId: The curve id.
\item CurveDescription: The curve description.
\item Currency: The currency of the curve.
\item Type: Must be set to Benchmark.
\item DayCounter: The day counter used to convert dates to times.
\item RecoveryRate [optional]: The recovery rate for the resulting default curve. Defaults to zero. The recovery rate
  can be a market quote as usual or also a fixed numeric value for this curve type.
\item BenchmarkCurve: The benchmark yield curve, typically this is the standard Ibor curve in the currence
  (e.g. EUR-EURIBOR-6M, USD-Libor-3M, ...)
\item SourceCurve: The all-in discounting curve.
\item Pillars: The pillars on which to match the source curve
\item SpotLag: The pillar dates are derived using the spot lag and the tenors as specified in the Pillars node using the
  specified calendar.
\item Calendar: The calendar used to derive the pillar dates.
\item Extrapolation [Optional]: If set to true, the curve is extrapoalted beyond the last pillar. Defaults to true.
\item AllowNegativeRates [Optional]: If set to true, the check for non-negative instantaneous hazard rate in the result
  curve is disabled, i.e. the relation $P_S(0,t) \leq P_B(0,t)$ is not enforced. This flag should be enabled with care,
  i.e.  a model consuming the resulting default curve must be able to handle negative hazard rates appropriately. On the
  other hand in some situations it is natural that the source curve rates are below the benchmark rates. Defaults to
  false.
\end{itemize}

\begin{longlisting}
%\hrule\medskip
\begin{minted}[fontsize=\footnotesize]{xml}
    <DefaultCurve>
      <CurveId>BOND_YIELD_EUR_OVER_OIS</CurveId>
      <CurveDescription>Default curve derived as bond yield curve over Eonia</CurveDescription>
      <Currency>EUR</Currency>
      <Type>Benchmark</Type>
      <DayCounter>A365</DayCounter>
      <RecoveryRate>RECOVERY_RATE/RATE//SNR/USD</RecoveryRate>
      <BenchmarkCurve>Yield/EUR/EUR6M</BenchmarkCurve>
      <SourceCurve>Yield/EUR/BOND_YIELD_EUR</SourceCurve>
      <Pillars>1Y,2Y,3Y,4Y,5Y,7Y,10Y</Pillars>
      <SpotLag>0</SpotLag>
      <Calendar>TARGET</Calendar>
      <Extrapolation>true</Extrapolation>
      <AllowNegativeRates>false</AllowNegativeRates>
    </DefaultCurve>
  </DefaultCurves>
\end{minted}
\caption{Benchmark default curve}
\label{lst:defaultcurve_benchmark}
\end{longlisting}

\subsubsection{Multi-Section Default Curve}\label{ss:multisection_default_curve}

Default curves can be build by stitching together instantaneous hazard rates from multiple source curves for multiple
date ranges as shown in listing \ref{lst:defaultcurve_multisection}.

The hazard rate of the resulting curve is taken from the $i$th input curve ($i=0,1,2,\ldots$) for dates before the $i$th
switch date and (if $i>0$) on or after the $i-1$th switch date. The day counter of all input curves should be equal to
the day counter of the result curve. The interpolation is hardcoded as backward flat in the hazard rate.

If not given, the recovery rate $R$ is assumed to be zero. The result default curve's survival probabiltiies are
computed as

\begin{equation}
  S(t) = \left[ \left(\frac{P_{S,n}(t)}{P_{S,n}(t_{n})}\right)^{(1-R_n)} \Pi_{i=0}^{n-1} \left(\frac{P_{S,i}(t_{i+1})}{P_{S,i}(t_{i})}\right)^{(1-R_i)} \right] ^ { \frac{1}{1-R} }
\end{equation}

where $P_{S,i}$ is the survival probability of the $i$th source curve, $R_i$ is the associated recovery rate for the
$i$th source curve, $n$ is chosen such that $P_{S,n}$ is the relevant source curve for time $t$ according to the given
switch dates and curve $i$ is relevant for times in $[t_i,t_{i+1}]$.

The meaning of each node is as follows:

\begin{itemize}
\item CurveId: The curve id.
\item CurveDescription: The curve description.
\item Currency: The currency of the curve.
\item Type: Must be set to MutliSection.
\item SourceCurves: The list of input default curves.
\item SwitchDates: The list of dates where we switch from one input curve to the next. The number of switch dates must
  be one less than the number of source curves.
\item DayCounter: The day counter used to convert dates to times.
\item RecoveryRate [optional]: The recovery rate for the resulting default curve. Defaults to zero. The recovery rate
  can be a market quote as usual or also a fixed numeric value for this curve type.
\item Extrapolation [Optional]: If set to true, the curve is extrapoalted beyond the last pillar. Defaults to true.
\end{itemize}


\begin{longlisting}
%\hrule\medskip
\begin{minted}[fontsize=\small]{xml}
<DefaultCurve>
   <CurveId>MyMultiSectionDefaultCurve</CurveId>
   <CurveDescription>Default curve with multiple sections</CurveDescription>
   <Currency>USD</Currency>
   <Type>MultiSection</Type>
   <SourceCurves>
     <SourceCurve>Default/USD/Generic_AA_Curve</SourceCurve>
     <SourceCurve>Default/USD/Generic_B_Curve</SourceCurve>
     <SourceCurve>Default/USD/Generic_C_Curve</SourceCurve>
   </SourceCurves>
   <SwitchDates>
     <SwitchDate>2020-10-01</SwitchDate>
     <SwitchDate>2021-12-01</SwitchDate>
   <SwitchDates>
   <Extrapolation>true</Extrapolation>
   <DayCounter>A365</DayCounter>
   <RecoveryRate>RECOVERY_RATE/RATE/NAME/SR/USD</RecoveryRate>
</DefaultCurve>
\end{minted}
\caption{Multi-Section default curve}
\label{lst:defaultcurve_multisection}
\end{longlisting}
