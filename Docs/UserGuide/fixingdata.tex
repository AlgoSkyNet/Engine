Historical fixings data in the {\tt fixings.txt} file is given in three columns;  Index Name,  Fixing Date and  Index value. Columns are separated by semicolons ";". Fixings are used in cases where the current coupon of a trade has fixed in the past, or other path dependent features.

\begin{itemize}

\item Index Name: The name of the Index. \\Allowable values are given in Tables \ref{tab:non-IR_indices} and \ref{tab:IR_indices}.

\item Fixing Date: The date of the fixing.  \\ Allowable values:  See \lstinline!Date! in Table \ref{tab:allow_stand_data}.

\item Index Value: The index value for the given fixing date. \\Allowable values: Any real number.
\end{itemize}

An excerpt of a fixings file is shown in Listing \ref{lst:fixings_file}. Note that alternative index name formats are used (Table \ref{tab:IR_indices}).

\begin{lstlisting}[label=lst:fixings_file,caption=Fixings File (fixings.txt)]

UK RPI;2011-10-02;238
UK RPI;2011-11-02;238.5
UK RPI;2012-10-02;245.6
UK RPI;2012-11-02;245.6

EU HICP;2012-10-02;113.91
EU HICP;2012-11-02;113.91
EU HICP;2012-12-02;113.91
EU HICP;2011-10-02;113.44
EU HICP;2011-11-02;113.53

EONIAON;2011-11-25;0.003240
EONIAON;2011-11-28;0.003180
EONIAON;2011-11-29;0.003320
EONIAON;2011-11-30;0.003340

EURLIBOR3M ACTUAL/360;2012-12-21;0.015000
EURLIBOR3M ACTUAL/360;2012-12-24;0.015000
EURLIBOR3M ACTUAL/360;2012-12-27;0.015100
EURLIBOR3M ACTUAL/360;2012-12-28;0.015200

   
\end{lstlisting}

\begin{table}[H]
\centering
  \begin{tabu} to 0.9\linewidth {| X[-1.5,l,m] | X[-5,l,m] |}
    \hline
    \bfseries{Index Type} & \bfseries{Allowable Values} \\
    \hline
    Inflation & \begin{tabular}[l]{@{}l@{}}  \\  \emph{UK RPI} \\ \emph{UK YYR\_RPI} \\ \emph{UK YY\_RPI} \\ \emph{EU HICP} \\ \emph{EU YY\_HICP} \\  \emph{EU YYR\_HICP} \\  \emph{EU HICPXT} \\  \emph{EU YY\_HICPXT} \\  \emph{EU YYR\_HICPXT} \\  \emph{FR HICPXT} \\  \emph{FR YY\_HICPXT} \\  \emph{FR YYR\_HICPXT} \\  \emph{US CPI} \\  \emph{US YY\_CPI} \\ \emph{US YYR\_CPI} \end{tabular}  \\ \hline
   Commodity & \begin{tabular}[l]{@{}l@{}}  \\  \emph{CMCI3M} \\ \emph{CMCI6M} \\ \emph{CMCI1Y} \\ \emph{CMCI2Y} \\ \emph{CMCI3Y} \end{tabular} \\ \hline
  \end{tabu}
  \caption{Allowable values for non-IR indices.}
  \label{tab:non-IR_indices}
\end{table}




\begin{table}[H]
\centering
\begin{tabular}{|l|l|}
\hline
\multicolumn{2}{|l|}{\textbf{IR Index on form CCY-INDEX-TENOR:}}                                                                                                                                                                                                                                               \\ \hline
\textbf{Index Component} & \textbf{Allowable Values}                                                                                                                                                                                                                                                           \\ \hline
CCY-INDEX                & \textit{\begin{tabular}[c]{@{}l@{}}EUR-EONIA\\ EUR-EURIBOR\\ EUR-LIBOR\\ USD-FedFunds\\ USD-LIBOR\\ GBP-SONIA\\ GBP-LIBOR\\ JPY-TONAR\\ JPY-LIBOR\\ CHF-LIBOR\\ AUD-LIBOR\\ AUD-BBSW\\ CAD-CDOR\\ CAD-BA\\ SEK-STIBOR\\ SEK-LIBOR\\ DKK-LIBOR\\ SGD-SIBOR\\ HKD-HIBOR\end{tabular}} \\ \hline
TENOR                    & An integer followed by \emph{D, W, M or Y}                                                                                                                                                                                                                                                 \\ \hline
\multicolumn{2}{|l|}{\textbf{IR Index on alternative IndexTenor form:}}                                                                                                                                                                                                                                        \\ \hline
\textbf{Index Component} & \textbf{Allowable Values}                                                                                                                                                                                                                                                           \\ \hline
Index                    & \textit{\begin{tabular}[c]{@{}l@{}}Eonia\\ Euribor\\ EURLibor\\ FedFunds\\ USDLibor\\ Sonia\\ GBPLibor\\ TONAR\\ JPYLibor\\ CHFLibor\\ AUDLibor\\ AUD-BBSW \\ CDOR\\ SEK-STIBOR\\ SEKLibor\\ DKKLibor\\ SGD-SIBOR\\ HKD-HIBOR\end{tabular}}                                                     \\ \hline
Tenor                    & \emph{ON, TN, SN} or an integer followed by \emph{D, W, M or Y}.                                                                                                                                                                                       \\ \hline
\end{tabular}
  \caption{Allowable values for IR indices.}
  \label{tab:IR_indices}
\end{table}



\begin{table}[H]
\centering
  \begin{tabu} to 0.9\linewidth {| X[-5,l,m] | X[-5,l,m] |}
    \hline
    \bfseries{Example IR indices on CCY-INDEX-TENOR form} & \bfseries{Corresponding indices on IndexTenor form} \\
    \hline
    EUR-EONIA-1D & EoniaON \\ \hline
    EUR-EURIBOR-3M & Euribor3M \\ \hline
    JPY-TONAR-1D & TONARON \\ \hline    
    HKD-HIBOR-6M & HKD-HIBOR6M \\ \hline    
  \end{tabu}
  \caption{Example IR indices.}
  \label{tab:example_IR_indices}
\end{table}


