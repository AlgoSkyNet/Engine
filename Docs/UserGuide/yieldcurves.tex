\subsubsection{Yield Curves}

The top level XML elements for each \lstinline!YieldCurve! node are shown in Listing \ref{lst:top_level_yc}.

\begin{listing}[H]
%\hrule\medskip
\begin{minted}[fontsize=\footnotesize]{xml}
<YieldCurve>
  <CurveId> </CurveId>
  <CurveDescription> </CurveDescription>
  <Currency> </Currency>
  <DiscountCurve> </DiscountCurve>
  <Segments> </Segments>
  <InterpolationVariable> </InterpolationVariable>
  <InterpolationMethod> </InterpolationMethod>
  <ZeroDayCounter> </ZeroDayCounter>
  <Tolerance> </Tolerance>
  <Extrapolation> </Extrapolation>
</YieldCurve>
\end{minted}
\caption{Top level yield curve node}
\label{lst:top_level_yc}
\end{listing}

The meaning of each of the top level elements in Listing \ref{lst:top_level_yc} is given below. If an element is labelled 
as 'Optional', then it may be excluded or included and left blank.
\begin{itemize}
\item CurveId: Unique identifier for the yield curve.
\item CurveDescription: A description of the yield curve. This field may be left blank.
\item Currency: The yield curve currency.
\item DiscountCurve: If the yield curve is being bootstrapped from market instruments, this gives the CurveId of the
yield curve used to discount cash flows during the bootstrap procedure. If this field is left blank or set equal to the
current CurveId, then this yield curve itself is used to discount cash flows during the bootstrap procedure.
\item Segments: This element contains child elements and is described in the following subsection.
\item InterpolationVariable [Optional]: The variable on which the interpolation is performed. The allowable values are
given in Table \ref{tab:allow_interp_variables}. If the element is omitted or left blank, then it defaults to
\emph{Discount}.
\item InterpolationMethod [Optional]: The interpolation method to use. The allowable values are given in Table
\ref{tab:allow_interp_methods}. If the element is omitted or left blank, then it defaults to \emph{LogLinear}.
\item ZeroDayCounter [Optional]: The day count basis used internally by the yield curve to calculate the time between
dates. In particular, if the curve is queried for a zero rate without specifying the day count basis, the zero rate that
is returned has this basis. If the element is omitted or left blank, then it defaults to \emph{A365}.
\item Tolerance [Optional]: The tolerance used by the root finding procedure in the bootstrapping algorithm. If the
element is omitted or left blank, then it defaults to \num[scientific-notation=true]{1.0e-12}.
\item Extrapolation [Optional]: Set to \emph{True} or \emph{False} to enable or disable extrapolation respectively. If
the element is omitted or left blank, then it defaults to \emph{True}.
\end{itemize}

\begin{table}[h]
\centering
  \begin{tabu} to 0.9\linewidth {| X[-1.5,l,m] | X[-5,l,m] |}
    \hline
    \bfseries{Variable} & \bfseries{Description} \\
    \hline
    Zero & The continuously compounded zero rate \\ \hline
    Discount & The discount factor \\ \hline
    Forward & The instantaneous forward rate \\ \hline
  \end{tabu}
  \caption{Allowable interpolation variables.}
  \label{tab:allow_interp_variables}
\end{table}

\begin{table}[h]
\centering
  \begin{tabu} to 0.9\linewidth {| X[-1.5,l,m] | X[-5,l,m] |}
    \hline
    \bfseries{Method} & \bfseries{Description} \\
    \hline
    Linear & Linear interpolation \\ \hline
    LogLinear & Linear interpolation on the natural log of the interpolation variable \\ \hline
    NaturalCubic & Monotonic Kruger cubic interpolation with second derivative at left and right \\ \hline
    FinancialCubic & Monotonic Kruger cubic interpolation with second derivative at left and 
                     first derivative at right \\ \hline
    ConvexMonotone & Convex Monotone Interpolation (Hagan, West) \\ \hline
  \end{tabu}
  \caption{Allowable interpolation methods.}
  \label{tab:allow_interp_methods}
\end{table}
%- - - - - - - - - - - - - - - - - - - - - - - - - - - - - - - - - - - - - - - -
\subsubsection*{Segments Node} \label{ss:segments_node}
%- - - - - - - - - - - - - - - - - - - - - - - - - - - - - - - - - - - - - - - -
The \lstinline!Segments! node gives the zero rates, discount factors and instruments that comprise the yield curve. This
node consists of a number of child nodes where the node name depends on the segment being described. Each node has a
\lstinline!Type! that determines its structure. The following sections describe the type of child nodes that are
available.

\subsubsection*{Direct Segment}
When the node name is \lstinline!Direct!, the \lstinline!Type! node has the value \emph{Zero} or \emph{Discount} and the
node has the structure shown in Listing \ref{lst:direct_segment}. We refer to this segment here as a direct segment
because the discount factors, or equivalently the zero rates, are given explicitly and do not need to be
bootstrapped. The \lstinline!Quotes! node contains a list of \lstinline!Quote! elements. Each \lstinline!Quote! element
contains an ID pointing to a line in the {\tt market.txt} file, i.e.\ in this case, pointing to a particular zero rate
or discount factor. The \lstinline!Conventions! node contains the ID of a node in the {\tt conventions.xml} file
described in section \ref{sec:conventions}. The \lstinline!Conventions! node associates conventions with the quotes.

\begin{listing}[H]
%\hrule\medskip
\begin{minted}[fontsize=\footnotesize]{xml}
<Direct>
  <Type> </Type>
  <Quotes>
    <Quote> </Quote>
    <Quote> </Quote>
     <!--...-->
  </Quotes>
  <Conventions> </Conventions>
</Direct>
\end{minted}
\caption{Direct yield curve segment}
\label{lst:direct_segment}
\end{listing}


\subsubsection*{Simple Segment}
When the node name is \lstinline!Simple!, the \lstinline!Type! node has the value \emph{Deposit}, \emph{FRA},
\emph{Future}, \emph{OIS} or \emph{Swap} and the node has the structure shown in Listing \ref{lst:simple_segment}. This
segment holds quotes for a set of deposit, FRA, Future, OIS or swap instruments corresponding to the value in the
\lstinline!Type! node. These quotes will be used by the bootstrap algorithm to imply a discount factor, or equivalently
a zero rate, curve. The only difference between this segment and the direct segment is that there is a
\lstinline!ProjectionCurve! node. This node allows us to specify the CurveId of another curve to project floating rates
on the instruments underlying the quotes listed in the \lstinline!Quote! nodes during the bootstrap procedure. This is
an optional node. If it is left blank or omitted, then the projection curve is assumed to equal the curve being
bootstrapped i.e.\ the current CurveId.

\begin{listing}[H]
%\hrule\medskip
\begin{minted}[fontsize=\footnotesize]{xml}
<Simple>
  <Type> </Type>
  <Quotes>
    <Quote> </Quote>
    <Quote> </Quote>
    <!--...-->
  </Quotes>
  <Conventions> </Conventions>
  <ProjectionCurve> </ProjectionCurve>
</Simple>
\end{minted}
\caption{Simple yield curve segment}
\label{lst:simple_segment}
\end{listing}

\subsubsection*{Average OIS Segment}
When the node name is \lstinline!AverageOIS!, the \lstinline!Type! node has the value \emph{Average OIS} and the node
has the structure shown in Listing \ref{lst:average_ois_segment}. This segment is used to hold quotes for Average OIS
swap instruments. The \lstinline!Quotes! node has the structure shown in Listing \ref{lst:average_ois_quotes}. Each
quote for an Average OIS instrument (a typical example in a USD Overnight Index Swap) consists of two quotes, a vanilla
IRS quote and an OIS-LIBOR basis swap spread quote.  The IDs of these two quotes are stored in the
\lstinline!CompositeQuote! node. The \lstinline!RateQuote! node holds the ID of the vanilla IRS quote and the
\lstinline!SpreadQuote! node holds the ID of the OIS-LIBOR basis swap spread quote.

\begin{listing}[H]
%\hrule\medskip
\begin{minted}[fontsize=\footnotesize]{xml}
<AverageOIS>
  <Type> </Type>
  <Quotes>
    <CompositeQuote> </CompositeQuote>
    <CompositeQuote> </CompositeQuote>
    <!--...-->
  </Quotes>
  <Conventions> </Conventions>
  <ProjectionCurve> </ProjectionCurve>
</AverageOIS>
\end{minted}
\caption{Average OIS yield curve segment}
\label{lst:average_ois_segment}
\end{listing}

\begin{listing}[H]
%\hrule\medskip
\begin{minted}[fontsize=\footnotesize]{xml}
<Quotes>
  <CompositeQuote>
    <SpreadQuote> </SpreadQuote>
    <RateQuote> </RateQuote>
  </CompositeQuote>
  <!--...-->
</Quotes>
\end{minted}
\caption{Average OIS segment's quotes section}
\label{lst:average_ois_quotes}
\end{listing}

\subsubsection*{Tenor Basis Segment}
When the node name is \lstinline!TenorBasis!, the \lstinline!Type! node has the value \emph{Tenor Basis Swap} or
\emph{Tenor Basis Two Swaps} and the node has the structure shown in Listing \ref{lst:tenor_basis_segment}. This segment
is used to hold quotes for tenor basis swap instruments. The quotes may be for a conventional tenor basis swap where
Ibor of one tenor is swapped for Ibor of another tenor plus a spread. In this case, the \lstinline!Type! node has the
value \emph{Tenor Basis Swap}. The quotes may also be for the difference in fixed rates on two fair swaps where one swap
is against Ibor of one tenor and the other swap is against Ibor of another tenor. In this case, the \lstinline!Type!
node has the value \emph{Tenor Basis Two Swaps}. Again, the structure is similar to the simple segment in Listing
\ref{lst:simple_segment} except that there are two projection curve nodes. There is a \lstinline!ProjectionCurveShort!
node for the index with the shorter tenor. This node holds the CurveId of a curve for projecting the floating rates on
the short tenor index. Similarly, there is a \lstinline!ProjectionCurveLong! node for the index with the longer
tenor. This node holds the CurveId of a curve for projecting the floating rates on the long tenor index. These are
optional nodes. If they are left blank or omitted, then the projection curve is assumed to equal the curve being
bootstrapped i.e.\ the current CurveId. However, at least one of the nodes needs to be populated to allow the bootstrap
to proceed.

\begin{listing}[H]
%\hrule\medskip
\begin{minted}[fontsize=\footnotesize]{xml}
<TenorBasis>
  <Type> </Type>
  <Quotes>
    <Quote> </Quote>
    <Quote> </Quote>
    <!--...-->
  </Quotes>
  <Conventions> </Conventions>
  <ProjectionCurveLong> </ProjectionCurveLong>
  <ProjectionCurveShort> </ProjectionCurveShort>
</TenorBasis>
\end{minted}
\caption{Tenor basis yield curve segment}
\label{lst:tenor_basis_segment}
\end{listing}

\subsubsection*{Cross Currency Segment}
When the node name is \lstinline!CrossCurrency!, the \lstinline!Type! node has the value \emph{FX Forward} or
\emph{Cross Currency Basis Swap}. When the \lstinline!Type! node has the value \emph{FX Forward}, the node has the
structure shown in Listing \ref{lst:fx_forward_segment}. This segment is used to hold quotes for FX forward
instruments. The \lstinline!DiscountCurve! node holds the CurveId of a curve used to discount cash flows in the other
currency i.e.\ the currency in the currency pair that is not equal to the currency in Listing
\ref{lst:top_level_yc}. The \lstinline!SpotRate! node holds the ID of a spot FX quote for the currency pair that is
looked up in the {\tt market.txt} file.

\begin{listing}[H]
%\hrule\medskip
\begin{minted}[fontsize=\footnotesize]{xml}
<CrossCurrency>
  <Type> </Type>
  <Quotes>
    <Quote> </Quote>
    <Quote> </Quote>
          ...
  </Quotes>
  <Conventions> </Conventions>
  <DiscountCurve> </DiscountCurve>
  <SpotRate> </SpotRate>
</CrossCurrency>
\end{minted}
\caption{FX forward yield curve segment}
\label{lst:fx_forward_segment}
\end{listing}

When the \lstinline!Type! node has the value \emph{Cross Currency Basis Swap} then the node has the structure shown in
Listing \ref{lst:xccy_basis_segment}. This segment is used to hold quotes for cross currency basis swap instruments. The
\lstinline!DiscountCurve! node holds the CurveId of a curve used to discount cash flows in the other currency i.e.\ the
currency in the currency pair that is not equal to the currency in Listing \ref{lst:top_level_yc}. The
\lstinline!SpotRate! node holds the ID of a spot FX quote for the currency pair that is looked up in the {\tt
  market.txt} file. The \lstinline!ProjectionCurveDomestic! node holds the CurveId of a curve for projecting the
floating rates on the index in this currency i.e.\ the currency in the currency pair that is equal to the currency in
Listing \ref{lst:top_level_yc}. It is an optional node and if it is left blank or omitted, then the projection curve is
assumed to equal the curve being bootstrapped i.e.\ the current CurveId. Similarly, the
\lstinline!ProjectionCurveForeign! node holds the CurveId of a curve for projecting the floating rates on the index in
the other currency. If it is left blank or omitted, then it is assumed to equal the CurveId provided in the
\lstinline!DiscountCurve! node in this segment.

\begin{listing}[H]
%\hrule\medskip
\begin{minted}[fontsize=\footnotesize]{xml}
<CrossCurrency>
  <Type> </Type>
  <Quotes>
    <Quote> </Quote>
    <Quote> </Quote>
          ...
  </Quotes>
  <Conventions> </Conventions>
  <DiscountCurve> </DiscountCurve>
  <SpotRate> </SpotRate>
  <ProjectionCurveDomestic> </ProjectionCurveDomestic>
  <ProjectionCurveForeign> </ProjectionCurveForeign>
</CrossCurrency>
\end{minted}
\caption{Cross currency basis yield curve segment}
\label{lst:xccy_basis_segment}
\end{listing}

\subsubsection*{Zero Spread Segment}

When the node name is \lstinline!ZeroSpread!, the \lstinline!Type!
node has the only allowable value \emph{Zero Spread},  and the node has the structure shown in 
Listing \ref{lst:zero_spread_segment}. This segment is used to build yield
curves which are expressed as a spread over some reference yield curve.

\begin{listing}[H]
%\hrule\medskip
\begin{minted}[fontsize=\footnotesize]{xml}
    <ZeroSpread>
          <Type>Zero Spread</Type>
          <Quotes>
            <Quote>ZERO/YIELD_SPREAD/EUR/BANK_EUR_LEND/A365/2Y</Quote>
            <Quote>ZERO/YIELD_SPREAD/EUR/BANK_EUR_LEND/A365/5Y</Quote>
            <Quote>ZERO/YIELD_SPREAD/EUR/BANK_EUR_LEND/A365/10Y</Quote>
            <Quote>ZERO/YIELD_SPREAD/EUR/BANK_EUR_LEND/A365/20Y</Quote>
          </Quotes>
          <Conventions>EUR-ZERO-CONVENTIONS-TENOR-BASED</Conventions>
          <ReferenceCurve>EUR1D</ReferenceCurve>
    </ZeroSpread>
\end{minted}
\caption{Zero spread yield curve segment}
\label{lst:zero_spread_segment}
\end{listing}
