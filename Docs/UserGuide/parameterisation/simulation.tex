\ifdefined\RiskCatalogue\newcommand{\simsection}{\section}\fi
\ifdefined\RiskCatalogue\newcommand{\simsubsection}{\subsection}\fi
\ifdefined\UserGuide\newcommand{\simsection}{\subsection}\fi
\ifdefined\UserGuide\newcommand{\simsubsection}{\subsubsection}\fi
%--------------------------------------------------------
\simsection{Simulation: {\tt simulation.xml}}\label{sec:simulation}
%--------------------------------------------------------

This file determines the behaviour of the risk factor simulation (scenario generation) module.
It is structured in three blocks of data.

\begin{listing}[H]
%\hrule\medskip
\begin{minted}[fontsize=\footnotesize]{xml}
<Simulation>
  <Parameters> ... </Parameters>
  <CrossAssetModel> ... </CrossAssetModel>
  <Market> ... </Market>
</Simulation>
\end{minted}
\caption{Simulation configuration}
\label{lst:simulation_configuration}
\end{listing}

Each of the three blocks is sketched in the following.

\simsubsection{Parameters}\label{sec:sim_params}

Let us discuss this section using the following example

\begin{listing}[H]
%\hrule\medskip
\begin{minted}[fontsize=\footnotesize]{xml}
<Parameters>
  <Grid>80,3M</Grid>
  <Calendar>EUR,USD,GBP,CHF</Calendar>
  <DayCounter>ACT/ACT</DayCounter>
  <Sequence>SobolBrownianBridge</Sequence>
  <Seed>42</Seed>
  <Samples>1000</Samples>
  <Ordering>Steps</Ordering>
  <DirectionIntegers>JoeKuoD7</DirectionIntegers>
  <!-- The following two nodes are optional -->
  <CloseOutLag>2W</CloseOutLag>
  <MporMode>StickyDate</MporMode>
</Parameters>
\end{minted}
\caption{Simulation configuration}
\label{lst:simulation_params_configuration}
\end{listing}

\begin{itemize}
\item {\tt Grid:} Specifies the simulation time grid, here 80 quarterly steps.\footnote{For exposure calculation under DIM, the second parameter has to match the Margin Period of Risk, i.e. if {\tt MarginPeriodOfRisk} is set to for instance {\tt 2W} in a netting set definition in {\tt netting.xml}, then one has to set {\tt Grid} to for instance {\tt 80,2W}.}
\item {\tt Calendar:} Calendar or combination of calendars used to adjust the dates of the grid. Date adjustment is
required because the simulation must step over 'good' dates on which index fixings can be stored.
%\item {\tt Scenario: } Choose between {\em Simple } and {\em Complex } implementations, the latter optimized for
% more efficient memory usage. \todo[inline]{Remove Scenario choice}
\item {\tt DayCounter:} Day count convention used to translate dates to times. Optional, defaults to ActualActual ISDA.
\item {\tt Sequence:} Choose random sequence generator ({\em MersenneTwister, MersenneTwisterAntithetic,
  Sobol,Burley2020Sobol, SobolBrownianBridge, Burley2020SobolBrownianBridge}).
\item {\tt Seed:} Random number generator seed
\item {\tt Samples:} Number of Monte Carlo paths to be produced
%\item {\tt Fixings: } Choose whether fixings should be simulated or not, and if so which fixing simulation method to
use ({\em Backward, Forward, BestOfForwardBackward, InterpolatedForwardBackward}), which number of forward horizon days
to use if one of the {\em Forward } related methods is chosen.
\item {\tt Ordering:} If the sequence type {\em SobolBrownianBridge} or {\em Burley2020SobolBrownianBridge} is used,
  ordering of variates ({\em Factors, Steps, Diagonal})
\item {\tt DirectionIntegers:} If the sequence type {\em SobolBrownianBridge}, {\em Burley2020SobolBrownianBridge}, {\em
  Sobol} or {\em Burley2020Sobol} is used, type of direction integers in Sobol generator ({\em Unit, Jaeckel,
  SobolLevitan, SobolLevitanLemieux, JoeKuoD5, JoeKuoD6, JoeKuoD7, Kuo, Kuo2, Kuo3})
\item {\tt CloseOutLag}: If this tag is present, this specifies the close-out period length (e.g. 2W) used; otherwise no close-out grid is built. The close-out grid is an auxiliary time grid that is offset from the main default date grid by the close-out period, typically set to the applicable margin period of risk. If present, it is used to evolve the portfolio value and determine close-out values associated with the preceding default date valuation.
\item {\tt MporMode}: This tag is expected if the previous one is present, permissible values are then {\tt StickyDate} and {\tt ActualDate}. {\tt StickyDate} means that only market data is evolved from the default date to close-out date for close-out date valuation, the valuation as of date remains unchanged and trades do not ``age'' over the period. As a consequence, exposure evolutions will not show spikes caused by cash flows within the close-out period. {\tt ActualDate} means that trades will also age over the close-out period so that one can experience exposure evolution spikes due to cash flows. 
\end{itemize}

\simsubsection{Model}\label{sec:sim_model}

The {\tt CrossAssetModel} section determines the cross asset model's number of currencies covered, composition, and each
component's calibration. It is currently made of 
\begin{itemize}
\item a sequence of LGM models for each currency (say $n_c$ currencies), 
\item $n_c-1$ FX models for each exchange rate to the base currency, 
\item $n_e$ equity models,
\item $n_i$ inflation models, 
\item $n_{cr}$ credit models, 
\item $n_{com}$ commodity models, 
\item a specification of the correlation structure between all components.
\end{itemize}

\medskip The simulated currencies are specified as follows, with clearly identifying the domestic currency which is also
the target currency for all FX models listed subsequently. If the portfolio requires more currencies to be simulated,
this will lead to an exception at run time, so that it is the user's responsibility to make sure that the list of
currencies here is sufficient. The list can be larger than actually required by the portfolio. This will not lead to any
exceptions, but add to the run time of ORE.

\begin{listing}[H]
%\hrule\medskip
\begin{minted}[fontsize=\footnotesize]{xml}
<CrossAssetModel>
  <DomesticCcy>EUR</DomesticCcy>
  <Currencies>
    <Currency>EUR</Currency>
    <Currency>USD</Currency>
    <Currency>GBP</Currency>
    <Currency>CHF</Currency>
    <Currency>JPY</Currency>
  </Currencies>
  <Equities>
	<!-- ... -->
  </Equities>
  <InflationIndices>
	<!-- ... -->
  </InflationIndices>
  <CreditNames>
	<!-- ... -->
  </CreditNames>
  <Commodities>
	<!-- ... -->
  </Commodities>
  <BootstrapTolerance>0.0001</BootstrapTolerance>
  <Measure>LGM</Measure><!-- Choices: LGM, BA -->
  <Discretization>Exact</Discretization>
  <!-- ... -->
</CrossAssetModel>
\end{minted}
\caption{Simulation model currencies configuration}
\label{lst:simulation_model_currencies_configuration}
\end{listing}
 
Bootstrap tolerance is a global parameter that applies to the calibration of all model components. If the calibration
error of any component exceeds this tolerance, this will trigger an exception at runtime, early in the ORE process.

The Measure tag allows switching between the LGM and the Bank Account (BA) measure for the risk-neutral market simulations using the Cross Asset Model. Note that within LGM one can shift the horizon (see ParameterTransformation below) to effectively switch to a T-Forward measure.

The Discretization tag chooses between time discretization schemes for the risk factor evolution. {\em Exact} means
exploiting the analytical tractability of the model to avoid any time discretization error. {\em Euler} uses a naive
time discretization scheme which has numerical error and requires small time steps for accurate results (useful for
testing purposes or if more sophisticated component models are used.)
 
\medskip

Each interest rate model is specified by a block as follows

\begin{listing}[H]
%\hrule\medskip
\begin{minted}[fontsize=\footnotesize]{xml}
<CrossAssetModel>	
  <!-- ... -->
  <InterestRateModels>
    <LGM ccy="default">
      <CalibrationType>Bootstrap</CalibrationType>
      <Volatility>
        <Calibrate>Y</Calibrate>
        <VolatilityType>Hagan</VolatilityType>
        <ParamType>Piecewise</ParamType>
        <TimeGrid>1.0,2.0,3.0,4.0,5.0,7.0,10.0</TimeGrid>
        <InitialValue>0.01,0.01,0.01,0.01,0.01,0.01,0.01,0.01<InitialValue>
      </Volatility>
      <Reversion>
        <Calibrate>N</Calibrate>
        <ReversionType>HullWhite</ReversionType>
        <ParamType>Constant</ParamType>
        <TimeGrid/>
        <InitialValue>0.03</InitialValue>
      </Reversion>
      <CalibrationSwaptions>
        <Expiries>1Y,2Y,4Y,6Y,8Y,10Y,12Y,14Y,16Y,18Y,19Y</Expiries>
        <Terms>19Y,18Y,16Y,14Y,12Y,10Y,8Y,6Y,4Y,2Y,1Y</Terms>
        <Strikes/>
      </CalibrationSwaptions>
      <ParameterTransformation>
        <ShiftHorizon>0.0</ShiftHorizon>
        <Scaling>1.0</Scaling>
      </ParameterTransformation>
    </LGM>
    <LGM ccy="EUR">
      <!-- ... -->
    </LGM>
    <LGM ccy="USD">
      <!-- ... -->
    </LGM>
  </InterestRateModels>	
  <!-- ... -->		
</CrossAssetModel>
\end{minted}
\caption{Simulation model IR configuration}
\label{lst:simulation_model_ir_configuration}
\end{listing}

We have LGM sections by currency, but starting with a section for currency 'default'. As the name implies, this is used
as default configuration for any currency in the currency list for which we do not provide an explicit
parametrisation. Within each LGM section, the interpretation of elements is as follows:

\begin{itemize}
\item {\tt CalibrationType: } Choose between {\em Bootstrap} and {\em BestFit}, where Bootstrap is chosen when we expect
to be able to achieve a perfect fit (as with calibration of piecewise volatility to a series of co-terminal Swaptions)
\item {\tt Volatility/Calibrate: } Flag to enable/disable calibration of this particular parameter
\item {\tt Volatility/VolatilityType: } Choose volatility parametrisation a la {\em HullWhite} or {\em Hagan}
\item {\tt Volatility/ParamType: } Choose between {\em Constant} and {\em Piecewise}
\item {\tt Volatility/TimeGrid: } Initial time grid for this parameter, can be left empty if ParamType is Constant
\item {\tt Volatility/InitialValue: } Vector of initial values, matching number of entries in time (for CalibrationType {\em BestFit} this should be one more entry than the {\tt Volatility/TimeGrid} entries, for {\em Bootstrap} this is ignored), or single value if the time grid is empty
\item {\tt Reversion/Calibrate: } Flag to enable/disable calibration of this particular parameter
\item {\tt Reversion/VolatilityType: } Choose reversion parametrisation a la {\em HullWhite} or {\em Hagan}
\item {\tt Reversion/ParamType: } Choose between {\em Constant} and {\em Piecewise}
\item {\tt Reversion/TimeGrid: } Initial time grid for this parameter, can be left empty if ParamType is Constant
\item {\tt Reversion/InitialValue: } Vector of initial values, matching number of entries in time, or single value if
the time grid is empty
\item {\tt CalibrationSwaptions: } Choice of calibration instruments by expiry, underlying Swap term and strike. There have to be at least one more calibration options configured than {\tt Volatility/TimeGrid} entries were given.
\item {\tt ParameterTransformation: } LGM model prices are invariant under scaling and shift transformations
\cite{Lichters} with advantages for numerical convergence of results in long term simulations. These transformations can
be chosen here. Default settings are shiftHorizon 0 (time in years) and scaling factor 1.
\end{itemize}

The reason for having to specify one more {\tt Volatility/InitialValue} entries than {\tt Volatility/TimeGrid} entries (and at least one more calibration option than {\tt Volatility/TimeGrid} entries) is the fact that the intervals defined by the {\tt Volatility/TimeGrid} entries are spanning from $[0,t_1],[t_1,t_2]\ldots[t_n,\infty]$, which results in $n+1$ intervals.

\medskip

Each FX model is specified by a block as follows

\begin{listing}[H]
%\hrule\medskip
\begin{minted}[fontsize=\footnotesize]{xml}
<CrossAssetModel>	
  <!-- ... -->
  <ForeignExchangeModels>
    <CrossCcyLGM foreignCcy="default">
      <DomesticCcy>EUR</DomesticCcy>
      <CalibrationType>Bootstrap</CalibrationType>
      <Sigma>
        <Calibrate>Y</Calibrate>
        <ParamType>Piecewise</ParamType>
        <TimeGrid>1.0,2.0,3.0,4.0,5.0,7.0,10.0</TimeGrid>
        <InitialValue>0.1,0.1,0.1,0.1,0.1,0.1,0.1,0.1</InitialValue>
      </Sigma>
      <CalibrationOptions>
        <Expiries>1Y,2Y,3Y,4Y,5Y,10Y</Expiries>
        <Strikes/>
      </CalibrationOptions>
    </CrossCcyLGM>
    <CrossCcyLGM foreignCcy="USD">
      <!-- ... -->
    </CrossCcyLGM>
    <CrossCcyLGM foreignCcy="GBP">
      <!-- ... -->
    </CrossCcyLGM>
    <!-- ... -->
  </ForeignExchangeModels>
  <!-- ... -->
<CrossAssetModel>	
\end{minted}
\caption{Simulation model FX configuration}
\label{lst:simulation_model_fx_configuration}
\end{listing}

CrossCcyLGM sections are defined by foreign currency, but we also support a default configuration as above for the IR
model parametrisations.  Within each CrossCcyLGM section, the interpretation of elements is as follows:

\begin{itemize}
\item {\tt DomesticCcy: } Domestic currency completing the FX pair
\item {\tt CalibrationType: } Choose between {\em Bootstrap} and {\em BestFit} as in the IR section
\item {\tt Sigma/Calibrate: } Flag to enable/disable calibration of this particular parameter
\item {\tt Sigma/ParamType: } Choose between {\em Constant} and {\em Piecewise}
\item {\tt Sigma/TimeGrid: } Initial time grid for this parameter, can be left empty if ParamType is Constant
\item {\tt Sigma/InitialValue: } Vector of initial values, matching number of entries in time (for CalibrationType {\em BestFit} this should be one more entry than the {\tt Sigma/TimeGrid} entries, for {\em Bootstrap} this is ignored), or single value if the time grid is empty
\item {\tt CalibrationOptions: } Choice of calibration instruments by expiry and strike, strikes can be empty (implying
the default, ATMF options), or explicitly specified (in terms of FX rates as absolute strike values, in delta notation
such as $\pm 25D$, $ATMF$ for at the money). There have to be at least one more calibration options configured than {\tt Sigma/TimeGrid} entries were given
\end{itemize}


\medskip

Each equity model is specified by a block as follows

\begin{listing}[H]
%\hrule\medskip
\begin{minted}[fontsize=\footnotesize]{xml}
<CrossAssetModel>	
  <!-- ... -->
  <EquityModels>
    <CrossAssetLGM name="default">
      <Currency>EUR</Currency>
      <CalibrationType>Bootstrap</CalibrationType>
      <Sigma>
        <Calibrate>Y</Calibrate>
        <ParamType>Piecewise</ParamType>
        <TimeGrid>1.0,2.0,3.0,4.0,5.0,7.0,10.0</TimeGrid>
        <InitialValue>0.1,0.1,0.1,0.1,0.1,0.1,0.1,0.1</InitialValue>
      </Sigma>
      <CalibrationOptions>
        <Expiries>1Y,2Y,3Y,4Y,5Y,10Y</Expiries>
        <Strikes/>
      </CalibrationOptions>
    </CrossAssetLGM>
    <CrossAssetLGM name="SP5">
      <!-- ... -->
    </CrossAssetLGM>
    <CrossAssetLGM name="Lufthansa">
      <!-- ... -->
    </CrossAssetLGM>
      <!-- ... -->
  </EquityModels>
  <!-- ... -->
<CrossAssetModel>	
\end{minted}
\caption{Simulation model equity configuration}
\label{lst:simulation_model_eq_configuration}
\end{listing}

CrossAssetLGM sections are defined by equity name, but we also support a default configuration as above for the IR and 
FX model parameterisations.  Within each CrossAssetLGM section, the interpretation of elements is as follows:

\begin{itemize}
	\item {\tt Currency: } Currency of denomination
	\item {\tt CalibrationType: } Choose between {\em Bootstrap} and {\em BestFit} as in the IR section
	\item {\tt Sigma/Calibrate: } Flag to enable/disable calibration of this particular parameter
	\item {\tt Sigma/ParamType: } Choose between {\em Constant} and {\em Piecewise}
	\item {\tt Sigma/TimeGrid: } Initial time grid for this parameter, can be left empty if ParamType is Constant
	\item {\tt Sigma/InitialValue: } Vector of initial values, matching number of entries in time (for CalibrationType {\em BestFit} this should be one more entry than the {\tt Sigma/TimeGrid} entries, for {\em Bootstrap} this is ignored), or single value if the time grid is empty
	\item {\tt CalibrationOptions: } Choice of calibration instruments by expiry and strike, strikes can be empty 
	(implying the default, ATMF options), or explicitly specified (in terms of equity prices as absolute strike values). There have to be at least one more calibration options configured than {\tt Sigma/TimeGrid} entries were given
\end{itemize}

\medskip

For the inflation model component, there is a choice between a Dodgson Kainth model and a Jarrow Yildrim model. The Dodgson Kainth 
model is specified in a \lstinline!LGM! or \lstinline!DodgsonKainth! node as outlined in Listing \ref{lst:simulation_model_dk_inflation_configuration}.
The inflation model parameterisation inherits from the LGM parameterisation for interest rate components, in particular the \lstinline!CalibrationType!, 
\lstinline!Volatility! and \lstinline!Reversion! elements. The \lstinline!CalibrationCapFloors! element specify the model's calibration to a selection of 
either CPI caps or CPI floors with specified strike.

\begin{listing}[H]
%\hrule\medskip
\begin{minted}[fontsize=\footnotesize]{xml}
<CrossAssetModel>	
  ...
  <InflationIndexModels>
    <LGM index="EUHICPXT">
      <Currency>EUR</Currency>
      <!-- As in the LGM parameterisation for any IR components -->
      <CalibrationType> ... </CalibrationType>
      <Volatility> ... </Volatility>
      <Reversion> ... </Reversion> 
      <ParameterTransformation> ... </ParameterTransformation>
      <!-- Inflation model specific -->
      <CalibrationCapFloors>
        <!-- not used yet, as there is only one strategy so far -->
        <CalibrationStrategy> ... </CalibrationStrategy> 
        <CapFloor> Floor </CapFloor> <!-- Cap, Floor -->
        <Expiries> 2Y, 4Y, 6Y, 8Y, 10Y </Expiries>
        <!-- can be empty, this will yield calibration to ATM -->
        <Strikes> 0.03, 0.03, 0.03, 0.03, 0.03 </Strikes> 
      </CalibrationCapFloors>
    </LGM>
    <LGM index="USCPI">
      ...
    </LGM>
    ...
  </InflationIndexModels>
  ...
<CrossAssetModel>	
\end{minted}
\caption{Simulation model DK inflation component configuration}
\label{lst:simulation_model_dk_inflation_configuration}
\end{listing}

The calibration instruments may be specified in an alternative way via a \lstinline!CalibrationBaskets! node. In general, a \lstinline!CalibrationBaskets! node 
can contain multiple \lstinline!CalibrationBasket! nodes each containing a list of calibration instruments of the same type. For Dodgson Kainth, only a single 
calibration basket is allowed and the instruments must be of type \lstinline!CpiCapFloor!. So, for example, the \lstinline!CalibrationCapFloors! node in 
Listing \ref{lst:simulation_model_dk_inflation_configuration} could be replaced with the \lstinline!CalibrationBaskets! node in \ref{lst:dk_inflation_calibration_basket}.

\begin{listing}[H]
\begin{minted}[fontsize=\footnotesize]{xml}
<CalibrationBaskets>
  <CalibrationBasket>
    <CpiCapFloor>
      <Type>Floor</Type>
      <Maturity>2Y</Maturity>
      <Strike>0.03</Strike>
    </CpiCapFloor>
    <CpiCapFloor>
      <Type>Floor</Type>
      <Maturity>4Y</Maturity>
      <Strike>0.03</Strike>
    </CpiCapFloor>
    <CpiCapFloor>
      <Type>Floor</Type>
      <Maturity>6Y</Maturity>
      <Strike>0.03</Strike>
    </CpiCapFloor>
    <CpiCapFloor>
      <Type>Floor</Type>
      <Maturity>8Y</Maturity>
      <Strike>0.03</Strike>
    </CpiCapFloor>
    <CpiCapFloor>
      <Type>Floor</Type>
      <Maturity>10Y</Maturity>
      <Strike>0.03</Strike>
    </CpiCapFloor>
  </CalibrationBasket>
</CalibrationBaskets>
\end{minted}
\caption{Calibration basket for DK inflation model component}
\label{lst:dk_inflation_calibration_basket}
\end{listing}

The Jarrow Yildrim model is specified in a \lstinline!JarrowYildirim! node as outlined in Listing \ref{lst:simulation_model_jy_inflation_configuration}. The \lstinline!RealRate! 
node describes the JY real rate process and has \lstinline!Volatility! and \lstinline!Reversion! nodes that follow those outlined in the interest rate LGM section above. The  
\lstinline!Index! node describes the JY index process and has a \lstinline!Volatility! component that follows the \lstinline!Sigma! component of the FX model above. The 
\lstinline!CalibrationBaskets! node is as outlined above for Dodgson Kainth but up to two baskets may be used and extra inflation instruments are supported in the calibration. More 
information is provided below.

The \lstinline!CalibrationType! determines the calibration approach, if any, that is used to calibrate the various parameters of the model i.e.\ the real rate reversion, the real 
rate volatility and the index volatility. If the \lstinline!CalibrationType! is \lstinline!None!, no calibration is attempted and all parameter values must be explicitly specified.
If the \lstinline!CalibrationType! is \lstinline!BestFit!, the parameters that have \lstinline!Calibrate! set to \lstinline!Y! will be calibrated to the instruments specified in 
the \lstinline!CalibrationBaskets! node. If the \lstinline!CalibrationType! is \lstinline!Bootstrap!, there are a number of options:

\begin{enumerate}
\item
The index volatility parameter may be calibrated, indicated by setting \lstinline!Calibrate! to \lstinline!Y! for that parameter, with both of the real rate parameters not calibrated 
and set explicitly in the \lstinline!RealRate! node. There should be exactly one \lstinline!CalibrationBasket! in the \lstinline!CalibrationBaskets! node and its \lstinline!parameter! 
attribute may be set to \lstinline!Index! or omitted.

\item
One of the real rate parameters may be calibrated, indicated by setting \lstinline!Calibrate! to \lstinline!Y! for that parameter, with the index volatility not calibrated and set 
explicitly in the \lstinline!Volatility! node. There should be exactly one \lstinline!CalibrationBasket! in the \lstinline!CalibrationBaskets! node and its \lstinline!parameter! 
attribute may be set to \lstinline!RealRate! or omitted.

\item
One of the real rate parameters and the index volatility parameter may be calibrated together. There should be exactly two \lstinline!CalibrationBasket! nodes in the \lstinline!CalibrationBaskets! 
node. The \lstinline!parameter! attribute should be set to \lstinline!RealRate! on the \lstinline!CalibrationBasket! node that should be used for the real rate parameter calibration. 
Similarly, the \lstinline!parameter! attribute should be set to \lstinline!Index! on the \lstinline!CalibrationBasket! node that should be used for the index volatility parameter calibration. 
The parameters are calibrated iteratively in turn until the root mean squared error over all calibration instruments in the two baskets is below the tolerance specified by the 
\lstinline!RmseTolerance! in the \lstinline!CalibrationConfiguration! node or until the maximum number of iterations as specified by the \lstinline!MaxIterations! in the 
\lstinline!CalibrationConfiguration! node has been reached. The \lstinline!CalibrationConfiguration! node is optional. If it is omitted, the \lstinline!RmseTolerance! defaults to 0.0001 and the 
\lstinline!MaxIterations! defaults to 50.

\end{enumerate}

Note that it is an error to attempt to calibrate both of the real rate parameters together when \lstinline!CalibrationType! is \lstinline!Bootstrap!. If a parameter is being calibrated 
with \lstinline!CalibrationType! set to \lstinline!Bootstrap!, the \lstinline!ParamType! should be \lstinline!Piecewise!. The \lstinline!TimeGrid! will be overridden for that parameter by 
the relevant calibration instrument times and the parameter's initial values are set to the first element of the \lstinline!InitialValue! list. So, leaving the \lstinline!TimeGrid! node 
empty and giving a single value in the \lstinline!InitialValue! node is the clearest XML setup in this case.

\begin{listing}[H]
\begin{minted}[fontsize=\footnotesize]{xml}
<JarrowYildirim index="EUHICPXT">
  <Currency>EUR</Currency>
  <CalibrationType>Bootstrap</CalibrationType>
  <RealRate>
    <Volatility>
      <Calibrate>Y</Calibrate>
      <VolatilityType>Hagan</VolatilityType>
      <ParamType>Piecewise</ParamType>
      <TimeGrid/>
      <InitialValue>0.0001</InitialValue>
    </Volatility>
    <Reversion>
      <Calibrate>N</Calibrate>
      <ReversionType>HullWhite</ReversionType>
      <ParamType>Constant</ParamType>
      <TimeGrid/>
      <InitialValue>0.5</InitialValue>
    </Reversion>
    <ParameterTransformation>
      <ShiftHorizon>0.0</ShiftHorizon>
      <Scaling>1.0</Scaling>
    </ParameterTransformation>
  </RealRate>
  <Index>
    <Volatility>
      <Calibrate>Y</Calibrate>
      <ParamType>Piecewise</ParamType>
      <TimeGrid/>
      <InitialValue>0.0001</InitialValue>
    </Volatility>
  </Index>
  <CalibrationBaskets>
    <CalibrationBasket parameter="Index">
      <CpiCapFloor>
        <Type>Floor</Type>
        <Maturity>2Y</Maturity>
        <Strike>0.0</Strike>
      </CpiCapFloor>
      ...
    </CalibrationBasket>
    <CalibrationBasket parameter="RealRate">
      <YoYSwap>
        <Tenor>2Y</Tenor>
      </YoYSwap>
      ...
    </CalibrationBasket>
  </CalibrationBaskets>
  <CalibrationConfiguration>
    <RmseTolerance>0.00000001</RmseTolerance>
    <MaxIterations>40</MaxIterations>
  </CalibrationConfiguration>
</JarrowYildirim>
\end{minted}
\caption{Simulation model JY inflation component configuration}
\label{lst:simulation_model_jy_inflation_configuration}
\end{listing}

The \lstinline!CpiCapFloor! and \lstinline!YoYSwap! calibration instruments can be seen in Listing \ref{lst:simulation_model_jy_inflation_configuration}. A \lstinline!YoYCapFloor! is 
also allowed and it has the structure shown in Listing \ref{lst:yoy_cf_calibration_inst}. The \lstinline!Type! may be \lstinline!Cap! or \lstinline!Floor!. The \lstinline!Tenor! should 
be a maturity period e.g.\ \lstinline!5Y!. The \lstinline!Strike! should be an absolute strike level for the year on year cap or floor e.g.\ \lstinline!0.01! for 1\%.

\begin{listing}[H]
\begin{minted}[fontsize=\footnotesize]{xml}
<YoYCapFloor>
  <Type>...</Type>
  <Tenor>...</Tenor>
  <Strike>...</Strike>
</YoYCapFloor>
\end{minted}
\caption{Layout for \lstinline!YoYCapFloor! calibration instrument.}
\label{lst:yoy_cf_calibration_inst}
\end{listing}

% credit models: todo

% commodity models

For commodity simulation we currently provide one model, as described in the methodology appendix. 
Commodity model components are specified by commodity name, by a block as follows

\begin{listing}[H]
\begin{minted}[fontsize=\footnotesize]{xml}
<CrossAssetModel>	
  <!-- ... -->
  <CommodityModels>
    <CommoditySchwartz name="default">
      <Currency>EUR</Currency>
      <CalibrationType>None</CalibrationType>
      <Sigma>
        <Calibrate>Y</Calibrate>
        <InitialValue>0.1</InitialValue>
      </Sigma>
      <Kappa>
        <Calibrate>Y</Calibrate>
        <InitialValue>0.1</InitialValue>
      </Kappa>
      <CalibrationOptions>
           ...
      </CalibrationOptions>
      <DriftFreeState>false</DriftFreeState>
    </CommoditySchwartz>
    <CommoditySchwartz name="WTI">
      <!-- ... -->
    </CommoditySchwartz>
    <CommoditySchwartz name="NG">
      <!-- ... -->
    </CommoditySchwartz>
      <!-- ... -->
  </CommodityModels>
  <!-- ... -->
<CrossAssetModel>	
\end{minted}
\caption{Simulation model commodity configuration}
\label{lst:simulation_model_com_configuration}
\end{listing}

CommoditySchwartz sections are defined by commodity name, but we also support a default configuration as above for the IR and 
FX model parameterisations.  Each component is parameterised in terms of two constant, non time-dependent parameters $\sigma$ and $\kappa$ so far (see appendix).
Within each CommoditySchwartz section, the interpretation of elements is as follows:

\begin{itemize}
\item {\tt Currency: } Currency of denomination
\item {\tt CalibrationType:} Choose between {\em BestFit} and {\em None}.  The choice {\em None} will deactivate calibration as usual. {\em BestFit} will attempt to set the model parameter(s) such that the error in matching calibration instrument prices is minimised.  The option  {\em Bootstrap} is not available here because the model parameters are not time-dependent and the model's degrees of freedom in general do not suffice to perfectly match the calibration instrument prices.
\item {\tt Sigma/Calibrate:} Flag to enable/disable calibration of this particular parameter
\item {\tt Sigma/InitialValue:} Initial value of the constant parameter
\item {\tt Kappa/Calibrate:} Flag to enable/disable calibration of this particular parameter
\item {\tt Kappa/InitialValue:} Initial value of the constant parameter
\item {\tt CalibrationOptions:} Choice of calibration instruments by expiry and strike, strikes can be empty 	(implying the default, ATMF options), or explicitly specified (in terms of commodity prices as absolute strike values). 
\item {\tt DriftFreeState[Optional]:} Boolean to switch between the two implementations of the state variable, see appendix. By default this is set to {\tt false}.        
\end{itemize}

\medskip
Finally, the instantaneous correlation structure is specified as follows.

\begin{listing}[H]
%\hrule\medskip
\begin{minted}[fontsize=\footnotesize]{xml}
<CrossAssetModel>
  <!-- ... -->
  <InstantaneousCorrelations>
    <Correlation factor1="IR:EUR" factor2="IR:USD">0.3</Correlation>
    <Correlation factor1="IR:EUR" factor2="IR:GBP">0.3</Correlation>
    <Correlation factor1="IR:USD" factor2="IR:GBP">0.3</Correlation>
    <Correlation factor1="IR:EUR" factor2="FX:USDEUR">0</Correlation>
    <Correlation factor1="IR:EUR" factor2="FX:GBPEUR">0</Correlation>
    <Correlation factor1="IR:GBP" factor2="FX:USDEUR">0</Correlation>
    <Correlation factor1="IR:GBP" factor2="FX:GBPEUR">0</Correlation>
    <Correlation factor1="IR:USD" factor2="FX:USDEUR">0</Correlation>
    <Correlation factor1="IR:USD" factor2="FX:GBPEUR">0</Correlation>
    <Correlation factor1="FX:USDEUR" factor2="FX:GBPEUR">0</Correlation>
    <!-- ... --> 
  </InstantaneousCorrelations>
</CrossAssetModel>
\end{minted}
\caption{Simulation model correlation configuration}
\label{lst:simulation_model_correlation_configuration}
\end{listing}

Any risk factor pair not specified explicitly here will be assumed to have zero correlation. Note that the commodity components can have non-zero correlations among each other, but correlations to all other CAM components must remain set to zero for the time being.

\simsubsection{Market}\label{sec:sim_market}

The last part of the simulation configuration file covers the specification of the simulated market.  Note that the
simulation model will yield the evolution of risk factors such as short rates which need to be translated into entire
yield curves that can be 'understood' by the instruments which we want to price under scenarios.  

Moreover we need to specify how volatility structures evolve even if we do not explicitly simulate volatility. This 
translation happens based on the information in the {\em simulation market} object, which is configured in the section 
within the enclosing tags {\tt <Market>} and {\tt </Market>}, as shown in the following small example.

It should be noted that equity volatilities are taken to be a curve by default. To simulate an equity volatility surface with smile the xml node {\tt <Surface> } must be supplied.
There are two methods in ORE for equity volatility simulation: 
\begin{itemize}
\item Simulating ATM volatilities only (and shifting other strikes relative to this using the $T_{0}$ smile). In this case set {\tt <SimulateATMOnly>} to true.
\item Simulating the full volatility surface. The node {\tt <SimulateATMOnly>} should be omitted or set to false, and explicit moneyness levels for simulation should be provided.
\end{itemize}

Swaption volatilities are taken to be a surface by default. To simulate a swaption volatility cube with smile the xml node {\tt <Cube> } must be supplied.
There are two methods in ORE for swaption volatility cube simulation: 
\begin{itemize}
\item Simulating ATM volatilities only (and shifting other strikes relative to this using the $T_{0}$ smile). In this case set {\tt <SimulateATMOnly>} to true.
\item Simulating the full volatility cube. The node {\tt <SimulateATMOnly>} should be omitted or set to false, and explicit strike spreads for simulation should be provided.
\end{itemize}

FX volatilities are taken to be a curve by default. To simulate an FX volatility cube with smile the xml node {\tt <Surface> } must be supplied. The surface node contains the moneyness levels to be simulated.

For Yield Curves, Swaption Volatilities, CapFloor Volatilities, Default Curves, Base Correlations and Inflation Curves, a DayCounter may be specified for each risk factor using the node {\tt <DayCounter name="EXAMPLE\_CURVE">}.  
If no day counter is specified for a given risk factor then the default Actual365 is used. To specify a new default for a risk factor type then use the daycounter node without any attribute,  {\tt <DayCounter>}.

For Yield Curves, there are several choices for the interpolation and extrapolation:
\begin{itemize}
\item Interpolation: This can be LogLinear or LinearZero. If not given, the value defaults to LogLinear.
\item Extrapolation: This can be FlatFwd or FlatZero. If not given, the value defaults to FlatFwd.
\end{itemize}

For Default Curve, there is a similar choice for the extrapolation:
\begin{itemize}
\item Extrapolation: This can be FlatFwd or FlatZero. If not given, the value defaults to FlatFwd.
\end{itemize}

\begin{longlisting}
%\hrule\medskip
\begin{minted}[fontsize=\footnotesize]{xml}
<Market>
  <BaseCurrency>EUR</BaseCurrency>
  <Currencies>
    <Currency>EUR</Currency>
    <Currency>USD</Currency>
  </Currencies>
  <YieldCurves>
    <Configuration>
      <Tenors>3M,6M,1Y,2Y,3Y,4Y,5Y,7Y,10Y,12Y,15Y,20Y</Tenors>
      <Interpolation>LogLinear</Interpolation>
      <Extrapolation>FlatFwd</Extrapolation>
      <DayCounter>ACT/ACT</DayCounter> <!-- Sets a new default for all yieldCurves -->
    </Configuration>
  </YieldCurves>
  <Indices>
    <Index>EUR-EURIBOR-6M</Index>
    <Index>EUR-EURIBOR-3M</Index>
    <Index>EUR-EONIA</Index>
    <Index>USD-LIBOR-3M</Index>
  </Indices>
  <SwapIndices>
    <SwapIndex>
      <Name>EUR-CMS-1Y</Name>
      <ForwardingIndex>EUR-EURIBOR-6M</ForwardingIndex>
      <DiscountingIndex>EUR-EONIA</DiscountingIndex>
    </SwapIndex>
  </SwapIndices>
  <DefaultCurves> 
      <Names> 
        <Name>CPTY1</Name> 
        <Name>CPTY2</Name> 
      </Names> 
      <Tenors>6M,1Y,2Y</Tenors> 
      <SimulateSurvivalProbabilities>true</SimulateSurvivalProbabilities> 
      <DayCounter name="CPTY1">ACT/ACT</DayCounter>
      <Extrapolation>FlatFwd</Extrapolation>
  </DefaultCurves> 
  <SwaptionVolatilities>
    <ReactionToTimeDecay>ForwardVariance</ReactionToTimeDecay>
    <Currencies>
      <Currency>EUR</Currency>
      <Currency>USD</Currency>
    </Currencies>
    <Expiries>6M,1Y,2Y,3Y,5Y,10Y,12Y,15Y,20Y</Expiries>
    <Terms>1Y,2Y,3Y,4Y,5Y,7Y,10Y,15Y,20Y,30Y</Terms>
    <Cube>
     <SimulateATMOnly>false</SimulateATMOnly>
     <StrikeSpreads>-0.02,-0.01,0.0,0.01,0.02</StrikeSpreads>
    </Cube>
    <!-- Sets a new daycounter for just the EUR swaptionVolatility surface -->
    <DayCounter ccy="EUR">ACT/ACT</DayCounter> 
  </SwaptionVolatilities> 
  <CapFloorVolatilities>
    <ReactionToTimeDecay>ConstantVariance</ReactionToTimeDecay>
    <Currencies>
      <Currency>EUR</Currency>
      <Currency>USD</Currency>
    </Currencies>
    <DayCounter ccy="EUR">ACT/ACT</DayCounter>
  </CapFloorVolatilities>
  <FxVolatilities>
    <ReactionToTimeDecay>ForwardVariance</ReactionToTimeDecay>
    <CurrencyPairs>
      <CurrencyPair>EURUSD</CurrencyPair>
    </CurrencyPairs>
    <Expiries>6M,1Y,2Y,3Y,4Y,5Y,7Y,10Y</Expiries>
    <Surface>
     <Moneyness>0.5,0.6,0.7,0.8,0.9</Moneyness>
    </Surface>
  </FxVolatilities>
  <EquityVolatilities>
      <Simulate>true</Simulate>
      <ReactionToTimeDecay>ForwardVariance</ReactionToTimeDecay>
      <!-- Alternative: ConstantVariance -->
      <Names>
        <Name>SP5</Name>
        <Name>Lufthansa</Name>
      </Names>
      <Expiries>6M,1Y,2Y,3Y,4Y,5Y,7Y,10Y</Expiries>
      <Surface>
        <SimulateATMOnly>false</SimulateATMOnly><!-- false -->
        <Moneyness>0.1,0.5,1.0,1.5,2.0,3.0</Moneyness><!-- omitted if SimulateATMOnly true -->
      </Surface>
      <TimeExtrapolation>Flat</TimeExtrapolation>
      <StrikeExtrapolation>Flat</StrikeExtrapolation>

  </EquityVolatilities>
  ...
  <BenchmarkCurves>
    <BenchmarkCurve>
      <Currency>EUR</Currency>
      <Name>BENCHMARK_EUR</Name>
  </BenchmarkCurve>
  ...
  </BenchmarkCurves>
  <Securities>
    <Simulate>true</Simulate>
    <Names>
      <Name>SECURITY_1</Name>
      ...
    </Names>
  </Securities>
  <ZeroInflationIndexCurves>
    <Names>
      <Name>EUHICP</Name>
      <Name>UKRPI</Name>
      <Name>USCPI</Name>
      ...
    </Names>
    <Tenors>6M,1Y,2Y,3Y,5Y,7Y,10Y,15Y,20Y</Tenors>
  </ZeroInflationIndexCurves>
  <YYInflationIndexCurves>
    <Names>
      <Name>EUHICPXT</Name>
      ...
    </Names>
    <Tenors>1Y,2Y,3Y,5Y,7Y,10Y,15Y,20Y</Tenors>
  </YYInflationIndexCurves>
  <DefaultCurves>
    <Names>
      <Name>ItraxxEuropeCrossoverS26V1</Name>
      ...
    </Names>
    <Tenors>1Y,2Y,3Y,5Y,10Y</Tenors>
    <SimulateSurvivalProbabilities>true</SimulateSurvivalProbabilities>
  </DefaultCurves>
  <BaseCorrelations/>
  <CDSVolatilities/>
  <Correlations>
    <Simulate>true</Simulate>
    <Pairs>
      <Pair>EUR-CMS-10Y,EUR-CMS-2Y</Pair>
    </Pairs>
    <Expiries>1Y,2Y</Expiries>
  </Correlations>
  <AdditionalScenarioDataCurrencies>
    <Currency>EUR</Currency>
    <Currency>USD</Currency>
  </AdditionalScenarioDataCurrencies>
  <AdditionalScenarioDataIndices>
    <Index>EUR-EURIBOR-3M</Index>
    <Index>EUR-EONIA</Index>
    <Index>USD-LIBOR-3M</Index>
  </AdditionalScenarioDataIndices>
</Market>
\end{minted}
\caption{Simulation market configuration}
\label{lst:simulation_market_configuration}
\end{longlisting}

\todo[inline]{Comment on cap/floor surface structure and reaction to time decay}
