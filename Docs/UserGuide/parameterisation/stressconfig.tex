%--------------------------------------------------------
\subsection{Stress Scenario Analysis: {\tt stressconfig.xml}}\label{sec:stress}
%--------------------------------------------------------

Stress tests can be applied in ORE to the same market segments and with same granularity as described in the sensitivity section \ref{sec:sensitivity}.

\medskip
This file {\tt stressconfig.xml} specifies how stress tests can be configured. The general structure is shown in listing
\ref{lst:stress_config}.

In this example, two stress scenarios ``parallel\_rates'' and ``twist'' are defined. Each scenario definition contains
the market components to be shifted in this scenario in a similar syntax that is also used for the sensitivity
configuration, see \ref{sec:sensitivity}. Components that should not be shifted, can just be omitted in the definition
of the scenario.

However, instead of specifying one shift size per market component, here a whole vector of shifts can be given, with
different shift sizes applied to each point of the curve (or surface / cube).

In case of the swaption volatility shifts, the single value given as {\tt Shift} (without the attributes {\tt expiry}
and {\tt term}) represents a default value that is used whenever no explicit value is given for a expiry / term pair.

In case of discount, index, yield and survival probability curves and cap floor surfaces the shifts can be defined as parRateShifts {\tt ParShift}. 
It's default to shifts on the zero rates. If a curve in a asset class (e.g. rates) is defined as par rate shift, 
all curves in this asset class has to be defined as par rate shift.

\begin{longlisting}
%\hrule\medskip
  \begin{minted}[fontsize=\scriptsize]{xml}
<StressTesting>
  <StressTest id="parallel_rates">
    <DiscountCurves>
      <DiscountCurve ccy="EUR">
        <ShiftType>Absolute</ShiftType>
        <ShiftTenors>6M,1Y,2Y,3Y,5Y,7Y,10Y,15Y,20Y</ShiftTenors>
        <Shifts>0.01,0.01,0.01,0.01,0.01,0.01,0.01,0.01,0.01</Shifts>
        <ParShift>true</ParShift>
      </DiscountCurve>
      ...
    </DiscountCurves>
    <IndexCurves>
      ...
    </IndexCurves>
    <YieldCurves>
      ...
    </YieldCurves>
    <FxSpots>
      <FxSpot ccypair="USDEUR">
        <ShiftType>Relative</ShiftType>
        <ShiftSize>0.01</ShiftSize>
      </FxSpot>
    </FxSpots>
    <FxVolatilities>
      ...
    </FxVolatilities>
    <SwaptionVolatilities>
      <SwaptionVolatility ccy="EUR">
        <ShiftType>Absolute</ShiftType>
        <ShiftExpiries>1Y,10Y</ShiftExpiries>
        <ShiftTerms>5Y</ShiftTerms>
        <Shifts>
          <Shift>0.0010</Shift>
          <Shift expiry="1Y" term="5Y">0.0010</Shift>
          <Shift expiry="1Y" term="5Y">0.0010</Shift>
          <Shift expiry="1Y" term="5Y">0.0010</Shift>
          <Shift expiry="10Y" term="5Y">0.0010</Shift>
          <Shift expiry="10Y" term="5Y">0.0010</Shift>
          <Shift expiry="10Y" term="5Y">0.0010</Shift>
        </Shifts>
      </SwaptionVolatility>
    </SwaptionVolatilities>
    <CapFloorVolatilities>
      <CapFloorVolatility ccy="EUR">
        <ShiftType>Absolute</ShiftType>
        <ShiftExpiries>6M,1Y,2Y,3Y,5Y,10Y</ShiftExpiries>
        <Shifts>0.001,0.001,0.001,0.001,0.001,0.001</Shifts>
        <ParShift>false</ParShift>
      </CapFloorVolatility>
    </CapFloorVolatilities>
  </StressTest>
  <StressTest id="twist">
    ...
  </StressTest>
</StressTesting>
  \end{minted}
\caption{Stress configuration}
\label{lst:stress_config}
\end{longlisting}
