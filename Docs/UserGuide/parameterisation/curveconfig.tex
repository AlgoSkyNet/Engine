%--------------------------------------------------------
\subsection{Curves: {\tt curveconfig.xml}}\label{sec:curveconfig}
%--------------------------------------------------------

The configuration of various term structures required to price a portfolio is covered in a single configuration file
which we will label {\tt curveconfig.xml} in the following though the file name can be chosen by the user. This
configuration determines the composition of 
\begin{itemize}
\item Yield curves % done
\item Default curves % done
\item Inflation curves % done
\item Equity forward price curves % done
\item Swaption volatility structures % done
\item Cap/Floor volatility structures % done
\item FX Option volatility structures % done
\item CDS volatility structures % done
\item Inflation Cap/Floor price surfaces % done
\item Equity volatility structures % done
\item Security spreads and recovery rates % done
\item Base correlation curves % done
\item Correlation termstructures % done
\end{itemize}

This file also contains other market objects such as FXSpots, Security Spreads and Security Rates which are necessary
for the construction of a market.

\subsubsection{Yield Curves}

The top level XML elements for each \lstinline!YieldCurve! node are shown in Listing \ref{lst:top_level_yc}.

\begin{listing}[H]
%\hrule\medskip
\begin{minted}[fontsize=\footnotesize]{xml}
<YieldCurve>
  <CurveId> </CurveId>
  <CurveDescription> </CurveDescription>
  <Currency> </Currency>
  <DiscountCurve> </DiscountCurve>
  <Segments> </Segments>
  <InterpolationVariable> </InterpolationVariable>
  <InterpolationMethod> </InterpolationMethod>
  <ZeroDayCounter> </ZeroDayCounter>
  <Tolerance> </Tolerance>
  <Extrapolation> </Extrapolation>
  <BootstrapConfig>
    ...
  </BootstrapConfig>
</YieldCurve>
\end{minted}
\caption{Top level yield curve node}
\label{lst:top_level_yc}
\end{listing}

The meaning of each of the top level elements in Listing \ref{lst:top_level_yc} is given below. If an element is labelled 
as 'Optional', then it may be excluded or included and left blank.
\begin{itemize}
\item CurveId: Unique identifier for the yield curve.
\item CurveDescription: A description of the yield curve. This field may be left blank.
\item Currency: The yield curve currency.
\item DiscountCurve: If the yield curve is being bootstrapped from market instruments, this gives the CurveId of the
yield curve used to discount cash flows during the bootstrap procedure. If this field is left blank or set equal to the
current CurveId, then this yield curve itself is used to discount cash flows during the bootstrap procedure.
\item Segments: This element contains child elements and is described in the following subsection.
\item InterpolationVariable [Optional]: The variable on which the interpolation is performed. The allowable values are
given in Table \ref{tab:allow_interp_variables}. If the element is omitted or left blank, then it defaults to
\emph{Discount}.
\item InterpolationMethod [Optional]: The interpolation method to use. The allowable values are given in Table
\ref{tab:allow_interp_methods}. If the element is omitted or left blank, then it defaults to \emph{LogLinear}.
\item ZeroDayCounter [Optional]: The day count basis used internally by the yield curve to calculate the time between
dates. In particular, if the curve is queried for a zero rate without specifying the day count basis, the zero rate that
is returned has this basis. If the element is omitted or left blank, then it defaults to \emph{A365}.

\item \lstinline!Tolerance! [Optional]: The tolerance used by the root finding procedure in the bootstrapping algorithm. If the
element is omitted or left blank, then it defaults to \num[scientific-notation=true]{1.0e-12}. It is preferable to use the 
\lstinline!Accuracy! node in the \lstinline!BootstrapConfig! node below for specifying this value. However, if this node is 
explicitly supplied, it takes precedence for backwards compatibility purposes.

\item Extrapolation [Optional]: Set to \emph{True} or \emph{False} to enable or disable extrapolation respectively. If
the element is omitted or left blank, then it defaults to \emph{True}.

\item \lstinline!BootstrapConfig! [Optional]: this node holds configuration details for the iterative bootstrap 
that are described in section \ref{sec:bootstrap_config}. If omitted, this node's default values described 
in section \ref{sec:bootstrap_config} are used.

\end{itemize}

\begin{table}[h]
\centering
  \begin{tabu} to 0.9\linewidth {| X[-1.5,l,m] | X[-5,l,m] |}
    \hline
    \bfseries{Variable} & \bfseries{Description} \\
    \hline
    Zero & The continuously compounded zero rate \\ \hline
    Discount & The discount factor \\ \hline
    Forward & The instantaneous forward rate \\ \hline
  \end{tabu}
  \caption{Allowable interpolation variables.}
  \label{tab:allow_interp_variables}
\end{table}

\begin{table}[h]
\centering
  \begin{tabu} to 0.9\linewidth {| X[-1.5,l,m] | X[-5,l,m] |}
    \hline
    \bfseries{Method} & \bfseries{Description} \\
    \hline
    Linear & Linear interpolation \\ \hline
    LogLinear & Linear interpolation on the natural log of the interpolation variable \\ \hline
    NaturalCubic & Monotonic Kruger cubic interpolation with second derivative at left and right \\ \hline
    FinancialCubic & Monotonic Kruger cubic interpolation with second derivative at left and 
                     first derivative at right \\ \hline
    ConvexMonotone & Convex Monotone Interpolation (Hagan, West) \\ \hline
    ExponentialSplines & Exponential Spline curve fitting, for Fitted Bond Curves only \\ \hline
    NelsonSiegel & Nelson-Siegel curve fitting, for Fitted Bond Curves only \\ \hline
    Svensson & Svensson curve fitting, for Fitted Bond Curves only \\ \hline
  \end{tabu}
  \caption{Allowable interpolation methods.}
  \label{tab:allow_interp_methods}
\end{table}
%- - - - - - - - - - - - - - - - - - - - - - - - - - - - - - - - - - - - - - - -
\subsubsection*{Segments Node} \label{ss:segments_node}
%- - - - - - - - - - - - - - - - - - - - - - - - - - - - - - - - - - - - - - - -
The \lstinline!Segments! node gives the zero rates, discount factors and instruments that comprise the yield curve. This
node consists of a number of child nodes where the node name depends on the segment being described. Each node has a
\lstinline!Type! that determines its structure. The following sections describe the type of child nodes that are
available. Note that for all segment types below, with the exception of \lstinline!DiscountRatio! and \lstinline!AverageOIS!, the 
\lstinline!Quote! elements within the \lstinline!Quotes! node may have an \lstinline!optional! attribute indicating whether or
not the quote is optional. Example:
%\hrule\medskip
\begin{minted}[fontsize=\footnotesize]{xml}
<Quotes>
  <Quote optional="true"></Quote>
</Quotes>
\end{minted}
%\hrule

\subsubsection*{Direct Segment}
When the node name is \lstinline!Direct!, the \lstinline!Type! node has the value \emph{Zero} or \emph{Discount} and the
node has the structure shown in Listing \ref{lst:direct_segment}. We refer to this segment here as a direct segment
because the discount factors, or equivalently the zero rates, are given explicitly and do not need to be
bootstrapped. The \lstinline!Quotes! node contains a list of \lstinline!Quote! elements. Each \lstinline!Quote! element
contains an ID pointing to a line in the {\tt market.txt} file, i.e.\ in this case, pointing to a particular zero rate
or discount factor. The \lstinline!Conventions! node contains the ID of a node in the {\tt conventions.xml} file
described in section \ref{sec:conventions}. The \lstinline!Conventions! node associates conventions with the quotes.

\begin{listing}[H]
%\hrule\medskip
\begin{minted}[fontsize=\footnotesize]{xml}
<Direct>
  <Type> </Type>
  <Quotes>
    <Quote> </Quote>
    <Quote> </Quote>
     <!--...-->
  </Quotes>
  <Conventions> </Conventions>
</Direct>
\end{minted}
\caption{Direct yield curve segment}
\label{lst:direct_segment}
\end{listing}


\subsubsection*{Simple Segment}
When the node name is \lstinline!Simple!, the \lstinline!Type! node has the value \emph{Deposit}, \emph{FRA},
\emph{Future}, \emph{OIS} or \emph{Swap} and the node has the structure shown in Listing \ref{lst:simple_segment}. This
segment holds quotes for a set of deposit, FRA, Future, OIS or swap instruments corresponding to the value in the
\lstinline!Type! node. These quotes will be used by the bootstrap algorithm to imply a discount factor, or equivalently
a zero rate, curve. The only difference between this segment and the direct segment is that there is a
\lstinline!ProjectionCurve! node. This node allows us to specify the CurveId of another curve to project floating rates
on the instruments underlying the quotes listed in the \lstinline!Quote! nodes during the bootstrap procedure. This is
an optional node. If it is left blank or omitted, then the projection curve is assumed to equal the curve being
bootstrapped i.e.\ the current CurveId.

\begin{listing}[H]
%\hrule\medskip
\begin{minted}[fontsize=\footnotesize]{xml}
<Simple>
  <Type> </Type>
  <Quotes>
    <Quote> </Quote>
    <Quote> </Quote>
    <!--...-->
  </Quotes>
  <Conventions> </Conventions>
  <ProjectionCurve> </ProjectionCurve>
</Simple>
\end{minted}
\caption{Simple yield curve segment}
\label{lst:simple_segment}
\end{listing}

\subsubsection*{Average OIS Segment}
When the node name is \lstinline!AverageOIS!, the \lstinline!Type! node has the value \emph{Average OIS} and the node
has the structure shown in Listing \ref{lst:average_ois_segment}. This segment is used to hold quotes for Average OIS
swap instruments. The \lstinline!Quotes! node has the structure shown in Listing \ref{lst:average_ois_quotes}. Each
quote for an Average OIS instrument (a typical example in a USD Overnight Index Swap) consists of two quotes, a vanilla
IRS quote and an OIS-LIBOR basis swap spread quote.  The IDs of these two quotes are stored in the
\lstinline!CompositeQuote! node. The \lstinline!RateQuote! node holds the ID of the vanilla IRS quote and the
\lstinline!SpreadQuote! node holds the ID of the OIS-LIBOR basis swap spread quote.

\begin{listing}[H]
%\hrule\medskip
\begin{minted}[fontsize=\footnotesize]{xml}
<AverageOIS>
  <Type> </Type>
  <Quotes>
    <CompositeQuote> </CompositeQuote>
    <CompositeQuote> </CompositeQuote>
    <!--...-->
  </Quotes>
  <Conventions> </Conventions>
  <ProjectionCurve> </ProjectionCurve>
</AverageOIS>
\end{minted}
\caption{Average OIS yield curve segment}
\label{lst:average_ois_segment}
\end{listing}

\begin{listing}[H]
%\hrule\medskip
\begin{minted}[fontsize=\footnotesize]{xml}
<Quotes>
  <CompositeQuote>
    <SpreadQuote> </SpreadQuote>
    <RateQuote> </RateQuote>
  </CompositeQuote>
  <!--...-->
</Quotes>
\end{minted}
\caption{Average OIS segment's quotes section}
\label{lst:average_ois_quotes}
\end{listing}

\subsubsection*{Tenor Basis Segment}
When the node name is \lstinline!TenorBasis!, the \lstinline!Type! node has the value \emph{Tenor Basis Swap} or
\emph{Tenor Basis Two Swaps} and the node has the structure shown in Listing \ref{lst:tenor_basis_segment}. This segment
is used to hold quotes for tenor basis swap instruments. The quotes may be for a conventional tenor basis swap where
Ibor of one tenor is swapped for Ibor of another tenor plus a spread. In this case, the \lstinline!Type! node has the
value \emph{Tenor Basis Swap}. The quotes may also be for the difference in fixed rates on two fair swaps where one swap
is against Ibor of one tenor and the other swap is against Ibor of another tenor. In this case, the \lstinline!Type!
node has the value \emph{Tenor Basis Two Swaps}. Again, the structure is similar to the simple segment in Listing
\ref{lst:simple_segment} except that there are two projection curve nodes. There is a \lstinline!ProjectionCurveShort!
node for the index with the shorter tenor. This node holds the CurveId of a curve for projecting the floating rates on
the short tenor index. Similarly, there is a \lstinline!ProjectionCurveLong! node for the index with the longer
tenor. This node holds the CurveId of a curve for projecting the floating rates on the long tenor index. These are
optional nodes. If they are left blank or omitted, then the projection curve is assumed to equal the curve being
bootstrapped i.e.\ the current CurveId. However, at least one of the nodes needs to be populated to allow the bootstrap
to proceed.

\begin{listing}[H]
%\hrule\medskip
\begin{minted}[fontsize=\footnotesize]{xml}
<TenorBasis>
  <Type> </Type>
  <Quotes>
    <Quote> </Quote>
    <Quote> </Quote>
    <!--...-->
  </Quotes>
  <Conventions> </Conventions>
  <ProjectionCurveLong> </ProjectionCurveLong>
  <ProjectionCurveShort> </ProjectionCurveShort>
</TenorBasis>
\end{minted}
\caption{Tenor basis yield curve segment}
\label{lst:tenor_basis_segment}
\end{listing}

\subsubsection*{Cross Currency Segment}
When the node name is \lstinline!CrossCurrency!, the \lstinline!Type! node has the value \emph{FX Forward} or
\emph{Cross Currency Basis Swap}. When the \lstinline!Type! node has the value \emph{FX Forward}, the node has the
structure shown in Listing \ref{lst:fx_forward_segment}. This segment is used to hold quotes for FX forward
instruments. The \lstinline!DiscountCurve! node holds the CurveId of a curve used to discount cash flows in the other
currency i.e.\ the currency in the currency pair that is not equal to the currency in Listing
\ref{lst:top_level_yc}. The \lstinline!SpotRate! node holds the ID of a spot FX quote for the currency pair that is
looked up in the {\tt market.txt} file.

\begin{listing}[H]
%\hrule\medskip
\begin{minted}[fontsize=\footnotesize]{xml}
<CrossCurrency>
  <Type> </Type>
  <Quotes>
    <Quote> </Quote>
    <Quote> </Quote>
          ...
  </Quotes>
  <Conventions> </Conventions>
  <DiscountCurve> </DiscountCurve>
  <SpotRate> </SpotRate>
</CrossCurrency>
\end{minted}
\caption{FX forward yield curve segment}
\label{lst:fx_forward_segment}
\end{listing}

When the \lstinline!Type! node has the value \emph{Cross Currency Basis Swap} then the node has the structure shown in
Listing \ref{lst:xccy_basis_segment}. This segment is used to hold quotes for cross currency basis swap instruments. The
\lstinline!DiscountCurve! node holds the CurveId of a curve used to discount cash flows in the other currency i.e.\ the
currency in the currency pair that is not equal to the currency in Listing \ref{lst:top_level_yc}. The
\lstinline!SpotRate! node holds the ID of a spot FX quote for the currency pair that is looked up in the {\tt
  market.txt} file. The \lstinline!ProjectionCurveDomestic! node holds the CurveId of a curve for projecting the
floating rates on the index in this currency i.e.\ the currency in the currency pair that is equal to the currency in
Listing \ref{lst:top_level_yc}. It is an optional node and if it is left blank or omitted, then the projection curve is
assumed to equal the curve being bootstrapped i.e.\ the current CurveId. Similarly, the
\lstinline!ProjectionCurveForeign! node holds the CurveId of a curve for projecting the floating rates on the index in
the other currency. If it is left blank or omitted, then it is assumed to equal the CurveId provided in the
\lstinline!DiscountCurve! node in this segment.

\begin{listing}[H]
%\hrule\medskip
\begin{minted}[fontsize=\footnotesize]{xml}
<CrossCurrency>
  <Type> </Type>
  <Quotes>
    <Quote> </Quote>
    <Quote> </Quote>
          ...
  </Quotes>
  <Conventions> </Conventions>
  <DiscountCurve> </DiscountCurve>
  <SpotRate> </SpotRate>
  <ProjectionCurveDomestic> </ProjectionCurveDomestic>
  <ProjectionCurveForeign> </ProjectionCurveForeign>
</CrossCurrency>
\end{minted}
\caption{Cross currency basis yield curve segment}
\label{lst:xccy_basis_segment}
\end{listing}

\subsubsection*{Zero Spread Segment}

When the node name is \lstinline!ZeroSpread!, the \lstinline!Type!
node has the only allowable value \emph{Zero Spread},  and the node has the structure shown in 
Listing \ref{lst:zero_spread_segment}. This segment is used to build yield
curves which are expressed as a spread over some reference yield curve.

\begin{listing}[H]
%\hrule\medskip
\begin{minted}[fontsize=\footnotesize]{xml}
    <ZeroSpread>
          <Type>Zero Spread</Type>
          <Quotes>
            <Quote>ZERO/YIELD_SPREAD/EUR/BANK_EUR_LEND/A365/2Y</Quote>
            <Quote>ZERO/YIELD_SPREAD/EUR/BANK_EUR_LEND/A365/5Y</Quote>
            <Quote>ZERO/YIELD_SPREAD/EUR/BANK_EUR_LEND/A365/10Y</Quote>
            <Quote>ZERO/YIELD_SPREAD/EUR/BANK_EUR_LEND/A365/20Y</Quote>
          </Quotes>
          <Conventions>EUR-ZERO-CONVENTIONS-TENOR-BASED</Conventions>
          <ReferenceCurve>EUR1D</ReferenceCurve>
    </ZeroSpread>
\end{minted}
\caption{Zero spread yield curve segment}
\label{lst:zero_spread_segment}
\end{listing}


\subsubsection*{Fitted Bond Segment}
\label{sec:fitted_bond_segment}

When the node name is \lstinline!FittedBond!, the \lstinline!Type! node has the only allowable value \emph{FittedBond},
and the node has the structure shown in Listing \ref{lst:fitted_bond_segment}. This segment is used to build yield
curves which are fitted to liquid bond prices. The segment has the following elements:

\begin{itemize}
\item Quotes: a list of bond price quotes, for each security in the list, reference data must be available
\item IborIndexCurves: for each Ibor index that is required by one of the bonds to which the curve is fitted, a mapping
  to an estimation curve for that index must be provided
\item ExtrapolateFlat: if true, the parametric curve is extrapolated flat in the instantaneous forward rate before the
  first and after the last maturity of the bonds in the calibration basket. This avoids unrealistic rates at the short
  end or for long maturities in the resulting curve.
\end{itemize}

The \lstinline!BootstrapConfig! has the following interpretation for a fitted bond curve:

\begin{itemize}
\item Accuracy [Optional, defaults to 1E-12]: the desired accuracy expressed as a weighted rmse in the implied quote,
  where 0.01 = 1 bp. Once this accuracy is reached in a calibration trial, the fit is accepted, no further calibration
  trials re run. In general, this parameter should be set to a higher than the default value for fitted bond curves.
\item GlobalAccuracy [Optional]: the acceptable accuracy. If the Accuracy is not reached in any calibration trial, but
  the GloablAccuracy is met, the best fit among the calibration trials is selected as a result of the calibration. If
  not given, the best calibration trial is compared to the Accuracy parameter instead.
\item DontThrow [Optional, defaults to false]: If true, the best calibration is always accepted as a result, i.e. no
  error is thrown even if the GlobalAccuracy is breached.
\item MaxAttempts [Optional, defaults to 5]: The maximum number of calibration trials. Each calibration trial is run with a random calibratio
  seed. Random calibration seeds are currently only supported for the NelsonSiegel interpolation method.
\end{itemize}

\begin{listing}[H]
%\hrule\medskip
\begin{minted}[fontsize=\footnotesize]{xml}
    <YieldCurve>
      ...
      <Segments>
        <FittedBond>
          <Type>FittedBond</Type>
          <Quotes>
            <Quote>BOND/PRICE/SECURITY_1</Quote>
            <Quote>BOND/PRICE/SECURITY_2</Quote>
            <Quote>BOND/PRICE/SECURITY_3</Quote>
            <Quote>BOND/PRICE/SECURITY_4</Quote>
            <Quote>BOND/PRICE/SECURITY_5</Quote>
          </Quotes>
          <!-- mapping of Ibor curves used in the bonds from which the curve is built -->
          <IborIndexCurves>
            <IborIndexCurve iborIndex="EUR-EURIBOR-6M">EUR-EURIBOR-6M</IborIndexCurve>
          </IborIndexCurves>
          <!-- flat extrapolation before first and after last bond maturity -->
          <ExtrapolateFlat>true</ExtrapolateFlat>
        </FittedBond>
      </Segments>
      <!-- NelsonSiegel, Svensson, ExponentialSplines -->
      <InterpolationMethod>NelsonSiegel</InterpolationMethod>
      <YieldCurveDayCounter>A365</YieldCurveDayCounter>
      <Extrapolation>true</Extrapolation>
      <BootstrapConfig>
        <!-- desired accuracy (in implied quote) -->
        <Accuracy>0.1</Accuracy>
        <!-- tolerable accuracy -->
        <GlobalAccuracy>0.5</GlobalAccuracy>
        <!-- do not throw even if tolerable accuracy is breached -->
        <DontThrow>false</DontThrow>
        <!-- max calibration trials to reach desired accuracy -->
        <MaxAttempts>20</MaxAttempts>
      </BootstrapConfig>
    </YieldCurve>
\end{minted}
\caption{Fitted bond yield curve segment}
\label{lst:fitted_bond_segment}
\end{listing}

\subsubsection*{Yield plus Default Segment}
\label{sec:yield_plus_default}

When the node name is \lstinline!YieldPlusDefault!, the \lstinline!Type! node has the only allowable value \emph{Yield
 Plus Default}, and the node has the structure shown in Listing \ref{lst:yield_plus_default_segment}. This segment is
used to build all-in discounting yield curves from a benchmark curve and (a weighted sum of) default curves. The
construction is in some sense inverse to the benchmark default curve construction, see \ref{ss:benchmark_default_curve}.

\begin{itemize}
\item ReferenceCurve: the benchmark yield curve serving as the basis of the resulting yield curve
\item DefaultCurves: a list of default curves whose weighted sum is added to the benchmark yield curve
\item Weights: a list of weights for the default curves, the number of weights must match the number of default curves
\end{itemize}

Notice that it is explicitly allowed to use default curves in different currencies than the benchmark yield curve. In
the construction, the hazard rate is reinterpreted as an instantaneous forward rate, and the sum of the curves is being
built in the instantaneous forward rate.

The definition takes into account the recovery rates associated to each default curve. The resulting discount factor is
computed as

\begin{equation}
P(0,t) = \prod_i  S_i(t)^{(1-R)w_i}
\end{equation}

where $S_i$ and $R_i$ are the survival probabilities and recovery rates of the source default curves, and $w_i$ are the
weights.

\begin{listing}[H]
%\hrule\medskip
\begin{minted}[fontsize=\footnotesize]{xml}
  <YieldCurve>
    <CurveId>BenchmarkPlusDefault</CurveId>
    <CurveDescription>USD Libor 3M + 0.5 x CDX.NA.HY + 0.5 x EUR.10BP</CurveDescription>
    <Currency>USD</Currency>
    <DiscountCurve/>
    <Segments>
      <YieldPlusDefault>
        <Type>Yield Plus Default</Type>
        <ReferenceCurve>USD3M</ReferenceCurve>
        <DefaultCurves>
          <DefaultCurve>Default/USD/CDX.NA.HY</DefaultCurve>
          <DefaultCurve>Default/EUR/EUR.10BP</DefaultCurve>
        </DefaultCurves>
        <Weights>
          <Weight>0.5</Weight>
          <Weight>0.5</Weight>
        </Weights>
      </YieldPlusDefault>
    </Segments>
  </YieldCurve>
</YieldCurves>
\end{minted}
\caption{Yield plus default curve segment}
\label{lst:yield_plus_default_segment}
\end{listing}

\subsubsection*{Weighted Average Segment}
\label{sec:yield_plus_default}

When the node name is \lstinline!WeightedAverage!, the \lstinline!Type! node has the only allowable value
\emph{WeightedAverage}, and the node has the structure shown in Listing \ref{lst:weighted_average_segment}. This segment
is used to build a curve with instantaneous forward rates that are the weighted sum of instantaneous forward rates of
reference curves. This way a projection curve for non-standard Ibor curves can be build, e.g. to project a Euribor2M
index using the curves for 1M and 3M.

\begin{itemize}
\item ReferenceCurve1: the first source curve
\item ReferenceCurve2: the second source curve
\item Weight1: the weight of the first curve
\item Weights: the weight of the second curve
\end{itemize}

If $P_1(0,t)$ and $P_2(0,t)$ denote the discount factors of the two reference curves, the discount factor $P(0,t)$ of
the resulting curve is defined as

\begin{equation}
P(0,t) = P_1(0,t)^{w_1}P_2(0,t)^{w_2}
\end{equation}

\begin{listing}[H]
%\hrule\medskip
\begin{minted}[fontsize=\footnotesize]{xml}
<YieldCurve>
  <CurveId>EUR2M</CurveId>
  <CurveDescription>Euribor2M forwarding curve, interpolated from 1M and 3M</CurveDescription>
  <Currency>EUR</Currency>
  <DiscountCurve>EUR1D</DiscountCurve>
  <Segments>
    <WeightedAverage>
      <Type>Weighted Average</Type>
      <ReferenceCurve1>EUR1M</ReferenceCurve1>
      <ReferenceCurve2>EUR3M</ReferenceCurve2>
      <Weight1>0.5</Weight1>
      <Weight2>0.5</Weight2>
    </WeightedAverage>
  </Segments>
</YieldCurve>
\end{minted}
\caption{Weighted Average yield curve segment}
\label{lst:weighted_average_segment}
\end{listing}

\subsubsection{Default Curves from CDS}

Default curves can be bootstrapped from credit default swap (CDS) market instruments. The CDS market quotes may be given as a par spread or as an upfront price. These market quotes are documented in Sections \ref{md:cds_spread_quote} and \ref{md:cds_price_quote} respectively. The bootstrap also requires a market recovery rate quote and this is documented in Section \ref{md:cds_recovery_rate_quote}.

Listing \ref{lst:defaultcurve_cds_configuration} outlines the configuration required to build a default curve from CDS quotes. The meaning of each of the nodes is as follows:

\begin{itemize}
\item
\lstinline!CurveId!: Unique identifier for the bootstrapped default curve. For index term curves a suffix \lstinline!_5Y! should be appended to the name indicating the index term, since this is the prefered name looked up by index cds and index cds option pricers. If such a curve is not found, the pricers will fall back to the specified credit curve id without suffix, i.e. following this naming convention is not mandatory, but recommended.

\item \lstinline!CurveDescription! [Optional]:
A description of the default curve. It is for information only and may be left blank.

\item \lstinline!Currency!:
The default curve's currency.

\item \lstinline!Type!:
For a default curve built from CDS, the \lstinline!Type! should be set to \lstinline!SpreadCDS! if the \lstinline!Quotes! reference CDS spread quotes or \lstinline!Price! if the \lstinline!Quotes! reference upfront price quotes.

\item \lstinline!DiscountCurve!:
A reference to a valid discount curve specification that will be used to discount cashflows during the bootstrap process. It should be of the form \lstinline!Yield/Currency/curve_name! where \lstinline!curve_name! is the name of a yield curve defined in the yield curve configurations.

\item \lstinline!DayCounter!:
The day counter used to convert from dates to times in the underlying structure. Allowable values are given in the Table \ref{tab:daycount}.

\item \lstinline!RecoveryRate!:
A valid recovery rate quote name as documented in Section \ref{md:cds_recovery_rate_quote}.

\item \lstinline!StartDate! [Optional]:
The \lstinline!StartDate! is optional and is used for index CDS to specify the start date of the index CDS. This is then used to determine the maturity associated with the index CDS spread quotes which are quoted with a tenor. For single name CDS, this should be omitted.

\item \lstinline!RunningSpread! [Optional]:
  The \lstinline!RunningSpread! is optional and is used for
  \begin(itemize}
    \item stripping cds curves from upfront quotes. Alternatively the upfront quote labels can contain the running spread.
    \item the calculation of the ATM level in cds and index cds volatility surfaces that are strike dependent
  \end{itemize}
  The value should be set whenever one of these use cases applies.

\item \lstinline!IndexTerm! [Optional]: The \lstinline!IndexTerm! is optional and is used to set up index cds curves for
  a specific term. If several quotes are specified explicitly or via wildcards, the quote matching the specified term is
  used to build a flat curve. If no quote is available for the specified term, an interpolated term quote will be built
  using the adjacent terms of the provided quotes.

\item \lstinline!Quotes!:
The \lstinline!Quotes! element should be populated with a list of valid \lstinline!Quote! elements. If the \lstinline!Type! is \lstinline!SpreadCDS!, the quotes should be CDS spread quote strings as documented in Section \ref{md:cds_spread_quote} and if \lstinline!Type! is \lstinline!Price!, the quotes should be CDS upfront price quote strings as documented in Section \ref{md:cds_price_quote}. The attribute \lstinline!optional! in the \lstinline!Quote! element should be set to \lstinline!true! if the associated quote is optional and set to \lstinline!false! if the associated quote is mandatory. If a quote is mandatory and not found in the market, the default curve building will fail. The attribute \lstinline!optional! may be omitted from the quote element. In this case, it defaults to \lstinline!false! and the quote is mandatory. Note also that instead of a list of explicit quotes, a single quote may be provided with the wildcard character \lstinline!*!. In this case, the market is searched for quotes matching the pattern. For example, \lstinline!CDS/CREDIT_SPREAD/JPM/SNRFOR/USD/XR14/*! would return all quotes in the market that start with \lstinline!CDS/CREDIT_SPREAD/JPM/SNRFOR/USD/XR14!.

\item \lstinline!Conventions!:
The name of a valid set of CDS conventions, as documented in Section \ref{sss:cds_conventions}, to use in the bootstrap.

\item \lstinline!Extrapolation! [Optional]:
A boolean value indicating if the bootstrapped default curve allows for extrapolation past the last pillar date. Allowable boolean values are given in the Table \ref{tab:boolean_allowable}. If omitted, it defaults to \lstinline!true!.

\item \lstinline!ImplyDefaultFromMarket! [Optional]:
A boolean value indicating if a reference entity's default should be implied from the market data. Allowable boolean values are given in the Table \ref{tab:boolean_allowable}. If omitted, it defaults to \lstinline!false!. When a default credit event has been determined for an entity, certain market data providers continue to supply a recovery rate from the credit event determination date up to the credit event auction settlement date. In this period, no CDS spreads or upfront prices are provided. When this flag is \lstinline!true!, we assume an entity is in default and awaiting a credit event auction if we find a recovery rate in the market but no CDS spreads or upfront prices. In this case, we build a survival probability curve with a value of close to but greater than 0.0 for one day after the valuation date. This will give an approximation to the correct price for CDS and index CDS in these cases. When this flag is \lstinline!false!, we make no such assumption and the default curve building will fail.

\item \lstinline!BootstrapConfig! [Optional]:
This node holds configuration details for the iterative bootstrap that are described in section \ref{sec:bootstrap_config}. If omitted, this node's default values described in section \ref{sec:bootstrap_config} are used.

\item AllowNegativeRates [Optional]: If set to false (default) negative instantaneous hazard rates implied by the CDS
  quotes lead to an exception or - if the DontThrow flag in the BootstrapConfig is set to true - to a zero instantaneous
  hazard rate in the relevant segment of the curve. In the latter case the market CDS instrument associated to the
  critical curve segment will not match the market quote exactly. If set to true, negative instantaneous hazard rates
  will be allowed during the bootstrap (in a range that is technically defined by the MaxFactor and MaxAttempts
  parameters for the survival probability in the bootstrap config).

\end{itemize}

\begin{longlisting}
%\hrule\medskip
\begin{minted}[fontsize=\footnotesize]{xml}
<DefaultCurve>
  <CurveId>...</CurveId>
  <CurveDescription>...</CurveDescription>
  <Currency>USD</Currency> 
  <Type>...</Type>
  <DiscountCurve>...</DiscountCurve>
  <DayCounter>...</DayCounter>
  <RecoveryRate>...</RecoveryRate>
  <StartDate>...</StartDate>
  <RunningSpread>...</RunningSpread>
  <IndexTerm>...</IndexTerm>
  <Quotes>
    <Quote optional="true">...</Quote>
    ...
  </Quotes>
  <Conventions>...</Conventions>
  <Extrapolation>...</Extrapolation>
  <ImplyDefaultFromMarket>...</ImplyDefaultFromMarket>
  <BootstrapConfig>
    ...
  </BootstrapConfig>
  <AllowNegativeRates>...</AllowNegativeRates>
</DefaultCurve>
\end{minted}
\caption{Default curve configuration based on CDS quotes}
\label{lst:defaultcurve_cds_configuration}
\end{longlisting}

\subsubsection{Benchmark Default Curve}\label{ss:benchmark_default_curve}

Default curves can be set up as a difference curve of two yield curves as shown in listing
\ref{lst:defaultcurve_benchmark}. A typical use case is to back out a default curve from an all-in discounting curve
fitted to a series of liquid bond prices (the ``source curve'') and a benchmark curve representing a benchmark funding
level. The default curve can then be used in models consuming a benchmark curve and a default curve.

If $P_B(0,t)$ and $P_S(0,t)$ denote the discount factors of the given benchmark
and source curve respectively the resulting default term structures has survival probabilities

\begin{equation}
S(t) = \left( P_S(0,t) / P_B(0,t) \right) ^ { 1/(1-R) }
\end{equation}

on the given pillar times. Her, $R$ is the specified recovery rate. If the recovery rate is zero, which is the usual
case, the formula simplifies to

\begin{equation}
  S(0,t) = P_S(0,t) / P_B(0,t)
\end{equation}

The interpolation is backward flat in the hazard rate. The meaning of each node is as follows:

\begin{itemize}
\item CurveId: The curve id.
\item CurveDescription: The curve description.
\item Currency: The currency of the curve.
\item Type: Must be set to Benchmark.
\item DayCounter: The day counter used to convert dates to times.
\item RecoveryRate [optional]: The recovery rate for the resulting default curve. Defaults to zero. The recovery rate
  can be a market quote as usual or also a fixed numeric value for this curve type.
\item BenchmarkCurve: The benchmark yield curve, typically this is the standard Ibor curve in the currence
  (e.g. EUR-EURIBOR-6M, USD-Libor-3M, ...)
\item SourceCurve: The all-in discounting curve.
\item Pillars: The pillars on which to match the source curve
\item SpotLag: The pillar dates are derived using the spot lag and the tenors as specified in the Pillars node using the
  specified calendar.
\item Calendar: The calendar used to derive the pillar dates.
\item Extrapolation [Optional]: If set to true, the curve is extrapoalted beyond the last pillar. Defaults to true.
\item AllowNegativeRates [Optional]: If set to true, the check for non-negative instantaneous hazard rate in the result
  curve is disabled, i.e. the relation $P_S(0,t) \leq P_B(0,t)$ is not enforced. This flag should be enabled with care,
  i.e.  a model consuming the resulting default curve must be able to handle negative hazard rates appropriately. On the
  other hand in some situations it is natural that the source curve rates are below the benchmark rates. Defaults to
  false.
\end{itemize}

\begin{longlisting}
%\hrule\medskip
\begin{minted}[fontsize=\footnotesize]{xml}
    <DefaultCurve>
      <CurveId>BOND_YIELD_EUR_OVER_OIS</CurveId>
      <CurveDescription>Default curve derived as bond yield curve over Eonia</CurveDescription>
      <Currency>EUR</Currency>
      <Type>Benchmark</Type>
      <DayCounter>A365</DayCounter>
      <RecoveryRate>RECOVERY_RATE/RATE//SNR/USD</RecoveryRate>
      <BenchmarkCurve>Yield/EUR/EUR6M</BenchmarkCurve>
      <SourceCurve>Yield/EUR/BOND_YIELD_EUR</SourceCurve>
      <Pillars>1Y,2Y,3Y,4Y,5Y,7Y,10Y</Pillars>
      <SpotLag>0</SpotLag>
      <Calendar>TARGET</Calendar>
      <Extrapolation>true</Extrapolation>
      <AllowNegativeRates>false</AllowNegativeRates>
    </DefaultCurve>
  </DefaultCurves>
\end{minted}
\caption{Benchmark default curve}
\label{lst:defaultcurve_benchmark}
\end{longlisting}

\subsubsection{Multi-Section Default Curve}\label{ss:multisection_default_curve}

Default curves can be build by stitching together instantaneous hazard rates from multiple source curves for multiple
date ranges as shown in listing \ref{lst:defaultcurve_multisection}.

The hazard rate of the resulting curve is taken from the $i$th input curve ($i=0,1,2,\ldots$) for dates before the $i$th
switch date and (if $i>0$) on or after the $i-1$th switch date. The day counter of all input curves should be equal to
the day counter of the result curve. The interpolation is hardcoded as backward flat in the hazard rate.

If not given, the recovery rate $R$ is assumed to be zero. The result default curve's survival probabiltiies are
computed as

\begin{equation}
  S(t) = \left[ \left(\frac{P_{S,n}(t)}{P_{S,n}(t_{n})}\right)^{(1-R_n)} \Pi_{i=0}^{n-1} \left(\frac{P_{S,i}(t_{i+1})}{P_{S,i}(t_{i})}\right)^{(1-R_i)} \right] ^ { \frac{1}{1-R} }
\end{equation}

where $P_{S,i}$ is the survival probability of the $i$th source curve, $R_i$ is the associated recovery rate for the
$i$th source curve, $n$ is chosen such that $P_{S,n}$ is the relevant source curve for time $t$ according to the given
switch dates and curve $i$ is relevant for times in $[t_i,t_{i+1}]$.

The meaning of each node is as follows:

\begin{itemize}
\item CurveId: The curve id.
\item CurveDescription: The curve description.
\item Currency: The currency of the curve.
\item Type: Must be set to MutliSection.
\item SourceCurves: The list of input default curves.
\item SwitchDates: The list of dates where we switch from one input curve to the next. The number of switch dates must
  be one less than the number of source curves.
\item DayCounter: The day counter used to convert dates to times.
\item RecoveryRate [optional]: The recovery rate for the resulting default curve. Defaults to zero. The recovery rate
  can be a market quote as usual or also a fixed numeric value for this curve type.
\item Extrapolation [Optional]: If set to true, the curve is extrapoalted beyond the last pillar. Defaults to true.
\end{itemize}


\begin{longlisting}
%\hrule\medskip
\begin{minted}[fontsize=\small]{xml}
<DefaultCurve>
   <CurveId>MyMultiSectionDefaultCurve</CurveId>
   <CurveDescription>Default curve with multiple sections</CurveDescription>
   <Currency>USD</Currency>
   <Type>MultiSection</Type>
   <SourceCurves>
     <SourceCurve>Default/USD/Generic_AA_Curve</SourceCurve>
     <SourceCurve>Default/USD/Generic_B_Curve</SourceCurve>
     <SourceCurve>Default/USD/Generic_C_Curve</SourceCurve>
   </SourceCurves>
   <SwitchDates>
     <SwitchDate>2020-10-01</SwitchDate>
     <SwitchDate>2021-12-01</SwitchDate>
   <SwitchDates>
   <Extrapolation>true</Extrapolation>
   <DayCounter>A365</DayCounter>
   <RecoveryRate>RECOVERY_RATE/RATE/NAME/SR/USD</RecoveryRate>
</DefaultCurve>
\end{minted}
\caption{Multi-Section default curve}
\label{lst:defaultcurve_multisection}
\end{longlisting}


\subsubsection{Swaption Volatility Structures}
\label{sss:swaptionconfig}

Listing \ref{lst:swaptionvol_configuration} shows an example of a Swaption volatility structure configuration.

\begin{longlisting}
%\hrule\medskip
\begin{minted}[fontsize=\footnotesize]{xml}
<SwaptionVolatilities>    
  <SwaptionVolatility>
    <CurveId>EUR_SW_N</CurveId>
    <CurveDescription>EUR normal swaption volatilities</CurveDescription>
    <Dimension>ATM</Dimension>
    <VolatilityType>Normal</VolatilityType>
    <Interpolation>Hagan2002NormalZeroBeta</Interpolation>
    <ParametricSmileConfiguration>
      <Parameters>
        <Parameter>
          <Name>alpha</Name>
          <InitialValue>0.0050</InitialValue>
          <IsFixed>false</IsFixed>
        </Parameter>
        <Parameter>
          <Name>beta</Name>
          <InitialValue>0.0</InitialValue>
          <IsFixed>true</IsFixed>
        </Parameter>
        <Parameter>
          <Name>nu</Name>
          <InitialValue>0.30</InitialValue>
          <IsFixed>false</IsFixed>
        </Parameter>
        <Parameter>
          <Name>rho</Name>
          <InitialValue>0.0</InitialValue>
          <IsFixed>false</IsFixed>
        </Parameter>
      </Parameters>
      <Calibration>
        <MaxCalibrationAttempts>10</MaxCalibrationAttempts>
        <ExitEarlyErrorThreshold>0.005</ExitEarlyErrorThreshold>
        <MaxAcceptableError>0.05</MaxAcceptableError>
      </Calibration>
    </ParametricSmileConfiguration>
    <Extrapolation>Flat</Extrapolation>
    <OutputVolatilityType>Normal</OutputVolatilityType>
    <DayCounter>Actual/365 (Fixed)</DayCounter>
    <Calendar>TARGET</Calendar>
    <BusinessDayConvention>Following</BusinessDayConvention>
    <!-- ATM matrix specification -->
    <OptionTenors>1M,3M,6M,1Y,2Y,3Y,4Y,5Y,7Y,10Y,15Y,20Y,25Y,30Y</OptionTenors>
    <SwapTenors>1Y,2Y,3Y,4Y,5Y,7Y,10Y,15Y,20Y,25Y,30Y</SwapTenors>
    <ShortSwapIndexBase>EUR-CMS-1Y</ShortSwapIndexBase>
    <SwapIndexBase>EUR-CMS-30Y</SwapIndexBase>
    <!-- Smile section specification -->
    <SmileOptionTenors>6M,1Y,10Y</SmileOptionTenors>
    <SmileSwapTenors>2Y,5Y</SmileSwapTenors>
    <SmileSpreads>-0.02,-0.01,0.01,0.02</SmileSpreads>
    <QuoteTag/>
  </SwaptionVolatility>
  ...
</SwaptionVolatilities>
\end{minted}
\caption{Swaption volatility configuration}
\label{lst:swaptionvol_configuration}
\end{longlisting}

The meaning of each of the elements in Listing \ref{lst:swaptionvol_configuration} is given below. 
%If an element is labelled as 'Optional', then it may be excluded or included and left blank.

\begin{itemize}
\item CurveId: Unique identifier of the swaption volatility structure
\item CurveDescription [Optional]: A description of the volatility structure, may be left blank.
\item Dimension: Distinguishes at-the-money matrices and full volatility cubes. \\ Allowable values: {\tt ATM, Smile}
\item VolatilityType: Specifies the type of market volatility inputs. \\ 
Allowable values: {\tt Normal, Lognormal, ShiftedLognormal} \\
In the case of {\tt ShiftedLognormal}, a matrix of shifts (by option and swap tenor) has to be provided in the market data input. 
\item Interpolation: Optional. Possible values: Linear, Hagan2002Lognormal, Hagan2002Normal, Hagan2002NormalZeroBeta,
  Antonov2015FreeBoundaryNormal, KienitzLawsonSwaynePde, FlochKennedy. If not given, defaults to Linear.
\item ParametricSmileConfiguration: Optional. Applies to SABR only. If not given, default values are used. Allows to
  specify initial values for calibrated parameters, to exclude single parameters from calibration and to set calibration
  parameters.
\item Extrapolation: Specifies the extrapolation behaviour in all dimensions. \\ Allowable values: {\tt Linear, Flat,
  None}
\item OutputVolatilityType: Optional, defaults to input volatility type and applies to SABR variants only. For
  Interpolation = Linear it must be set to the input VolatilityType or (better) omitted. Possible values: Normal,
  Lognormal (alias for ShiftedLognormal, shift is always implied from input data), ShiftedLognormal.
\item DayCounter: The term structure's day counter used in date to time conversions
\item Calendar: The term structure's calendar used in option tenor to date conversions
\item BusinessDayConvention: The term structure's business day convention used in option tenor to date conversion
\item ATM Matrix specification, required for both Dimension choices:
  \begin{itemize}
  \item OptionTenors: Option expiry in period form
  \item SwapTenors: Underlying Swap term in period form
  \item ShortSwapIndexBase: Swap index (ORE naming convention, e.g. EUR-CMS-1Y) used to compute ATM strikes for tenors up to and including the tenor given in the index (1Y in this example)
  \item SwapIndexBase: Swap index used to compute ATM strikes for tenors longer than the one defined by the short index 
  \end{itemize}
\item Smile section specification, this part is required when Dimension is set to {\tt Smile}, otherwise it can be omitted:
  \begin{itemize}
  \item SmileOptionTenors: Option expiries, in period form, where smile section data is to be taken into account
  \item SmileSwapTenors: Underlying Swap term, in period form, where smile section data is to be taken into account
  \item SmileSpreads: Strikes in smile direction expressed as strike spreads, relative to the ATM strike at the expiry/term point of the ATM matrix. Note that trailing 0s are not ignored.
  \end{itemize}
\item QuoteTag [Optional]: If non-empty, a tag will be included in the market datum labels. This can be used to set up
  underlying specific volatility date. For example, if the quote tag is set to EUR-EURIBOR-3M, the market datum labels
  will be \verb+SWAPTION/RATE_LNVOL/EUR/EUR-EURIBOR-3M/5Y/10Y/ATM+ instead of
  \verb+SWAPTION/RATE_LNVOL/EUR/5Y/10Y/ATM+. See section \ref{ss:swaptionvolatilitydata}.
\end{itemize}

\subsubsection{Cap Floor Volatility Structures}

The cap volatility structure parameterisation allows the user to pick out term cap volatilities in the market data and define how they should be stripped to create an optionlet volatility structure. The parameterisation allows for three separate types of input term cap volatility structures:

\begin{enumerate}
\item A strip of at-the-money (ATM) cap volatilities.
\item A cap maturity tenor by absolute cap strike grid of cap volatilities.
\item A combined structure containing both the ATM cap volatilities and the maturity by strike grid of cap volatilities.
\end{enumerate}

The input cap volatilities may be normal, lognormal or shifted lognormal. The structure of the market quotes is provided in Table \ref{tab:capfloor_implvol_quote}.

The structure of the XML, i.e.\ the nodes that are necessary, used and ignored, and the way that the optionlet volatilities are stripped hinges on the value of the \lstinline!InterpolateOn! node. This node may be set to \lstinline!TermVolatilities! or \lstinline!OptionletVolatilities!.

When set to \lstinline!TermVolatilities!, a column of sequential caps or floors, are created for each strike level out to the maximum cap maturity configured. In other words, if the index tenor is 6M, the first cap created would have a maturity of 1Y, the second cap 18M, the third cap 2Y and so on until we have a cap with maturity equal to the maximum maturity tenor in the configuration. The volatility for each of these caps or floors is then interpolated from the term cap volatility surface using the configured interpolation. Finally, the optionlet volatility at each cap or floor maturity, starting from the first, is derived in turn such that the column of cap or floor volatilities are matched.

When set to \lstinline!OptionletVolatilities!, the optionlet volatility structure pillar dates are set to the fixing dates on the last caplet on each of the configured caps or floors i.e.\ caps or floors with the maturities in the configured \lstinline!Tenors! or \lstinline!AtmTenors!. The optionlet volatilities on these pillar dates are then solved for such that the configured cap or floor volatilities are matched. In the following sections, we describe four XML configurations separately for clarity:

\begin{enumerate}
\item ATM curve with interpolation on term volatilities.
\item ATM curve with interpolation on optionlet volatilities.
\item Surface, possibly including an ATM column, with interpolation on term volatilities.
\item Surface, possibly including an ATM column, with interpolation on optionlet volatilities.
\end{enumerate}

Listing \ref{lst:capfloorvol_atm_configuration_term} shows the layout for parameterising an ATM cap volatility curve with interpolation on term volatilities. Nodes that have no effect for this parameterisation but that are allowed by the schema are not referenced. The meaning of each of the nodes is as follows:

\begin{itemize}
\item
\lstinline!CurveId!: Unique identifier for the cap floor volatility structure.

\item \lstinline!CurveDescription! [Optional]:
A description of the volatility structure. It is for information only and may be left blank.

\item \lstinline!VolatilityType!:
Indicates the cap floor volatility type. It may be \lstinline!Normal!, \lstinline!Lognormal! or \lstinline!ShiftedLognormal!. Note that this then determines which market data points are looked up in the market when creating the ATM cap floor curve and how they are interpreted when stripping the optionlets. In particular, the market will be searched for market data points of the form \lstinline!CAPFLOOR/RATE_NVOL/Currency/Tenor/IndexTenor/1/1/0!, \lstinline!CAPFLOOR/RATE_LNVOL/Currency/Tenor/IndexTenor/1/1/0! or \lstinline!CAPFLOOR/RATE_SLNVOL/Currency/Tenor/IndexTenor/1/1/0! respectively.

\item \lstinline!Extrapolation!:
Indicates the extrapolation in the time direction before the first optionlet volatility and after the last optionlet volatility. The extrapolation occurs on the stripped optionlet volatilities. The allowable values are \lstinline!None!, \lstinline!Flat! and \lstinline!Linear!. If set to \lstinline!None!, extrapolation is turned off and an exception is thrown if the optionlet surface is queried outside the allowable times. If set to \lstinline!Flat!, the first optionlet volatility is used before the first time and the last optionlet volatility is used after the last time. If set to \lstinline!Linear!, the interpolation method configured in \lstinline!InterpolationMethod! is used to extrapolate.

\item \lstinline!InterpolationMethod! [Optional]:
Indicates the interpolation in the time direction. As \lstinline!InterpolateOn! is set to \lstinline!TermVolatilities! here, the interpolation is used in the stripping process to interpolate the term cap floor volatility curve as explained above. It is also used to interpolate the optionlet volatilities when an optionlet volatility is queried from the stripped optionlet structure. The allowable values are \lstinline!Bilinear! and \lstinline!BicubicSpline!. If not set, \lstinline!BicubicSpline! is assumed. Obviously, as we are describing an ATM curve here, there is no interpolation in the strike direction so when \lstinline!Bilinear! is set the time interpolation is linear and when \lstinline!BicubicSpline! is set the time interpolation is cubic spline.

\item \lstinline!IncludeAtm!:
A boolean value indicating if an ATM curve should be used. Allowable boolean values are given in the Table \ref{tab:boolean_allowable}. As we are describing an ATM curve here, this node should be set to \lstinline!true! as shown in \ref{lst:capfloorvol_atm_configuration_term}.

\item \lstinline!DayCounter!:
The day counter used to convert from dates to times in the underlying structure. Allowable values are given in the Table \ref{tab:daycount}.

\item \lstinline!Calendar!:
The calendar used to advance dates by periods in the underlying structure. In particular, it is used in deriving the cap maturity dates from the configured cap tenors. Allowable values are given in the Table \ref{tab:calendar}.

\item \lstinline!BusinessDayConvention!:
The business day convention used to advance dates by periods in the underlying structure. In particular, it is used in deriving the cap maturity dates from the configured cap tenors. Allowable values are given in the Table \ref{tab:allow_stand_data} under \lstinline!Roll Convention!.

\item \lstinline!Tenors! [Optional]:
A comma separated list of valid tenor strings giving the cap floor maturity tenors to be used in the ATM curve. If omitted, the tenors for the ATM curve must be provided in the \lstinline!AtmTenors! node instead. If the tenors are provided here, the \lstinline!AtmTenors! node may be omitted.

\item \lstinline!OptionalQuotes! [Optional]:
A boolean flag to indicate whether market data quotes for all tenors are required. If true, we attempt to build the curve from whatever quotes are provided. If false, the curve will fail to build if any quotes are missing. This also applies to quotes for the \lstinline!AtmTenors!. Default value is false.

\item \lstinline!IborIndex!:
A valid interest rate index name giving the index underlying the cap floor quotes. Allowable values are given in the Table \ref{tab:indices}.

\item \lstinline!DiscountCurve!:
A reference to a valid discount curve specification that will be used to discount cashflows during the stripping process. It should be of the form \lstinline!Yield/Currency/curve_name! where \lstinline!curve_name! is the name of a yield curve defined in the yield curve configurations.

\item \lstinline!AtmTenors! [Optional]:
A comma separated list of valid tenor strings giving the cap floor maturities to be used in the ATM curve. If omitted, the tenors for the ATM curve must be provided in the \lstinline!Tenors! node instead. If the tenors are provided here, the \lstinline!Tenors! node may be omitted.

\item \lstinline!SettlementDays! [Optional]:
Any non-negative integer is allowed here. If omitted, it is assumed to be 0. If provided the reference date of the term volatility curve and the stripped optionlet volatility structure will be calculated by advancing the valuation date by this number of days using the configured calendar and business day convention. In general, this should be omitted or set to 0.

\item \lstinline!InterpolateOn!:
As referenced above, the allowable values are \lstinline!TermVolatilities! or \lstinline!OptionletVolatilities!. As we are describing here an ATM curve with interpolation on term volatilities, this should be set to \lstinline!TermVolatilities! as shown in Listing \ref{lst:capfloorvol_atm_configuration_term}.

\item \lstinline!BootstrapConfig! [Optional]:
This node holds configuration details for the iterative bootstrap that are described in section \ref{sec:bootstrap_config}. If omitted, this node's default values described in section \ref{sec:bootstrap_config} are used.

\end{itemize}

\begin{longlisting}
\begin{minted}[fontsize=\footnotesize]{xml}
<CapFloorVolatility>
  <CurveId>...</CurveId>
  <CurveDescription>...</CurveDescription>
  <VolatilityType>...</VolatilityType>
  <Extrapolation>...</Extrapolation>
  <InterpolationMethod>...</InterpolationMethod>
  <IncludeAtm>true</IncludeAtm>
  <DayCounter>...</DayCounter>
  <Calendar>...</Calendar>
  <BusinessDayConvention>...</BusinessDayConvention>
  <Tenors>...</Tenors>
  <OptionalQuotes>...</OptionalQuotes>
  <IborIndex>...</IborIndex>
  <DiscountCurve>...</DiscountCurve>
  <AtmTenors>...</AtmTenors>
  <SettlementDays>...</SettlementDays>
  <InterpolateOn>TermVolatilities</InterpolateOn>
  <BootstrapConfig>...</BootstrapConfig>
</CapFloorVolatility>
\end{minted}
\caption{ATM cap floor configuration with interpolation on term volatilities.}
\label{lst:capfloorvol_atm_configuration_term}
\end{longlisting}

Listing \ref{lst:capfloorvol_atm_configuration_opt} shows the layout for parameterising an ATM cap volatility curve with interpolation on optionlet volatilities. Nodes that have no effect for this parameterisation but that are allowed by the schema are not referenced. The meaning of each of the nodes is as follows:

\begin{itemize}
\item
\lstinline!CurveId!: Unique identifier for the cap floor volatility structure.

\item \lstinline!CurveDescription! [Optional]:
A description of the volatility structure. It is for information only and may be left blank.

\item \lstinline!VolatilityType!:
Indicates the cap floor volatility type. It may be \lstinline!Normal!, \lstinline!Lognormal! or \lstinline!ShiftedLognormal!. Note that this then determines which market data points are looked up in the market when creating the ATM cap floor curve and how they are interpreted when stripping the optionlets. In particular, the market will be searched for market data points of the form \lstinline!CAPFLOOR/RATE_NVOL/Currency/Tenor/IndexTenor/1/1/0!, \lstinline!CAPFLOOR/RATE_LNVOL/Currency/Tenor/IndexTenor/1/1/0! or \lstinline!CAPFLOOR/RATE_SLNVOL/Currency/Tenor/IndexTenor/1/1/0! respectively.

\item \lstinline!Extrapolation!:
The allowable values are \lstinline!None!, \lstinline!Flat! and \lstinline!Linear!. If set to \lstinline!None!, extrapolation is turned off and an exception is thrown if the optionlet surface is queried outside the allowable times. Otherwise, extrapolation is allowed and the type of extrapolation is determined by the \lstinline!TimeInterpolation! node value described below.

\item \lstinline!IncludeAtm!:
A boolean value indicating if an ATM curve should be used. Allowable boolean values are given in the Table \ref{tab:boolean_allowable}. As we are describing an ATM curve here, this node should be set to \lstinline!true! as shown in \ref{lst:capfloorvol_atm_configuration_opt}.

\item \lstinline!DayCounter!:
The day counter used to convert from dates to times in the underlying structure. Allowable values are given in the Table \ref{tab:daycount}.

\item \lstinline!Calendar!:
The calendar used to advance dates by periods in the underlying structure. In particular, it is used in deriving the cap maturity dates from the configured cap tenors. Allowable values are given in the Table \ref{tab:calendar}.

\item \lstinline!BusinessDayConvention!:
The business day convention used to advance dates by periods in the underlying structure. In particular, it is used in deriving the cap maturity dates from the configured cap tenors. Allowable values are given in the Table \ref{tab:allow_stand_data} under \lstinline!Roll Convention!.

\item \lstinline!Tenors! [Optional]:
A comma separated list of valid tenor strings giving the cap floor maturity tenors to be used in the ATM curve. If omitted, the tenors for the ATM curve must be provided in the \lstinline!AtmTenors! node instead. If the tenors are provided here, the \lstinline!AtmTenors! node may be omitted.

\item \lstinline!OptionalQuotes! [Optional]:
A boolean flag to indicate whether market data quotes for all tenors are required. If true, we attempt to build the curve from whatever quotes are provided. If false, the curve will fail to build if any quotes are missing. This also applies to quotes for the \lstinline!AtmTenors!. Default value is false.

\item \lstinline!IborIndex!:
A valid interest rate index name giving the index underlying the cap floor quotes. Allowable values are given in the Table \ref{tab:indices}.

\item \lstinline!DiscountCurve!:
A reference to a valid discount curve specification that will be used to discount cashflows during the stripping process. It should be of the form \lstinline!Yield/Currency/curve_name! where \lstinline!curve_name! is the name of a yield curve defined in the yield curve configurations.

\item \lstinline!AtmTenors! [Optional]:
A comma separated list of valid tenor strings giving the cap floor maturities to be used in the ATM curve. If omitted, the tenors for the ATM curve must be provided in the \lstinline!Tenors! node instead. If the tenors are provided here, the \lstinline!Tenors! node may be omitted.

\item \lstinline!SettlementDays! [Optional]:
Any non-negative integer is allowed here. If omitted, it is assumed to be 0. If provided the reference date of the term volatility curve and the stripped optionlet volatility structure will be calculated by advancing the valuation date by this number of days using the configured calendar and business day convention. In general, this should be omitted or set to 0.

\item \lstinline!InterpolateOn!:
As referenced above, the allowable values are \lstinline!TermVolatilities! or \lstinline!OptionletVolatilities!. As we are describing here an ATM curve with interpolation on optionlet volatilities, this should be set to \lstinline!OptionletVolatilities! as shown in Listing \ref{lst:capfloorvol_atm_configuration_opt}.

\item \lstinline!TimeInterpolation! [Optional]:
Indicates the interpolation and extrapolation, if allowed by the \lstinline!Extrapolation! node, in the time direction. As \lstinline!InterpolateOn! is set to \lstinline!OptionletVolatilities! here, the interpolation is used to interpolate the optionlet volatilities only i.e.\ there is no interpolation on the term cap floor volatility curve. The allowable values are \lstinline!Linear!, \lstinline!LinearFlat!, \lstinline!BackwardFlat!, \lstinline!Cubic! and \lstinline!CubicFlat!. If not set, \lstinline!LinearFlat! is assumed. Note that \lstinline!Linear! indicates linear interpolation and linear extrapolation. \lstinline!LinearFlat! indicates linear interpolation and flat extrapolation. Analogous meanings apply for \lstinline!Cubic! and \lstinline!CubicFlat!.

\item \lstinline!BootstrapConfig! [Optional]:
This node holds configuration details for the iterative bootstrap that are described in section \ref{sec:bootstrap_config}. If omitted, this node's default values described in section \ref{sec:bootstrap_config} are used.

\end{itemize}

\begin{longlisting}
\begin{minted}[fontsize=\footnotesize]{xml}
<CapFloorVolatility>
  <CurveId>...</CurveId>
  <CurveDescription>...</CurveDescription>
  <VolatilityType>...</VolatilityType>
  <Extrapolation>...</Extrapolation>
  <IncludeAtm>true</IncludeAtm>
  <DayCounter>...</DayCounter>
  <Calendar>...</Calendar>
  <BusinessDayConvention>...</BusinessDayConvention>
  <Tenors>...</Tenors>
  <OptionalQuotes>...</OptionalQuotes>
  <IborIndex>...</IborIndex>
  <DiscountCurve>...</DiscountCurve>
  <AtmTenors>...</AtmTenors>
  <SettlementDays>...</SettlementDays>
  <InterpolateOn>OptionletVolatilities</InterpolateOn>
  <TimeInterpolation>...</TimeInterpolation>
  <BootstrapConfig>...</BootstrapConfig>
</CapFloorVolatility>
\end{minted}
\caption{ATM cap floor configuration with interpolation on optionlet volatilities.}
\label{lst:capfloorvol_atm_configuration_opt}
\end{longlisting}

Listing \ref{lst:capfloorvol_surface_configuration_term} shows the layout for parameterising a cap tenor by absolute cap strike volatility surface with interpolation on term volatilities. This parameterisation also allows for the inclusion of a cap floor ATM curve in combination with the surface. Nodes that have no effect for this parameterisation but that are allowed by the schema are not referenced. The meaning of each of the nodes is as follows:

\begin{itemize}
\item
\lstinline!CurveId!: Unique identifier for the cap floor volatility structure.

\item \lstinline!CurveDescription! [Optional]:
A description of the volatility structure. It is for information only and may be left blank.

\item \lstinline!VolatilityType!:
Indicates the cap floor volatility type. It may be \lstinline!Normal!, \lstinline!Lognormal! or \lstinline!ShiftedLognormal!. Note that this then determines which market data points are looked up in the market when creating the cap floor surface and how they are interpreted when stripping the optionlets. In particular, the market will be searched for market data points of the form \lstinline!CAPFLOOR/RATE_NVOL/Currency/Tenor/IndexTenor/0/0/Strike!, \lstinline!CAPFLOOR/RATE_LNVOL/Currency/Tenor/IndexTenor/0/0/Strike! or \lstinline!CAPFLOOR/RATE_SLNVOL/Currency/Tenor/IndexTenor/0/0/Strike! respectively.

\item \lstinline!Extrapolation!:
Indicates the extrapolation in the time and strike direction. The extrapolation occurs on the stripped optionlet volatilities. The allowable values are \lstinline!None!, \lstinline!Flat! and \lstinline!Linear!. If set to \lstinline!None!, extrapolation is turned off and an exception is thrown if the optionlet surface is queried outside the allowable times or strikes. If set to \lstinline!Flat!, the optionlet volatility on the time strike boundary is used if the optionlet surface is queried outside the allowable times or strikes. If set to \lstinline!Linear!, the interpolation method configured in \lstinline!InterpolationMethod! is used to extrapolate either time or strike direction.

\item \lstinline!InterpolationMethod! [Optional]:
Indicates the interpolation in the time and strike direction. As \lstinline!InterpolateOn! is set to \lstinline!TermVolatilities! here, the interpolation is used in the stripping process to interpolate the term cap floor volatility surface as explained above. It is also used to interpolate the optionlet volatilities when an optionlet volatility is queried from the stripped optionlet structure. The allowable values are \lstinline!Bilinear! and \lstinline!BicubicSpline!. If not set, \lstinline!BicubicSpline! is assumed.

\item \lstinline!IncludeAtm!:
A boolean value indicating if an ATM curve should be used in combination with the surface. Allowable boolean values are given in the Table \ref{tab:boolean_allowable}. If set to \lstinline!true!, the \lstinline!AtmTenors! node needs to be populated with the ATM tenors to use. The ATM quotes that are searched for are as outlined in the previous two ATM sections above. The original stripped optionlet surface is amended by inserting the optionlet volatilities at the successive ATM strikes that reproduce the sequence of ATM cap volatilities.

\item \lstinline!DayCounter!:
The day counter used to convert from dates to times in the underlying structure. Allowable values are given in the Table \ref{tab:daycount}.

\item \lstinline!Calendar!:
The calendar used to advance dates by periods in the underlying structure. In particular, it is used in deriving the cap maturity dates from the configured cap tenors. Allowable values are given in the Table \ref{tab:calendar}.

\item \lstinline!BusinessDayConvention!:
The business day convention used to advance dates by periods in the underlying structure. In particular, it is used in deriving the cap maturity dates from the configured cap tenors. Allowable values are given in the Table \ref{tab:allow_stand_data} under \lstinline!Roll Convention!.

\item \lstinline!Tenors!:
A comma separated list of valid tenor strings giving the cap floor maturity tenors to be used in the tenor by strike surface. In this case, i.e.\ configuring a surface, they must be provided.

\item \lstinline!OptionalQuotes! [Optional]:
A boolean flag to indicate whether market data quotes for all tenors are required. If true, we attempt to build the curve from whatever quotes are provided. If false, the curve will fail to build if any quotes are missing. This also applies to quotes for the \lstinline!AtmTenors!. Default value is false.

\item \lstinline!IborIndex!:
A valid interest rate index name giving the index underlying the cap floor quotes. Allowable values are given in the Table \ref{tab:indices}.

\item \lstinline!DiscountCurve!:
A reference to a valid discount curve specification that will be used to discount cashflows during the stripping process. It should be of the form \lstinline!Yield/Currency/curve_name! where \lstinline!curve_name! is the name of a yield curve defined in the yield curve configurations.

\item \lstinline!AtmTenors! [Optional]:
A comma separated list of valid tenor strings giving the cap floor maturity tenors to be used in the ATM curve. It must be provided when \lstinline!IncludeAtm! is \lstinline!true! and omitted when \lstinline!IncludeAtm! is \lstinline!false!.

\item \lstinline!SettlementDays! [Optional]:
Any non-negative integer is allowed here. If omitted, it is assumed to be 0. If provided the reference date of the term volatility curve and the stripped optionlet volatility structure will be calculated by advancing the valuation date by this number of days using the configured calendar and business day convention. In general, this should be omitted or set to 0.

\item \lstinline!InterpolateOn!:
As referenced above, the allowable values are \lstinline!TermVolatilities! or \lstinline!OptionletVolatilities!. As we are describing here a surface with interpolation on term volatilities, this should be set to \lstinline!TermVolatilities! as shown in Listing \ref{lst:capfloorvol_surface_configuration_term}.

\item \lstinline!BootstrapConfig! [Optional]:
This node holds configuration details for the iterative bootstrap that are described in section \ref{sec:bootstrap_config}. If omitted, this node's default values described in section \ref{sec:bootstrap_config} are used.

\end{itemize}

\begin{longlisting}
\begin{minted}[fontsize=\footnotesize]{xml}
<CapFloorVolatility>
  <CurveId>...</CurveId>
  <CurveDescription>...</CurveDescription>
  <VolatilityType>...</VolatilityType>
  <Extrapolation>...</Extrapolation>
  <InterpolationMethod>...</InterpolationMethod>
  <IncludeAtm>...</IncludeAtm>
  <DayCounter>...</DayCounter>
  <Calendar>...</Calendar>
  <BusinessDayConvention>...</BusinessDayConvention>
  <Tenors>...</Tenors>
  <OptionalQuotes>...</OptionalQuotes>
  <IborIndex>...</IborIndex>
  <DiscountCurve>...</DiscountCurve>
  <AtmTenors>...</AtmTenors>
  <SettlementDays>...</SettlementDays>
  <InterpolateOn>TermVolatilities</InterpolateOn>
  <BootstrapConfig>...</BootstrapConfig>
</CapFloorVolatility>
\end{minted}
\caption{Cap floor surface with interpolation on term volatilities.}
\label{lst:capfloorvol_surface_configuration_term}
\end{longlisting}

Listing \ref{lst:capfloorvol_surface_configuration_opt} shows the layout for parameterising a cap tenor by absolute cap strike volatility surface with interpolation on optionlet volatilities. This parameterisation also allows for the inclusion of a cap floor ATM curve in combination with the surface. Nodes that have no effect for this parameterisation but that are allowed by the schema are not referenced. The meaning of each of the nodes is as follows:

\begin{itemize}
\item
\lstinline!CurveId!: Unique identifier for the cap floor volatility structure.

\item \lstinline!CurveDescription! [Optional]:
A description of the volatility structure. It is for information only and may be left blank.

\item \lstinline!VolatilityType!:
Indicates the cap floor volatility type. It may be \lstinline!Normal!, \lstinline!Lognormal! or \lstinline!ShiftedLognormal!. Note that this then determines which market data points are looked up in the market when creating the cap floor surface and how they are interpreted when stripping the optionlets. In particular, the market will be searched for market data points of the form \lstinline!CAPFLOOR/RATE_NVOL/Currency/Tenor/IndexTenor/0/0/Strike!, \lstinline!CAPFLOOR/RATE_LNVOL/Currency/Tenor/IndexTenor/0/0/Strike! or \lstinline!CAPFLOOR/RATE_SLNVOL/Currency/Tenor/IndexTenor/0/0/Strike! respectively.

\item \lstinline!Extrapolation!:
The allowable values are \lstinline!None!, \lstinline!Flat! and \lstinline!Linear!. If set to \lstinline!None!, extrapolation is turned off and an exception is thrown if the optionlet surface is queried outside the allowable times or strikes. Otherwise, extrapolation is allowed and the type of extrapolation is determined by the \lstinline!TimeInterpolation! and \lstinline!StrikeInterpolation! node values described below.

\item \lstinline!IncludeAtm!:
A boolean value indicating if an ATM curve should be used in combination with the surface. Allowable boolean values are given in the Table \ref{tab:boolean_allowable}. If set to \lstinline!true!, the \lstinline!AtmTenors! node needs to be populated with the ATM tenors to use. The ATM quotes that are searched for are as outlined in the previous two ATM sections above. The original stripped optionlet surface is amended by inserting the optionlet volatilities at the configured ATM strikes that reproduce the configured ATM cap volatilities.

\item \lstinline!DayCounter!:
The day counter used to convert from dates to times in the underlying structure. Allowable values are given in the Table \ref{tab:daycount}.

\item \lstinline!Calendar!:
The calendar used to advance dates by periods in the underlying structure. In particular, it is used in deriving the cap maturity dates from the configured cap tenors. Allowable values are given in the Table \ref{tab:calendar}.

\item \lstinline!BusinessDayConvention!:
The business day convention used to advance dates by periods in the underlying structure. In particular, it is used in deriving the cap maturity dates from the configured cap tenors. Allowable values are given in the Table \ref{tab:allow_stand_data} under \lstinline!Roll Convention!.

\item \lstinline!Tenors!:
A comma separated list of valid tenor strings giving the cap floor maturity tenors to be used in the tenor by strike surface. In this case, i.e.\ configuring a surface, they must be provided.

\item \lstinline!OptionalQuotes! [Optional]:
A boolean flag to indicate whether market data quotes for all tenors and strikes are required. If true, we attempt to build the curve from whatever quotes are provided. If false, the curve will fail to build if any quotes are missing. This also applies to quotes for the \lstinline!AtmTenors!. Default value is false.

\item \lstinline!IborIndex!:
A valid interest rate index name giving the index underlying the cap floor quotes. Allowable values are given in the Table \ref{tab:indices}.

\item \lstinline!DiscountCurve!:
A reference to a valid discount curve specification that will be used to discount cashflows during the stripping process. It should be of the form \lstinline!Yield/Currency/curve_name! where \lstinline!curve_name! is the name of a yield curve defined in the yield curve configurations.

\item \lstinline!AtmTenors! [Optional]:
A comma separated list of valid tenor strings giving the cap floor maturity tenors to be used in the ATM curve. It must be provided when \lstinline!IncludeAtm! is \lstinline!true! and omitted when \lstinline!IncludeAtm! is \lstinline!false!.

\item \lstinline!SettlementDays! [Optional]:
Any non-negative integer is allowed here. If omitted, it is assumed to be 0. If provided the reference date of the term volatility curve and the stripped optionlet volatility structure will be calculated by advancing the valuation date by this number of days using the configured calendar and business day convention. In general, this should be omitted or set to 0.

\item \lstinline!InterpolateOn!:
As referenced above, the allowable values are \lstinline!TermVolatilities! or \lstinline!OptionletVolatilities!. As we are describing here a surface with interpolation on optionlet volatilities, this should be set to \lstinline!OptionletVolatilities! as shown in Listing \ref{lst:capfloorvol_surface_configuration_opt}.

\item \lstinline!TimeInterpolation!:
Indicates the interpolation and extrapolation, if allowed by the \lstinline!Extrapolation! node, in the time direction. As \lstinline!InterpolateOn! is set to \lstinline!OptionletVolatilities! here, the interpolation is used to interpolate the optionlet volatilities only i.e.\ there is no interpolation on the term cap floor volatility curve. The allowable values are \lstinline!Linear!, \lstinline!LinearFlat!, \lstinline!BackwardFlat!, \lstinline!Cubic! and \lstinline!CubicFlat!. If not set, \lstinline!LinearFlat! is assumed. Note that \lstinline!Linear! indicates linear interpolation and linear extrapolation. \lstinline!LinearFlat! indicates linear interpolation and flat extrapolation. Analogous meanings apply for \lstinline!Cubic! and \lstinline!CubicFlat!.

\item \lstinline!StrikeInterpolation!:
Indicates the interpolation and extrapolation, if allowed by the \lstinline!Extrapolation! node, in the strike direction. Again, as \lstinline!InterpolateOn! is set to \lstinline!OptionletVolatilities! here, the interpolation is used to interpolate the optionlet volatilities in the strike direction. The allowable values are \lstinline!Linear!, \lstinline!LinearFlat!, \lstinline!Cubic! and \lstinline!CubicFlat!. If not set, \lstinline!LinearFlat! is assumed.

\item \lstinline!BootstrapConfig! [Optional]:
This node holds configuration details for the iterative bootstrap that are described in section \ref{sec:bootstrap_config}. If omitted, this node's default values described in section \ref{sec:bootstrap_config} are used.

\end{itemize}

\begin{longlisting}
\begin{minted}[fontsize=\footnotesize]{xml}
<CapFloorVolatility>
  <CurveId>...</CurveId>
  <CurveDescription>...</CurveDescription>
  <VolatilityType>...</VolatilityType>
  <Extrapolation>...</Extrapolation>
  <InterpolationMethod>...</InterpolationMethod>
  <IncludeAtm>...</IncludeAtm>
  <DayCounter>...</DayCounter>
  <Calendar>...</Calendar>
  <BusinessDayConvention>...</BusinessDayConvention>
  <Tenors>...</Tenors>
  <OptionalQuotes>...</OptionalQuotes>
  <IborIndex>...</IborIndex>
  <DiscountCurve>...</DiscountCurve>
  <AtmTenors>...</AtmTenors>
  <SettlementDays>...</SettlementDays>
  <InterpolateOn>OptionletVolatilities</InterpolateOn>
  <TimeInterpolation>...</TimeInterpolation>
  <StrikeInterpolation>...</StrikeInterpolation>
  <BootstrapConfig>...</BootstrapConfig>
</CapFloorVolatility>
\end{minted}
\caption{Cap floor surface with interpolation on optionlet volatilities.}
\label{lst:capfloorvol_surface_configuration_opt}
\end{longlisting}


\subsubsection{FX Volatility Structures}

Listings \ref{lst:fxoptionvol_configuration_atm}, \ref{lst:fxoptionvol_configuration_smile_vv},
\ref{lst:fxoptionvol_configuration_smile_delta}, \ref{lst:fxoptionvol_configuration_smile_bfrr}, \ref{lst:fxoptionvol_configuration_smile_absolute},
\ref{lst:fxoptionvol_configuration_atm_triangulated} shows examples of FX volatility structure configurations.

\begin{longlisting}
\begin{minted}[fontsize=\footnotesize]{xml}
  <FXVolatility>
    <CurveId>EURUSD</CurveId>
    <CurveDescription />
    <Dimension>ATM</Dimension> 
    <Expiries>1M,3M,6M,1Y,2Y,3Y,10Y</Expiries>
    <FXSpotID>FX/EUR/USD</FXSpotID>
    <FXForeignCurveID>Yield/EUR/EUR-IN-USD</FXForeignCurveID>
    <FXDomesticCurveID>Yield/USD/USD1D</FXDomesticCurveID>
    <DayCounter>A365</DayCounter>
    <Calendar>US,TARGET</Calendar>
    <Conventions>EUR-USD-FXOPTION</Conventions>
  </FXVolatility>
\end{minted}
\caption{FX option volatility configuration ATM}
\label{lst:fxoptionvol_configuration_atm}
\end{longlisting}

\begin{longlisting}
\begin{minted}[fontsize=\footnotesize]{xml}
  <FXVolatility>
    <CurveId>USDJPY</CurveId>
    <CurveDescription />
    <Dimension>Smile</Dimension> 
    <SmileType>VannaVolga</SmileType> 
    <SmileInterpolation>VannaVolga2</SmileInterpolation>
    <Expiries>1M,3M,6M,1Y,2Y,3Y,10Y</Expiries>
    <SmileDelta>25</SmileDelta>
    <FXSpotID>FX/USD/JPY</FXSpotID>
    <FXForeignCurveID>Yield/USD/USD1D</FXForeignCurveID>
    <FXDomesticCurveID>Yield/JPY/JPY-IN-USD</FXDomesticCurveID>
    <DayCounter>A365</DayCounter>
    <Calendar>US,JP</Calendar>
    <Conventions>USD-JPY-FXOPTION</Conventions>
  </FXVolatility>
\end{minted}
\caption{FX option volatility configuration Smile / VannaVolga}
\label{lst:fxoptionvol_configuration_smile_vv}
\end{longlisting}

\begin{longlisting}
\begin{minted}[fontsize=\footnotesize]{xml}
  <FXVolatility>
    <CurveId>USDJPY</CurveId>
    <CurveDescription />
    <Dimension>Smile</Dimension> 
    <SmileType>Delta</SmileType> 
    <SmileInterpolation>Linear</SmileInterpolation>
    <Expiries>1M,3M,6M,1Y,2Y,3Y,10Y</Expiries>
    <Deltas>10P,20P,30P,40P,ATM,40C,30C,20C,10C</Deltas>
    <FXSpotID>FX/USD/JPY</FXSpotID>
    <FXForeignCurveID>Yield/USD/USD1D</FXForeignCurveID>
    <FXDomesticCurveID>Yield/JPY/JPY-IN-USD</FXDomesticCurveID>
    <DayCounter>A365</DayCounter>
    <Calendar>US,JP</Calendar>
    <Conventions>USD-JPY-FXOPTION</Conventions>
  </FXVolatility>
\end{minted}
\caption{FX option volatility configuration Smile / Delta}
\label{lst:fxoptionvol_configuration_smile_delta}
\end{longlisting}

\begin{longlisting}
\begin{minted}[fontsize=\footnotesize]{xml}
  <FXVolatility>
    <CurveId>USDJPY</CurveId>
    <CurveDescription />
    <Dimension>Smile</Dimension> 
    <SmileType>BFRR</SmileType> 
    <SmileInterpolation>Cubic</SmileInterpolation>
    <Expiries>1M,3M,6M,1Y,2Y,3Y,10Y</Expiries>
    <SmileDelta>10,25</SmileDelta>
    <FXSpotID>FX/USD/JPY</FXSpotID>
    <FXForeignCurveID>Yield/USD/USD1D</FXForeignCurveID>
    <FXDomesticCurveID>Yield/JPY/JPY-IN-USD</FXDomesticCurveID>
    <DayCounter>A365</DayCounter>
    <Calendar>US,JP</Calendar>
    <Conventions>USD-JPY-FXOPTION</Conventions>
  </FXVolatility>
\end{minted}
\caption{FX option volatility configuration Smile / BFRR with 10 and 25 BF and RR}
\label{lst:fxoptionvol_configuration_smile_bfrr}
\end{longlisting}

\begin{longlisting}
\begin{minted}[fontsize=\footnotesize]{xml}
  <FXVolatility>
    <CurveId>USDJPY</CurveId>
    <CurveDescription />
    <Dimension>Smile</Dimension> 
    <SmileType>Absolute</SmileType> 
    <SmileInterpolation>Cubic</SmileInterpolation>
    <Expiries>1M,3M,6M,1Y,2Y,3Y,10Y</Expiries>
    <FXSpotID>FX/USD/JPY</FXSpotID>
    <FXForeignCurveID>Yield/USD/USD1D</FXForeignCurveID>
    <FXDomesticCurveID>Yield/JPY/JPY-IN-USD</FXDomesticCurveID>
    <DayCounter>A365</DayCounter>
    <Calendar>US,JP</Calendar>
    <Conventions>USD-JPY-FXOPTION</Conventions>
  </FXVolatility>
\end{minted}
\caption{FX option volatility configuration Smile / Absolute vols}
\label{lst:fxoptionvol_configuration_smile_absolute}
\end{longlisting}

\begin{longlisting}
\begin{minted}[fontsize=\footnotesize]{xml}
  <FXVolatility>
    <CurveId>EURJPY</CurveId>
    <CurveDescription />
    <Dimension>ATMTriangulated</Dimension> 
    <FXSpotID>FX/EUR/JPY</FXSpotID>
    <DayCounter>A365</DayCounter>
    <Calendar>US,JP</Calendar>
    <BaseVolatility1>EURUSD</BaseVolatility1>
    <BaseVolatility2>USDJPY</BaseVolatility2>
  </FXVolatility>
\end{minted}
\caption{FX option volatility configuration ATM Triangulated}
\label{lst:fxoptionvol_configuration_atm_triangulated}
\end{longlisting}

The meaning of each of the elements in Listings \ref{lst:fxoptionvol_configuration_atm},
\ref{lst:fxoptionvol_configuration_smile_vv}, \ref{lst:fxoptionvol_configuration_smile_delta},
\ref{lst:fxoptionvol_configuration_smile_bfrr}, \ref{lst:fxoptionvol_configuration_smile_absolute}, \ref{lst:fxoptionvol_configuration_atm_triangulated} is given below.

\begin{itemize}
\item CurveId: Unique identifier of the FX volatility structure
\item CurveDescription [Optional]: A description of the volatility structure, may be left blank.
\item Dimension: Distinguishes at-the-money volatility curves from volatility surfaces. An `ATMTriangulated' value
  denotes a curve triangulated from two other surfaces.\\ Allowable values: {\tt ATM, Smile, ATMTriangulated}
\item SmileType [Optional]: Required field in case of Dimension {\tt Smile}, otherwise it can be omitted. \\ Allowable
  values: {\tt VannaVolga} as per (Castagna \& Mercurio - 2006), {\tt Delta}, {\tt BFRR}, {\tt Absolute}, with default value {\tt
    VannaVolga} if left blank.
\item SmileInterpolation [Optional]: Smile interpolation method applied, required field in case of Dimension {\tt
  Smile}, otherwise it can be omitted. \\ Allowable values:
\begin{itemize}
\item {\tt VannaVolga1} or {\tt VannaVolga2} in case of SmileType {\tt VannaVolga} with default VannaVolga2 if left
  blank.  VannaVolga1/VannaVolga2 refer to the first/second approximation in (eq. 13) and (eq. 14) of the reference
  above.
\item {\tt Linear} or {\tt Cubic} in case of SmileType {\tt Delta} or {\tt BFRR} with default {\tt Linear} for SmileType
  Delta and {\tt Cubic} for SmileType BFRR and Absolute if left blank
\end{itemize}
\item Expiries: Option expiries in period form. A wildcard may also be used. In the wildcard case, it will look for any
  matching quotes provided in the loader, and construct the curve from these. This is currently only supported for {\tt
    ATM} or {\tt Delta} or {\tt BFRR} or {\tt Absolute} curves.
\item Deltas [Optional]: Strike grid, required in case of SmileType {\tt Delta} \\ Allowable values: {\tt ATM, *P, *C},
  see example in Listing \ref{lst:fxoptionvol_configuration_smile_delta}
\item SmileDelta [Optional]: Strike grid for SmileType {\tt VannaVolga} and {\tt BFRR}, defaults to 25 for VannaVolga
  resp. 10,25 for BFRR \\ Allowable values: a list of integers, see example in Listing
  \ref{lst:fxoptionvol_configuration_smile_bfrr}
\item FXSpotID: ORE representation of the relevant FX spot quote in the form FX/CCY1/CCY2 
\item FXForeignCurveID [Optional]: Yield curve, in ORE format Yield/CCY/ID, used as foreign yield curve in smile curves,
  may be omitted for ATM curves.
\item FXDomesticCurveID [Optional]: Yield curve, in ORE format Yield/CCY/ID, used as domestic yield curve in smile
  curves, may be omitted for ATM curves.
\item DayCounter: The term structure's day counter used in date to time conversions. Optional, defaults to A365.
\item Calendar: The term structure's calendar used in option tenor to date conversions. Optional, defaults to source ccy
  + target ccy default calendars.
\item Conventions [Optional]: FX conventions object ID that is used to determine the at-the-money type and delta type of
  the volatility quotes, these default to {\tt AtmDeltaNeutral} and {\tt Spot} for option tenors <= 2Y and {\tt
    AtmDeltaNeutral} and {\tt Fwd} for option tenors > 2Y if the conventions ID is omitted or left blank. Furthermore,
  the conventions hold information on the side of risk reversals (RiskReversalInFavorOf, defaults to {\tt Call}) and the
  quote style of butterflies (ButterflyStyle, defaults to {\tt Broker}), these latter two are relevant for BF, RR market
  data input. See \ref{sss:fx_option_conv} for more details.
\item BaseVolatility1: For `ATMTriangulated' this denotes one of the surfaces we want to triangulate from
\item BaseVolatility2: For `ATMTriangulated' this denotes one of the surfaces we want to triangulate from
\end{itemize}

\subsubsection{Equity Curve Structures}

Listing \ref{lst:eqcurve_configuration} shows the configuration of equity forward price curves.

\begin{longlisting}
%\hrule\medskip
\begin{minted}[fontsize=\footnotesize, texcomments]{xml}
<EquityCurves>
  <EquityCurve>
    <CurveId>SP5</CurveId>
    <CurveDescription>SP 500 equity price projection curve</CurveDescription>
    <Currency>USD</Currency>
    <ForecastingCurve>EUR1D</ForecastingCurve>
    <!-- DividendYield, ForwardPrice, OptionPremium, NoDividends, ForwardDividendPrice -->
    <Type>DividendYield</Type> 
    <!-- Optional node, only used when Type is OptionPremium -->
    <ExerciseStyle>American</ExerciseStyle>
    <!-- Spot quote from the market data file -->
    <SpotQuote>EQUITY/PRICE/SP5/USD</SpotQuote>
    <!-- Note: do not provide Quotes if Type is NoDividends -->
    <Quotes>
      <Quote>EQUITY_DIVIDEND/RATE/SP5/USD/3M</Quote>
      <Quote>EQUITY_DIVIDEND/RATE/SP5/USD/20160915</Quote>
      <Quote>EQUITY_DIVIDEND/RATE/SP5/USD/1Y</Quote>
      <Quote>EQUITY_DIVIDEND/RATE/SP5/USD/20170915</Quote>
    </Quotes>
    <!-- Optional interpolation options, default Zero and Linear -->
    <!-- Note: do not provide DividendInterpolation if Type is NoDividends -->
    <DividendInterpolation>
      <!-- Zero, Discount -->
      <InterpolationVariable>Zero</InterpolationVariable>
      <!-- See Table \ref{tab:allow_interp_methods} for allowed interpolation methods -->
      <InterpolationMethod>Linear</InterpolationMethod>
    </DividendInterpolation>
    <DividendExtrapolation>False</DividendExtrapolation>
    <!-- Optional node, defaults to false -->
    <Extrapolation>False</Extrapolation>
    <DayCounter>A365</DayCounter>
  </EquityCurve>
  <EquityCurve>
    ...
  </EquityCurve> 
  </EquityCurves>
\end{minted}
\caption{Equity curve configuration}
\label{lst:eqcurve_configuration}
\end{longlisting}

The meaning of each of the elements is given below. 

\begin{itemize}
\item CurveId: Unique identifier of the equity curve structure.
\item CurveDescription [Optional]: A description of the equity curve structure, may be left blank.
\item Currency: Currency of the equity.
\item Calendar [Optional]: The term structure's calendar used in tenor to date conversions. Defaults to the calendar corresponding to \lstinline!Currency!.
\item ForecastingCurve: CurveId of the curve used for discounting equity fixing forecasts.
\item Type: The quote types in \lstinline!Quote! (e.g.\ option premium, forward equity price) and whether dividends are taken into account. Allowable values: {\tt DividendYield, ForwardPrice, ForwardDividendPrice, OptionPremium, NoDividends}
\item ExerciseStyle [Optional]: Exercise type of the underlying option quotes. Only required if \lstinline!Type! is \emph{OptionPremium}. Allowable values: {\tt American, European}
\item SpotQuote: Market datum ID/name of the current spot rate for the equity underlying.
\item Quotes [Optional]: Market datum IDs/names to be used in building the curve structure.
\item DayCounter [Optional]: The term structure's day counter used in date to time conversions. Defaults to {\tt A365F}.
\item DividendInterpolation [Optional]: This node contains an \lstinline!InterpolationVariable! and \lstinline!InterpolationMethod! sub-node, which define the variable on which the interpolation is performed and the interpolation method for the dividend curve, respectively. The allowable values are found in Table \ref{tab:allow_interp_variables} and Table \ref{tab:allow_interp_methods}, respectively. This should not be provided if \lstinline!Type! is {\tt NoDividends}.
\item DividendExtrapolation [Optional]: Boolean flag indicating whether extrapolation in the dividend curve is allowed. If True the dividend curve is extrapolated forward at the risk free rate beyond the last date of the curve - this is only considered when \lstinline!Extrapolation! is False. Defaults to {\tt False}.
\item Extrapolation [Optional]: Boolean flag indicating whether extrapolation in the forward price is allowed. Defaults to {\tt False}.
\end{itemize}

The equity curves here consists of a spot equity price, as well as a set of either forward prices or else dividend 
yields. If the index is a dividend futures index then curve type should be entered as ForwardDividendPrice. In this case the curve will be built from forward prices as normal, but excluded from the SIMM calculations as required by the SIMM methodology.
Upon construction, ORE stores internally an equity spot price quote, a forecasting curve and a dividend yield 
term structure, which are then used together for projection of forward prices.

\subsubsection{Equity Volatility Structures}

When configuring volatility structures for equities, there are four options:
\begin{enumerate}
\item a constant volatility for all expiries and strikes.
\item a volatility curve with a dependency on expiry but no strike dimension.
\item a volatility surface with an expiry and strike dimension.
\item a proxy surface - point to another surface as a proxy.
\end{enumerate}

There are a number of fields common to all configurations:
\begin{itemize}
\item CurveId: Unique identifier for the curve.
\item CurveDescription [Optional]: A description of the curve. This field may be left blank.
\item EquityId: [Optional] Identifies the underlying equity name, this is used in the construction of the \lstinline!Quote! strings. If omitted the \lstinline!CurveId! is used instead.
\item Currency: Currency of the equity.
\item Calendar [Optional]: allowable value is any valid calendar. Defaults to \lstinline!NullCalendar!.
\item DayCounter [Optional]: allowable value is any valid day counter. Defaults to \lstinline!A365!.
\item OneDimSolverConfig [Optional]: A configuration for the one dimensional solver used in deriving volatilities from prices. This node is described in detail in Section \ref{sec:one_dim_solver_config}. If not provided, a default step based configuration is used. This is only used when volatilities are stripped from prices.
\item PreferOutOfTheMoney [Optional]: allowable value is any boolean value. Defaults to \lstinline!false! for backwards compatibility. It is used, when building the volatility surface, to choose between a call and put option price or volatility when both are provided. When set to \lstinline!true!, the out of the money option is chosen by comparing the quote's strike with the forward price at the associated expiry. Conversely, when set to \lstinline!false!, the in the money option is chosen.
\end{itemize}

Listing \ref{lst:eqoptionvol_constant} shows the configuration of equity volatility structures with constant volatility. A node \lstinline!Constant! takes one \lstinline!Quote!, as described in Section \ref{md:equity_option_iv}, which is held constant for all strikes and expiries.

\begin{longlisting}
%\hrule\medskip
\begin{minted}[fontsize=\footnotesize]{xml}
<EquityVolatilities>
  <EquityVolatility>
    <CurveId>SP5</CurveId>
    <CurveDescription>Lognormal option implied vols for SP 500</CurveDescription>
    <EquityId>RIC:.SP5</EquityId>
    <Currency>USD</Currency>
    <DayCounter>Actual/365 (Fixed)</DayCounter>
    <Constant>
      <Quote>EQUITY_OPTION/RATE_LNVOL/RIC:.SP5/USD/5Y/ATMF</Quote>
    </Constant>
  </EquityVolatility>
  <EquityVolatility>
    ...
  </EquityVolatility>
</EquityVolatilities>
\end{minted}
\caption{Equity option volatility configuration - constant}
\label{lst:eqoptionvol_constant}
\end{longlisting}

Secondly, the volatility curve configuration layout is given in Listing \ref{lst:eqoptionvol_curve}. With this curve construction the volatility is held constant in the strike direction, and quotes of varying expiry can be provided, for examlple if only ATM volatility quotes are available. The volatility quote IDs in the \lstinline!Quotes! node should be Equity option volatility quotes as described in Section \ref{md:equity_option_iv}. An explicit list of quotes can be provided, or a single quote with a wildcard replacing the expiry/strike. In the wildcard case, it will look for any matching quotes provided in the loader, and construct the curve from these. The \lstinline!Interpolation! node supports \lstinline!Linear!, \lstinline!Cubic! and \lstinline!LogLinear! interpolation. The \lstinline!Extrapolation! node supports either \lstinline!None! for no extrapolation or \lstinline!Flat! for flat extrapolation in the volatility.

\begin{longlisting}
%\hrule\medskip
\begin{minted}[fontsize=\footnotesize]{xml}
<EquityVolatilities>
  <EquityVolatility>
    <CurveId>SP5</CurveId>
    <CurveDescription>Lognormal option implied vols for SP 500</CurveDescription>
    <EquityId>RIC:.SP5</EquityId>
    <Currency>USD</Currency>
    <DayCounter>Actual/365 (Fixed)</DayCounter>
    <Curve>
      <QuoteType>ImpliedVolatility</QuoteType>
      <VolatilityType>Lognormal</VolatilityType>
      <Quotes>
        <Quote>EQUITY_OPTION/RATE_LNVOL/RIC:.SP5/USD/*</Quote>
      </Quotes>
      <Interpolation>LinearFlat</Interpolation>
      <Extrapolation>Flat</Extrapolation>
    </Curve>
  </EquityVolatility>
  <EquityVolatility>
    ...
  </EquityVolatility>
</EquityVolatilities>
\end{minted}
\caption{Equity option volatility configuration - curve}
\label{lst:eqoptionvol_curve}
\end{longlisting}

The volatility strike surface configuration layout is given in Listing \ref{lst:eqoptionvol_surface}. This allows a full surface of \lstinline!Strikes! and \lstinline!Expiries! to be defined. The following are the valid nodes:

\begin{itemize}
\item
\lstinline!QuoteType!: either \lstinline!ImpliedVolatility! of \lstinline!Premium!, indicating the type of quotes provided in the market.

\item
\lstinline!ExerciseType! [Optional]: only valid when \lstinline!QuoteType! is \lstinline!Premium!. Valid types are \lstinline!European! and \lstinline!American!.

\item
\lstinline!VolatilityType! [Optional]: only valid when \lstinline!QuoteType! is \lstinline!ImpliedVolatility!. Valid types are \lstinline!Lognormal!, \lstinline!ShiftedLognormal! and \lstinline!Normal!.

\item
\lstinline!Strikes!: comma separated list of strikes, representing the absolute strike values for the option. In other words, A single wildcard character, \lstinline!*!, can be used here also to indicate that all strikes found in the market data for this equity volatility configuration should be used when building the equity volatility surface.

\item
\lstinline!Expiries!: comma separated list of expiry tenors and or expiry dates. A single wildcard character, \lstinline!*!, can be used here also to indicate that all expiries found in the market data for this equity volatility configuration should be used when building the equity volatility surface.

\item
\lstinline!TimeInterpolation!: interpolation in the option expiry direction. If either \lstinline!Strikes! or \lstinline!Expiries! are configured with a wildcard character, \lstinline!Linear! is used. If both \lstinline!Strikes! and \lstinline!Expiries! are configured explicitly, \lstinline!Linear! or \lstinline!Cubic! is allowed here but the value must agree with the value for \lstinline!StrikeInterpolation!.

\item
\lstinline!StrikeInterpolation!: interpolation in the strike direction. If either \lstinline!Strikes! or \lstinline!Expiries! are configured with a wildcard character, \lstinline!Linear! is used. If both \lstinline!Strikes! and \lstinline!Expiries! are configured explicitly, \lstinline!Linear! or \lstinline!Cubic! is allowed here but the value must agree with the value for \lstinline!TimeInterpolation!.

\item
\lstinline!Extrapolation!: boolean value. If \lstinline!true!, extrapolation is allowed. If \lstinline!false!, extrapolation is not allowed.

\item
\lstinline!TimeExtrapolation!: extrapolation in the option expiry direction. If both \lstinline!Strikes! and \lstinline!Expiries! are configured explicitly, the extrapolation in the time direction is flat in volatility regardless of the setting here. If either \lstinline!Strikes! or \lstinline!Expiries! are configured with a wildcard character, \lstinline!Linear!, \lstinline!UseInterpolator!, \lstinline!Flat! or \lstinline!None! are allowed. If \lstinline!Linear! or \lstinline!UseInterpolator! is specified, the extrapolation is linear. If \lstinline!Flat! is specified, the extrapolation is flat. If \lstinline!None! is specified, it is ignored and the extrapolation is flat since extrapolation in the time direction cannot be turned off in isolation i.e.\ extrapolation can only be turned off for the surface as a whole using the \lstinline!Extrapolation! flag.

\item
\lstinline!StrikeExtrapolation!: extrapolation in the strike direction. The allowable values are \lstinline!Linear!, \lstinline!UseInterpolator!, \lstinline!Flat! or \lstinline!None!. If \lstinline!Linear! or \lstinline!UseInterpolator! is specified, the extrapolation uses the strike interpolation setting for extrapolation i.e. linear or cubic in this case. If \lstinline!Flat! is specified, the extrapolation is flat. If \lstinline!None! is specified, it is ignored and the extrapolation is flat since extrapolation in the strike direction cannot be turned off in isolation i.e.\ extrapolation can only be turned off for the surface as a whole using the \lstinline!Extrapolation! flag.

\end{itemize}

When this configuration is used, the market is searched for quote strings of the form \lstinline!EQUITY_OPTION/PRICE/[NAME]/[CURRENCY]/[EXPIRY]/[STRIKE]! or \lstinline!EQUITY_OPTION/RATE_LNVOL/[NAME]/[CURRENCY]/[EXPIRY]/[STRIKE]!, depending on the \lstinline!QuoteType!. When both the \lstinline!Strikes! and \lstinline!Expiries! are configured explicitly, it is clear that the \lstinline![EXPIRY]! field is populated from the list of expiries in turn and the \lstinline![STRIKE]! field is populated from the list of strikes in turn. If there are $m$ expiries in the \lstinline!Expiries! list and $n$ strikes in the \lstinline!Strikes! list, there will be $m \times n$ quotes created and searched for in the market data. If \lstinline!Expiries! are configured via the wildcard character, \lstinline!*!, all quotes in the market data matching the pattern \lstinline!EQUITY_OPTION/RATE_LNVOL/[NAME]/[CURRENCY]/*/[STRIKE]!. Similarly for \lstinline!Strikes! configured via the wildcard character, \lstinline!*!.

\begin{longlisting}
%\hrule\medskip
\begin{minted}[fontsize=\footnotesize]{xml}
<EquityVolatilities>
  <EquityVolatility>
    <CurveId>SP5</CurveId>
    <CurveDescription>Lognormal option implied vols for SP 500</CurveDescription>
    <EquityId>RIC:.SP5</EquityId>
    <Currency>USD</Currency>
    <DayCounter>Actual/365 (Fixed)</DayCounter>
    <StrikeSurface>
      <QuoteType>Premium</QuoteType>
      <ExerciseType>European</ExerciseType>
      <Strikes>*</Strikes>
      <Expiries>*</Expiries>
      <TimeInterpolation>Linear</TimeInterpolation>
      <StrikeInterpolation>Linear</StrikeInterpolation>
      <Extrapolation>true</Extrapolation>
      <TimeExtrapolation>UseInterpolator</TimeExtrapolation>
      <StrikeExtrapolation>Flat</StrikeExtrapolation>
    </StrikeSurface>
  </EquityVolatility>
  <EquityVolatility>
    ...
  </EquityVolatility>
</EquityVolatilities>
\end{minted}
\caption{Equity option volatility configuration - strike surface}
\label{lst:eqoptionvol_surface}
\end{longlisting}

A volatility surface can also be given in terms of moneyness levels as shown in listing
\ref{lst:eqoptionvol_mny_surface}. The nodes have the same meaning as in the case of a strike surface with the following
exceptions:

\begin{itemize}
\item \lstinline!QuoteType!: only \lstinline!ImpliedVolatility! is allowed
\item \lstinline!VolatilityType! [Optional]: only \lstinline!Lognormal! is allowed
\item \lstinline!MoneynessType!: \lstinline!Spot! or \lstinline!Fwd!, indicating the type of moneyness. See
  \ref{md:equity_option_iv} for the definition of moneyness types.
\item \lstinline!MoneynessLevels!: comma separated list of moneyness levels, no wild cards are allowed.
\item \lstinline!Expiries!: comma separated list of expiry tenors and or expiry dates. A single wildcard character, \lstinline!*!, can be used here also to indicate that all expiries found in the market data for this equity volatility configuration should be used when building the equity volatility surface.
\end{itemize}

Notice that the market data for the moneyness level $1.0$ must be given as a moneyness quote, not an ATM or ATMF quote,
see \ref{md:equity_option_iv} for details of the market data.

\begin{longlisting}
%\hrule\medskip
\begin{minted}[fontsize=\footnotesize]{xml}
<EquityVolatilities>
  <EquityVolatility>
    <CurveId>SP5</CurveId>
    <CurveDescription>Lognormal option implied vols for SP 500</CurveDescription>
    <EquityId>RIC:.SP5</EquityId>
    <Currency>USD</Currency>
    <DayCounter>Actual/365 (Fixed)</DayCounter>
    <MoneynessSurface>
      <QuoteType>ImpliedVolatility</QuoteType>
      <VolatilityType>Lognormal</VolatilityType>
      <MoneynessType>Fwd</MoneynessType>
      <MoneynessLevels>0.5,0.6,0.7,0.8,0.9,1.0,1.1,1.2,1.3,1.4,1.5</MoneynessLevels>
      <Expiries>*</Expiries>
      <TimeInterpolation>Linear</TimeInterpolation>
      <StrikeInterpolation>Linear</StrikeInterpolation>
      <Extrapolation>true</Extrapolation>
      <TimeExtrapolation>UseInterpolator</TimeExtrapolation>
      <StrikeExtrapolation>Flat</StrikeExtrapolation>
    </MoneynessSurface>
  </EquityVolatility>
  <EquityVolatility>
    ...
  </EquityVolatility>
</EquityVolatilities>
\end{minted}
\caption{Equity option volatility configuration - moneyness surface}
\label{lst:eqoptionvol_mny_surface}
\end{longlisting}

Finally, the volatility proxy surface configuration layout is given in Listing \ref{lst:eqoptionvol_proxy}. This allows us to use any other surface as a proxy, in cases where there is no option data for a given equity. We provide a name in the \lstinline!EquityVolatilityCurve! field, which must match the \lstinline!CurveId! of another configuration.  \lstinline!FXVolatilityCurve! and \lstinline!CorrelationCurve! must be provided if the currency of the proxy surface is different to that of current surface, that can be omitted otherwise. The proxy surface looks up the volatility in the reference surface based on moneyness.
 
\begin{longlisting}
%\hrule\medskip
\begin{minted}[fontsize=\footnotesize]{xml}
<EquityVolatilities>
  <EquityVolatility>
    <CurveId>ABC</CurveId>
    <CurveDescription>Lognormal option implied vols for APC - proxied from SP5</CurveDescription>
    <EquityId>RIC:.SP5</EquityId>
    <Currency>USD</Currency>
    <DayCounter>Actual/365 (Fixed)</DayCounter>
    <ProxySurface>
      <EquityVolatilityCurve>RIC:.SPX</EquityVolatilityCurve>
      <FXVolatilityCurve>GBPUSD</FXVolatilityCurve>
      <CorrelationCurve>FX-GENERIC-GBP-USD&amp;EQ-RIC:VOD.L</CorrelationCurve>
    </ProxySurface>
  </EquityVolatility>
  <EquityVolatility>
    ...
  </EquityVolatility>
</EquityVolatilities>
\end{minted}
\caption{Equity option volatility configuration - proxy surface}
\label{lst:eqoptionvol_proxy}
\end{longlisting}

\subsubsection{Inflation Curves}

Listing \ref{lst:inflationcurve_configuration} shows the configuration of an inflation curve. The inflation curve specific
elements are the following:

\begin{longlisting}
%\hrule\medskip
\begin{minted}[fontsize=\footnotesize]{xml}
<InflationCurves>
   <InflationCurve>
       <CurveId>USCPI_ZC_Swaps</CurveId>
       <CurveDescription>Estimation Curve for USCPI</CurveDescription>
       <NominalTermStructure>Yield/USD/USD1D</NominalTermStructure>
       <Type>ZC</Type>
       <Quotes>
           <Quote>ZC_INFLATIONSWAP/RATE/USCPI/1Y</Quote>
           <Quote>ZC_INFLATIONSWAP/RATE/USCPI/2Y</Quote>
           ...
           <Quote>ZC_INFLATIONSWAP/RATE/USCPI/30Y</Quote>
           <Quote>ZC_INFLATIONSWAP/RATE/USCPI/40Y</Quote>
       </Quotes>
       <Conventions>USCPI_INFLATIONSWAP</Conventions>
       <Extrapolation>true</Extrapolation>
       <Calendar>US</Calendar>
       <DayCounter>A365</DayCounter>
       <Lag>3M</Lag>
       <Frequency>Monthly</Frequency>
       <BaseRate>0.01</BaseRate>
       <Tolerance>0.000000000001</Tolerance>
       <Seasonality>
           <BaseDate>20160101</BaseDate>
           <Frequency>Monthly</Frequency>
           <Factors>
               <Factor>SEASONALITY/RATE/MULT/USCPI/JAN</Factor>
               <Factor>SEASONALITY/RATE/MULT/USCPI/FEB</Factor>
               <Factor>SEASONALITY/RATE/MULT/USCPI/MAR</Factor>
               <Factor>SEASONALITY/RATE/MULT/USCPI/APR</Factor>
               <Factor>SEASONALITY/RATE/MULT/USCPI/MAY</Factor>
               <Factor>SEASONALITY/RATE/MULT/USCPI/JUN</Factor>
               <Factor>SEASONALITY/RATE/MULT/USCPI/JUL</Factor>
               <Factor>SEASONALITY/RATE/MULT/USCPI/AUG</Factor>
               <Factor>SEASONALITY/RATE/MULT/USCPI/SEP</Factor>
               <Factor>SEASONALITY/RATE/MULT/USCPI/OCT</Factor>
               <Factor>SEASONALITY/RATE/MULT/USCPI/NOV</Factor>
               <Factor>SEASONALITY/RATE/MULT/USCPI/DEC</Factor>
           </Factors>
           <OverrideFactors>
               1.00,1.00,1.00,1.00,1.00,1.00,1.00,1.00,1.00,1.00,1.00,1.00
           </OverrideFactors>
       </Seasonality>
   </InflationCurve>
</InflationCurves>
\end{minted}
\caption{Inflation Curve Configuration}
\label{lst:inflationcurve_configuration}
\end{longlisting}

\begin{itemize}
\item {\tt NominalTermStructure}: The interest rate curve to be used to strip the inflation curve.
\item {\tt Type}: The type of the curve, {\tt ZC} for zero coupon, {\tt YY} for year on year.
\item {\tt Quotes}: The instruments' market quotes from which to bootstrap the curve.
\item {\tt Conventions}: The conventions applicable to the curve instruments.
\item {\tt Lag}: The observation lag used in the term structure.
\item {\tt Frequency}: The frequency of index fixings.
\item {\tt BaseRate}: The rate at $t=0$, this introduces an additional degree of freedom to get a smoother curve. If not
  given, it is defaulted to the first market rate.
\end{itemize}

The optional seasonality block defines a multiplicative seasonality and contains the following elements:

\begin{itemize}
\item {\tt BaseDate}: Defines the first inflation period to which to apply the seasonality correction, only day and
  month matters, the year is ignored.
\item {\tt Frequency:} Defines the frequency of the factors (usually identical to the index's fixing frequency).
\item {\tt Factors:} Multiplicative seasonality correction factors, must be part of the market data.
\item {\tt OverrideFactors:} A numeric list of seasonality correction factors, replacing the Factors. This allows to
  specify a static seasonality correction that does not require market data quotes. If both Factors and OverrideFactors
  are given, the OverrideFactors are used. Otherwise only one of Factors, OverrideFactors is required in a seasonality
  block.
\end{itemize}

We note that if zero coupon swap market quotes are given, but the type is set to YY, the zero coupon swap quotes will be
converted to year on year swap quotes on the fly, using the plain forward rates, i.e. no convexity adjustment is
applied.

\subsubsection{Inflation Cap/Floor Volatility Surfaces}

Listing \ref{lst:inflationcapfloorvolsurface_configuration} shows the configuration of an zero coupon inflation cap
floor price surface.

\begin{longlisting}
%\hrule\medskip
\begin{minted}[fontsize=\footnotesize]{xml}
      <InflationCapFloorVolatility>
          <CurveId>EUHICPXT_ZC_CF</CurveId>
          <CurveDescription>
              EUHICPXT CPI Cap/Floor vol surface derived from price quotes
          </CurveDescription>
          <Type>ZC</Type>
          <QuoteType>Price</QuoteType>
          <VolatilityType>Normal</VolatilityType>
          <Extrapolation>Y</Extrapolation>
          <DayCounter>ACT</DayCounter>
          <Calendar>TARGET</Calendar>
          <BusinessDayConvention>MF</BusinessDayConvention>
          <Tenors>1Y,2Y,3Y,4Y,5Y,6Y,7Y,8Y,9Y,10Y,12Y,15Y,20Y,30Y</Tenors>
          <!-- QuoteType 'Volatility' requires <Strikes>: -->
          <!-- <Strikes>-0.02,-0.01,-0.005,0.00,0.01,0.015,0.02,0.025,0.03</Strikes> -->
          <!-- QuoteType 'Price' requires <CapStrikes> and/or <FloorStrikes>: -->
          <CapStrikes/> 
          <FloorStrikes>-0.02,-0.01,-0.005,0.00,0.01,0.015,0.02,0.025,0.03</FloorStrikes>
          <Index>EUHICPXT</Index>
          <IndexCurve>Inflation/EUHICPXT/EUHICPXT_ZC_Swaps</IndexCurve>
          <IndexInterpolated>false</IndexInterpolated>
          <ObservationLag>3M</ObservationLag>
          <YieldTermStructure>Yield/EUR/EUR1D</YieldTermStructure>
          <QuoteIndex>...</QuoteIndex>
      </InflationCapFloorVolatility>
\end{minted}
\caption{Inflation ZC cap floor volatility surface configuration}
\label{lst:inflationcapfloorvolsurface_configuration}
\end{longlisting}

\begin{itemize}
\item {\tt Type}: The type of the surface, {\tt ZC} for zero coupon, {\tt YY} for year on year. 
\item {\tt QuoteType}: The type of the quotes used to build the surface, {\tt Volatility} for volatility quotes, {\tt Price} for bootstrap from option premia. 
\item {\tt VolatilityType}: If QuoteType is {\tt Volatility}, this specifies whether the input is {\tt Normal}, {\tt Lognormal} or {\tt ShiftedLognormal}. 
\item {\tt Extrapolation}: Boolean to indicate whether the surface should allow extrapolation. 
\item {\tt DayCounter}: The term structure's day counter
\item {\tt Calendar}: The term structure's calendar 
\item {\tt BusinessDayConvention}: The term structure's business day convention
\item {\tt Tenors:} The maturities for which cap and floor prices are quoted
\item {\tt Strikes}: In the case of QuoteType {\tt Volatility}, the strikes for which floor prices are quoted (may, and will usually, overlap with the cap
  strike region); neither CapStrikes nor FloorStrikes are expected in this case. 
\item {\tt CapStrikes}: The strikes for which cap prices are quoted (may, and will usually, overlap with the floor
  strike region); if the QuoteType above is {\tt Price}, either CapStrikes or FloorStrikes or both are required.
\item {\tt FloorStrikes}: The strikes for which floor prices are quoted (may and will usually) overlap with the cap
  strike region); if the QuoteType above is {\tt Price}, either CapStrikes or FloorStrikes or both are required. 
\item {\tt Index}: The underlying zero inflation index.
\item {\tt IndexCurve}: The curve id of the index's projection curve used to determine the ATM levels for the surface.
\item {\tt IndexInterpolated}: Flag indicating whether the index should be interpolating.
\item {\tt Observation Lag}: The observation lag applicable to the term structure.
\item {\tt YieldTermStructure}: The nominal term structure.
\item \lstinline!QuoteIndex!: An optional node allowing the user to provide an alternative index name for forming the quotes that will be used in building the cap floor surface. If this node is not provided, the \lstinline!Index! node value is used in quote construction. For example, quotes must be created from each strike and each tenor and these quotes are subsequently looked up in the market data when building the cap floor volatility surface. The quotes are formed by concatenating \lstinline![Type]_INFLATIONCAPFLOOR!, \lstinline!PRICE! or \lstinline!RATE_[Vol_Type]VOL!, \lstinline![Index_Name]!, \lstinline![Tenor]!, \lstinline!C! or \lstinline!F! and \lstinline![Strike]! delimited by \lstinline!/!. If \lstinline!QuoteIndex! is provided, it is used as the \lstinline![Index_Name]! token. If it is not provided \lstinline!Index! is used as usual.
\end{itemize}

\subsubsection{CDS Volatilities}

When configuring volatility structures for CDS and index CDS options, there are three options:
\begin{enumerate}
\item a constant volatility for all expiries, strikes and terms.
\item a volatility curve with a dependency on expiry and term, but no strike dimension.
\item a volatility surface with an expiry, term and strike dimension.
\end{enumerate}

Firstly, the constant volatility configuration layout is given in Listing \ref{lst:cdsvol_constant_config}. The single volatility quote ID, \lstinline!constant_quote_id!, in the \lstinline!Quote! node should be a CDS option volatility quote as described in Section \ref{md:cds_option_iv}. The \lstinline!DayCounter! node is optional and defaults to \lstinline!A365F!. The \lstinline!Calendar! node is optional and defaults to \lstinline!NullCalendar!. The \lstinline!DayCounter! and \lstinline!Calendar! nodes are common to all three CDS volatility configurations.

\begin{longlisting}
\begin{minted}[fontsize=\footnotesize]{xml}
<CDSVolatility>
  <CurveId>..<CurveId>
  <CurveDescription>...</CurveDescription>
  <Constant>
    <Quote>constant_quote_id</Quote>
  </Constant>
  <DayCounter>...</DayCounter>
  <Calendar>...</Calendar>
</CDSVolatility>
\end{minted}
\caption{Constant CDS volatility configuration}
\label{lst:cdsvol_constant_config}
\end{longlisting}

Secondly, the volatility curve configuration layout is given in Listing \ref{lst:cdsvol_curve_config}. The volatility quote IDs, \lstinline!quote_id_1!, \lstinline!quote_id_2!, etc., in the \lstinline!Quotes! node should be CDS option volatility quotes as described in Section \ref{md:cds_option_iv}. The \lstinline!Interpolation! node supports \lstinline!Linear!, \lstinline!Cubic! and \lstinline!LogLinear! interpolation. The \lstinline!Extrapolation! node supports either \lstinline!None! for no extrapolation or \lstinline!Flat! for flat extrapolation in the volatility.

The optional boolean parameter \lstinline!EnforceMontoneVariance! should be set to \lstinline!true! to raise an exception if the curve implied variance is not montone increasing with time and should be set to \lstinline!false! if you want to suppress such an exception. The default value for \lstinline!EnforceMontoneVariance! is \lstinline!true!.

\begin{longlisting}
\begin{minted}[fontsize=\footnotesize]{xml}
<CDSVolatility>
  <CurveId>..<CurveId>
  <CurveDescription>...</CurveDescription>
  <Terms>
    <Term>
      <Label>...</Label>
      <Curve>...</Curve>
    </Term>
  </Terms>
  <Curve>
    <Quotes>
      <Quote>quote_id_1</Quote>
      <Quote>quote_id_2</Quote>
      ...
    </Quotes>
    <Interpolation>...</Interpolation>
    <Extrapolation>...</Extrapolation>
    <EnforceMontoneVariance></EnforceMontoneVariance>
  </Curve>
  <DayCounter>...</DayCounter>
  <Calendar>...</Calendar>
</CDSVolatility>
\end{minted}
\caption{CDS volatility curve configuration}
\label{lst:cdsvol_curve_config}
\end{longlisting}

For backwards compatibility, the volatility curve configuration can also be given using a single \lstinline!Expiries! node as shown in Listing \ref{lst:cdsvol_curve_config_alt}. Note that this configuration is deprecated and the configuration in \ref{lst:cdsvol_curve_config} is preferred. The \lstinline!Expiries! node should take a comma separated list of tenors and or expiry dates e.g.\ \lstinline!<Expiries>1M,3M,6M</Expiries>!. The list of expiries are extracted and a set of quotes are created of the form \lstinline!INDEX_CDS_OPTION/RATE_LNVOL/[NAME]/[EXPIRY]! or \lstinline!INDEX_CDS_OPTION/RATE_LNVOL/[NAME]/[TERM]/[EXPIRY]!. There is one quote for each expiry in the list where the \lstinline![EXPIRY]! field is understood to be replaced with the expiry string extracted from the list.

The \lstinline![NAME]! field is populated with the curve id or with the \lstinline![QuoteName]! if that is specified. The rules for including market quotes into the volatility surface construction are as follows:

\begin{itemize}
\item All quotes explicitly specified with their full name are loaded (applies to configs of type constant or curve
  without wildcards)
\item If a quote does not contain a term, we only load it if at most one term is specified in the vol curve config. The
  quote gets the unique term specified in the vol curve configs assigned or 5Y if the config does not specify any terms.
\item If a quote contains a term if this matches one of the configured terms in the curve configuration or if the curve
  configuration does not specify any terms.
\end{itemize}

The \lstinline![Terms]! node specifies a list of term labels ``5Y'', ``7Y'', ... and associated credit curve spec names
representing the curve suitable to estimate the ATM level for that term.

If only one expiry is provided in the list, there is only one quote and a constant volatility structure is configured as in Listing \ref{lst:cdsvol_constant_config}. If more than one expiry is provided, a curve is configured as in \ref{lst:cdsvol_curve_config}. The interpolation is \lstinline!Linear!, the extrapolation is \lstinline!Flat! and \lstinline!EnforceMontoneVariance! is \lstinline!true!.

\begin{longlisting}
\begin{minted}[fontsize=\footnotesize]{xml}
<CDSVolatility>
  <CurveId>..<CurveId>
  <CurveDescription>...</CurveDescription>
  <Terms>
    <Term>
      <Label>...</Label>
      <Curve>...</Curve>
    </Term>
  </Terms>
  <Expiries>...</Expiries>
  <DayCounter>...</DayCounter>
  <Calendar>...</Calendar>
  <QuoteName>...</QuoteName>
</CDSVolatility>
\end{minted}
\caption{Legacy deprecated CDS volatility curve configuration}
\label{lst:cdsvol_curve_config_alt}
\end{longlisting}

Thirdly, the volatility surface configuration layout is given in Listing \ref{lst:cdsvol_surface_config}. The nodes have the following meanings and supported values:
\begin{itemize}
\item
\lstinline!Strikes!: comma separated list of strikes. The strikes may be in terms of spread or price. However, it is important to ensure that the trade XML for a CDS option or index CDS option provides the strike in the same way. In other words, if the strike is in terms of spread on the trade XML, the strike must be in terms of spread in the CDS volatility configuration here. Similarly for strikes in terms of price. A single wildcard character, \lstinline!*!, can be used here also to indicate that all strikes found in the market data for this CDS volatility configuration should be used when building the CDS volatility surface.

\item
\lstinline!Expiries!: comma separated list of expiry tenors and or expiry dates. A single wildcard character, \lstinline!*!, can be used here also to indicate that all expiries found in the market data for this CDS volatility configuration should be used when building the CDS volatility surface.

\item
\lstinline!TimeInterpolation!: interpolation in the option expiry direction. If either \lstinline!Strikes! or \lstinline!Expiries! are configured with a wildcard character, \lstinline!Linear! is used. If both \lstinline!Strikes! and \lstinline!Expiries! are configured explicitly, \lstinline!Linear! or \lstinline!Cubic! is allowed here but the value must agree with the value for \lstinline!StrikeInterpolation!.

\item
\lstinline!StrikeInterpolation!: interpolation in the strike direction. If either \lstinline!Strikes! or \lstinline!Expiries! are configured with a wildcard character, \lstinline!Linear! is used. If both \lstinline!Strikes! and \lstinline!Expiries! are configured explicitly, \lstinline!Linear! or \lstinline!Cubic! is allowed here but the value must agree with the value for \lstinline!TimeInterpolation!.

\item
\lstinline!Extrapolation!: boolean value. If \lstinline!true!, extrapolation is allowed. If \lstinline!false!, extrapolation is not allowed.

\item
\lstinline!TimeExtrapolation!: extrapolation in the option expiry direction. If both \lstinline!Strikes! and \lstinline!Expiries! are configured explicitly, the extrapolation in the time direction is flat in volatility regardless of the setting here. If either \lstinline!Strikes! or \lstinline!Expiries! are configured with a wildcard character, \lstinline!Linear!, \lstinline!UseInterpolator!, \lstinline!Flat! or \lstinline!None! are allowed. If \lstinline!Linear! or \lstinline!UseInterpolator! is specified, the extrapolation is linear. If \lstinline!Flat! is specified, the extrapolation is flat. If \lstinline!None! is specified, it is ignored and the extrapolation is flat since extrapolation in the time direction cannot be turned off in isolation i.e.\ extrapolation can only be turned off for the surface as a whole using the \lstinline!Extrapolation! flag.

\item
\lstinline!StrikeExtrapolation!: extrapolation in the strike direction. The allowable values are \lstinline!Linear!, \lstinline!UseInterpolator!, \lstinline!Flat! or \lstinline!None!. If \lstinline!Linear! or \lstinline!UseInterpolator! is specified, the extrapolation uses the strike interpolation setting for extrapolation i.e. linear or cubic in this case. If \lstinline!Flat! is specified, the extrapolation is flat. If \lstinline!None! is specified, it is ignored and the extrapolation is flat since extrapolation in the strike direction cannot be turned off in isolation i.e.\ extrapolation can only be turned off for the surface as a whole using the \lstinline!Extrapolation! flag.

\item
\lstinline!DayCounter!: allowable value is any valid day count fraction. As stated above, this node is optional and defaults to \lstinline!A365F!.

\item
\lstinline!Calendar!: allowable value is any valid calendar. As stated above, this node is optional and defaults to \lstinline!NullCalendar!.

\item
\lstinline!StrikeType!: allowable value is either \lstinline!Price! or \lstinline!Spread!. This flag denotes if the strikes are in terms of spread or price. Currently, this is merely informational and as outlined in the \lstinline!Strikes! section above, it is the responsibility of the user to ensure that the strike type in trades aligns with the configured strike type in the CDS volatility surfaces.

\item
\lstinline!QuoteName!: this node is optional and the allowable value is any string. This value can be used in determining the name and term that appears in the quote strings that are searched for in the market data to feed into the CDS volatility surface construction. How it is used has been outlined above when describing the deprecated CDS volatility curve configuration.

\item
\lstinline!StrikeFactor!: this node is optional and the allowable value is any positive real number. It defaults to 1. The strikes configured and found in the market data quote strings may not be in absolute terms. For example, a quote string such as \lstinline!INDEX_CDS_OPTION/RATE_LNVOL/CDXIGS33V1/5Y/1M/115! could be given to indicate an index CDS option volatility quote for CDX IG Series 33 Version 1, with underlying index term 5Y expiring in 1M with a strike spread of 115 bps. The strike here is in bps and needs to be divided by 10,000 before being used in the ORE volatility objects. The \lstinline!StrikeFactor! would be set to \lstinline!10000! here.

\end{itemize}

When the CDS volatility surface is configured as in Listing \ref{lst:cdsvol_surface_config}, the market is searched for quote strings of the form \lstinline!INDEX_CDS_OPTION/RATE_LNVOL/[NAME]/[EXPIRY]/[STRIKE]! or \lstinline!INDEX_CDS_OPTION/RATE_LNVOL/[NAME]/[TERM]/[EXPIRY]/[STRIKE]!. The population of the \lstinline![NAME]! field, and possibly the \lstinline![TERM]! field, and how they depend on the \lstinline!QuoteName! and \lstinline!ParseTerm! nodes has been discussed at length above when describing the deprecated CDS volatility curve configuration. When both the \lstinline!Strikes! and \lstinline!Expiries! are configured explicitly, it is clear that the \lstinline![EXPIRY]! field is populated from the list of expiries in turn and the \lstinline![STRIKE]! field is populated from the list of strikes in turn. If there are $m$ expiries in the \lstinline!Expiries! list and $n$ strikes in the \lstinline!Strikes! list, there will be $m \times n$ quotes created and searched for in the market data. If \lstinline!Expiries! are configured via the wildcard character, \lstinline!*!, all quotes in the market data matching the pattern \lstinline!INDEX_CDS_OPTION/RATE_LNVOL/[NAME]/*/[STRIKE]! will be used if \lstinline![TERM]! has not been populated and all quotes in the market data matching the pattern \lstinline!INDEX_CDS_OPTION/RATE_LNVOL/[NAME]/[TERM]/*/[STRIKE]! will be used if \lstinline![TERM]! has been populated. Similarly for \lstinline!Strikes! configured via the wildcard character, \lstinline!*!.

\begin{longlisting}
\begin{minted}[fontsize=\footnotesize]{xml}
<CDSVolatility>
  <CurveId/>
  <CurveDescription/>
  <Terms>
    <Term>
      <Label>...</Label>
      <Curve>...</Curve>
    </Term>
  </Terms>
  <StrikeSurface>
    <Strikes>...</Strikes>
    <Expiries>...</Expiries>
    <TimeInterpolation>...</TimeInterpolation>
    <StrikeInterpolation>...</StrikeInterpolation>
    <Extrapolation>...</Extrapolation>
    <TimeExtrapolation>...</TimeExtrapolation>
    <StrikeExtrapolation>...</StrikeExtrapolation>
  </StrikeSurface>
  <DayCounter>...</DayCounter>
  <Calendar>...</Calendar>
  <StrikeType>...</StrikeType>
  <QuoteName>...</QuoteName>
  <StrikeFactor>..</StrikeFactor>
  <ParseTerm>...</ParseTerm>
</CDSVolatility>
\end{minted}
\caption{CDS volatility surface configuration}
\label{lst:cdsvol_surface_config}
\end{longlisting}

\subsubsection{Base Correlations}

Listing \ref{lst:basecorr_configuration} shows the configuration of a Base Correlation curve.

\begin{longlisting}
%\hrule\medskip
\begin{minted}[fontsize=\footnotesize]{xml}
  <BaseCorrelations>
    <BaseCorrelation>
      <CurveId>CDXIG</CurveId>
      <CurveDescription>CDX IG Base Correlations</CurveDescription>
      <Terms>1D</Terms>
      <DetachmentPoints>0.03, 0.06, 0.10, 0.20, 1.0</DetachmentPoints>
      <SettlementDays>0</SettlementDays>
      <Calendar>US</Calendar>
      <BusinessDayConvention>F</BusinessDayConvention>
      <DayCounter>A365</DayCounter>
      <Extrapolate>Y</Extrapolate>
    </BaseCorrelation>
  </BaseCorrelations>
\end{minted}
\caption{Base Correlation Configuration}
\label{lst:basecorr_configuration}
\end{longlisting}

The meaning of each of the elements in Listing \ref{lst:basecorr_configuration} is given below. 

\begin{itemize}
\item CurveId: Unique identifier of the base correlation structure
\item CurveDescription [Optional]: A description of the base correlation structure, may be left blank.
\item Terms: Comma-separated list of tenors, sorted in increasing order, possibly a single term to represent a flat term structure in time-direction
\item DetachmentPoints: Comma-separated list of equity tranche detachment points, sorted in increasing order\\
Allowable values: Any positive number less than one
\item SettlementDays: The floating term structure's settlement days argument used in the reference date calculation
\item DayCounter: The term structure's day counter used in date to time conversions
\item Calendar: The term structure's calendar used in tenor to date conversions
\item BusinessDayConvention: The term structure's business day convention used in tenor to date conversion
\item Extrapolate: Boolean to indicate whether the correlation curve shall be extrapolated or not
\end{itemize}

\subsubsection{FXSpots}

Listing \ref{lst:fxspot_configuration} shows the configuration of the fxSpots. It is assumed that each FXSpot CurveId is of
the form "Ccy1Ccy2".
\begin{longlisting}
%\hrule\medskip
\begin{minted}[fontsize=\footnotesize]{xml}
<FXSpots>
  <FXSpot>
    <CurveId>EURUSD</CurveId>
    <CurveDescription/>
  </FXSpot>
  <FXSpot>
    <CurveId>EURGBP</CurveId>
    <CurveDescription/>
  </FXSpot>
  <FXSpot>
    <CurveId>EURCHF</CurveId>
    <CurveDescription/>
  </FXSpot>
  <FXSpot>
    <CurveId>EURJPY</CurveId>
    <CurveDescription/>
  </FXSpot>
</FXSpots>
\end{minted}
\caption{FXSpot Configuration}
\label{lst:fxspot_configuration}
\end{longlisting}

\subsubsection{Securities}

Listing \ref{lst:security_configuration} shows the configuration of the Securities. Each Security name is associated with

\begin{itemize}
\item an optional SpreadQuote
\item an optional RecoveryRateQuote. Usually a pricer will fall back on the recovery rate associated to the credit curve
  involved in the pricing if no security specific recovery rate is given. If no credit curve is given either, a zero
  recovery rate will be assumed.
\end{itemize}

If no configuration is given for a security, in general a pricer will assume as zero spread and recovery rate. Notice
that in this case the spread and recovery will not be simulated and therefore also be excluded from the sensitivity and
stress analysis.

\begin{longlisting}
%\hrule\medskip
\begin{minted}[fontsize=\footnotesize]{xml}
<Securities>
  <Security>
    <CurveId>SECURITY_1</CurveId>
    <CurveDescription>Security</CurveDescription>
    <SpreadQuote>BOND/YIELD_SPREAD/SECURITY_1</SpreadQuote>
    <RecoveryRateQuote>RECOVERY_RATE/RATE/SECURITY_1</RecoveryRateQuote>
  </Security>
</Securities>
\end{minted}
\caption{Security Configuration}
\label{lst:security_configuration}
\end{longlisting}

\subsubsection{Correlations}

Listing \ref{lst:correlation_configuration} shows the configuration of the Correlations. The Correlation type can be either CMSSpread or Generic. The former one is to configure the correlation between two CMS indexes, the latter one is to  generally configure the correlation between two indexes, e.g. between a CMS index and a IBOR index. Currently only ATM correlation curves or Flat correlation structures are supported. Correlation quotes may be loaded directly (by setting QuoteType to RATE) or calibrated from prices (set QuoteType to PRICE).  Moreover a flat zero correlation curve can be loaded (by setting  QuoteType to NULL). In this case market quotes are not needed to be provided. Currently only CMSSpread correlations can be calibrated. This is done using CMS Spread Options, and requires a CMSSpreadOption convention, SwaptionVolatility and DiscountCurve to be provided. OptionTenors can be a comma separated list of periods, 1Y,2Y etc, or a \lstinline!*! to indicate a wildcard. If a wildcard is provided, all relevant market data quotes are used.

\begin{longlisting}
%\hrule\medskip
\begin{minted}[fontsize=\footnotesize]{xml}
  <Correlations>
    <Correlation>
      <CurveId>EUR-CORR</CurveId>
      <CurveDescription>EUR CMS correlations</CurveDescription>
      <!--CMSSpread, Generic-->
      <CorrelationType>CMSSpread</CorrelationType>
      <Index1>EUR-CMS-10Y</Index1>
      <Index2>EUR-CMS-2Y</Index2>
      <!--Conventions, SwaptionVolatility and DiscountCurve only required when QuoteType = PRICE-->
      <Conventions>EUR-CMS-10Y-2Y-CONVENTION</Conventions>
      <SwaptionVolatility>EUR</SwaptionVolatility>
      <DiscountCurve>EUR-EONIA</DiscountCurve>
      <Currency>EUR</Currency>
      <!-- ATM, Constant -->
      <Dimension>ATM</Dimension>
      <!-- RATE, PRICE, NULL -->
      <QuoteType>PRICE</QuoteType>
      <Extrapolation>true</Extrapolation>
      <!-- Day counter for date to time conversion -->
      <DayCounter>Actual/365 (Fixed)</DayCounter>
      <!--Ccalendar and Business day convention for option tenor to date conversion -->
      <Calendar>TARGET</Calendar>
      <BusinessDayConvention>Following</BusinessDayConvention>
      <OptionTenors>1Y,2Y</OptionTenors>
    </Correlation>
  </Correlations>\end{minted}
\caption{Correlation Configuration}
\label{lst:correlation_configuration}
\end{longlisting}

\subsubsection{Commodity Curves}

Commodity Curves are setup as price curves in ORE, meaning that they return a price for a given time $t$ rather than a rate or discount factor, these curves are common in commodities and can be populated with futures prices directly.

Listing \ref{lst:commodity_curve_configuration_1} shows the configuration of Commodity curves built from futures prices, in this example WTI Oil prices, note there is no spot price in this configuration, rather we have a set of futures prices only.

\begin{longlisting}
\begin{minted}[fontsize=\footnotesize]{xml}
<CommodityCurve>
  <CurveId>WTI_USD</CurveId>
  <CurveDescription>WTI USD price curve</CurveDescription>
  <Currency>USD</Currency>
  <Quotes>
    <Quote>COMMODITY_FWD/PRICE/WTI/USD/2016-06-30</Quote>
    <Quote>COMMODITY_FWD/PRICE/WTI/USD/2016-07-31</Quote>
    ...
  </Quotes>
  <DayCounter>A365</DayCounter>
  <InterpolationMethod>Linear</InterpolationMethod>
  <Extrapolation>true</Extrapolation>
</CommodityCurve>
\end{minted}
\caption{Commodity Curve Configuration for WTI Oil}
\label{lst:commodity_curve_configuration_1}
\end{longlisting}

Listing \ref{lst:commodity_curve_configuration_2} shows the configuration for a Precious Metal curve using FX style spot and forward point quotes, note that SpotQuote is used in this case. The different interpretation of the forward quotes is controlled by the conventions.

\begin{longlisting}
\begin{minted}[fontsize=\footnotesize]{xml}
<CommodityCurve>
  <CurveId>XAU</CurveId>
  <CurveDescription>Gold USD price curve</CurveDescription>
  <Currency>USD</Currency>
  <SpotQuote>COMMODITY/PRICE/XAU/USD</SpotQuote>
  <Quotes>
    <Quote>COMMODITY_FWD/PRICE/XAU/USD/ON</Quote>
    <Quote>COMMODITY_FWD/PRICE/XAU/USD/TN</Quote>
    <Quote>COMMODITY_FWD/PRICE/XAU/USD/SN</Quote>
    <Quote>COMMODITY_FWD/PRICE/XAU/USD/1W</Quote>
    ...
  </Quotes>
  <DayCounter>A365</DayCounter>
  <InterpolationMethod>Linear</InterpolationMethod>
  <Conventions>XAU</Conventions>
  <Extrapolation>true</Extrapolation>
</CommodityCurve>
\end{minted}
\caption{Commodity Curve Configuration for Gold in USD}
\label{lst:commodity_curve_configuration_2}
\end{longlisting}

The meaning of each of the top level elements is given below. If an element is labelled
as 'Optional', then it may be excluded or included and left blank.
\begin{itemize}
\item CurveId: Unique identifier for the curve.
\item CurveDescription: A description of the curve. This field may be left blank.
\item Currency: The commodity curve currency.
\item SpotQuote [Optional]: The spot price quote, if omitted then the spot value will be interpolated.
\item Quotes: forward price quotes. These can be a futures price or forward points.
\item DayCounter: The day count basis used internally by the curve to calculate the time between dates.
\item InterpolationMethod [Optional]: The variable on which the interpolation is performed. The allowable values are
Linear, LogLinear, Cubic, Hermite, LinearFlat, LogLinearFlat, CubicFlat, HermiteFlat, BackwardFlat. This is different to yield curves above in that Flat versions
of the standard methods are defined, with each of these if there is no Spot price than any extrapolation between $T_0$ and the
first future price will be flat (i.e. the first future price will be copied back "Flat" to $T_0$). 
If the element is omitted or left blank, then it defaults to \emph{Linear}.
\item Conventions [Optional]: The conventions to use, if omited it is assumed that these quotes are Outright prices.
\item Extrapolation [Optional]: Set to \emph{True} or \emph{False} to enable or disable extrapolation respectively. If
the element is omitted or left blank, then it defaults to \emph{True}.
\end{itemize}


Alternatively commodity curves can be set up as a commodity curve times the ratio of two yield curves as shown in listing
\ref{lst:commodity_curve_configuration_3}. This can be used to setup commodity curves in different currencies, for example Gold in EUR (XAUEUR) can be built from a Gold in USD curve and then the ratio of the EUR and USD discount factors at each pillar. This is akin to crossing FX forward points to get FX forward prices for any pair.

\begin{longlisting}
\begin{minted}[fontsize=\footnotesize]{xml}
<CommodityCurve>
  <CurveId>XAUEUR</CurveId>
  <CurveDescription>Gold EUR price curve</CurveDescription>
  <Currency>EUR</Currency>
  <BasePriceCurve>XAU</BasePriceCurve>
  <BaseYieldCurve>USD-FedFunds</BaseYieldCurve>
  <YieldCurve>EUR-IN-USD</YieldCurve>
  <Extrapolation>true</Extrapolation>
</CommodityCurve>
\end{minted}
\caption{Commodity Curve Configuration for Gold in EUR}
\label{lst:commodity_curve_configuration_3}
\end{longlisting}

Commodity curves may also be set up using a base future price curve and a set of future basis quotes to give an outright price curve. There are a number of options here depending on whether the base future and basis future are averaging or not averaging. Whether or not the base future and basis future is averaging is determined from the conventions supplied in the \lstinline!BasePriceConventions! and \lstinline!BasisConventions! fields. 
\begin{itemize}
\item
The base future is not averaging and the basis future is not averaging. The commodity curve that is built gives the outright price of the non-averaging future. An example of this is the CME Henry Hub future contract, symbol NG, and the various locational gas basis future contracts, e.g.\ ICE Waha Basis Future, symbol WAH. Listing \ref{lst:commodity_crv_config_ice_waha} demonstrates an example set-up for this curve. The curve that is built will give the ICE Waha outright price on the basis contract's expiry dates.

\item
The base future is not averaging and the basis future is averaging. The commodity curve that is built gives the outright price of the averaging future. In this case, if the \lstinline!AverageBase! field is \lstinline!true! the base price will be averaged from and excluding one basis future expiry to and including the next basis future expiry. An example of this is the CME Light Sweet Crude Oil future contract, symbol CL, and the various locational oil basis future contracts, e.g.\ CME WTI Midland (Argus) Future, symbol FF. Listing \ref{lst:commodity_crv_config_cme_ff} demonstrates an example set-up for this curve. The curve that is built will give the outright average price of WTI Midland (Argus) over the calendar month. If the \lstinline!AverageBase! field is \lstinline!false!, the base price is not averaged and the basis is added to the base price to give a curve that can be queried on an expiry date for an average price. An example of this is a curve built for the average of the daily prices published by Gas Daily using the ICE futures that reference the difference between the Inside FERC price and the average Gas Daily price.

\item
The base future is averaging and the basis future is averaging. The commodity curve that is built gives the outright price of the averaging future. The set-up is identical to that outlined in Listings \ref{lst:commodity_crv_config_ice_waha} and \ref{lst:commodity_crv_config_cme_ff}.

\item
The base future is averaging and the basis future is not averaging. This combination is not currently supported.
\end{itemize}

\begin{longlisting}
\begin{minted}[fontsize=\footnotesize]{xml}
<CommodityCurve>
  <CurveId>ICE:WAH</CurveId>
  <Currency>USD</Currency>
  <BasisConfiguration>
    <BasePriceCurve>NYMEX:NG</BasePriceCurve>
    <BasePriceConventions>NYMEX:NG</BasePriceConventions>
    <BasisQuotes>
      <Quote>COMMODITY_FWD/PRICE/ICE:WAH/*</Quote>
    </BasisQuotes>
    <BasisConventions>ICE:WAH</BasisConventions>
    <DayCounter>A365</DayCounter>
    <InterpolationMethod>LinearFlat</InterpolationMethod>
    <AddBasis>true</AddBasis>
  </BasisConfiguration>
  <Extrapolation>true</Extrapolation>
</CommodityCurve>
\end{minted}
\caption{Commodity curve configuration for ICE Waha}
\label{lst:commodity_crv_config_ice_waha}
\end{longlisting}

\begin{longlisting}
\begin{minted}[fontsize=\footnotesize]{xml}
<CommodityCurve>
  <CurveId>NYMEX:FF</CurveId>
  <Currency>USD</Currency>
  <BasisConfiguration>
    <BasePriceCurve>NYMEX:CL</BasePriceCurve>
    <BasePriceConventions>NYMEX:CL</BasePriceConventions>
    <BasisQuotes>
      <Quote>COMMODITY_FWD/PRICE/NYMEX:FF/*</Quote>
    </BasisQuotes>
    <BasisConventions>NYMEX:FF</BasisConventions>
    <DayCounter>A365</DayCounter>
    <InterpolationMethod>LinearFlat</InterpolationMethod>
    <AddBasis>true</AddBasis>
    <AverageBase>true</AverageBase>
    <PriceAsHistoricalFixing>true</PriceAsHistoricalFixing>
  </BasisConfiguration>
  <Extrapolation>true</Extrapolation>
</CommodityCurve>
\end{minted}
\caption{Commodity curve configuration for WTI Midland (Argus)}
\label{lst:commodity_crv_config_cme_ff}
\end{longlisting}

The meaning of the fields in the \lstinline!BasisConfiguration! node in Listings \ref{lst:commodity_crv_config_ice_waha} and \ref{lst:commodity_crv_config_cme_ff} are as follows:
\begin{itemize}
\item \lstinline!BasePriceCurve!: The identifier for the base future commodity price curve.
\item \lstinline!BasePriceConventions!: The identifier for the base future contract conventions.
\item \lstinline!BasisQuotes!: The set of basis quotes to look for in the market. Note that this can be a single wildcard string as shown in the Listings or a list of explicit \lstinline!COMMODITY_FWD! \lstinline!PRICE! quote strings.
\item \lstinline!BasisConventions!: The identifier for the basis future contract conventions.
\item \lstinline!DayCounter!: Has the meaning given previously in this section.
\item \lstinline!InterpolationMethod! [Optional]: Has the meaning given previously in this section.
\item \lstinline!AddBasis! [Optional]: This is a boolean flag where \lstinline!true!, the default value, indicates that the basis value should be added to the base price to give the outright price and \lstinline!false! indicates that the basis value should be subtracted from the base price to give the outright price.
\item \lstinline!MonthOffset! [Optional]: This is an optional non-negative integer value. In general, the basis contract period and the base contract period are equal, i.e.\ the value of the March basis contract for ICE Waha will be added to the value of thr March contract for CME NG. If for contracts with a monthly cycle or greater, the base contract month is $n$ months prior to the basis contract month, the \lstinline!MonthOffset! should be set to $n$. The default value if omitted is 0.
\item \lstinline!PriceAsHistoricalFixing! [Optional]: This is a boolean flag where \lstinline!true!, the default value, indicates that the historical fixings are prices of the underlying. If set to false, the fixings are basis spreads and ORE will convert them into prices by adding the corresponding base index fixings.

\end{itemize}

A commodity curve may also be set up as a piecewise price curve involving sets of quotes e.g. direct future price quotes, future price quotes that are the average of other future prices over a period, future price quotes that are the average of spot price over a period etc. This is particularly useful for cases where there are future contracts with different cycles. For example, in the power markets, there are daily future contracts at the short end and monthly future contracts that average the daily prices over the month at the long end. An example of such a set-up is shown in Listing \ref{lst:commodity_crv_config_ice_pdq}.

\begin{longlisting}
\begin{minted}[fontsize=\footnotesize]{xml}
<CommodityCurve>
  <CurveId>ICE:PDQ</CurveId>
  <Currency>USD</Currency>
  <PriceSegments>
    <PriceSegment>
      <Type>Future</Type>
      <Priority>1</Priority>
      <Conventions>ICE:PDQ</Conventions>
      <Quotes>
        <Quote>COMMODITY_FWD/PRICE/ICE:PDQ/*</Quote>
      </Quotes>
    </PriceSegment>
    <PriceSegment>
      <Type>AveragingFuture</Type>
      <Priority>2</Priority>
      <Conventions>ICE:PMI</Conventions>
      <Quotes>
        <Quote>COMMODITY_FWD/PRICE/ICE:PMI/*</Quote>
      </Quotes>
    </PriceSegment>
  </PriceSegments>
  <DayCounter>A365</DayCounter>
  <InterpolationMethod>LinearFlat</InterpolationMethod>
  <Extrapolation>true</Extrapolation>
  <BootstrapConfig>...</BootstrapConfig>
</CommodityCurve>
\end{minted}
\caption{Commodity curve configuration for PJM Real Time Peak}
\label{lst:commodity_crv_config_ice_pdq}
\end{longlisting}

The \lstinline!BootstrapConfig! node is described in Section \ref{sec:bootstrap_config}. The remaining nodes in Listing \ref{lst:commodity_crv_config_ice_pdq} have been described already in this section. The meaning of each of the fields in the \lstinline!PriceSegment! node in Listing \ref{lst:commodity_crv_config_ice_pdq} is as follows:
\begin{itemize}
\item \lstinline!Type!:
The type of the future contract for which quotes are being provided in the current \lstinline!PriceSegment!. The allowable values are:
    \begin{itemize}
    \item \lstinline!Future!: This indicates that the quote is a straight future contract price quote.
    \item \lstinline!AveragingFuture!: This indicates that the quote is for a future contract price that is the average of other future contract prices over a given period. The averaging period for each quote and other conventions are given in the associated conventions determined by the \lstinline!Conventions! node.
    \item \lstinline!AveragingSpot!: This indicates that the quote is for a future contract price that is the average of spot prices over a given period. The averaging period for each quote and other conventions are given in the associated conventions determined by the \lstinline!Conventions! node.
    \end{itemize}

\item \lstinline!Priority! [Optional]:
An optional non-negative integer giving the priority of the current \lstinline!PriceSegment! relative to the other \lstinline!PriceSegment!s when there are quotes for contracts with the same expiry dates in those segments. Values closer to zero indicate higher priority i.e.\ quotes in this segment are given priority in the event of clashes. If omitted, the \lstinline!PriceSegment!s are currently read in the order that they are provided in the XML so that quotes in segments appearing earlier in the XML will be given preference in the case of clashes.

\item \lstinline!Conventions!:
The identifier for the future contract conventions associated with the quotes in the \lstinline!PriceSegment!. Details on these conventions are given in Section \ref{sec:commodity_future_conventions}.

\item \lstinline!Quotes!:
The set of future price quotes to look for in the market. Note that this can be a single wildcard string as shown in the Listing \ref{lst:commodity_crv_config_ice_pdq} or a list of explicit \lstinline!COMMODITY_FWD! \lstinline!PRICE! quote strings.

\end{itemize}


\subsubsection{Commodity Volatilities}

The following types of commodity volatility structures are supported in ORE:
\begin{itemize}
\item A constant volatility structure giving the same single volatility for all expiry times and strikes.
\item A one-dimensional expiry dependent volatility structure i.e.\ the volatility returned is dependent on the time to option expiry but does not change with option strike.
\item A two-dimensional volatility structure with a dependence on both expiry and strike. There is support for absolute strikes, delta strikes and moneyness strikes.
\item An average price option (APO) volatility surface. In particular, this structure returns the volatility of an average price that can then be used directly in the Black 76 formula to give the value of the APO.
\end{itemize}

Listing \ref{lst:comm_vol_const_config} outlines the configuration for a constant volatility structure.

\begin{longlisting}
\begin{minted}[fontsize=\footnotesize]{xml}
<CommodityVolatility>
  <CurveId>...</CurveId>
  <CurveDescription>...</CurveDescription>
  <Currency>...</Currency>
  <Constant>
    <Quote>...</Quote>
  </Constant>
  <DayCounter>...</DayCounter>
  <Calendar>...</Calendar>
</CommodityVolatility>
\end{minted}
\caption{Constant commodity volatility configuration}
\label{lst:comm_vol_const_config}
\end{longlisting}

The meaning of each of the elements is as follows:
\begin{itemize}
\item \lstinline!CurveId!: Unique identifier for the curve.
\item \lstinline!CurveDescription!: A description of the curve. This field may be left blank.
\item \lstinline!Currency!: The commodity curve currency.
\item \lstinline!Quote!: The single quote giving the constant volatility.
\item \lstinline!DayCounter! [Optional]: The day count basis used internally by the curve to calculate the time between dates. If omitted it defaults to \lstinline!A365!.
\item \lstinline!Calendar! [Optional]: The calendar used internally by the volatility structure to amend dates generated from option tenors i.e.\ if a volatility is requested from the surface using an expiry tenor. If omitted it defaults to \lstinline!NullCalendar! meaning there is no adjustment to the expiry on applying the option tenor.
\end{itemize}

Listing \ref{lst:comm_vol_curve_config} outlines the configuration for the one-dimensional expiry dependent volatility curve.

\begin{longlisting}
\begin{minted}[fontsize=\footnotesize]{xml}
<CommodityVolatility>
  <CurveId>...</CurveId>
  <CurveDescription>...</CurveDescription>
  <Currency>...</Currency>
  <Curve>
    <QuoteType>...</QuoteType>
    <VolatilityType>...</VolatilityType>
    <ExerciseType>...</ExerciseType>
    <Quotes>
      <Quote>...</Quote>
      <Quote>...</Quote>
      ...
    </Quotes>
    <Interpolation>...</Interpolation>
    <Extrapolation>...</Extrapolation>
    <EnforceMontoneVariance>...</EnforceMontoneVariance>
  </Curve>
  <DayCounter>...</DayCounter>
  <Calendar>...</Calendar>
  <FutureConventions>...</FutureConventions>
  <OptionExpiryRollDays>...</OptionExpiryRollDays>
</CommodityVolatility>
\end{minted}
\caption{Commodity volatility curve configuration}
\label{lst:comm_vol_curve_config}
\end{longlisting}

The meaning of each of the elements is given below. Elements that were defined for the previous configuration and are common to all of the configurations are not repeated.
\begin{itemize}
\item \lstinline!QuoteType! [Optional]: The allowable values in general for \lstinline!QuoteType! are \lstinline!ImpliedVolatility! and \lstinline!Premium!. Currently, only \lstinline!ImpliedVolatility! is supported for commodity volatility curves. This is the default for \lstinline!QuoteType! so this node may be omitted.
\item \lstinline!VolatilityType! [Optional]: The allowable values in general for \lstinline!VolatilityType! are \lstinline!Lognormal!, \lstinline!ShiftedLognormal! and \lstinline!Normal!. Currently, only \lstinline!Lognormal! is supported for commodity volatility curves. This is the default for \lstinline!VolatilityType! so this node may be omitted.
\item \lstinline!ExerciseType! [Optional]: This node is described below in the context of surfaces. For commodity volatility curves, it is ignored and should be omitted.
\item \lstinline!Quotes!: A list of commodity option volatility quotes with different expiries to use in the commodity curve building. The commodity option volatility quotes are explained in Section \ref{md:commodity_option_iv}. As indicated above, any quote string used here much start with \lstinline!COMMODITY_OPTION/RATE_LNVOL!. A single regular expression \lstinline!Quote! is also supported here in place of a list of explicit \lstinline!Quote! strings. Note that if a list of explicit \lstinline!Quote! strings are provided, it is an error to have a duplicated option expiry date. If a regular expression is used, the first quote found is used and subsequent qutoes with the same expiry are discarded with a warning issued.
\item \lstinline!Interpolation!: The interpolation to use to give volatilities between option expiry times. The allowable values are \lstinline!Linear!, \lstinline!Cubic! and \lstinline!LogLinear!. Note that the interpolation here is on the variance.
\item \lstinline!Extrapolation!: The extrapolation to use to give volatilities after the last expiry date in the variance curve. The allowable values are \lstinline!None!, \lstinline!UseInterpolator!, \lstinline!Linear! and \lstinline!Flat!. However, all options except \lstinline!None! yield the same extrapolation i.e.\ flat extrapolation in the volatility. \lstinline!None! disables extrapolation so that an exception is raised if the curve is queried after the last expiry for a volatility. Note that as the curve is parameterised in variance, interpolation is used to interpolate between time zero where the variance is zero and the first expiry time.
\item \lstinline!EnforceMontoneVariance! [Optional]: Boolean parameter that should be set to \lstinline!true! to raise an exception if the implied variance curve is not montone increasing with time. It should be set to \lstinline!false! to suppress such an exception. The default value if omitted is \lstinline!true!.
\item \lstinline!FutureConventions! [Optional]: Depending on the quotes that are provided in the \lstinline!Quotes! section, a \lstinline!CommodityFuture! convention may be needed in order to derive an option expiry date from the \textit{Expiry} portion of the commodity option quote. In particular, as outlined in Section \ref{md:commodity_option_iv}, the \textit{Expiry} portion of a commodity option quote allows for continuation expiries of the form \lstinline!cN!. The \lstinline!N! is a positive integer meaning the \lstinline!N!-th next expiry after the valuation date on which we are building the commodity volatility curve. When a continuation expiry is used in a quote, the \lstinline!FutureConventions! is needed and gives the ID of the conventions associated with the commodity for which we are trying to build the volatility curve. These conventions are used to determine the explicit expiry date for the given option quote from the continuation expiry.
\item \lstinline!OptionExpiryRollDays! [Optional]: The \lstinline!OptionExpiryRollDays! can be any non-negative integer and may be needed when deriving an option expiry date from the \textit{Expiry} portion of the commodity option quote. If the \textit{Expiry} portion of the commodity option quote is a continuation expiry, an explicit expiry date is deduced as explained in the previous bullet point. Additionally, in some cases, the option quotes for the next option expiry may stop a number of business days before that option expiry and the \lstinline!cN! expiry in this period begins referring to the \lstinline!N+1!-th next option expiry. As an example, assume $d_v$ is the valuation date and $e_1 = d_v$ is the next option expiry date. If \lstinline!OptionExpiryRollDays! is \lstinline!0! then a commodity option quote with an \textit{Expiry} portion equal to \lstinline!c1! on $d_v$ indicates that the option quote is for an option with expiry date equal to $e_1$. However, if \lstinline!OptionExpiryRollDays! is \lstinline!1!, a commodity option quote with an \textit{Expiry} portion equal to \lstinline!c1! on $d_v$ indicates that the option quote is for an option with expiry date equal to $e_2$ where $e_2$ is the next option expiry date after $e_1$. In other words, with \lstinline!OptionExpiryRollDays! set to \lstinline!1! the option quotes for expiry date $e_1$ stopped on the business day before $e_1$. If omitted, \lstinline!OptionExpiryRollDays! defaults to \lstinline!0!.
\end{itemize}

Listing \ref{lst:comm_vol_strike_surface_config} outlines the configuration for the two-dimensional expiry and absolute strike commodity option surface.

\begin{longlisting}
\begin{minted}[fontsize=\footnotesize]{xml}
<CommodityVolatility>
  <CurveId>...</CurveId>
  <CurveDescription>...</CurveDescription>
  <Currency>...</Currency>
  <StrikeSurface>
    <QuoteType>...</QuoteType>
    <VolatilityType>...</VolatilityType>
    <ExerciseType>...</ExerciseType>
    <Strikes>...</Strikes>
    <Expiries>...</Expiries>
    <TimeInterpolation>...</TimeInterpolation>
    <StrikeInterpolation>...</StrikeInterpolation>
    <Extrapolation>...</Extrapolation>
    <TimeExtrapolation>...</TimeExtrapolation>
    <StrikeExtrapolation>...</StrikeExtrapolation>
  </StrikeSurface>
  <DayCounter>...</DayCounter>
  <Calendar>...</Calendar>
  <FutureConventions>..</FutureConventions>
  <OptionExpiryRollDays>...</OptionExpiryRollDays>
  <PriceCurveId>...</PriceCurveId>
  <YieldCurveId>...</YieldCurveId>
  <OneDimSolverConfig>
    <MaxEvaluations>100</MaxEvaluations>
    <InitialGuess>0.35</InitialGuess>
    <Accuracy>0.0001</Accuracy>
    <MinMax>
      <Min>0.0001</Min>
      <Max>2.0</Max>
    </MinMax>
  </OneDimSolverConfig>
  <PreferOutOfTheMoney>...</PreferOutOfTheMoney>
  <QuoteSuffix>...</QuoteSuffix>
</CommodityVolatility>
\end{minted}
\caption{Expiry and absolute strike commodity option surface configuration}
\label{lst:comm_vol_strike_surface_config}
\end{longlisting}

The meaning of each of the elements is given below. Again, nodes explained in the previous configuration are not repeated here.
\begin{itemize}
\item \lstinline!QuoteType! [Optional]:
As above, the allowable values for \lstinline!QuoteType! are \lstinline!ImpliedVolatility! and \lstinline!Premium!. If omitted, the default is \lstinline!ImpliedVolatility!. If the \lstinline!QuoteType! is \lstinline!Premium!, a volatility surface will be stripped from option premium quotes. Note that \lstinline!Premium! is only allowed if one or both of \lstinline!Strikes! or \lstinline!Expiries! below is set to the single wildcard value \lstinline!*!. In other words, if we explicitly specify all of the strikes and expiries, we can only build a volatility surface directly and the \lstinline!QuoteType! must be \lstinline!ImpliedVolatility!.

\item \lstinline!VolatilityType! [Optional]:
As above, the allowable values for \lstinline!VolatilityType! are \lstinline!Lognormal!, \lstinline!ShiftedLognormal! and \lstinline!Normal!. This is only needed if \lstinline!QuoteType! is \lstinline!ImpliedVolatility!. Currently, only \lstinline!Lognormal! is supported for commodity volatility surfaces. This is the default for \lstinline!VolatilityType! so this node may be omitted.

\item \lstinline!ExerciseType! [Optional]:
The allowable values for \lstinline!ExerciseType! are \lstinline!European! and \lstinline!American!. This is only needed if \lstinline!QuoteType! is \lstinline!Premium! and indicates if the option premium quotes are American or European exercise. If omitted the default is \lstinline!European!.

\item \lstinline!Strikes!:
This can be a single wildcard value \lstinline!*! or a comma separated list of explicit strike prices. We explain below how these strikes are combined with the other parameters in the configuration to give a list of commodity option quotes to search for in the market data.

\item \lstinline!Expiries!:
This can be a single wildcard value \lstinline!*! or a comma separated list of expiry strings. We explain below how these expiries are combined with the other parameters in the configuration to give a list of commodity option quotes to search for in the market data. Note that as outlined in Section \ref{md:commodity_option_iv}, the \textit{Expiry} portion of the commodity option quote may be an explicit expiry date, an expiry tenor or a continuation expiry of the form \lstinline!cN! explained in the volatility curve section above.

\item \lstinline!TimeInterpolation!:
Indicates the interpolation in the time direction. There are quite a number of restrictions here. If either \lstinline!Strikes! or \lstinline!Expiries! use the single wildcard value \lstinline!*!, the interpolation in both the time and strike direction is linear regardless of the value passed here in the \lstinline!TimeInterpolation! node. If neither \lstinline!Strikes! nor \lstinline!Expiries! use the single wildcard value \lstinline!*!, \lstinline!TimeInterpolation! may be set to \lstinline!Linear! or \lstinline!Cubic! but \lstinline!StrikeInterpolation! must have the same value. If it does not, then \lstinline!Linear! is used for both. In other words, if neither \lstinline!Strikes! nor \lstinline!Expiries! use the single wildcard value \lstinline!*!, we can configure bilinear or bicubic interpolation. Again, in all cases, the interpolation is carried out on the variance.

\item \lstinline!StrikeInterpolation!:
Indicates the interpolation in the strike direction. The requirements are exactly as outlined for the \lstinline!TimeInterpolation! node.

\item \lstinline!Extrapolation!:
A boolean value indicating if extrapolation is allowed.

\item \lstinline!TimeExtrapolation!:
Indicates the extrapolation in the time direction. The allowable values are \lstinline!None!, \lstinline!UseInterpolator!, \lstinline!Linear! and \lstinline!Flat!. If neither \lstinline!Strikes! nor \lstinline!Expiries! use the single wildcard value \lstinline!*!, the extrapolation in the time direction is flat regardless of the value passed here. If either \lstinline!Strikes! or \lstinline!Expiries! use the single wildcard value \lstinline!*!, both \lstinline!Flat! and \lstinline!None! give flat extrapolation in the time direction whereas \lstinline!UseInterpolator! and \lstinline!Linear! indicate that the configured interpolation (linear or cubic) should be continued in the time direction in order to extrapolate. \lstinline!Linear! is only allowable here for backward compatibility and \lstinline!UseInterpolator! should be preferred for clarity.

\item \lstinline!StrikeExtrapolation!:
Indicates the extrapolation in the strike direction. The allowable values are \lstinline!None!, \lstinline!UseInterpolator!, \lstinline!Linear! and \lstinline!Flat!. Both \lstinline!Flat! and \lstinline!None! give flat extrapolation in the strike direction. \lstinline!UseInterpolator! and \lstinline!Linear! indicate that the configured interpolation (linear or cubic) should be continued in the strike direction in order to extrapolate. \lstinline!Linear! is only allowable here for backward compatibility and \lstinline!UseInterpolator! should be preferred for clarity.

\item \lstinline!PriceCurveId! [Optional]:
The ID of a price curve for the commodity of the form \lstinline!Commodity/{CCY}/{NAME}!. This is needed if the \lstinline!QuoteType! is \lstinline!Premium!. It is also needed when the \lstinline!QuoteType! is \lstinline!ImpliedVolatility! if either \lstinline!Strikes! or \lstinline!Expiries! use the single wildcard value \lstinline!*! and both call and put quotes are found in the market for the same expiry and strike pair. In this case, it is needed to determine which quotes to use based on the value of the \lstinline!PreferOutOfTheMoney! node.

\item \lstinline!YieldCurveId! [Optional]:
The ID of a yield curve in the currency of the commodity of the form \lstinline!Yield/{CCY}/{NAME}!. This is needed if the \lstinline!QuoteType! is \lstinline!Premium! in the stripping of the volatilities from premia.

\item \lstinline!OneDimSolverConfig! [Optional]:
This is used if the \lstinline!QuoteType! is \lstinline!Premium!. It provides the options for the root search in the stripping of the volatilities from premia. If omitted, the default set of options shown in Listing \ref{lst:comm_vol_strike_surface_config} are used. The \lstinline!MinMax! node can be replaced with a single \lstinline!Step! node that accepts a double giving the step size to use in the root search.

\item \lstinline!PreferOutOfTheMoney! [Optional]:
A node accepting a boolean value. If set to \lstinline!true!, quotes for out of the money options are preferred where a call and a put quote are found for the same expiry strike pair. If set to \lstinline!false!, quotes for in the money options are preferred where a call and a put quote are found for the same expiry strike pair. If omitted, \lstinline!true! is assumed.

\item \lstinline!QuoteSuffix! [Optional]:
The allowable values are \lstinline!C! and \lstinline!P! indicating \lstinline!Call! and \lstinline!Put! respectively. If given, they are used in the construction of the commodity option quote strings as explained below. They are useful in cases where the market data contains both call and put volatility quotes for the same expiry strike pair and you want to use only the calls (set \lstinline!QuoteSuffix! to \lstinline!C!) or the puts (set \lstinline!QuoteSuffix! to \lstinline!P!).

\end{itemize}

As mentioned above, a number of parameters from the two-dimensional expiry and absolute strike configuration are used in constructing the commodity option quote strings that are looked up in the market data. There are two cases:
\begin{enumerate}
\item
Both the \lstinline!Strikes! and \lstinline!Expiries! node provide a comma separated list of values. As mentioned above, we can only use a \lstinline!QuoteType! of \lstinline!ImpliedVolatility! in this case where we have explicit expiries and strikes and the \lstinline!VolatilityType! must be \lstinline!Lognormal!. For example, assume the \lstinline!Expiries! node has the set of values \lstinline!e_1,e_2,...,e_N! and that the \lstinline!Strikes! node has the set of values \lstinline!s_1,s_2,...,s_M!. For each of the $N \times M$ expiry strike pairs $(e_n,s_m)$, a quote of the form \lstinline!COMMODITY_OPTION/RATE_LNVOL/{N}/{C}/e_n/s_m[/{S}]! is created to be looked up in the market data. \lstinline!{N}! is the value in the \lstinline!CurveId! node, \lstinline!{C}! is the value in the \lstinline!Currency! node and \lstinline!{S}! is the value in the \lstinline!QuoteSuffix! node if given. This explicit grid of volatility quotes must be present in the market for the commodity volatility surface to be constructed.

\item
One or both of the \lstinline!Strikes! and \lstinline!Expiries! node use a single wildcard value \lstinline!*!. As mentioned above, the \lstinline!QuoteType! can be \lstinline!ImpliedVolatility! or \lstinline!Premium! in this case. As above, assume the \lstinline!Expiries! node has the set of values \lstinline!e_1,e_2,...,e_N! and that the \lstinline!Strikes! node has the set of values \lstinline!s_1,s_2,...,s_M!. The additional constraint here is that $N=1$ and \lstinline!e_1! is \lstinline!*! or that $M=1$ and \lstinline!s_1! is \lstinline!*!, or both. For each of the $N \times M$ expiry strike pairs $(e_n,s_m)$, a quote of the form \lstinline!COMMODITY_OPTION/{T}/{N}/{C}/e_n/s_m[/{S}]! is created to be looked up in the market data. \lstinline!{T}! is \lstinline!PRICE! when \lstinline!QuoteType! is \lstinline!Premium! and is \lstinline!RATE_LNVOL! when \lstinline!QuoteType! is \lstinline!ImpliedVolatility!, \lstinline!{N}! is the value in the \lstinline!CurveId! node, \lstinline!{C}! is the value in the \lstinline!Currency! node and \lstinline!{S}! is the value in the \lstinline!QuoteSuffix! node if given. Any quote in the market with a name matching any of the quote strings formed in this way are then included in the commodity volatility curve building. Note that the \lstinline!QuoteSuffix! has no effect in this case and should be omitted i.e.\ it is only used in the case of an explicit grid of quotes above.
\end{enumerate}

Listing \ref{lst:comm_vol_mny_surface_config} outlines the configuration for the two-dimensional expiry and moneyness strike commodity option surface. This is similar to the absolute strike surface configuration above but currently only supports a \lstinline!QuoteType! of \lstinline!ImpliedVolatility! i.e.\ \lstinline!QuoteType! of \lstinline!Premium! is not supported. Also, the \lstinline!VolatilityType! must be \lstinline!Lognormal!. Both forward and spot moneyness is supported.

\begin{longlisting}
\begin{minted}[fontsize=\footnotesize]{xml}
<CommodityVolatility>
  <CurveId>...</CurveId>
  <CurveDescription>...</CurveDescription>
  <Currency>...</Currency>
  <MoneynessSurface>
    <QuoteType>...</QuoteType>
    <VolatilityType>...</VolatilityType>
    <ExerciseType>...</ExerciseType>
    <MoneynessType>...</MoneynessType>
    <MoneynessLevels>...</MoneynessLevels>
    <Expiries>...</Expiries>
    <TimeInterpolation>...</TimeInterpolation>
    <StrikeInterpolation>...</StrikeInterpolation>
    <Extrapolation>...</Extrapolation>
    <TimeExtrapolation>...</TimeExtrapolation>
    <StrikeExtrapolation>...</StrikeExtrapolation>
    <FuturePriceCorrection>...</FuturePriceCorrection>
  </MoneynessSurface>
  <DayCounter>...</DayCounter>
  <Calendar>...</Calendar>
  <FutureConventions>..</FutureConventions>
  <OptionExpiryRollDays>...</OptionExpiryRollDays>
  <PriceCurveId>...</PriceCurveId>
  <YieldCurveId>...</YieldCurveId>
</CommodityVolatility>
\end{minted}
\caption{Expiry and moneyness strike commodity option surface configuration}
\label{lst:comm_vol_mny_surface_config}
\end{longlisting}

The meaning of each of the elements is given below. Again, nodes explained in the previous configuration are not repeated here.
\begin{itemize}
\item \lstinline!MoneynessType!:
The allowable values are \lstinline!Spot! for spot moneyness and \lstinline!Fwd! for forward moneyness.

\item \lstinline!MoneynessLevels!:
This must be a comma separated list of moneyness values. A moneyness value of $1$ indicates a strike equal to spot or forward depending on the value given in the \lstinline!MoneynessType! node.

\item \lstinline!TimeInterpolation!:
Only \lstinline!Linear! is currently supported here.

\item \lstinline!StrikeInterpolation!:
Only \lstinline!Linear! is currently supported here.

\item \lstinline!Extrapolation!:
A boolean value indicating if extrapolation is allowed.

\item \lstinline!TimeExtrapolation!:
Only \lstinline!Flat! is currently supported here giving flat extrapolation of the volatility.

\item \lstinline!StrikeExtrapolation!:
Indicates the extrapolation in the strike direction. The allowable values are \lstinline!None!, \lstinline!UseInterpolator!, \lstinline!Linear! and \lstinline!Flat!. Both \lstinline!Flat! and \lstinline!None! give flat extrapolation in the strike direction. \lstinline!UseInterpolator! and \lstinline!Linear! indicate that the configured interpolation (linear) should be continued in the strike direction in order to extrapolate.

\item \lstinline!FuturePriceCorrection! [Optional]:
This is a boolean flag that defaults to \lstinline!true!. In most cases, for options on futures, the option expiry date is a short period before the future expiry. If there is an arbitrary interpolation applied to the future price curve, the future price on the option expiry date may not equal the associated future price. If \lstinline!FuturePriceCorrection! is \lstinline!true!, this is corrected i.e.\ the future price on option expiry is the associated future price for the future expiry date. Note that a valid \lstinline!FutureConventions! is needed for the correction to be applied.

\item \lstinline!PriceCurveId!:
This is required for both a spot and forward moneyness surface.

\item \lstinline!YieldCurveId!:
This is required for a forward moneyness surface.

\end{itemize}

Note that, similar to the procedure outlined above for the absolute strike surface, quote strings of the form \lstinline!COMMODITY_OPTION/RATE_LNVOL/{N}/{C}/e_n/MNY/{T}/l_m! are created from the moneyness configuration to be looked up in the market. Here, \lstinline!l_m! are the moneyness levels for $m=1,\ldots,M$ and \lstinline!{T}! is the moneyness type i.e.\ either \lstinline!Spot! or \lstinline!Fwd!.

Listing \ref{lst:comm_vol_delta_surface_config} outlines the configuration for the two-dimensional expiry and delta strike commodity option surface. Similar to the moneyness strike surface configuration above, this currently only supports a \lstinline!QuoteType! of \lstinline!ImpliedVolatility! i.e.\ \lstinline!QuoteType! of \lstinline!Premium! is not supported. Also, the \lstinline!VolatilityType! must be \lstinline!Lognormal!. Various delta and ATM types are supported.

\begin{longlisting}
\begin{minted}[fontsize=\footnotesize]{xml}
<CommodityVolatility>
  <CurveId>...</CurveId>
  <CurveDescription>...</CurveDescription>
  <Currency>...</Currency>
  <DeltaSurface>
    <QuoteType>...</QuoteType>
    <VolatilityType>...</VolatilityType>
    <ExerciseType>...</ExerciseType>
    <DeltaType>...</DeltaType>
    <AtmType>...</AtmType>
    <AtmDeltaType>...</AtmDeltaType>
    <PutDeltas>...</PutDeltas>
    <CallDeltas>...</CallDeltas>
    <Expiries>...</Expiries>
    <TimeInterpolation>...</TimeInterpolation>
    <StrikeInterpolation>...</StrikeInterpolation>
    <Extrapolation>...</Extrapolation>
    <TimeExtrapolation>...</TimeExtrapolation>
    <StrikeExtrapolation>...</StrikeExtrapolation>
    <FuturePriceCorrection>...</FuturePriceCorrection>
  </DeltaSurface>
  <DayCounter>...</DayCounter>
  <Calendar>...</Calendar>
  <FutureConventions>..</FutureConventions>
  <OptionExpiryRollDays>...</OptionExpiryRollDays>
  <PriceCurveId>...</PriceCurveId>
  <YieldCurveId>...</YieldCurveId>
</CommodityVolatility>
\end{minted}
\caption{Expiry and delta strike commodity option surface configuration}
\label{lst:comm_vol_delta_surface_config}
\end{longlisting}

The meaning of each of the elements is given below. Again, nodes explained in the previous configuration are not repeated here.
\begin{itemize}
\item \lstinline!DeltaType!:
The allowable supported values are \lstinline!Spot! for spot delta \lstinline!Fwd! for forward delta.

\item \lstinline!AtmType!:
The allowable supported values are \lstinline!AtmSpot!, \lstinline!AtmFwd! and \lstinline!AtmDeltaNeutral!.

\item \lstinline!AtmDeltaType! [Optional]:
This is only needed if the \lstinline!AtmType! is \lstinline!AtmDeltaNeutral!.

\item \lstinline!PutDeltas!:
A comma separated list of one or more put deltas to use in the volatility surface. Note that the put deltas should be given without a sign e.g.\ \lstinline!<PutDeltas>0.10,0.20,0.30,0.40</PutDeltas>! would be an example.

\item \lstinline!PutDeltas!:
A comma separated list of one or more call deltas to use in the volatility surface.

\item \lstinline!TimeInterpolation!:
Only \lstinline!Linear! is currently supported here.

\item \lstinline!StrikeInterpolation!:
Allowable values are \lstinline!Linear!, \lstinline!NaturalCubic! and \lstinline!FinancialCubic!.

\item \lstinline!Extrapolation!:
A boolean value indicating if extrapolation is allowed.

\item \lstinline!TimeExtrapolation!:
Only \lstinline!Flat! is currently supported here giving flat extrapolation of the volatility.

\item \lstinline!StrikeExtrapolation!:
Indicates the extrapolation in the strike direction. The allowable values are \lstinline!None!, \lstinline!UseInterpolator!, \lstinline!Linear! and \lstinline!Flat!. Both \lstinline!Flat! and \lstinline!None! give flat extrapolation in the strike direction. \lstinline!UseInterpolator! and \lstinline!Linear! indicate that the configured interpolation should be continued in the strike direction in order to extrapolate.

\item \lstinline!PriceCurveId!:
This is required for a delta surface.

\item \lstinline!YieldCurveId!:
This is required for a delta surface.

\end{itemize}

Note that, similar to the procedure outlined above for the absolute strike surface, quote strings are created from the configuration to be looked up in the market. For the put deltas, quote strings of the form \lstinline!COMMODITY_OPTION/RATE_LNVOL/{N}/{C}/e_n/DEL/{T}/Put/d_m! are created. Here, \lstinline!d_m! are the \lstinline!PutDeltas! and \lstinline!{T}! is the delta type i.e.\ either \lstinline!Spot! or \lstinline!Fwd!. Similarly for the call deltas, quote strings of the form \lstinline!COMMODITY_OPTION/RATE_LNVOL/{N}/{C}/e_n/DEL/{T}/Call/d_j! are created where \lstinline!d_j! are the \lstinline!CallDeltas!. For ATM, quote strings of the form \lstinline!COMMODITY_OPTION/RATE_LNVOL/{N}/{C}/e_n/DEL/ATM/{A}[/DEL/{T}]! are created where \lstinline!{A}! is the \lstinline!AtmType! i.e.\ \lstinline!AtmSpot!, \lstinline!AtmFwd! or \lstinline!AtmDeltaNeutral! and \lstinline!{T}! is the optional delta type.

Also, it is worth adding a note here on the interpolation for a delta based surface. Assume we want the volatility at time $t$ and absolute strike $s$ i.e. at the $(t, s)$ node. For the maturity time $t$, a delta "slice" i.e. a set of (delta, vol) pairs for that time $t$, is obtained by interpolating, or extrapolating, the variance in the time direction on each delta column. Then for each (delta, vol) pair at time $t$, an absolute strike value is deduced to give a slice at time $t$ in terms of absolute strike i.e. a set of (strike, vol) pairs at time $t$. This strike versus volatility curve is then interpolated, or extrapolated, to give the vol at the $(t, s)$.

Listing \ref{lst:comm_vol_apo_surface_config} outlines the configuration for the APO volatility surface. This currently only supports a \lstinline!QuoteType! of \lstinline!ImpliedVolatility! and \lstinline!VolatilityType! must be \lstinline!Lognormal!. This configuration takes a base commodity volatility surface and builds a surface that can be queried for volatilities to price APOs directly i.e.\ using the volatility directly in a Black 76 formula along with the average future price. It uses the approach described in the Section entitled \textit{Commodity Average Price Option - Future Settlement Prices} in the Product Catalogue to go from future option volatilities to APO volatilities.

We describe here briefly a motivating example encountered in practice. We have commodity APOs where the underlying is WTI Midland Argus averaged over the calendar month. We do not have direct volatilities for these APO contracts. We have a price curve for the average of WTI Midland Argus over the calendar month from the futures market. We can use the volatility surface that we have for CME WTI to build an APO surface for WTI Midland Argus. Listing \ref{lst:comm_vol_apo_surface_config} shows the configuration used in this context.

\begin{longlisting}
\begin{minted}[fontsize=\footnotesize]{xml}
<CommodityVolatility>
  <CurveId>WTI_MIDLAND</CurveId>
  <CurveDescription>WTI Midland (CAL) APO surface</CurveDescription>
  <Currency>USD</Currency>
  <ApoFutureSurface>
    <QuoteType>ImpliedVolatility</QuoteType>
    <VolatilityType>Lognormal</VolatilityType>
    <MoneynessLevels>0.50,0.75,1.00,1.25,1.50</MoneynessLevels>
    <VolatilityId>CommodityVolatility/USD/WTI</VolatilityId>
    <PriceCurveId>Commodity/USD/WTI</PriceCurveId>
    <FutureConventions>WTI</FutureConventions>
    <TimeInterpolation>Linear</TimeInterpolation>
    <StrikeInterpolation>Linear</StrikeInterpolation>
    <Extrapolation>true</Extrapolation>
    <TimeExtrapolation>Flat</TimeExtrapolation>
    <StrikeExtrapolation>Flat</StrikeExtrapolation>
    <MaxTenor>2Y</MaxTenor>
    <Beta>0</Beta>
  </ApoFutureSurface>
  <DayCounter>A365</DayCounter>
  <Calendar>CME</Calendar>
  <FutureConventions>WTI_MIDLAND</FutureConventions>
  <PriceCurveId>Commodity/USD/WTI_MIDLAND</PriceCurveId>
  <YieldCurveId>Yield/USD/USD-FedFunds</YieldCurveId>
</CommodityVolatility>
\end{minted}
\caption{APO surface configuration}
\label{lst:comm_vol_apo_surface_config}
\end{longlisting}

The meaning of each of the elements is given below.
\begin{itemize}
\item \lstinline!MoneynessLevels!:
A comma separated list of the moneyness levels used in the APO surface construction. Forward moneyness is assumed with a value of $1$ indicating a strike equal to the future price.

\item \lstinline!VolatilityId!:
The ID of an existing commodity option surface for options on the future settlement price referenced in the APOs used in the generation of the volatilities for this surface.

\item \lstinline!PriceCurveId!:
The ID of an existing commodity price curve for the future settlement price referenced in the APOs used in the generation of the volatilities for this surface.

\item \lstinline!FutureConventions!:
This ID of the commodity future conventions for the future settlement price referenced in the APOs used in the generation of the volatilities for this surface.

\item \lstinline!TimeInterpolation!:
Only \lstinline!Linear! is currently supported here. Note that the interpolation is in terms of variance.

\item \lstinline!StrikeInterpolation!:
Only \lstinline!Linear! is supported here. Note that the interpolation is in terms of variance.

\item \lstinline!Extrapolation!:
A boolean value indicating if extrapolation is allowed.

\item \lstinline!TimeExtrapolation!:
Only \lstinline!Flat! is currently supported here. The flat extrapolation is in terms of the volatility.

\item \lstinline!StrikeExtrapolation!:
Indicates the extrapolation in the strike direction. The allowable values are \lstinline!None!, \lstinline!UseInterpolator!, \lstinline!Linear! and \lstinline!Flat!. Both \lstinline!Flat! and \lstinline!None! give flat extrapolation in the strike direction. \lstinline!UseInterpolator! and \lstinline!Linear! indicate that the configured interpolation should be continued in the strike direction in order to extrapolate.

\item \lstinline!PriceCurveId!:
The ID of an existing commodity price curve giving the average price for the APO period.

\item \lstinline!YieldCurveId!:
This ID of a yield curve in the currency of the commodity used for discounting.

\end{itemize}

\subsubsection{Bootstrap Configuration}
\label{sec:bootstrap_config}
The \lstinline!BootstrapConfig! node, outlined in listing \ref{lst:bootstrap_config_outline}, can be added to curve configurations that use a bootstrap algorithm to alter the default behaviour of the bootstrap algorithm.

\begin{longlisting}
\begin{minted}[fontsize=\footnotesize]{xml}
<BootstrapConfig>
  <Accuracy>...</Accuracy>
  <GlobalAccuracy>...</GlobalAccuracy>
  <DontThrow>...</DontThrow>
  <MaxAttempts>...</MaxAttempts>
  <MaxFactor>...</MaxFactor>
  <MinFactor>...</MinFactor>
  <DontThrowSteps>...</DontThrowSteps>
</BootstrapConfig>
\end{minted}
\caption{\lstinline!BootstrapConfig! node outline}
\label{lst:bootstrap_config_outline}
\end{longlisting}

The meaning of each of the elements is:
\begin{itemize}
\item \lstinline!Accuracy! [Optional]:
The accuracy with which the implied quote must match the market quote for each instrument in the curve bootstrap. This node should hold a positive real number. If omitted, the default value is \num[scientific-notation=true]{1.0e-12}.

\item \lstinline!GlobalAccuracy! [Optional]:
If the interpolation method in the bootstrap is global, e.g. cubic spline, the bootstrap routine needs to perform multiple iterative bootstraps of the curve to converge. After each such bootstrap of the full curve, the absolute value of the change between the current bootstrap and previous bootstrap for the curve value at each pillar is calculated. The global bootstrap is deemed to have converged if the maximum of these changes is less than the global accuracy or the accuracy from the \lstinline!Accuracy! node. This node should hold a positive real number. If omitted, the global accuracy is set equal to the accuracy from the \lstinline!Accuracy! node. This node is useful in some cases where a slightly less restrictive accuracy, than that given by the \lstinline!Accuracy! node, is needed for the global bootstrap.

\item \lstinline!DontThrow! [Optional]:
If this node is set to \lstinline!true!, the curve bootstrap does not throw an error when the bootstrap fails at a pillar. Instead, a curve value is sought at the failing pillar that minimises the absolute value of the difference between the implied quote and the market quote at that pillar. The minimum is sought between the minimum and maximum curve value that was used in the root finding routine that failed at the pillar. The number of steps used in this search is given by the \lstinline!DontThrowSteps! node below. This node should hold a boolean value. If omitted, the default value is \lstinline!false! i.e. the bootstrap throws an exception at the first pillar where the bootstrap fails.

\item \lstinline!MaxAttempts! [Optional]:
At each pillar, the bootstrap routine searches between a minimum curve value and a maximum curve value for a curve value that gives an implied quote that matches the market quote at that pillar. In some cases, the minimum curve value and maximum curve value are too restrictive and the bootstrap fails at a pillar. This node determines how many times the bootstrap should be attempted at each pillar. For example, if the node is set to 1, the bootstrap uses the minimum curve value and maximum curve value implied in the code and fails if a root is not found. If this node is set to 2 and the first attempt fails, the minimum curve value is reduced by a factor specified in the node \lstinline!MinFactor!, the maximum curve value is increased by a factor specified in the node \lstinline!MaxFactor! and a second attempt is made to find a root between the enlarged bounds. If no root is found, the bootstrap then fails at this pillar. This node should hold a positive integer. If omitted, the default value is 5.

\item \lstinline!MaxFactor! [Optional]:
This node is used only if \lstinline!MaxAttempts! is greater than 1. The meaning of this node is given in the description of the \lstinline!MaxAttempts! node. This node should hold a positive real number. If omitted, the default value is 2.0.

\item \lstinline!MinFactor! [Optional]:
This node is used only if \lstinline!MaxAttempts! is greater than 1. The meaning of this node is given in the description of the \lstinline!MaxAttempts! node. This node should hold a positive real number. If omitted, the default value is 2.0.

\item \lstinline!DontThrowSteps! [Optional]:
This node is used only if \lstinline!DontThrow! is \lstinline!true!. The meaning of this node is given in the description of the \lstinline!DontThrow! node. This node should hold a positive integer. If omitted, the default value is 10.

\end{itemize}

\subsubsection{One Dimensional Solver Configuration}
\label{sec:one_dim_solver_config}
The \lstinline!OneDimSolverConfig! node, outlined in Listing \ref{lst:one_dim_solver_config_outline}, can be added to certain curve configurations that lead to a one dimensional solver being used in the curve construction. For example, the \lstinline!EquityVolatility! curve configuration can lead to equity volatilities being stripped from equity option premiums. In this case, the \lstinline!OneDimSolverConfig! node can be added to the \lstinline!EquityVolatility! curve configuration to indicate how the solver should behave i.e.\ maximum number of evaluations, initial guess, accuracy etc. The various options are outlined below.

\begin{longlisting}
\begin{minted}[fontsize=\footnotesize]{xml}
<OneDimSolverConfig>
  <MaxEvaluations>...</MaxEvaluations>
  <InitialGuess>...</InitialGuess>
  <Accuracy>...</Accuracy>
  <MinMax>
    <Min>...</Min>
    <Max>...</Max>
  </MinMax>
  <!-- Step only needed if MinMax not provided. -->
  <Step>...</Step>
  <LowerBound>...</LowerBound>
  <UpperBound>...</UpperBound>
</OneDimSolverConfig>
\end{minted}
\caption{\lstinline!OneDimSolverConfig! node outline}
\label{lst:one_dim_solver_config_outline}
\end{longlisting}

The meaning of each of the elements is:
\begin{itemize}
\item \lstinline!MaxEvaluations!:
This node should hold a positive integer. The maximum number of function evaluations that can be made by the solver.

\item \lstinline!InitialGuess!:
This node should hold a real number. The initial guess used by the solver.

\item \lstinline!Accuracy!:
This node should hold a positive real number. The accuracy used by the solver in the root find.

\item \lstinline!MinMax! [Optional]:
A node that holds a \lstinline!Min! and a \lstinline!Max! node each of which should hold a real number. This indicates that the solver should search for a root between the value in \lstinline!Min! and the value in \lstinline!Max!. The value in \lstinline!Min! should obviously be less than the value in \lstinline!Max!. This node is optional. If not provided, the \lstinline!Step! node below should be provided to set up a step based solver.

\item \lstinline!Step! [Optional]:
This node should hold a real number. The validation is a choice between \lstinline!MinMax! and \lstinline!Step! so that \lstinline!Step! can only be provided if \lstinline!MinMax! is not and vice versa. The value in \lstinline!Step! provides the solver with a step size to use in its search for a root.

\item \lstinline!LowerBound! [Optional]:
This node should hold a real number. It provides a lower bound for the search domain. If omitted, no lower bound is applied to the search domain.

\item \lstinline!UpperBound! [Optional]:
This node should hold a real number. It provides an upper bound for the search domain. If omitted, no upper bound is applied to the search domain. Obviously, if both \lstinline!LowerBound! and \lstinline!UpperBound! are provided, the value in \lstinline!LowerBound! should be less than the value in \lstinline!UpperBound!.

\end{itemize}
