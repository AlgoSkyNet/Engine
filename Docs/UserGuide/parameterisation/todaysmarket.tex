%--------------------------------------------------------
\subsection{Market: {\tt todaysmarket.xml}}\label{sec:market}
%--------------------------------------------------------

This configuration file determines the subset of the 'market' universe which is going to be built by ORE. It is the
user's responsibility to make sure that this subset is sufficient to cover the portfolio to be analysed. If it is not,
the application will complain at run time and exit.

\medskip We assume that the market configuration is provided in file {\tt todaysmarket.xml}, however, the file name can
be chosen by the user. The file name needs to be entered into the master configuration file {\tt ore.xml}, see section
\ref{sec:master_input}.

\medskip 
The file starts and ends with the opening and closing tags {\tt <TodaysMarket>} 
and {\tt </TodaysMarket>}. The file then contains configuration blocks for
\begin{itemize}
\item Discounting curves
\item Index curves (to project index fixings)
\item Yield curves (for other purposes, e.g. as benchmark curve for bond pricing)
\item Swap index curves (to project Swap rates)
\item FX spot rates
\item Inflation index curves (to project zero or yoy inflation fixings)
\item Equity curves (to project forward prices)
\item Default curves
\item Swaption volatility structures
\item Cap/Floor volatility structures
\item FX volatility structures
\item Inflation Cap/Floor volatility surfaces
\item Equity volatility structures
\item CDS volatility structures
\item Base correlation structures
\item Correlation structures
\item Securities
\end{itemize}

There can be alternative versions of each block each labeled with a unique identifier (e.g. Discount curve block with ID
'default', discount curve block with ID 'ois', another one with ID 'xois', etc). The purpose of these IDs will be
explained at the end of this section. We now discuss each block's layout.

\subsubsection{Discounting Curves} 

We pick one discounting curve block as an example here (see {\tt Examples/Input/todaysmarket.xml}), the one with ID 'ois' 

\begin{listing}[H]
%\hrule\medskip
\begin{minted}[fontsize=\footnotesize]{xml}
  <DiscountingCurves id="ois">
    <DiscountingCurve currency="EUR">Yield/EUR/EUR1D</DiscountingCurve>
    <DiscountingCurve currency="USD">Yield/USD/USD1D</DiscountingCurve>
    <DiscountingCurve currency="GBP">Yield/GBP/GBP1D</DiscountingCurve>
    <DiscountingCurve currency="CHF">Yield/CHF/CHF6M</DiscountingCurve>
    <DiscountingCurve currency="JPY">Yield/JPY/JPY6M</DiscountingCurve>
    <!-- ... -->
  </DiscountingCurves>
\end{minted}
\caption{Discount curve block with ID 'ois'}
\label{lst:discountcurve_spec}
\end{listing}

This block instructs ORE to build five discount curves for the indicated currencies. The string within the tags,
e.g. Yield/EUR/EUR1D, uniquely identifies the curve to be built.  Curve Yield/EUR/EUR1D is defined in the curve
configuration file explained in section \ref{sec:curveconfig} below. In this case ORE is instructed to build an Eonia
Swap curve made of Overnight Deposit and Eonia Swap quotes. The right most token of the string Yield/EUR/EUR1D (EUR1D)
is user defined, the first two tokens Yield/EUR have to be used to point to a yield curve in currency EUR.
 
\subsubsection{Index Curves} 

See an excerpt of the index curve block with ID 'default' from the same example file:

\begin{listing}[H]
%\hrule\medskip
\begin{minted}[fontsize=\footnotesize]{xml}
<IndexForwardingCurves id="default">
  <Index name="EUR-EURIBOR-3M">Yield/EUR/EUR3M</Index>
  <Index name="EUR-EURIBOR-6M">Yield/EUR/EUR6M</Index>
  <Index name="EUR-EURIBOR-12M">Yield/EUR/EUR12M</Index>
  <Index name="EUR-EONIA">Yield/EUR/EUR1D</Index>
  <Index name="USD-LIBOR-3M">Yield/USD/USD3M</Index>
  <!-- ... -->
</IndexForwardingCurves>
\end{minted}
\caption{Index curve block with ID 'default'}
\label{lst:indexcurve_spec}
\end{listing}

This block of curve specifications instructs ORE to build another set of yield curves, unique strings
(e.g. Yield/EUR/EUR6M etc.) point to the {\tt curveconfig.xml} file where these curves are defined. Each curve is then
associated with an index name (of format Ccy-IndexName-Tenor, e.g. EUR-EURIBOR-6M) so that ORE will project the
respective index using the selected curve (e.g. Yield/EUR/EUR6M).

\subsubsection{Yield Curves}

See an excerpt of the yield curve block with ID 'default' from the same example file:

\begin{listing}[H]
%\hrule\medskip
\begin{minted}[fontsize=\footnotesize]{xml}
<YieldCurves id="default">
  <YieldCurve name="BANK_EUR_LEND">Yield/EUR/BANK_EUR_LEND</YieldCurve>
  <YieldCurve name="BANK_EUR_BORROW">Yield/EUR/BANK_EUR_BORROW</YieldCurve>
  <!-- ... -->
</YieldCurves>
\end{minted}
\caption{Yield curve block with ID 'default'}
\label{lst:yieldcurve_spec}
\end{listing}

This block of curve specifications instructs ORE to build another set of yield curves, unique strings
(e.g. Yield/EUR/EUR6M etc.) point to the {\tt curveconfig.xml} file where these curves are defined. Other than
discounting and index curves the yield curves in this block are not tied to a particular purpose. The curves defined in
this block typically include

\begin{itemize}
\item additional curves needed in the XVA post processor, e.g. for the FVA calculation
\item benchmark curves used for bond pricing
\end{itemize}

\subsubsection{Swap Index Curves}

The following is an excerpt of the swap index curve block with ID 'default' from the same example file:

\begin{listing}[H]
%\hrule\medskip
\begin{minted}[fontsize=\footnotesize]{xml}
<SwapIndexCurves id="default">
  <SwapIndex name="EUR-CMS-1Y">
    <Index>EUR-EURIBOR-6M</Index>
    <Discounting>EUR-EONIA</Discounting>
  </SwapIndex>
  <SwapIndex name="EUR-CMS-30Y">
    <Index>EUR-EURIBOR-6M</Index>
    <Discounting>EUR-EONIA</Discounting>
  </SwapIndex>
  <!-- ... -->
</SwapIndexCurves>
\end{minted}
\caption{Swap index curve block with ID 'default'}
\label{lst:swapindexcurve_spec}
\end{listing}

These instructions do not build any additional curves. They only build the respective swap index objects and associate
them with the required index forwarding and discounting curves already built above. This enables a swap index to project
the fair rate of forward starting Swaps. Swap indices are also containers for conventions. Swaption volatility surfaces
require two swap indices each available in the market object, a long term and a short term swap index. The curve
configuration file below will show that in particular the required short term index has term 1Y, and the required long
term index has 30Y term. This is why we build these two indices at this point.

\subsubsection{FX Spot}

The following is an excerpt of the FX spot block with ID 'default' from the same example file:

\begin{listing}[H]
%\hrule\medskip
\begin{minted}[fontsize=\footnotesize]{xml}
<FxSpots id="default">
  <FxSpot pair="EURUSD">FX/EUR/USD</FxSpot>
  <FxSpot pair="EURGBP">FX/EUR/GBP</FxSpot>
  <FxSpot pair="EURCHF">FX/EUR/CHF</FxSpot>
  <FxSpot pair="EURJPY">FX/EUR/JPY</FxSpot>
  <!-- ... -->
</FxSpots>
\end{minted}
\caption{FX spot block with ID 'default'}
\label{lst:fxspot_spec}
\end{listing}

This block instructs ORE to provide four FX quotes, all quoted with target currency EUR so
that foreign currency amounts can be converted into EUR via multiplication with that rate.
 
\subsubsection{FX Volatilities}

The following is an excerpt of the FX Volatilities block with ID 'default' from the same example file:

\begin{listing}[H]
%\hrule\medskip
\begin{minted}[fontsize=\footnotesize]{xml}
<FxVolatilities id="default">
  <FxVolatility pair="EURUSD">FXVolatility/EUR/USD/EURUSD</FxVolatility>
  <FxVolatility pair="EURGBP">FXVolatility/EUR/GBP/EURGBP</FxVolatility>
  <FxVolatility pair="EURCHF">FXVolatility/EUR/CHF/EURCHF</FxVolatility>
  <FxVolatility pair="EURJPY">FXVolatility/EUR/JPY/EURJPY</FxVolatility>
  <!-- ... -->
</FxVolatilities>
\end{minted}
\caption{FX volatility block with ID 'default'}
\label{lst:fxvol_spec}
\end{listing}

This instructs ORE to build four FX volatility structures for all FX pairs with target currency EUR, see curve
configuration file for the definition of the volatility structure.

\subsubsection{Swaption Volatilities}

The following is an excerpt of the Swaption Volatilities block with ID 'default' from the same example file:

\begin{listing}[H]
%\hrule\medskip
\begin{minted}[fontsize=\footnotesize]{xml}
<SwaptionVolatilities id="default">
  <SwaptionVolatility currency="EUR">SwaptionVolatility/EUR/EUR_SW_N</SwaptionVolatility>
  <SwaptionVolatility currency="USD">SwaptionVolatility/USD/USD_SW_N</SwaptionVolatility>
  <SwaptionVolatility currency="GBP">SwaptionVolatility/GBP/GBP_SW_N</SwaptionVolatility>
  <SwaptionVolatility currency="CHF">SwaptionVolatility/CHF/CHF_SW_N</SwaptionVolatility>
  <SwaptionVolatility currency="JPY">SwaptionVolatility/CHF/JPY_SW_N</SwaptionVolatility>
</SwaptionVolatilities>
\end{minted}
\caption{Swaption volatility block with ID 'default'}
\label{lst:swaptionvol_spec}
\end{listing}

This instructs ORE to build five Swaption volatility structures, see the curve configuration file for the definition of
the volatility structure. The latter token (e.g. EUR\_SW\_N) is user defined and will be found in the curve
configuration's CurveId tag.

\subsubsection{Cap/Floor Volatilities}

The following is an excerpt of the Cap/Floor Volatilities block with ID 'default' from the same example file:

\begin{listing}[H]
%\hrule\medskip
\begin{minted}[fontsize=\footnotesize]{xml}
<CapFloorVolatilities id="default">
  <CapFloorVolatility currency="EUR">CapFloorVolatility/EUR/EUR_CF_N</CapFloorVolatility>
  <CapFloorVolatility currency="USD">CapFloorVolatility/USD/USD_CF_N</CapFloorVolatility>
  <CapFloorVolatility currency="GBP">CapFloorVolatility/GBP/GBP_CF_N</CapFloorVolatility>
</CapFloorVolatilities>
\end{minted}
\caption{Cap/Floor volatility block with ID 'default'}
\label{lst:capfloorvol_spec}
\end{listing}

This instructs ORE to build three Cap/Floor volatility structures, see the curve configuration file for the definition
of the volatility structure. The latter token (e.g. EUR\_CF\_N) is user defined and will be found in the curve
configuration's CurveId tag.

\subsubsection{Default Curves}

The following is an excerpt of the Default Curves block with ID 'default' from the same example file:

\begin{listing}[H]
%\hrule\medskip
\begin{minted}[fontsize=\footnotesize]{xml}
<DefaultCurves id="default">
  <DefaultCurve name="BANK">Default/USD/BANK_SR_USD</DefaultCurve>
  <DefaultCurve name="CPTY_A">Default/USD/CUST_A_SR_USD</DefaultCurve>
  <DefaultCurve name="CPTY_B">Default/USD/CUST_A_SR_USD</DefaultCurve>
  <!-- ... -->
</DefaultCurves>
\end{minted}
\caption{Default curves block with ID 'default'}
\label{lst:defaultcurve_spec}
\end{listing}

This instructs ORE to build a set of default probability curves, again defined in the curve configuration file. Each
curve is then associated with a name (BANK, CUST\_A) for subsequent lookup.  As before, the last token
(e.g. BANK\_SR\_USD) is user defined and will be found in the curve configuration's CurveId tag.

\subsubsection{Securities}\label{sssec:securities}

The following is an excerpt of the Security block with ID 'default' from the same example file:

\begin{listing}[H]
	%\hrule\medskip
	\begin{minted}[fontsize=\footnotesize]{xml}
<Securities id="default">
  <Security name="SECURITY_1">Security/SECURITY_1</Security>
</Securities>
	\end{minted}
	\caption{Securities block with ID 'default'}
	\label{lst:secspread_spec}
\end{listing}

The pricing of bonds includes (among other components) a security specific spread and rate. 
This block links a security name to a spread and rate pair defined in the curve configuration file. This name may then be referenced 
as the security id in the bond trade definition.

\subsubsection{Equity Curves}
The following is an excerpt of the Equity curves block with ID 'default' from the same example file:

\begin{listing}[H]
%\hrule\medskip
\begin{minted}[fontsize=\footnotesize]{xml}
<EquityCurves id="default">
  <EquityCurve name="SP5">Equity/USD/SP5</EquityCurve>
  <EquityCurve name="Lufthansa">Equity/EUR/Lufthansa</EquityCurve>
</EquityCurves>
\end{minted}
\caption{Equity curves block with ID 'default'}
\label{lst:eqcurve_spec}
\end{listing}

This instructs ORE to build a set of equity curves, again defined in the curve configuration file. Each equity curve 
after construction will consist of a spot equity price, as well as a term structure of dividend yields, which can be 
used to determine forward prices. This object is then associated with a name (e.g. SP5) for subsequent lookup. 

\subsubsection{Equity Volatilities}

The following is an excerpt of the equity volatilities block with ID 'default' from the same example file:

\begin{listing}[H]
%\hrule\medskip
\begin{minted}[fontsize=\footnotesize]{xml}
<EquityVolatilities id="default">
  <EquityVolatility name="SP5">EquityVolatility/USD/SP5</EquityVolatility>
  <EquityVolatility name="Lufthansa">EquityVolatility/EUR/Lufthansa</EquityVolatility>
</EquityVolatilities>
\end{minted}
\caption{EQ volatility block with ID 'default'}
\label{lst:eqvol_spec}
\end{listing}

This instructs ORE to build two equity volatility structures for SP5 and Lufthansa, respectively. See the curve
configuration file for the definition of the equity volatility structure.


\subsubsection{Inflation Index Curves}

The following is an excerpt of the Zero Inflation Index Curves block with ID 'default' from the sample example file:

\begin{listing}[H]
%\hrule\medskip
\begin{minted}[fontsize=\footnotesize]{xml}
<ZeroInflationIndexCurves id="default">
    <ZeroInflationIndexCurve name="EUHICPXT">
        Inflation/EUHICPXT/EUHICPXT_ZC_Swaps
    </ZeroInflationIndexCurve>
    <ZeroInflationIndexCurve name="FRHICP">
        Inflation/FRHICP/FRHICP_ZC_Swaps
    </ZeroInflationIndexCurve>
    <ZeroInflationIndexCurve name="UKRPI">
        Inflation/UKRPI/UKRPI_ZC_Swaps
    </ZeroInflationIndexCurve>
    <ZeroInflationIndexCurve name="USCPI">
        Inflation/USCPI/USCPI_ZC_Swaps
    </ZeroInflationIndexCurve>
    ...
</ZeroInflationIndexCurves>
\end{minted}
\caption{Zero Inflation Index Curves block with ID 'default'}
\label{lst:zeroinflationindexcurve_spec}
\end{listing}

This instructs ORE to build a set of zero inflation index curves, which are defined in the curve configuration
file. Each curve is then associated with an index name (like e.g. EUHICPXT or UKRPI). The last token
(e.g. EUHICPXT\_ZC\_Swap) is user defined and will be found in the curve configuration's CurveId tag.

In a similar way, Year on Year index curves are specified:

\begin{listing}[H]
%\hrule\medskip
\begin{minted}[fontsize=\footnotesize]{xml}
  <YYInflationIndexCurves id="default">
      <YYInflationIndexCurve name="EUHICPXT">
          Inflation/EUHICPXT/EUHICPXT_YY_Swaps
      </YYInflationIndexCurve>
      ...
  </YYInflationIndexCurves>
\end{minted}
\caption{YoY Inflation Index Curves block with ID 'default'}
\label{lst:yoyinflationindexcurve_spec}
\end{listing}

Note that the index name is the same as in the corresponding zero index curve definition, but the token corresponding to
the CurveId tag is different. This is because the actual underlying index (and in particular its fixings) are shared
between the two index types, while different projection curves are used to forecast future index realisations.

\subsubsection{Inflation Cap/Floor Volatility Surfaces}

The following is an excerpt of the Inflation Cap/Floor Volatility Surfaces blocks with ID 'default' from the sample example
file:

{
\begin{listing}[H]
%\hrule\medskip
\begin{minted}[fontsize=\footnotesize]{xml}
  <YYInflationCapFloorVolatilities id="default">
    <YYInflationCapFloorVolatility name="EUHICPXT">
        InflationCapFloorVolatility/EUHICPXT/EUHICPXT_YY_CF
    </InflationCapFloorVolatility>
  </YYInflationCapFloorVolatilities>

  <ZeroInflationCapFloorVolatilities id="default">
    <ZeroInflationCapFloorVolatility name="UKRPI">
        InflationCapFloorVolatility/UKRPI/UKRPI_ZC_CF
    </ZeroInflationCapFloorVolatility>
    <ZeroInflationCapFloorVolatility name="EUHICPXT">
        InflationCapFloorVolatility/EUHICPXT/EUHICPXT_ZC_CF
    </ZeroInflationCapFloorVolatility>
    <ZeroInflationCapFloorVolatility name="USCPI">
        InflationCapFloorVolatility/USCPI/USCPI_ZC_CF
    </ZeroInflationCapFloorVolatility>
  </ZeroInflationCapFloorVolatilities>
\end{minted}
\caption{Inflation Cap/Floor Volatility Surfaces block with ID 'default'}
\label{lst:inflation_cap_floor_surface_spec}
\end{listing}

This instructs ORE to build a set of year-on-year and zero inflation cap floor volatility surfaces, which are defined in the curve
configuration file. Each surface is associated with an index name. The last token (e.g. EUHICPXT\_ZC\_CF) is user defined
and will be found in the curve configuration's CurveId tag.

\subsubsection{CDS Volatility Structures}

CDS volatility structures are configured as follows
\begin{listing}[H]
%\hrule\medskip
\begin{minted}[fontsize=\footnotesize]{xml}
  <CDSVolatilities id="default">
   <CDSVolatility name="CDSVOL_A">CDSVolatility/CDXIG</CDSVolatility>
   <CDSVolatility name="CDSVOL_B">CDSVolatility/CDXHY</CDSVolatility>
  </CDSVolatilities>
\end{minted}
\caption{CDS volatility structure block with ID 'default'}
\label{lst:cdsvol_spec}
\end{listing}

The composition of the CDS volatility structures is defined in the curve configuration.

\subsubsection{Base Correlation Structures}

Base correlation structures are configured as follows
\begin{listing}[H]
%\hrule\medskip
\begin{minted}[fontsize=\footnotesize]{xml}
  <BaseCorrelations id="default">
   <BaseCorrelation name="CDXIG">BaseCorrelation/CDXIG</BaseCorrelation>
  </BaseCorrelations>
\end{minted}
\caption{Base Correlations block with ID 'default'}
\label{lst:basecorr_spec}
\end{listing}

The composition of the base correlation structure is defined in the curve configuration.

\subsubsection{Correlation Structures}

Correlation structures are configured as follows
\begin{listing}[H]
%\hrule\medskip
\begin{minted}[fontsize=\footnotesize]{xml}
 <Correlations id="default">
      <Correlation name="EUR-CMS-10Y:EUR-CMS-1Y">Correlation/EUR-CORR</Correlation>  
      <Correlation name="USD-CMS-10Y:USD-CMS-1Y">Correlation/USD-CORR</Correlation>
 </Correlations>
\end{minted}
\caption{Correlations block with ID 'default'}
\label{lst:corr_spec}
\end{listing}

The composition of the correlation structure is defined in the curve configuration.
\subsubsection{Market Configurations}

Finally, representatives of each type of block (Discount Curves, Index Curves, Volatility structures, etc, up to
Inflation Cap/Floor Price Surfaces) can be bundled into a market configuration. This is done by adding the following to
the {\tt todaysmarket.xml} file:

\begin{listing}[H]
%\hrule\medskip
\begin{minted}[fontsize=\footnotesize]{xml}
<Configuration id="default">
  <DiscountingCurvesId>xois_eur</DiscountingCurvesId>
</Configuration>
<Configuration id="collateral_inccy">
  <DiscountingCurvesId>ois</DiscountingCurvesId>
</Configuration>
<Configuration id="collateral_eur">
  <DiscountingCurvesId>xois_eur</DiscountingCurvesId>
</Configuration>
<Configuration id="libor">
  <DiscountingCurvesId>inccy_swap</DiscountingCurvesId>
</Configuration>
\end{minted}
\caption{Market configurations}
\label{lst:config_spec}
\end{listing}

When ORE constructs the market object, all market configurations will be build and labelled using the 'Configuration
Id'.  This allows configuring a market setup for different alternative purposes side by side in the same {\tt
  todaysmarket.xml} file. Typical use cases are
\begin{itemize}
\item different discount curves needed for model calibration and risk factor evolution, respectively
\item different discount curves needed for collateralised and uncollateralised derivatives pricing.
\end{itemize}
The former is actually used throughout the {\tt Examples} section. Each master input file {\tt ore.xml} has a Markets
section (see \ref{sec:master_input}) where four market configuration IDs have to be provided - the ones used for
'lgmcalibration', 'fxcalibration', 'pricing' and 'simulation' (i.e. risk factor evolution).

\medskip The configuration ID concept extends across all curve and volatility objects though currently used only to
distinguish discounting.