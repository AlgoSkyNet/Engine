

%- - - - - - - - - - - - - - - - - - - - - - - - - - - - - - - - - - - - - - - -
\subsection{Trade Specific Data}
%- - - - - - - - - - - - - - - - - - - - - - - - - - - - - - - - - - - - - - - -

After the envelope node, trade-specific data for each trade type supported by
ORE is included. 
Each trade type has its own trade data container which is defined by an XML node containing a trade-specific
configuration of individual XML tags - called elements - and trade components. The trade components are defined by XML
sub-nodes that can be used within multiple trade data containers, i.e.  by multiple trade types.

\vspace{1em}

Details of  trade-specific data for all trade types follow below.

\subsubsection{Swap}

The \lstinline!SwapData! node is the trade data container for the Swap trade type. A Swap must have at least one leg,
and can have an unlimited number of legs. Each leg is represented by a \lstinline!LegData! trade component sub-node,
described in section \ref{ss:leg_data}. An example structure of a two-legged \lstinline!SwapData!
node is shown in Listing \ref{lst:swap_data}.

\begin{listing}[H]
%\hrule\medskip
\begin{minted}[fontsize=\footnotesize]{xml}
<SwapData>
  <LegData>
    ...
  </LegData>
  <LegData>
    ...
  </LegData>
</SwapData>
\end{minted}
\caption{Swap data}
\label{lst:swap_data}
\end{listing}

\subsubsection{Cap/Floor}

The \lstinline!CapFloorData! node is the trade data container for trade type CapFloors.  It's a cap, floor or collar
(i.e. a portfolio of a long cap and a short floor for a long position in the collar) on a series of Ibor or CMS rates. The
\lstinline!CapFloorData! node contains a \lstinline!LongShort! sub-node which indicates whether the cap (floor, collar)
is long or short, and a \lstinline!LegData!  sub-node where the LegType element must be set to \emph{Floating} or \emph{CMS}, plus
elements for the Cap and Floor rates. An example structure with Cap rates is shown in in Listing
\ref{lst:capfloor_data}. A \lstinline!CapFloorData! node must have either \lstinline!Caps! or \lstinline!Floors!
elements, or both.

\begin{listing}[H]
%\hrule\medskip
\begin{minted}[fontsize=\footnotesize]{xml}
<CapFloorData>
  <LongShort>Long</LongShort>
  <LegData>
    <Payer>false</Payer>
    <LegType>Floating</LegType>
     ...
  </LegData>
  <Caps>
    <Rate>0.05</Rate>
  </Caps>
</CapFloorData>
\end{minted}
\caption{Cap/Floor data}
\label{lst:capfloor_data}
\end{listing}

The meanings and allowable values of the elements in the \lstinline!CapFloorData!  node follow below.

\begin{itemize}

\item LongShort: This node defines the position in the cap (floor, collar) and can take values \lstinline!Long! or
  \lstinline!Short!

\item LegData: This is a trade component sub-node outlined in section \ref{ss:leg_data}. Exactly
  one \lstinline!LegData! node is allowed and the LegType element must be set to \emph{Floating} or \emph{CMS}.

\item Caps: This node has child elements of type \lstinline!Rate!
  capping the floating leg. The first rate value corresponds to the
  first coupon, the second rate value corresponds to the second
  coupon, etc. If the number of coupons exceeds the number of rate
  values, the rate will be kept flat at the value of last entered rate
  for the remaining coupons. For a fixed cap rate over all coupons,
  one single rate value is sufficient. The number of entered rate
  values cannot exceed the number of coupons. 

  Allowable values for each \lstinline!Rate! element: Any real number. The rate is expressed in decimal form, eg 0.05 is
  a rate of 5\%

\item Floors: This node has child elements of type
  \lstinline!Rate! flooring the floating leg.  The first rate value
  corresponds to the first coupon, the second rate value corresponds
  to the second coupon, etc. If the number of coupons exceeds the
  number of rate values, the rate will be kept flat at the value of
  last entered rate for the remaining coupons. For a fixed floor rate
  over all coupons, one single rate value is sufficient. The number of
  entered rate values cannot exceed the number of coupons.

  Allowable values for each \lstinline!Rate! element: Any real number. The rate is expressed in decimal form, eg 0.05 is
  a rate of 5\%

\end{itemize}

\subsubsection{Swaption}

The \lstinline!SwaptionData!  node is the trade data container for the Swaption trade type. The \lstinline!SwaptionData!
node has one and exactly one \lstinline!OptionData! trade component sub-node, and at least one \lstinline!LegData! trade
component sub-node.  These trade components are outlined in section \ref{ss:option_data} and section
\ref{ss:leg_data}.\\
\vspace{5mm}
Supported swaption exercise styles are European and Bermudan. European swaptions can have unlimited number of legs, with
each leg represented by a \lstinline!LegData! sub-node. Bermudan swaptions must have two legs, i.e. two
\lstinline!LegData! sub-nodes. See Table \ref{tab:bermudan_requirements} for further details on requirements for
Bermudan swaptions. Cross currency swaptions are not supported for either exercise style, i.e. the Currency element must
have the same value for all \lstinline!LegData! sub-nodes of a swaption.\\
\vspace{5mm}
The structure of an example \lstinline!SwaptionData!  node of a two-legged European swaption is shown in Listing
\ref{lst:swaption_data}.

\begin{listing}[H]
%\hrule\medskip
\begin{minted}[fontsize=\footnotesize]{xml}
<SwaptionData>
    <OptionData>
        <Style>European</Style>
        ...
    </OptionData>
    <LegData>
        <Currency>GBP</Currency>
        ...
    </LegData>
    <LegData>
        <Currency>GBP</Currency>
        ...
    </LegData>
</SwaptionData>
\end{minted}
\caption{Swaption data}
\label{lst:swaption_data}
\end{listing}

\begin{table}[H]
\centering
\begin{tabu} to 0.9\linewidth {| X[-1.5,l,m] | X[-5,l,m] |}
    \hline
        & \bfseries{A Bermudan Swaption requires:} \\  \hline
    \lstinline!OptionData! & One \lstinline!OptionData! sub-node  \\  \hline
   \lstinline!Style! &  \emph{Bermudan}\\ \hline
    \lstinline!ExerciseDates! & At least two \lstinline!ExerciseDate! child elements.\\ \hline
    \lstinline!LegData! &  Two \lstinline!LegData! sub-nodes \\ \hline
    \lstinline!LegType! & \emph{Fixed} on one node and \emph{Floating} on the other.\\ \hline    
    \lstinline!Currency! & The same currency for both nodes.\\ \hline 
%    \lstinline!Notionals! & No accretion or amortisation, just a constant notional. Exactly one \lstinline!Notional! child element for each node.\\ \hline
%   \lstinline!Rates! & A constant rate. The fixed rate node should have exactly one \lstinline!Rate! child element.\\ \hline
%    \lstinline!Spreads! &  A constant spread. The floating rate node should have exactly one \lstinline!Spread! child element.\\ \hline
%    \bfseries{ScheduleData} &   \\ \hline   
%    \lstinline!IsRuleBased! & Must be \emph{true} for both nodes. TBC: Fixed by Henning?\\ \hline      
  \end{tabu}
  \caption{Requirements for Bermudan Swaptions}
  \label{tab:bermudan_requirements}
\end{table}

\subsubsection{FX Forward}

The \lstinline!FXForwardData!  node is the trade data container for the FX Forward trade type.  The structure -
including example values - of the \lstinline!FXForwardData!  node is shown in Listing \ref{lst:fxforward_data}. It
contains no sub-nodes.

\begin{listing}[H]
%\hrule\medskip
\begin{minted}[fontsize=\footnotesize]{xml}
        <FxForwardData>
            <ValueDate>2023-04-09</ValueDate>
            <BoughtCurrency>EUR</BoughtCurrency>
            <BoughtAmount>1000000</BoughtAmount>
            <SoldCurrency>USD</SoldCurrency>
            <SoldAmount>1500000</SoldAmount>
        </FxForwardData>
\end{minted}
\caption{FX Forward data}
\label{lst:fxforward_data}
\end{listing}

The meanings and allowable values of the various elements in the \lstinline!FXForwardData!  node follow below.  All elements are required.

\begin{itemize}
\item ValueDate: The value date of the FX Forward. \\ Allowable values:  See \lstinline!Date! in Table \ref{tab:allow_stand_data}.
\item BoughtCurrency: The currency to be bought on value date.  \\ Allowable values:  See \lstinline!Currency! in Table \ref{tab:allow_stand_data}.
\item BoughtAmount: The amount to be bought on value date.  \\ Allowable values:  Any positive real number.
\item SoldCurrency: The currency to be sold on value date.  \\ Allowable values:  See \lstinline!Currency! in Table \ref{tab:allow_stand_data}.
\item SoldAmount: The amount to be sold on value date.  \\ Allowable values:  Any positive real number.
\item Settlement [Optional]: Delivery type.  Note that Non-Deliverable Forwards can be represented by \emph{Cash} settlement. Delivery type does not impact pricing in ORE.  \\Allowable values: \emph{Cash} or \emph{Physical}.  Defaults to \emph{Physical} if left blank or omitted.


\end{itemize}

\subsubsection{FX Swap}

The \lstinline!FXSwapData!  node is the trade data container for the FX Swap trade type.  The structure -
including example values - of the \lstinline!FXSwapData!  node is shown in Listing 
\ref{lst:fxswap_data}. 
It contains no sub-nodes.

\begin{listing}[H]
%\hrule\medskip
\begin{minted}[fontsize=\footnotesize]{xml}
        <FxSwapData>
            <NearDate>2018-09-01</NearDate>
            <NearBoughtCurrency>EUR</NearBoughtCurrency>
            <NearBoughtAmount>1000000</NearBoughtAmount>
            <NearSoldCurrency>USD</NearSoldCurrency>
            <NearSoldAmount>1140000</NearSoldAmount>
            <FarDate>2028-09-01</FarDate>
            <FarBoughtAmount>1300000</FarBoughtAmount>
            <FarSoldAmount>1000000</FarSoldAmount>
        </FxSwapData>
\end{minted}
\caption{FX Swap data}
\label{lst:fxswap_data}
\end{listing}

The meanings and allowable values of the various elements in the \lstinline!FXSwapData!  node follow below.  All elements are required.

\begin{itemize}
\item NearDate: The date of the initial fx exchange of the FX Swap. \\ Allowable values:  See \lstinline!Date! in Table \ref{tab:allow_stand_data}.
\item NearBoughtCurrency: The currency to be bought in the initial exchange at near date, and sold in the final exchange at far date.  \\ Allowable values:  See \lstinline!Currency! in Table \ref{tab:allow_stand_data}.
\item NearBoughtAmount: The amount to be bought on near date.  \\ Allowable values:  Any positive real number.
\item NearSoldCurrency: The currency to be sold in the initial fx exchange at near date, and bought in the final exchange at far date.   \\ Allowable values:  See \lstinline!Currency! in Table \ref{tab:allow_stand_data}.
\item NearSoldAmount: The amount to be sold on near date.  \\ Allowable values:  Any positive real number.
\item FarDate: The date of the final fx exchange of the FX Swap. \\ Allowable values:  Any date further into the future than NearDate. See \lstinline!Date! in Table  \ref{tab:allow_stand_data}.
\item FarBoughtAmount: The amount to be bought on far date.  \\ Allowable values:  Any positive real number.
\item FarSoldAmount: The amount to be sold on far date.  \\ Allowable values:  Any positive real number.
\end{itemize}

\subsubsection{FX Option}

The \lstinline!FXOptionData!  node is the trade data container for the FX Option trade type.  Vanilla FX options are
supported, the exercise style must be \emph{European}. The strike rate of an FX option is: SoldAmount / BoughtAmount. The
\lstinline!FXOptionData!  node includes one and only one \lstinline!OptionData! trade component sub-node plus elements
specific to the FX Option. The structure of an example \lstinline!FXOptionData! node for a FX Option is shown in Listing
\ref{lst:fxoption_data}.

\begin{listing}[H]
%\hrule\medskip
\begin{minted}[fontsize=\footnotesize]{xml}
        <FxOptionData>
            <OptionData>
             ...
            </OptionData>
            <BoughtCurrency>EUR</BoughtCurrency>
            <BoughtAmount>1000000</BoughtAmount>
            <SoldCurrency>USD</SoldCurrency>
            <SoldAmount>1350000</SoldAmount>
        </FxOptionData>
\end{minted}
\caption{FX Option data}
\label{lst:fxoption_data}
\end{listing}

The meanings and allowable values of the elements in the \lstinline!FXOptionData!  node follow below.

\begin{itemize}
\item OptionData: This is a trade component sub-node outlined in section \ref{ss:option_data}. Note that the
  FX option type allows for \emph{European} option style only.

\item BoughtCurrency: The bought currency of the FX option.  

Allowable values:  See Currency in Table \ref{tab:allow_stand_data}.

\item BoughtAmount: The amount in the BoughtCurrency.  

Allowable values:  Any positive real number.

\item SoldCurrency: The sold currency of the FX option.  

Allowable values:  See Currency in Table \ref{tab:allow_stand_data}.

\item SoldAmount: The amount in the SoldCurrency.  

Allowable values:  Any positive real number.

\end{itemize}



\subsubsection{Equity Option}

The \lstinline!EquityOptionData!  node is the trade data container for the equity option trade type.  Vanilla equity 
options are supported, the exercise style must be \emph{European}. The \lstinline!EquityOptionData!  node includes one and 
only one \lstinline!OptionData! trade component sub-node plus elements specific to the equity option. The structure of 
an example \lstinline!EquityOptionData! node for an equity option is shown in Listing
\ref{lst:eqoption_data}.

\begin{listing}[H]
%\hrule\medskip
\begin{minted}[fontsize=\footnotesize]{xml}
<EquityOptionData>
    <OptionData>
      ...
    </OptionData>
    <Name>ISIN:US78378X1072</Name>
    <Currency>USD</Currency>
    <Strike>2147.56</Strike>
    <Quantity>17000</Quantity>
</EquityOptionData>
\end{minted}
\caption{Equity Option data}
\label{lst:eqoption_data}
\end{listing}

The meanings and allowable values of the elements in the \lstinline!EquityOptionData!  node follow below.

\begin{itemize}
	\item OptionData: This is a trade component sub-node outlined in section \ref{ss:option_data} Option Data. Note 
	that the equity option type allows for \emph{European} option style only.	
	\item Name: The identifier of the underlying equity. \\
	Allowable values:  Any string (provided it is the ID of an equity in the market configuration). Typically an ISIN-code with the \emph{ISIN:} prefix.
	\item Currency: The currency of the equity option. \\
	Allowable values:  See \lstinline!Currency! in Table \ref{tab:allow_stand_data}.	
	\item Strike: The option strike price.\\
	Allowable values:  Any positive real number.	
	\item Quantity: The number of units of the underlying covered by the transaction. \\
	Allowable values:  Any positive real number.
\end{itemize}

\subsubsection{Equity Forward}

The \lstinline!EquityForwardData!  node is the trade data container for the equity forward trade type.  Vanilla equity 
forwards are supported. The structure of an example \lstinline!EquityForwardData! node for an equity option is shown in 
Listing \ref{lst:eqfwd_data}.

\begin{listing}[H]
%\hrule\medskip
\begin{minted}[fontsize=\footnotesize]{xml}
<EquityForwardData>
  <LongShort>Long</LongShort>
  <Maturity>2018-06-30</Maturity>
  <Name>ISIN:US78378X1072</Name>
  <Currency>USD</Currency>
  <Strike>2147.56</Strike>
  <Quantity>17000</Quantity>
</EquityForwardData>
\end{minted}
\caption{Equity Forward data}
\label{lst:eqfwd_data}
\end{listing}

The meanings and allowable values of the elements in the \lstinline!EquityForwardData!  node follow below.

\begin{itemize}
	\item LongShort: Defines whether the underlying equity will be bought (long) or sold (short). \\
	Allowable values:  \emph{Long}, \emph{Short}.
	\item Maturity: The maturity date of the forward contract, i.e. the date when the underlying equity will be bought/sold. \\
	Allowable values: Any date string.
	\item Name: The identifier of the underlying equity or equity index. \\
	Allowable values:  Any string (provided it is the ID of an equity in the market configuration). Typically an ISIN-code with the \emph{ISIN:} prefix.
	\item Currency: The  currency of the equity forward. \\
	Allowable values:  See \lstinline!Currency! in Table \ref{tab:allow_stand_data}.	
	\item Strike: The agreed buy/sell price of the equity forward. \\
	Allowable values:  Any positive real number.	
	\item Quantity: The number of units of the underlying equity to be bought/sold. \\
	Allowable values:  Any positive real number.
\end{itemize}


\subsubsection{CPI Swap}

A CPI swap is set up as a swap, with one leg of type {\tt CPI}. Listing \ref{lst:cpiswap} shows an example. The
CPI leg contains an additional {\tt CPILegData} block. See \ref{ss:cpilegdata} for details on the CPI leg specification.

\begin{listing}[H]
%\hrule\medskip
\begin{minted}[fontsize=\footnotesize]{xml}
    <SwapData>
      <LegData>
        <LegType>Floating</LegType>
        <Payer>true</Payer>
         ...
      </LegData>
      <LegData>
        <LegType>CPI</LegType>
        <Payer>false</Payer>
        ...
        <CPILegData>
        ...
        </CPILegData>
      </LegData>
    </SwapData>
\end{minted}
\caption{CPI Swap Data}
\label{lst:cpiswap}
\end{listing}

\subsubsection{Year on Year Inflation Swap}

A Year on Year inflation swap is set up as a swap, with one leg of type {\tt YY}. Listing \ref{lst:yyswap} shows an
example. The YY leg contains an additional {\tt YYLegData} block. See \ref{ss:yylegdata} for details on the YY leg
specification.

\begin{listing}[H]
%\hrule\medskip
\begin{minted}[fontsize=\footnotesize]{xml}
    <SwapData>
      <LegData>
        <LegType>Floating</LegType>
        <Payer>true</Payer>
        ...
      </LegData>
      <LegData>
        <LegType>YY</LegType>
        <Payer>false</Payer>
        ...
        <YYLegData>
        ...
        </YYLegData>
      </LegData>
    </SwapData>
\end{minted}
\caption{Year on Year Swap Data}
\label{lst:yyswap}
\end{listing}

\subsubsection{Zero Coupon Swap}

A Zero Coupon swap is set up as a swap, with one leg of type {\tt ZeroCouponFixed}. Listing \ref{lst:zeroswap} shows an
example. The ZeroCouponFixed leg contains an additional {\tt ZeroCouponFixedLegData} block. See \ref{ss:zerolegdata} for details on the ZeroCouponFixed leg
specification.

\begin{listing}[H]
%\hrule\medskip
\begin{minted}[fontsize=\footnotesize]{xml}
    <SwapData>
      <LegData>
        <LegType>Floating</LegType>
        <Payer>true</Payer>
        ...
      </LegData>
      <LegData>
        <LegType>ZeroCouponFixed</LegType>
        <Payer>false</Payer>
        ...
        <ZeroCouponFixedLegData>
            <Rates>
                <Rate>0.02</Rate>
            </Rates>
            <Compounding>Simple</Compounding>
        </ZeroCouponFixedLegData>
      </LegData>
    </SwapData>
\end{minted}
\caption{Zero Coupon Swap Data}
\label{lst:zeroswap}
\end{listing}

\subsubsection{Bond}

A vanilla Bond is set up using a {\tt BondData} block as shown in listing \ref{lst:bonddata}. The bond specific elements
are

\begin{itemize}
\item IssuerId: A unique identifier for the issuer of the bond
\item CreditCurveId: The entity defining the default curve used for pricing, via the default curves block in {\tt
    todaysmarket.xml}
% \item LGD (optional): If given, this LGD is used for pricing, overriding the default LGD of the default curve
\item SecurityId: A unique identifier for the security, this defines the security specific spread to be used for pricing.
\item ReferenceCurveId: The benchmark curve to be used for pricing, this must match a name of a yield curve in the market data configuration.
\item SettlementDays: The settlement delay applicable to the security.
\item Calendar: The calendar associated to the settlement lag.
\item IssueDate: The issue date of the security.
\end{itemize}

A LegData block then defines the cashflow structure of the bond, this can be of type fixed, floating etc.

\begin{listing}[H]
%\hrule\medskip
\begin{minted}[fontsize=\footnotesize]{xml}
    <BondData>
        <IssuerId>CPTY_C</IssuerId>
        <CreditCurveId>CPTY_C</CreditCurveId>
        <SecurityId>SECURITY_1</SecurityId>
        <ReferenceCurveId>EUR-EURIBOR-6M</ReferenceCurveId>
        <SettlementDays>2</SettlementDays>
        <Calendar>TARGET</Calendar>
        <IssueDate>20160203</IssueDate>
        <LegData>
            <LegType>Fixed</LegType>
            <Payer>false</Payer>
            ...
        </LegData>
    </BondData>
\end{minted}
\caption{Bond Data}
\label{lst:bonddata}
\end{listing}

The bond pricing requires a recovery rate that can be specified per SecurityId in the market data configuration.

\subsubsection{Credit Default Swap}

A credit default swap is set up using a {\tt CreditDefaultSwapData} block as shown in listing \ref{lst:cdsdata}. The CDS specific elements
are

\begin{itemize}
\item IssuerId: An identifier for the reference entity of the CDS. For informational purposes and not used for pricing.
\item CreditCurveId: Typically the ISIN-code of the reference entity defining the default curve used for pricing. Other identifiers may be used as well, provided they are supported in the market data configuration.
\item SettlesAccrual: Whether or not the accrued coupon is due in the event of a default.
\item PaysAtDefaultTime: If set to true, any payments triggered by a default event are due at default time. If set to false, they are due at the end of the accrual period.
\item ProtectionStart: The first date where a default event will trigger the contract.
\item UpfrontDate [Optional]: Settlement date for the upfront payment.
\item UpfrontFee [Optional]: The upfront payment, expressed as a rate, to be multiplied by notional amount.
\end{itemize}

A LegData block then defines the cashflow structure of the credit default swap, this must be be of type \emph{Fixed}.

\begin{listing}[H]
%\hrule\medskip
\begin{minted}[fontsize=\footnotesize]{xml}
    <CreditDefaultSwapData>
      <IssuerId>CPTY_A</IssuerId>
      <CreditCurveId>ISIN:XS0982710740</CreditCurveId>
      <SettlesAccrual>Y</SettlesAccrual>
      <PaysAtDefaultTime>Y</PaysAtDefaultTime>
      <ProtectionStart>20160206</ProtectionStart>
      <UpfrontDate>20160208</UpfrontDate>
      <UpfrontFee>0.0</UpfrontFee>
      <LegData>
            <LegType>Fixed</LegType>
            <Payer>false</Payer>
            ...
        </LegData>
    </CreditDefaultSwapData>
\end{minted}
\caption{CreditDefaultSwap Data}
\label{lst:cdsdata}
\end{listing}

\subsubsection{Forward Rate Agreement}

A forward rate agreement is set up using a {\tt ForwardRateAgreementData} block as shown in listing \ref{lst:ForwardRateAgreementdata}. The forward rate agreement specific elements
are

\begin{itemize}
\item StartDate: A FRA expires/settles on the startDate. \\
Allowable values:  See \lstinline!Date! in Table \ref{tab:allow_stand_data}.

\item EndDate: EndDate is the date when the forward loan or deposit ends. It follows that (EndDate - StartDate) is the tenor/term of the underlying loan or deposit.
\item Currency: The currency of the FRA notional. \\
Allowable values:  See \lstinline!Currency! in Table \ref{tab:allow_stand_data}.	
\item Index: The name of the interest rate index the FRA is benchmarked against.
\item LongShort: Specifies whether the FRA position is long (one receives the agreed rate) or short (one pays the agreed rate).
\item Strike: The agreed forward interest rate.
\item Notional: No accretion or amortisation, just a constant notional.
\end{itemize}


\begin{listing}[H]
%\hrule\medskip
\begin{minted}[fontsize=\footnotesize]{xml}
    <ForwardRateAgreementData>
        <StartDate>20161028</StartDate>
        <EndDate>20351028</EndDate>
        <Currency>EUR</Currency>
        <Index>EUR-EURIBOR-6M</Index>
        <LongShort>Long</LongShort>
        <Strike>0.00001</Strike>
        <Notional>1000000000</Notional>
    </ForwardRateAgreementData>
\end{minted}
\caption{Forward Rate Agreement Data}
\label{lst:ForwardRateAgreementdata}
\end{listing}


\subsubsection{Commodity Option}

The \lstinline!CommodityOptionData!  node is the trade data container for the commodity option trade type.  Vanilla commodity 
options are supported, the exercise style must be \emph{European}. The \lstinline!CommodityOptionData!  node includes one and 
only one \lstinline!OptionData! trade component sub-node plus elements specific to the commodity option. The structure of 
an example \lstinline!CommodityOptionData! node for a commodity option is shown in Listing
\ref{lst:comoption_data}.

\begin{listing}[H]
%\hrule\medskip
\begin{minted}[fontsize=\footnotesize]{xml}
<CommodityOptionData>
    <OptionData>
      ...
    </OptionData>
    <Name>COMDTY_WTI_USD</Name>
    <Currency>USD</Currency>
    <Strike>70</Strike>
    <Quantity>500000</Quantity>
</CommodityOptionData>
\end{minted}
\caption{Commodity Option data}
\label{lst:comoption_data}
\end{listing}

The meanings and allowable values of the elements in the \lstinline!CommodityOptionData!  node follow below.

\begin{itemize}
	\item OptionData: This is a trade component sub-node outlined in section \ref{ss:option_data} Option Data. Note 
	that the commodity option type allows for \emph{European} option style only.	
	\item Name: The name of the underlying commodity. \\
	Allowable values:  Any string (provided it is the ID of a commodity in the market configuration).
	\item Currency: The currency of the commodity option. \\
	Allowable values:  See \lstinline!Currency! in Table \ref{tab:allow_stand_data}.	
	\item Strike: The option strike price.\\
	Allowable values:  Any positive real number.	
	\item Quantity: The number of units of the underlying commodity covered by the transaction. \\
	Allowable values:  Any positive real number.
\end{itemize}

\subsubsection{Commodity Forward}

The \lstinline!CommodityForwardData!  node is the trade data container for the commodity forward trade type.   The structure of an example \lstinline!CommodityForwardData! node for an equity option is shown in 
Listing \ref{lst:comfwd_data}.

\begin{listing}[H]
%\hrule\medskip
\begin{minted}[fontsize=\footnotesize]{xml}
<CommodityForwardData>
  <LongShort>Long</LongShort>
  <Maturity>2018-06-30</Maturity>
  <Name>COMDTY_GOLD_USD</Name>
  <Currency>USD</Currency>
  <Strike>1355</Strike>
  <Quantity>1000</Quantity>
</CommodityForwardData>
\end{minted}
\caption{Commodity Forward data}
\label{lst:comfwd_data}
\end{listing}

The meanings and allowable values of the elements in the \lstinline!CommodityForwardData!  node follow below.

\begin{itemize}
	\item LongShort: Defines whether the underlying commodity will be bought (long) or sold (short). \\
	Allowable values: \emph{Long, Short}
	\item Maturity: The maturity date of the forward contract, i.e. the date when the underlying commodity will be bought/sold. \\
	Allowable values: Any date string.
	\item Name: The name of the underlying commodity. \\
	Allowable values:  Any string (provided it is the ID of an equity in the market configuration).
	\item Currency: The  currency of the commodity forward. \\
	Allowable values:  See \lstinline!Currency! in Table \ref{tab:allow_stand_data}.	
	\item Strike: The agreed buy/sell price of the commodity forward. \\
	Allowable values:  Any positive real number.	
	\item Quantity: The number of units of the underlying commodity to be bought/sold. \\
	Allowable values:  Any positive real number.
\end{itemize}

