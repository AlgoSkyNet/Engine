%--------------------------------------------------------
\subsection{Conventions: {\tt conventions.xml}}
\label{sec:conventions}
%--------------------------------------------------------

The conventions to associate with a set market quotes in the construction of termstructures are specified in another xml
file which we will refer to as {\tt conventions.xml} in the following though the file name can be chosen by the user.
Each separate set of conventions is stored in an XML node. The type of conventions that a node holds is determined by
the node name. Every node has an \lstinline!Id! node that gives a unique identifier for the convention set. The
following sections describe the type of conventions that can be created and the allowed values.

%- - - - - - - - - - - - - - - - - - - - - - - - - - - - - - - - - - - - - - - -
\subsubsection{Zero Conventions}
%- - - - - - - - - - - - - - - - - - - - - - - - - - - - - - - - - - - - - - - -
A node with name \emph{Zero} is used to store conventions for direct zero rate quotes. Direct zero rate quotes can be
given with an explicit maturity date or with a tenor and a set of conventions from which the maturity date is
deduced. The node for a zero rate quote with an explicit maturity date is shown in Listing
\ref{lst:zero_conventions_date}. The node for a tenor based zero rate is shown in Listing
\ref{lst:zero_conventions_tenor}.

\begin{listing}[H]
%\hrule\medskip
\begin{minted}[fontsize=\footnotesize]{xml}
<Zero>
  <Id> </Id>
  <TenorBased>False</TenorBased>
  <DayCounter> </DayCounter>
  <CompoundingFrequency> </CompoundingFrequency>
  <Compounding> </Compounding>
</Zero>
\end{minted}
\caption{Zero conventions}
\label{lst:zero_conventions_date}
\end{listing}

\begin{listing}[H]
%\hrule\medskip
\begin{minted}[fontsize=\footnotesize]{xml}
<Zero>
  <Id> </Id>
  <TenorBased>True</TenorBased>
  <DayCounter> </DayCounter>
  <CompoundingFrequency> </CompoundingFrequency>
  <Compounding> </Compounding>
  <TenorCalendar> </TenorCalendar>
  <SpotLag> </SpotLag>
  <SpotCalendar> </SpotCalendar>
  <RollConvention> </RollConvention>
  <EOM> </EOM>
</Zero>
\end{minted}
\caption{Zero conventions, tenor based}
\label{lst:zero_conventions_tenor}
\end{listing}

The meanings of the various elements in this node are as follows:
\begin{itemize}
\item TenorBased: True if the conventions are for a tenor based zero quote and False if they are
for a zero quote with an explicit maturity date.
\item DayCounter: The day count basis associated with the zero rate quote (for choices see section
\ref{sec:allowable_values})
\item CompoundingFrequency: The frequency of compounding (Choices are {\em Once, Annual, Semiannual, Quarterly,
Bimonthly, Monthly, Weekly, Daily}).
\item Compounding: The type of compounding for the zero rate (Choices are {\em Simple, Compounded, Continuous,
SimpleThenCompounded}).
\item TenorCalendar: The calendar used to advance from the spot date to the maturity date by the zero rate tenor (for
choices see section \ref{sec:allowable_values}).
\item SpotLag [Optional]: The number of business days to advance from the valuation date before applying the zero rate
tenor. If not provided, this defaults to 0.
\item SpotCalendar [Optional]: The calendar to use for business days when applying the \lstinline!SpotLag!. If not
provided, it defaults to a calendar with no holidays.
\item RollConvention [Optional]: The roll convention to use when applying the zero rate tenor. If not provided, it
defaults to Following (Choices are {\em Backward, Forward, Zero, ThirdWednesday, Twentieth, TwentiethIMM, CDS, ThirdThursday, ThirdFriday, MondayAfterThirdFriday, TuesdayAfterThirdFriday, LastWednesday}).
\item EOM [Optional]: Whether or not to use the end of month convention when applying the zero rate tenor. If not
provided, it defaults to false.
\end{itemize}

%- - - - - - - - - - - - - - - - - - - - - - - - - - - - - - - - - - - - - - - -
\subsubsection{Deposit Conventions}
%- - - - - - - - - - - - - - - - - - - - - - - - - - - - - - - - - - - - - - - -

A node with name \emph{Deposit} is used to store conventions for deposit or index fixing quotes. The conventions can be
index based, in which case all necessary conventions are deduced from a given index family. The structure of the index
based node is shown in Listing \ref{lst:deposit_conventions_index}. Alternatively, all the necessary conventions can be
given explicitly without reference to an index family. The structure of this node is shown in Listing
\ref{lst:deposit_conventions_explicit}.

\begin{listing}[H]
%\hrule\medskip
\begin{minted}[fontsize=\footnotesize]{xml}
<Deposit>
  <Id> </Id>
  <IndexBased>True</IndexBased>
  <Index> </Index>
</Deposit>
\end{minted}
\caption{Deposit conventions}
\label{lst:deposit_conventions_index}
\end{listing}

\begin{listing}[H]
%\hrule\medskip
\begin{minted}[fontsize=\footnotesize]{xml}
<Deposit>
  <Id> </Id>
  <IndexBased>False</IndexBased>
  <Calendar> </Calendar>
  <Convention> </Convention>
  <EOM> </EOM>
  <DayCounter> </DayCounter>
</Deposit>
\end{minted}
\caption{Deposit conventions}
\label{lst:deposit_conventions_explicit}
\end{listing}


The meanings of the various elements in this node are as follows:
\begin{itemize}
\item IndexBased: \emph{True} if the deposit conventions are index based and \emph{False} if the conventions are given
explicitly.
\item Index: The index family from which to imply the conventions for the deposit quote. For example, this could be
EUR-EURIBOR, USD-LIBOR etc.
\item Calendar: The business day calendar for the deposit quote.
\item Convention: The roll convention for the deposit quote.
\item EOM: \emph{True} if the end of month roll convention is to be used for the deposit quote and \emph{False} if not.
\item DayCounter: The day count basis associated with the deposit quote.
\end{itemize}

%- - - - - - - - - - - - - - - - - - - - - - - - - - - - - - - - - - - - - - - -
\subsubsection{Future Conventions}\label{ss:conventions_future}
%- - - - - - - - - - - - - - - - - - - - - - - - - - - - - - - - - - - - - - - -

A node with name \emph{Future} is used to store conventions for money market (MM) or overnight index (OI) interest rate
future quotes, for example futures on Euribor 3M or SOFR 3M underlyings. The structure of this node is shown in Listing
\ref{lst:future_conventions}. The fields have the following meaning:

\begin{itemize}
\item Id: The name of the convention.
\item Index: The underlying index of the futures, this is either a MM (i.e. Ibor) index like e.g. EUR-EURIBOR-3M or an
  overnight index like e.g. USD-SOFR.
\item DateGenerationRule [Optional]: This should be set to 'IMM' when the start and end dates of the future are
  following the IMM date logic or 'FirstDayOfMonth' when the start and end date are the first day of a month. If not
  given this field defaults to 'IMM'.
  \begin{itemize}
  \item For MM futures only 'IMM' is allowed and the expiry date is determined as the next 3rd Wednesday of the expiry
    month of a future.
  \item For an overnight index future 'IMM' means that the end date of the future is set to the 3rd Wednesday of the
    expiry month and the start date is set to the 3rd Wednesday of the expiry month minus the future tenor. The setting
    'IMM' applies to SOFR-3M futures for example. 'FirstDayOfMonth' on the other hand means that the end date of the
    future is set to the first day in the month following the future's expiry month and the start date is set to the
    first day of the month lying $n$ months before the end date's month where $n$ is the number of months of the
    future's underlying tenor. The setting 'FirstDayOfMonth' applies to SOFR-1M futures for example. This tenor is
    derived from the market quote, see \ref{ss:market_data_oi_index_future_prices}.
  \end{itemize}
\item OvernightIndexFutureNettingType [Optional]: Only relevant for OI futures. Can be 'Compounding' (which is also the
  default value if no value is given) or 'Averaging'. For example, SOFR 3M futures are compounding while SOFR 1M futures
  are averaging the daily overnight fixings over the calculation period of the future.
\end{itemize}

Listings \ref{lst:future_conventions_euribor_3m}, \ref{lst:future_conventions_sofr_3m},
\ref{lst:future_conventions_sofr_1m} show examples for Euribor-3M, SOFR-3M and SOFR-1M future conventions.

\begin{listing}[H]
%\hrule\medskip
\begin{minted}[fontsize=\footnotesize]{xml}
<Future>
  <Id> </Id>
  <Index> </Index>
  <DateGenerationRule> </DateGenerationRule>
  <OvernightIndexFutureNettingType> </OvernightIndexFutureNettingType>
</Future>
\end{minted}
\caption{Future conventions}
\label{lst:future_conventions}
\end{listing}

\begin{listing}[H]
%\hrule\medskip
\begin{minted}[fontsize=\footnotesize]{xml}
<Future>
  <Id>EURIBOR-3M-FUTURES</Id>
  <Index>EUR-EURIBOR-3M</Index>
</Future>
\end{minted}
\caption{Euribor 3M MM Future conventions}
\label{lst:future_conventions_euribor_3m}
\end{listing}

\begin{listing}[H]
%\hrule\medskip
\begin{minted}[fontsize=\footnotesize]{xml}
  <Future>
    <Id>USD-SOFR-3M-FUTURES</Id>
    <Index>USD-SOFR</Index>
    <DateGenerationRule>IMM</DateGenerationRule>
    <OvernightIndexFutureNettingType>Compounding</OvernightIndexFutureNettingType>
  </Future>
\end{minted}
\caption{USD SOFR 3M OI Future conventions}
\label{lst:future_conventions_sofr_3m}
\end{listing}

\begin{listing}[H]
%\hrule\medskip
\begin{minted}[fontsize=\footnotesize]{xml}
  <Future>
    <Id>USD-SOFR-1M-FUTURES</Id>
    <Index>USD-SOFR</Index>
    <DateGenerationRule>FirstDayOfMonth</DateGenerationRule>
    <OvernightIndexFutureNettingType>Averaging</OvernightIndexFutureNettingType>
  </Future>
\end{minted}
\caption{USD SOFR 1M OI Future conventions}
\label{lst:future_conventions_sofr_1m}
\end{listing}


%- - - - - - - - - - - - - - - - - - - - - - - - - - - - - - - - - - - - - - - -
\subsubsection{FRA Conventions}
%- - - - - - - - - - - - - - - - - - - - - - - - - - - - - - - - - - - - - - - -
A node with name \emph{FRA} is used to store conventions for FRA quotes. The structure of this node is shown in Listing 
\ref{lst:fra_conventions}. The only piece of information needed is the underlying index name and this is given in the 
\lstinline!Index! node. For example, this could be EUR-EURIBOR-6M, CHF-LIBOR-6M etc.

\begin{listing}[H]
%\hrule\medskip
\begin{minted}[fontsize=\footnotesize]{xml}
<FRA>
  <Id> </Id>
  <Index> </Index>
</FRA>
\end{minted}
\caption{FRA conventions}
\label{lst:fra_conventions}
\end{listing}

%- - - - - - - - - - - - - - - - - - - - - - - - - - - - - - - - - - - - - - - -
\subsubsection{OIS Conventions}
%- - - - - - - - - - - - - - - - - - - - - - - - - - - - - - - - - - - - - - - -

A node with name \emph{OIS} is used to store conventions for Overnight Indexed Swap (OIS) quotes. The structure of this
node is shown in Listing \ref{lst:ois_conventions}.

\begin{listing}[H]
%\hrule\medskip
\begin{minted}[fontsize=\footnotesize]{xml}
<OIS>
  <Id> </Id>
  <SpotLag> </SpotLag>
  <Index> </Index>
  <FixedDayCounter> </FixedDayCounter>
  <PaymentLag> </PaymentLag>
  <EOM> </EOM>
  <FixedFrequency> </FixedFrequency>
  <FixedConvention> </FixedConvention>
  <FixedPaymentConvention> </FixedPaymentConvention>
  <Rule> </Rule>
  <PaymentCalendar> </PaymentCalendar>
</OIS>
\end{minted}
\caption{OIS conventions}
\label{lst:ois_conventions}
\end{listing}

The meanings of the various elements in this node are as follows:
\begin{itemize}
\item SpotLag: The number of business days until the start of the OIS.
\item Index: The name of the overnight index. For example, this could be EUR-EONIA, USD-FedFunds etc.
\item FixedDayCounter: The day count basis on the fixed leg of the OIS.
\item PaymentLag [Optional]: The payment lag, as a number of business days, on both legs. If not provided, this defaults
to 0.
\item EOM [Optional]: \emph{True} if the end of month roll convention is to be used when generating the OIS schedule and
\emph{False} if not. If not provided, this defaults to \emph{False}.
\item FixedFrequency [Optional]: The frequency of payments on the fixed leg. If not provided, this defaults to
\emph{Annual}.
\item FixedConvention [Optional]: The roll convention for accruals on the fixed leg. If not provided, this defaults to
\emph{Following}.
\item FixedPaymentConvention [Optional]: The roll convention for payments on the fixed leg. If not provided, this
defaults to \emph{Following}.
\item Rule [Optional]: The rule used for generating the OIS dates schedule i.e.\ \emph{Backward} or \emph{Forward}. If
not provided, this defaults to \emph{Backward}.
\item PaymentCalendar [Optional]: The business day calendar used for determining coupon payment dates.
If not specified, this defaults to the fixing calendar defined on the overnight index.
\end{itemize}

%- - - - - - - - - - - - - - - - - - - - - - - - - - - - - - - - - - - - - - - -
\subsubsection{Swap Conventions}
%- - - - - - - - - - - - - - - - - - - - - - - - - - - - - - - - - - - - - - - -
A node with name \emph{Swap} is used to store conventions for vanilla interest rate swap (IRS) quotes. The structure of
this node is shown in Listing \ref{lst:swap_conventions}.

\begin{listing}[H]
%\hrule\medskip
\begin{minted}[fontsize=\footnotesize]{xml}
<Swap>
  <Id> </Id>
  <FixedCalendar> </FixedCalendar>
  <FixedFrequency> </FixedFrequency>
  <FixedConvention> </FixedConvention>
  <FixedDayCounter> </FixedDayCounter>
  <Index> </Index>
  <FloatFrequency> </FloatFrequency>
  <SubPeriodsCouponType> </SubPeriodsCouponType>
</Swap>
\end{minted}
\caption{Swap conventions}
\label{lst:swap_conventions}
\end{listing}

The meanings of the various elements in this node are as follows:
\begin{itemize}
\item FixedCalendar: The business day calendar on the fixed leg.
\item FixedFrequency: The frequency of payments on the fixed leg.
\item FixedConvention: The roll convention on the fixed leg.
\item FixedDayCounter: The day count basis on the fixed leg.
\item Index: The Ibor index on the floating leg.
\item FloatFrequency [Optional]: The frequency of payments on the floating leg, to be used if the frequency is different to the tenor of the index (e.g. CAD swaps for BA-3M have a 6M or 1Y payment frequency with a Compounding coupon)
\item SubPeriodsCouponType [Optional]: Defines how coupon rates should be calculated when the float frequency is different to that of the index. Possible values are "Compounding" and "Averaging".
\end{itemize}

%- - - - - - - - - - - - - - - - - - - - - - - - - - - - - - - - - - - - - - - -
\subsubsection{Average OIS Conventions}
%- - - - - - - - - - - - - - - - - - - - - - - - - - - - - - - - - - - - - - - -
A node with name \emph{AverageOIS} is used to store conventions for average OIS quotes. An average OIS is a swap where a
fixed rate is swapped against a daily averaged overnight index plus a spread. The structure of this node is shown in
Listing \ref{lst:average_ois_conventions}.

\begin{listing}[H]
%\hrule\medskip
\begin{minted}[fontsize=\footnotesize]{xml}
<AverageOIS>
  <Id> </Id>
  <SpotLag> </SpotLag>
  <FixedTenor> </FixedTenor>
  <FixedDayCounter> </FixedDayCounter>
  <FixedCalendar> </FixedCalendar>
  <FixedConvention> </FixedConvention>
  <FixedPaymentConvention> </FixedPaymentConvention>
  <FixedFrequency> </FixedFrequency>
  <Index> </Index>
  <OnTenor> </OnTenor>
  <RateCutoff> </RateCutoff>
</AverageOIS>
\end{minted}
\caption{Average OIS conventions}
\label{lst:average_ois_conventions}
\end{listing}


The meanings of the various elements in this node are as follows:
\begin{itemize}
\item SpotLag: Number of business days until the start of the average OIS.
\item FixedTenor: The frequency of payments on the fixed leg.
\item FixedDayCounter: The day count basis on the fixed leg.
\item FixedCalendar: The business day calendar on the fixed leg.
\item FixedFrequency: The frequency of payments on the fixed leg.
\item FixedConvention: The roll convention for accruals on the fixed leg.
\item FixedPaymentConvention: The roll convention for payments on the fixed leg.
\item FixedFrequency [Optional]: The frequency of payments on the fixed leg. If not provided, this defaults to \emph{Annual}.
\item Index: The name of the overnight index.
\item OnTenor: The frequency of payments on the overnight leg.
\item RateCutoff: The rate cut-off on the overnight leg. Generally, the overnight fixing is only observed up to a
certain number of days before the payment date and the last observed rate is applied for the remaining days in the
period. This rate cut-off gives the number of days e.g.\ 2 for Fed Funds average OIS.
\end{itemize}

%- - - - - - - - - - - - - - - - - - - - - - - - - - - - - - - - - - - - - - - -
\subsubsection{Tenor Basis Swap Conventions}
%- - - - - - - - - - - - - - - - - - - - - - - - - - - - - - - - - - - - - - - -
A node with name \emph{TenorBasisSwap} is used to store conventions for tenor basis swap quotes. The structure of this 
node is shown in Listing \ref{lst:tenor_basis_conventions}.

\begin{listing}[H]
%\hrule\medskip
\begin{minted}[fontsize=\footnotesize]{xml}
<TenorBasisSwap>
  <Id> </Id>
  <LongIndex> </LongIndex>
  <ShortIndex> </ShortIndex>
  <ShortPayTenor> </ShortPayTenor>
  <SpreadOnShort> </SpreadOnShort>
  <IncludeSpread> </IncludeSpread>
  <SubPeriodsCouponType> </SubPeriodsCouponType>
</TenorBasisSwap>
\end{minted}
\caption{Tenor basis swap conventions}
\label{lst:tenor_basis_conventions}
\end{listing}


The meanings of the various elements in this node are as follows:
\begin{itemize}
\item LongIndex: The name of the long tenor Ibor index.
\item ShortIndex: The name of the short tenor Ibor index.
\item ShortPayTenor [Optional]: The frequency of payments on the short tenor Ibor leg. This is usually the same as the
short tenor Ibor index's tenor. However, it can also be longer e.g.\ USD tenor basis swaps where the short tenor Ibor
index is compounded and paid on the same frequency as the long tenor Ibor index. If not provided, this defaults to the
short tenor Ibor index's tenor.
\item SpreadOnShort [Optional]: \emph{True} if the tenor basis swap quote has the spread on the short tenor Ibor index
leg and \emph{False} if not. If not provided, this defaults to \emph{True}.
\item IncludeSpread [Optional]: \emph{True} if the tenor basis swap spread is to be included when compounding is
performed on the short tenor Ibor index leg and \emph{False} if not. If not provided, this defaults to \emph{False}.
\item SubPeriodsCouponType [Optional]: This field can have the value \emph{Compounding} or \emph{Averaging} and it only
applies when the frequency of payments on the short tenor Ibor leg does not equal the short tenor Ibor index's tenor. If
\emph{Compounding} is specified, then the short tenor Ibor index is compounded and paid on the frequency specified in
the \lstinline!ShortPayTenor! node. If \emph{Averaging} is specified, then the short tenor Ibor index is averaged and
paid on the frequency specified in the \lstinline!ShortPayTenor! node. If not provided, this defaults to
\emph{Compounding}.
\end{itemize}

%- - - - - - - - - - - - - - - - - - - - - - - - - - - - - - - - - - - - - - - -
\subsubsection{Tenor Basis Two Swap Conventions}
%- - - - - - - - - - - - - - - - - - - - - - - - - - - - - - - - - - - - - - - -
A node with name \emph{TenorBasisTwoSwap} is used to store conventions for tenor basis swap quotes where the quote is
the spread between the fair fixed rate on two swaps against Ibor indices of different tenors. We call the swap against
the Ibor index of longer tenor the long swap and the remaining swap the short swap. The structure of the tenor basis two
swap conventions node is shown in Listing \ref{lst:tenor_basis_two_conventions}.

\begin{listing}[H]
%\hrule\medskip
\begin{minted}[fontsize=\footnotesize]{xml}
<TenorBasisTwoSwap>
  <Id> </Id>
  <Calendar> </Calendar>
  <LongFixedFrequency> </LongFixedFrequency>
  <LongFixedConvention> </LongFixedConvention>
  <LongFixedDayCounter> </LongFixedDayCounter>
  <LongIndex> </LongIndex>
  <ShortFixedFrequency> </ShortFixedFrequency>
  <ShortFixedConvention> </ShortFixedConvention>
  <ShortFixedDayCounter> </ShortFixedDayCounter>
  <ShortIndex> </ShortIndex>
  <LongMinusShort> </LongMinusShort>
</TenorBasisTwoSwap>
\end{minted}
\caption{Tenor basis two swap conventions}
\label{lst:tenor_basis_two_conventions}
\end{listing}

The meanings of the various elements in this node are as follows:
\begin{itemize}
\item Calendar: The business day calendar on both swaps.
\item LongFixedFrequency: The frequency of payments on the fixed leg of the long swap.
\item LongFixedConvention: The roll convention on the fixed leg of the long swap.
\item LongFixedDayCounter: The day count basis on the fixed leg of the long swap.
\item LongIndex: The Ibor index on the floating leg of the long swap.
\item ShortFixedFrequency: The frequency of payments on the fixed leg of the short swap.
\item ShortFixedConvention: The roll convention on the fixed leg of the short swap.
\item ShortFixedDayCounter: The day count basis on the fixed leg of the short swap.
\item ShortIndex: The Ibor index on the floating leg of the short swap.
\item LongMinusShort [Optional]: \emph{True} if the basis swap spread is to be interpreted as the fair rate on the long
swap minus the fair rate on the short swap and \emph{False} if the basis swap spread is to be interpreted as the fair
rate on the short swap minus the fair rate on the long swap. If not provided, it defaults to \emph{True}.
\end{itemize}

%- - - - - - - - - - - - - - - - - - - - - - - - - - - - - - - - - - - - - - - -
\subsubsection{FX Conventions}\label{sss:fx_convention}
%- - - - - - - - - - - - - - - - - - - - - - - - - - - - - - - - - - - - - - - -
A node with name \emph{FX} is used to store conventions for FX spot and forward quotes for a given currency pair. The
structure of this node is shown in Listing \ref{lst:fx_conventions}.

\begin{listing}[H]
%\hrule\medskip
\begin{minted}[fontsize=\footnotesize]{xml}
<FX>
  <Id> </Id>
  <SpotDays> </SpotDays>
  <SourceCurrency> </SourceCurrency>
  <TargetCurrency> </TargetCurrency>
  <PointsFactor> </PointsFactor>
  <AdvanceCalendar> </AdvanceCalendar>
  <SpotRelative> </SpotRelative>
</FX>
\end{minted}
\caption{FX conventions}
\label{lst:fx_conventions}
\end{listing}


The meanings of the various elements in this node are as follows:
\begin{itemize}
\item SpotDays: The number of business days to spot for the currency pair.
\item SourceCurrency: The source currency of the currency pair. The FX quote is assumed to give the number of units of
target currency per unit of source currency.
\item TargetCurrency: The target currency of the currency pair.
\item PointsFactor: The number by which a points quote for the currency pair should be divided before adding it to the
spot quote to obtain the forward rate.
\item AdvanceCalendar [Optional]: The business day calendar(s) used for advancing dates for both spot and forwards. If
not provided, it defaults to a calendar with no holidays.
\item SpotRelative [Optional]: \emph{True} if the forward tenor is to be interpreted as being relative to the spot date.
\emph{False} if the forward tenor is to be interpreted as being relative to the valuation date. If not provided, it
defaults to \emph{True}.
\end{itemize}

%- - - - - - - - - - - - - - - - - - - - - - - - - - - - - - - - - - - - - - - -
\subsubsection{Cross Currency Basis Swap Conventions}
%- - - - - - - - - - - - - - - - - - - - - - - - - - - - - - - - - - - - - - - -
A node with name \emph{CrossCurrencyBasis} is used to store conventions for cross currency basis swap quotes. The
structure of this node is shown in Listing \ref{lst:xccy_basis_conventions}.

\begin{listing}[H]
%\hrule\medskip
\begin{minted}[fontsize=\footnotesize]{xml}
<CrossCurrencyBasis>
  <Id> </Id>
  <SettlementDays> </SettlementDays>
  <SettlementCalendar> </SettlementCalendar>
  <RollConvention> </RollConvention>
  <FlatIndex> </FlatIndex>
  <SpreadIndex> </SpreadIndex>
  <EOM> </EOM>
  <IsResettable> </IsResettable>
  <FlatIndexIsResettable> </FlatIndexIsResettable>>
  <PaymentLag> </PaymentLag>
  <FlatPaymentLag> </FlatPaymentLag>
  <!-- for OIS only -->
  <IncludeSpread> </IncludeSpread>
  <Lookback> </Lookback>
  <FixingDays> </FixingDays>
  <RateCutoff> </RateCutoff>
  <IsAveraged> </IsAveraged>
  <FlatIncludeSpread> </FlatIncludeSpread>
  <FlatLookback> </FlatLookback>
  <FlatFixingDays> </FlatFixingDays>
  <FlatRateCutoff> </FlatRateCutoff>
  <FlatIsAveraged> </FlatIsAveraged>
</CrossCurrencyBasis>
\end{minted}
\caption{Cross currency basis swap conventions}
\label{lst:xccy_basis_conventions}
\end{listing}


The meanings of the various elements in this node are as follows:
\begin{itemize}
\item SettlementDays: The number of business days to the start of the cross currency basis swap.
\item SettlementCalendar: The business day calendar(s) for both legs and to arrive at the settlement date using the
SettlementDays above.
\item RollConvention: The roll convention for both legs.
\item FlatIndex: The name of the index on the leg that does not have the cross currency basis spread.
\item SpreadIndex: The name of the index on the leg that has the cross currency basis spread.
\item EOM [Optional]: \emph{True} if the end of month convention is to be used when generating the schedule on both legs, and \emph{False} if not. If not provided, it defaults to \emph{False}.
\item IsResettable [Optional]: \emph{True} if the swap is mark-to-market resetting, and \emph{False} otherwise. If not provided, it defaults to \emph{False}.
\item FlatIndexIsResettable [Optional]: \emph{True} if it is the notional on the leg paying the flat index that resets, and \emph{False} otherwise. If not provided, it defaults to \emph{True}.
\item FlatTenor [Optional]: the flat leg period length (typical value is 3M), defaults to index tenor except for ON indices for which it defaults to 3M
\item SpreadTenor [Optional]: the spread leg period length (typical value is 3M), defaults to index tenor except for ON indices for which it defaults to 3M
\item SpreadPaymentLag [Optional]: the payment lag for the spread leg, allowable values are 0, 1, 2, ..., defaults to 0 if not given
\item FlatPaymentLag [Optional]: the payment lag for the flat leg, allowable values are 0, 1, 2, ..., defaults to 0 if nove given
\item SpreadIncludeSpread [Optional]: Only relevant if spread leg is OIS, allowable values are true, false, defaults to false if not given
\item SpreadLookback [Optional]: Only relevant if spread leg is OIS, allowable values are 0D, 1D, ..., defaults to 0D if not given
\item SpreadFixingDays [Optional]: Only relevant if spread leg is OIS, allowable values are 0, 1, 2, ..., defaults to 0 if not given
\item SpreadRateCutoff [Optional]: Only relevant if spread leg is OIS, allowable values are 0, 1, 2, ..., defaults to 0 if not given
\item SpreadIsAveraged [Optional]: Only relevant if spread leg is OIS, allowable values are true, false, defaults to false if not given
\item FlatIncludeSpread [Optional]: Only relevant if spread leg is OIS, allowable values are true, false, defaults to false if not given
\item FlatLookback [Optional]: Only relevant if spread leg is OIS, allowable values are 0D, 1D, ..., defaults to 0D if not given
\item FlatFixingDays [Optional]: Only relevant if spread leg is OIS, allowable values are 0, 1, 2, ..., defaults to 0 if not given
\item FlatRateCutoff [Optional]: Only relevant if spread leg is OIS, allowable values are 0, 1, 2, ..., defaults to 0 if not given
\item FlatIsAveraged [Optional]: Only relevant if spread leg is OIS, allowable values are true, false, defaults to false if not given
\end{itemize}

\subsubsection{Inflation Swap Conventions}
A node with name \lstinline!InflationSwap! is used to store conventions for zero or year on year inflation swap quotes. The structure of this node is shown in Listing \ref{lst:inflation_conventions}

\begin{listing}[H]
%\hrule\medskip
\begin{minted}[fontsize=\footnotesize]{xml}
<InflationSwap>
  <Id>EUHICPXT_INFLATIONSWAP</Id>
  <FixCalendar>TARGET</FixCalendar>
  <FixConvention>MF</FixConvention>
  <DayCounter>30/360</DayCounter>
  <Index>EUHICPXT</Index>
  <Interpolated>false</Interpolated>
  <ObservationLag>3M</ObservationLag>
  <AdjustInflationObservationDates>false</AdjustInflationObservationDates>
  <InflationCalendar>TARGET</InflationCalendar>
  <InflationConvention>MF</InflationConvention>
</InflationSwap>
\end{minted}
\caption{Inflation swap conventions}
\label{lst:inflation_conventions}
\end{listing}

The meaning of the elements is as follows:

\begin{itemize}
\item \lstinline!FixCalendar!: The calendar for the fixed rate leg of the swap.
\item \lstinline!FixConvention!: The rolling convention for the fixed rate leg of the swap.
\item \lstinline!DayCounter!: The payoff or coupon day counter, applied to both legs.
\item \lstinline!Index!: The underlying inflation index.
\item \lstinline!Interpolated!: Flag indicating interpolation of the index in the swap's payoff calculation.
\item \lstinline!ObservationLag!: The index observation lag to be applied.
\item \lstinline!AdjustInflationObservationDates!: Flag indicating whether index observation dates should be adjusted or not.
\item \lstinline!InflationCalendar!: The calendar for the inflation leg of the swap.
\item \lstinline!InflationConvention!: The rolling convention for the inflation leg of the swap.

\item \lstinline!PublicationRoll!:
This is an optional node taking the values \lstinline!None!, \lstinline!OnPublicationDate! or \lstinline!AfterPublicationDate!. If omitted, the value \lstinline!None! is used. Currently, our only known use case for a value other than \lstinline!None! is for Australian zero coupon inflation indexed swaps (ZCIIS). Here, the index is published quarterly on the last Wednesday of the month following the end of the reference quarter. The start date and maturity date of the market quoted ZCIIS roll to the next quarterly date after the publication date of the index. For example, the AU CPI value for Q3 2020, i.e.\ 1 Jul 2020 to 30 Sep 2020 was released on 28 Oct 2020. On 27 Oct 2020, before the index publication date, the market 5Y ZCIIS would start on 15 Sep 2020 and end on 15 Sep 2025 and reference the Q2 inflation index value. On 29 Oct 2020, after the index publication date, the market 5Y ZCIIS would start on 15 Dec 2020 and end on 15 Dec 2025 and reference the Q3 inflation index value. On the release date, i.e. 28 Oct 2020, the market ZCIIS that is set up is determined by whether the \lstinline!PublicationRoll! value is \lstinline!OnPublicationDate! or \lstinline!AfterPublicationDate!. If it is set to \lstinline!OnPublicationDate!, the swap rolls on this date and hence the market 5Y ZCIIS would start on 15 Dec 2020 and end on 15 Dec 2025 and reference the Q3 inflation index value. If it is set to \lstinline!AfterPublicationDate!, the swap does not roll on the publication date and instead rolls on the next day, and hence the market 5Y ZCIIS would start on 15 Sep 2020 and end on 15 Sep 2025 and reference the Q2 inflation index value. The publication schedule for the index must be provided in the \lstinline!PublicationSchedule! node if \lstinline!PublicationRoll! is not \lstinline!None!. An example of the AU CPI conventions set up is given in Listing \ref{lst:aucpi_inflation_conventions}.

\item \lstinline!PublicationSchedule!:
This is an optional node and is not used if \lstinline!PublicationRoll! is \lstinline!None!. If \lstinline!PublicationRoll! is not \lstinline!None!, it must be provided and gives the publication dates for the inflation index. The node fields are the same fields that are described in the Section \ref{ss:schedule_data}, i.e.\ they are \lstinline!ScheduleData! elements. An example of the AU CPI conventions set up is given in Listing \ref{lst:aucpi_inflation_conventions}. The \lstinline!PublicationSchedule! must cover the dates on which you intend to perform valuations, i.e. the first publication schedule date must be less than the smallest valuation date that you intend to use and the last publication schedule date must be greater than the largest valuation date that you intend to use.

\end{itemize}

\begin{listing}[H]
\begin{minted}[fontsize=\footnotesize]{xml}
<InflationSwap>
  <Id>AUCPI_INFLATIONSWAP</Id>
  <FixCalendar>AUD</FixCalendar>
  <FixConvention>F</FixConvention>
  <DayCounter>30/360</DayCounter>
  <Index>AUCPI</Index>
  <Interpolated>false</Interpolated>
  <ObservationLag>3M</ObservationLag>
  <AdjustInflationObservationDates>false</AdjustInflationObservationDates>
  <InflationCalendar>AUD</InflationCalendar>
  <InflationConvention>F</InflationConvention>
  <PublicationRoll>AfterPublicationDate</PublicationRoll>
  <PublicationSchedule>
    <Rules>
      <StartDate>2001-01-24</StartDate>
      <EndDate>2030-01-30</EndDate>
      <Tenor>3M</Tenor>
      <Calendar>AUD</Calendar>
      <Convention>Preceding</Convention>
      <TermConvention>Unadjusted</TermConvention>
      <Rule>LastWednesday</Rule>
    </Rules>
  </PublicationSchedule>
</InflationSwap>
\end{minted}
\caption{AU CPI inflation swap conventions}
\label{lst:aucpi_inflation_conventions}
\end{listing}

%- - - - - - - - - - - - - - - - - - - - - - - - - - - - - - - - - - - - - - - -
\subsubsection{CMS Spread Option Conventions}
%- - - - - - - - - - - - - - - - - - - - - - - - - - - - - - - - - - - - - - - -

A node with name \emph{CmsSpreadOption} is used to store the conventions.

\begin{listing}[H]
%\hrule\medskip
\begin{minted}[fontsize=\footnotesize]{xml}
  <CmsSpreadOption>
    <Id>EUR-CMS-10Y-2Y-CONVENTION</Id>
    <ForwardStart>0M</ForwardStart>
    <SpotDays>2D</SpotDays>
    <SwapTenor>3M</SwapTenor>
    <FixingDays>2</FixingDays>
    <Calendar>TARGET</Calendar>
    <DayCounter>A360</DayCounter>
    <RollConvention>MF</RollConvention>
  </CmsSpreadOption>
\end{minted}
\caption{Inflation swap conventions}
\label{lst:cms_spread_option_conventions}
\end{listing}

The meaning of the elements is as follows:

\begin{itemize}
\item ForwardStart: The calendar for the fixed rate leg of the swap.
\item SpotDays: The number of business days to spot for the CMS Spread Index.
\item SwapTenor: The frequency of payments on the CMS Spread leg.
\item FixingDays: The number of fixing days.
\item Calendar: The calendar for the CMS Spread leg.
\item DayCounter: The day counter for the CMS Spread leg.
\item RollConvention: The rolling convention for the CMS Spread Leg.
\end{itemize}

%- - - - - - - - - - - - - - - - - - - - - - - - - - - - - - - - - - - - - - - -
\subsubsection{Ibor Index Conventions}
%- - - - - - - - - - - - - - - - - - - - - - - - - - - - - - - - - - - - - - - -

A node with name \emph{IborIndex} is used to store conventions for Ibor indices. This can be used to define new Ibor
indices without the need of adding them to the C++ code, or also to override the conventions of existing Ibor indices.

\begin{listing}[H]
%\hrule\medskip
\begin{minted}[fontsize=\footnotesize]{xml}
  <IborIndex>
    <Id>EUR-EURIBOR_ACT365-3M</Id>
    <FixingCalendar>TARGET</FixingCalendar>
    <DayCounter>A365F</DayCounter>
    <SettlementDays>2</SettlementDays>
    <BusinessDayConvention>MF</BusinessDayConvention>
    <EndOfMonth>true</EndOfMonth>
  </IborIndex>
\end{minted}
\caption{Ibor index convention}
\label{lst:ibor_index_conventions}
\end{listing}

The meaning of the elements is as follows:

\begin{itemize}
\item Id: The index name. This must be of the form ``CCY-NAME-TENOR'' with a currency ``CCY'', an index name ``NAME''
  and a string ``TENOR'' representing a period. The name should not be ``GENERIC'', since this is reserved.
\item FixingCalendar: The fixing calendar of the index.
\item DayCounter: The day count convention used by the index.
\item SettlementDays: The settlement days for the index. This must be a non-negative whole number.
\item BusinessDayConvention: The business day convention used by the index.
\item EndOfMonth: A flag indicating whether the index employs the end of month convention.
\end{itemize}

Notice that if another convention depends on an Ibor index convention (because it contains the Ibor index name defined
in the latter convention), the Ibor index convention must appear before the convention that depends on it in the
convention input file.

Also notice that customised indices can not be used in cap / floor volatility surface configurations.

%- - - - - - - - - - - - - - - - - - - - - - - - - - - - - - - - - - - - - - - -
\subsubsection{Overnight Index Conventions}
%- - - - - - - - - - - - - - - - - - - - - - - - - - - - - - - - - - - - - - - -

A node with name \emph{OvernightIndex} is used to store conventions for Overnight indices. This can be used to define
new Overnight indices without the need of adding them to the C++ code, or also to override the conventions of existing
Overnight indices.

\begin{listing}[H]
%\hrule\medskip
\begin{minted}[fontsize=\footnotesize]{xml}
  <OvernightIndex>
    <Id>EUR-ESTER</Id>
    <FixingCalendar>TARGET</FixingCalendar>
    <DayCounter>A360</DayCounter>
    <SettlementDays>0</SettlementDays>
  </OvernightIndex>
\end{minted}
\caption{Overnight index convention}
\label{lst:overnight_index_conventions}
\end{listing}

The meaning of the elements is as follows:

\begin{itemize}
\item Id: The index name. This must be of the form ``CCY-NAME'' with a currency ``CCY'' and an index name ``NAME''. The
  name should not be ``GENERIC'', since this is reserved.
\item FixingCalendar: The fixing calendar of the index.
\item DayCounter: The day count convention used by the index.
\item SettlementDays: The settlement days for the index. This must be a non-negative whole number.
\end{itemize}

Notice that if another convention depends on an Overnight index convention (because it contains the Overnight index name
defined in the latter convention), the Overnight index convention must appear before the convention that depends on it
in the convention input file.

Also notice that customised indices can not be used in cap / floor volatility surface configurations.

\subsubsection{Inflation Index Conventions}
A node with the name \lstinline!ZeroInflationIndex! is used to store data for the creation of a new inflation index. This avoids having to add the index definition to the C++ code and recompile. Note that the \lstinline!ZeroInflationIndex! node should be placed before its use in any other convention, e.g.\ in an \lstinline!InflationSwap! convention, to avoid an error due to the new index itself not being created. If the \lstinline!Id! node matches an existing inflation index, the newly created index will take precedence and its defintion will be used in the code for the given \lstinline!Id!.

\begin{listing}[H]
\begin{minted}[fontsize=\footnotesize]{xml}
<ZeroInflationIndex>
  <Id>...</Id>
  <RegionName>...</RegionName>
  <RegionCode>...</RegionCode>
  <Revised>...</Revised>
  <Frequency>...</Frequency>
  <AvailabilityLag>...</AvailabilityLag>
  <Currency>...</Currency>
</ZeroInflationIndex>
\end{minted}
\caption{\emph{ZeroInflationIndex} node}
\label{lst:zero_inflation_index_conventions}
\end{listing}

The meaning of each element is as follows:
\begin{itemize}
\item \lstinline!Id!: The new inflation index name.
\item \lstinline!RegionName!: The name of the region with which the inflation index is associated.
\item \lstinline!RegionCode!: A code for the region with which the inflation index is associated.
\item \lstinline!Revised!: A boolean flag indicating whether the index is a revised index or not. This is generally set to \lstinline!false! but is left as an option to align with the C++ \lstinline!InflationIndex! class definition.
\item \lstinline!Frequency!: A valid frequency indicating the publication frequency of the inflation index, generally \lstinline!Monthly!, \lstinline!Quarterly! or \lstinline!Annual!.
\item \lstinline!AvailabilityLag!: A valid period indicating the lag between the inflation index publication for a given period and the period itself. For example, if March's inflation index value is published in April, the \lstinline!AvailabilityLag! would be \lstinline!1M!.
\item \lstinline!Currency!: The ISO currency code of the currency associated with the inflation index, generally the currency of the region.
\end{itemize}

%- - - - - - - - - - - - - - - - - - - - - - - - - - - - - - - - - - - - - - - -
\subsubsection{Swap Index Conventions}
%- - - - - - - - - - - - - - - - - - - - - - - - - - - - - - - - - - - - - - - -

A node with name \emph{SwapIndex} is used to store conventions for Swap indices (also known as ``CMS'' indices).

\begin{listing}[H]
%\hrule\medskip
\begin{minted}[fontsize=\footnotesize]{xml}
  <SwapIndex>
    <Id>EUR-CMS-2Y</Id>
    <Conventions>EUR-EURIBOR-6M-SWAP</Conventions>
    <FixingCalendar>TARGET</FixingCalendar>
  </SwapIndex>
\end{minted}
\caption{Swap index convention}
\label{lst:swap_index_conventions}
\end{listing}

The meaning of the elements is as follows:

\begin{itemize}
\item Id: The index name. This must be of the form ``CCY-CMS-TENOR'' with a currency ``CCY'' and a string ``TENOR''
  representing a period. The index name can contain an optional tag ``CCY-CMS-TAG-TENOR'' which is an arbitrary label
  that allows to define more than one swap index per currency.
\item Conventions: A swap convention defining the index conventions.
\item FixingCalendar [Optional]: The fixing calendar for the swap index fixings publication. If not given, the fixed leg
  calendar from the swap conventions will be used as a fall back.
\end{itemize}

%- - - - - - - - - - - - - - - - - - - - - - - - - - - - - - - - - - - - - - - - 
\subsubsection{FX Option Conventions}\label{sss:fx_option_conv}
%- - - - - - - - - - - - - - - - - - - - - - - - - - - - - - - - - - - - - - - - 
A node with name \emph{FxOption} is used to store conventions for FX option quotes for a given currency pair. The 
structure of this node is shown in Listing \ref{lst:fx_option_conventions}. 
 
\begin{listing}[H] 
%\hrule\medskip 
\begin{minted}[fontsize=\footnotesize]{xml}
<FxOption>
  <Id>EUR-USD-FXOPTION</Id>
  <FXConventionID>EUR-USD-FX</FXConventionID>
  <AtmType>AtmDeltaNeutral</AtmType>
  <DeltaType>Spot</DeltaType>
  <SwitchTenor>2Y</SwitchTenor>
  <LongTermAtmType>AtmDeltaNeutral</LongTermAtmType>
  <LongTermDeltaType>Fwd</LongTermDeltaType>
  <RiskReversalInFavorOf>Call</RiskReversalInFavorOf>
  <ButterflyStyle>Broker</ButterflyStyle>
</FxOption>
\end{minted} 
\caption{FX option conventions} 
\label{lst:fx_option_conventions} 
\end{listing} 
 
 
The meanings of the various elements in this node are as follows: 
\begin{itemize}
\item FXConventionID: The FX convention for the currency pair (see \ref{sss:fx_convention}). Optional, if not given, the
  FX spot days default to $2$ and the advance calendar defaults to source ccy + target ccy default calendars.
\item AtmType: Convention of ATM option quote (Choices are {\em AtmNull, AtmSpot, AtmFwd, 
AtmDeltaNeutral, AtmVegaMax, AtmGammaMax, AtmPutCall50}). 
\item DeltaType: Convention of Delta option quote (Choices are {\em Spot, Fwd, PaSpot, 
    PaFwd}).
\item SwitchTenor [Optional]: If given, different ATM and Delta conventions will be used if the option tenor is greater
  or equal the switch tenor (``long term'' atm and delta type)
\item LongTermAtmType [Mandatory if and only if SwitchTenor is given]: ATM type to use for options with tenor > switch
  point, if SwitchTenor is given
\item LongTermDeltaType [Mandatory if and only if SwitchTenor is given]: Delta type to use for options with tenor >
  switch point, if SwitchTenor is given
\item RiskReversalInFavorOf [Optional]: Call (default), Put. Only relevant for BF, RR market data input.
\item ButterflyStyle [Optional]: Broker (default), Smile. Only relevant for BF, RR market data input.
\end{itemize} 

%- - - - - - - - - - - - - - - - - - - - - - - - - - - - - - - - - - - - - - - -
\subsubsection{Commodity Forward Conventions}
%- - - - - - - - - - - - - - - - - - - - - - - - - - - - - - - - - - - - - - - -
A node with name \lstinline!CommodityForward! is used to store conventions for commodity forward price quotes. The
structure of this node is shown in Listing \ref{lst:commodity_forward_conventions}.

\begin{listing}[H]
\begin{minted}[fontsize=\footnotesize]{xml}
<CommodityForward>
  <Id>...</Id>
  <SpotDays>...</SpotDays>
  <PointsFactor>...</PointsFactor>
  <AdvanceCalendar>...</AdvanceCalendar>
  <SpotRelative>...</SpotRelative>
  <BusinessDayConvention>...</BusinessDayConvention>
  <Outright>...</Outright>
</CommodityForward>
\end{minted}
\caption{Commodity forward conventions}
\label{lst:commodity_forward_conventions}
\end{listing}

The meanings of the various elements in this node are as follows:
\begin{itemize}
\item \lstinline!Id!: The identifier for the commodity forward convention. The identifier here should match the \lstinline!Name! that would be provided for the commodity in the trade XML as described in Table \ref{tab:commodity_data}.
\item \lstinline!SpotDays! [Optional]: The number of business days to spot for the commodity. Any non-negative integer is allowed here. If omitted, this takes a default value of 2.
\item \lstinline!PointsFactor! [Optional]: This is only used if \lstinline!Outright! is \lstinline!false!. Any positive real number is allowed here. When \lstinline!Outright! is \lstinline!false!, the commodity forward quotes are provided as points i.e. a number that should be added to the commodity spot to give the outright commodity forward rate. The \lstinline!PointsFactor! is the number by which the points quote should be divided before adding it to the spot quote to obtain the forward price. If omitted, this takes a default value of 1.
\item \lstinline!AdvanceCalendar! [Optional]: The business day calendar(s) used for advancing dates for both spot and forwards. The allowable values are given in Table \ref{tab:calendar}. If omitted, it defaults to \lstinline!NullCalendar! i.e. a calendar where all days are considered good business days.
\item \lstinline!SpotRelative! [Optional]: The allowable values are \lstinline!true! and \lstinline!false!. If \lstinline!true!, the forward tenor is interpreted as being relative to the spot date. If \lstinline!false!, the forward tenor is interpreted as being relative to the valuation date. If omitted, it defaults to \lstinline!True!.
\item \lstinline!BusinessDayConvention! [Optional]: The business day roll convention used to adjust dates when getting from the valuation date to the spot date and the forward maturity date. The allowable values are given in Table \ref{tab:allow_stand_data}. If omitted, it defaults to \lstinline!Following!.
\item \lstinline!Outright! [Optional]: The allowable values are \lstinline!true! and \lstinline!false!. If \lstinline!true!, the forward quotes are interpreted as outright forward prices. If \lstinline!false!, the forward quotes are interpreted as points i.e. as a number that must be added to the spot price to get the outright forward price. If omitted, it defaults to \lstinline!true!.
\end{itemize}

\subsubsection{Commodity Future Conventions}
\label{sec:commodity_future_conventions}
A node with name \lstinline!CommodityFuture! is used to store conventions for commodity future contracts and options on them. These conventions are used in commodity derivative trades and commodity curve construction to calculate contract expiry dates. The structure of this node is shown in Listing \ref{lst:commodity_future_conventions}.

\begin{listing}[h!]
\begin{minted}[fontsize=\footnotesize,breaklines]{xml}
<CommodityFuture>
  <Id>...</Id>
  <AnchorDay>
    ...
  </AnchorDay>
  <ContractFrequency>...</ContractFrequency>
  <Calendar>...</Calendar>
  <ExpiryCalendar>...</ExpiryCalendar>
  <ExpiryMonthLag>...</ExpiryMonthLag>
  <OneContractMonth>...</OneContractMonth>
  <OffsetDays>...</OffsetDays>
  <BusinessDayConvention>...</BusinessDayConvention>
  <AdjustBeforeOffset>...</AdjustBeforeOffset>
  <IsAveraging>...</IsAveraging>
  <OptionExpiryOffset>...</OptionExpiryOffset>
  <ProhibitedExpiries>
    <Dates>
      <Date forFuture="true" convention="Preceding" forOption="true" optionConvention="Preceding">...</Date>
        ...
    </Dates>
  </ProhibitedExpiries>
  <OptionExpiryMonthLag>...</OptionExpiryMonthLag>
  <OptionExpiryDay>...</OptionExpiryDay>
  <OptionContractFrequency>...</OptionContractFrequency>
  <OptionNthWeekday>
    <Nth>...</Nth>
    <Weekday>...</Weekday>
  </OptionNthWeekday>
  <OptionExpiryLastWeekdayOfMonth>...</OptionExpiryLastWeekdayOfMonth>
  <OptionExpiryWeeklyDayOfTheWeek>...</OptionExpiryWeeklyDayOfTheWeek>
  <OptionBusinessDayConvention>...</OptionBusinessDayConvention>
  <FutureContinuationMappings>
    <ContinuationMapping>
      <From>...</From>
      <To>...</To>
    </ContinuationMapping>
    ...
  </FutureContinuationMappings>
  <OptionContinuationMappings>
    <ContinuationMapping>
      <From>...</From>
      <To>...</To>
    </ContinuationMapping>
    ...
  </OptionContinuationMappings>
  <AveragingData>
    ...
  </AveragingData>
  <HoursPerDay>...</HoursPerDay>
</CommodityFuture>
\end{minted}
\caption{Commodity future conventions}
\label{lst:commodity_future_conventions}
\end{listing}

The meanings of the various elements in this node are as follows:
\begin{itemize}
\item \lstinline!Id!: The identifier for the commodity future convention. The identifier here should match the \lstinline!Name! that would be provided for the commodity in the trade XML as described in Table \ref{tab:commodity_data}.
\item \lstinline!AnchorDay! [Optional]: This node is not applicable for daily future contracts and hence is optional. It is necessary for future contracts with a monthly cycle or greater or if the option contracts cycle is monthly or greater.  This node is used to give a date in the future contract month to use as a base date for calculating the expiry date. It can contain a \lstinline!DayOfMonth! node, a \lstinline!CalendarDaysBefore! node or an \lstinline!NthWeekday! node:
    \begin{itemize}
    \item The \lstinline!DayOfMonth! This node can contain any integer in the range $1,\ldots,31$ indicating the day of the month. A value of 31 will guarantee that the last day in the month is used a base date.
    \item The \lstinline!CalendarDaysBefore! This node can contain any non-negative integer. The contract expiry date is this number of calendar days before the first calendar day of the contract month.
    \item The \lstinline!NthWeekday! This node has the elements shown in Listing \ref{lst:nth_weekday_node}. This node is used to indicate a date in a given month in the form of the n-th named weekday of that month e.g. 3rd Wednesday. The allowable values for \lstinline!Nth! are ${1,2,3,4}$. The \lstinline!Weekday! node takes a weekday in the form of the first three characters of the weekday with the first character capitalised.
    \item The \lstinline!LastWeekday! [Optional]: This node is used to indicate a date in a given month in the form of the last named weekday of that month e.g. last Wednesday. The node takes a weekday in the form of the first three characters of the weekday with the first character capitalised.
    \item The \lstinline!BusinessDaysAfter! This node can contain any integer. If the number is positive the contract expiry is the n-th business day of the contract month. If the number is negative the contract expiry date is this number of business days before the first calendar day of the contract month.
    \item The \lstinline!WeeklyDayOfTheWeek! [Optional]: This node is used to indicate a date in a given week in the form of the named weekday, e.g. Wednesday. This node is mandatory for weekly contract frequencies and is not allowed with any other frequency.  The node takes a weekday in the form of the first three characters of the weekday with the first character capitalised.
    \end{itemize}
\item \lstinline!ContractFrequency!: This node indicates the frequency of the commodity future contracts. The value here is usually \lstinline!Monthly! or \lstinline!Quarterly!, but allowed values are \lstinline!Daily!, \lstinline!Weekly!, \lstinline!Monthly!, \lstinline!Quaterly! and \lstinline!Annual!.
\item \lstinline!Calendar!: The business day trading calendar(s) applicable for the commodity future contract.
\item \lstinline!ExpiryCalendar! [Optional]: The business day expiry calendar(s) applicable for the commodity future contract. This calendar is used when deriving expiry dates. If omitted, this defaults to the trading day calendar specified in the \lstinline!Calendar! node.
\item \lstinline!ExpiryMonthLag! [Optional]: The allowable values are any integer. This value indicates the number of months from the month containing the expiry date to the contract month. If 0, the commodity future contract expiry date is in the contract month. If the value of \lstinline!ExpiryMonthLag! is $n > 0$, the commodity future contract expires in the $n$-th month prior to the contract month. If the value of \lstinline!ExpiryMonthLag! is $n < 0$, the commodity future contract expires in the $n$-th month after the contract month. The value of \lstinline!ExpiryMonthLag! is generally 0, 1 or 2. For example, \lstinline!NYMEX:CL! has an \lstinline!ExpiryMonthLag! of 1 and \lstinline!ICE:B! has an \lstinline!ExpiryMonthLag! of 2. If omitted, it defaults to 0.
\item \lstinline!OneContractMonth! [Optional]: This node takes a calendar month in the form of the first three characters of the month with the first character capitalised. The month provided should be an arbitrary valid future contract month. It is used in cases where the \lstinline!ContractFrequency! is not \lstinline!Monthly! in order to determine the valid contract months. If omitted, it defaults to January.
\item \lstinline!OffsetDays! [Optional]: The number of business days that the expiry date is before the base date where the base date is implied by the \lstinline!AnchorDay! node above. Any non-negative integer is allowed here. If omitted, this takes a default value of zero.
\item \lstinline!BusinessDayConvention! [Optional]: The business day roll convention used to adjust the expiry date. The allowable values are given in Table \ref{tab:allow_stand_data}. If omitted, it defaults to \lstinline!Preceding!.
\item \lstinline!AdjustBeforeOffset! [Optional]: The allowable values are \lstinline!true! and \lstinline!false!. If \lstinline!true!, if the base date implied by the \lstinline!AnchorDay! node above is not a good business day according to the calendar provided in the \lstinline!Calendar! node, this date is adjusted before the offset specified in the \lstinline!OffsetDays! is applied. If \lstinline!false!, this adjustment does not happen. If omitted, it defaults to \lstinline!true!. 
\item \lstinline!IsAveraging! [Optional]: The allowable values are \lstinline!true! and \lstinline!false!. This node indicates if the future contract is based on the average commodity price of the contract period. If omitted, it defaults to \lstinline!false!.
\item \lstinline!OptionExpiryOffset! [Optional]: The number of business days that the option expiry date is before the future expiry date. Any non-negative integer is allowed here. If omitted, this takes a default value of zero and the expiry date of an option on the future contract is assumed to equal the expiry date of the future contract.
\item \lstinline!ProhibitedExpiries! [Optional]: This node can be used to specify explicit dates which are not allowed as future contract expiry dates or as option expiry dates. A useful example of this is the ICE Brent contract which has the following constraint on expiry dates: \emph{If the day on which trading is due to cease would be either: (i) the Business Day preceding Christmas Day, or (ii) the Business Day preceding New Year’s Day, then trading shall cease on the next preceding Business Day}. Each \lstinline!Date! node can take optional attributes. The default values of these attributes is shown in Listing \ref{lst:commodity_future_conventions}. The \lstinline!convention! attribute accepts a valid business day convention in the list \lstinline!Preceding!, \lstinline!ModifiedPreceding!, \lstinline!Following! and \lstinline!ModifiedFollowing!. This \lstinline!convention! indicates how the future expiry date should be adjusted if it lands on the prohibited expiry \lstinline!Date!. If ommitted, the default is \lstinline!Preceding!. Both \lstinline!Preceding! and \lstinline!ModifiedPreceding! indicate that the next available business day before the date is tested. \lstinline!Following! and \lstinline!ModifiedFollowing! indicate that the next available business day after the date is tested. The \lstinline!optionConvention! attribute allows the same values and behaves in the same way to determine how the option expiry date should be adjusted if it lands on the prohibited expiry \lstinline!Date!. The \lstinline!forFuture! and \lstinline!forOption! boolean attributes enable the prohibited expiry to apply only for the future expiry date or the option expiry date respectively by setting the value to \lstinline!false!.
\item \lstinline!OptionExpiryMonthLag! [Optional]: The allowable values are any integer. This value indicates the number of months from the month containing the option expiry date to the month containing the expiry date. If 0, the commodity future option contract expiry date is anchored in the same month as the commodity future contract expiry date. If the value of \lstinline!OptionExpiryMonthLag! is $n > 0$, the commodity option future contract expires in the $n$-th month prior to the commodity future contract expiry month. If the value of \lstinline!OptionExpiryMonthLag! is $n < 0$, the commodity option future contract expires in the $n$-th month after the the commodity future contract expiry month. The value of \lstinline!OptionExpiryMonthLag! should be equal to \lstinline!ExpiryMonthLag! when \lstinline!OptionExpiryOffset! is used. The \lstinline!OptionExpiryMonthLag! is rarely used. An example is the Crude Palm Oil contract \lstinline!XKLS:FCPO! where the future contract expiry is in the delivery month and the option expiry is in the month that is 2 months prior to this. In this case, \lstinline!OptionExpiryMonthLag! is 2. If omitted, \lstinline!OptionExpiryMonthLag! defaults to 0.
\item \lstinline!OptionExpiryDay! [Optional]: This node can contain any integer in the range $1,\ldots,31$ indicating the day of the month on which an option expiry date is anchored. A value of 31 will guarantee that the last day in the month is used a base date. If omitted, this is not used. Setting this field takes precedence over \lstinline!OptionExpiryOffset!.\item \lstinline!OptionBusinessDayConvention! [Optional]: The business day convention used to adjust the option expiry date to a good business day if \lstinline!OptionExpiryDay! is used.
\item \lstinline!OptionContractFrequency! [Optional]: This node indicates the frequency of the commodity future options if it differs from the frequency of the underlying future contract. The value here is usually \lstinline!Monthly!
\item \lstinline!OptionNthWeekday! [Optional]: This node has the elements shown in Listing \ref{lst:nth_weekday_node}. This node is used to indicate a date in a given month in the form of the n-th named weekday of that month e.g. 3rd Wednesday. The allowable values for \lstinline!Nth! are ${1,2,3,4}$. The \lstinline!Weekday! node takes a weekday in the form of the first three characters of the weekday with the first character capitalised.
\item \lstinline!OptionBusinessDayConvention! [Optional]: The business day convention used to adjust the option expiry date to a good business day if \lstinline!OptionExpiryDay! is used.
\item \lstinline!OptionExpiryLastWeekdayOfMonth! [Optional]: This node is used to indicate a date in a given month in the form of the last named weekday of that month e.g. last Wednesday. The node takes a weekday in the form of the first three characters of the weekday with the first character capitalised.
\item \lstinline!OptionExpiryWeeklyDayOfTheWeek! [Optional]: This node is used to indicate a date in a given week in the form of the named weekday, e.g. Wednesday. The node takes a weekday in the form of the first three characters of the weekday with the first character capitalised. This node is mandatory for weekly expiring options. The node is not allowed to use with any other option contract frequency.
\item \lstinline!FutureContinuationMappings! [Optional]: When building future curves, we may use market data that has a continuation expiry, i.e. \lstinline!c1!, \lstinline!c2!, etc. , as opposed to an explicit expiry date or tenor. In some cases, the continuation expiries coming from the market data provider may skip serial months and therefore we use the mapping here to map from the market data provider index to the relevant serial month.
\item \lstinline!OptionContinuationMappings! [Optional]: When building option volatility structures, we may use market data that has a continuation expiry, i.e. \lstinline!c1!, \lstinline!c2!, etc. , as opposed to an explicit expiry date or tenor. In some cases, the continuation expiries coming from the market data provider may skip serial months and therefore we use the mapping here to map from the market data provider index to the relevant serial month. For example, for the Crude Palm Oil contract \lstinline!XKLS:FCPO!, the option expiry months are serial up to the 9th month and then alternate months. So, we would add a mapping from 10 to 11, 11 to 13 and so on so that the correct option expiry is arrived at when reading the market data quotes and constructing the option volatility structure.
\item \lstinline!AveragingData! [Optional]: This node is needed for future contracts that are used in a piecewise commodity curve \lstinline!PriceSegment! and whose underlying is the average of other future prices or spot prices over a given period. An example is the ICE PMI power contract with contract specifications outlined \href{https://www.theice.com/products/6590369/PJM-Western-Hub-Real-Time-Peak-1-MW-Fixed-Price-Future}{here}. It is described in detail below.
\item \lstinline!HoursPerDay! [Optional]: For power derivatives, quantities are sometimes given as a quantity per hour. To deduce the quantity for the day which is multiplied by that day's future price, one needs to know the number of hours in the day associated with the future price. For example ICE PDQ is the daily PJM Western Hub Real Time Peak future contract. The price each day for this contract is the average of the locational marginal prices (LMPs) for all hours ending 08:00 to 23:00 Eastern Pacific Time. In other words, there are 16 hours in the day that feed in to the average yielding this settlement price. For this contract, \lstinline!HoursPerDay! would be \lstinline!16!. This field is only needed if a trade XML references this commodity contract, has \lstinline!CommodityQuantityFrequency! set to \lstinline!PerHour! and has no \lstinline!HoursPerDay! value set directly in the XML.
\end{itemize}

\begin{listing}[h!]
\begin{minted}[fontsize=\footnotesize]{xml}
<NthWeekday>
  <Nth>...</Nth>
  <Weekday>...</Weekday>
</NthWeekday>
\end{minted}
\caption{\textnormal{\lstinline!NthWeekday!} node outline}
\label{lst:nth_weekday_node}
\end{listing}

An example \lstinline!CommodityFuture! node for the NYMEX WTI future contract, specified \href{https://www.cmegroup.com/trading/energy/crude-oil/light-sweet-crude_contract_specifications.html}{here}, is provided in Listing \ref{lst:ex_wti_comm_future_convention}.

\begin{listing}[h!]
\begin{minted}[fontsize=\footnotesize]{xml}
<CommodityFuture>
  <Id>NYMEX:CL</Id>
  <AnchorDay>
    <DayOfMonth>25</DayOfMonth>
  </AnchorDay>
  <ContractFrequency>Monthly</ContractFrequency>
  <Calendar>US-NYSE</Calendar>
  <ExpiryMonthLag>1</ExpiryMonthLag>
  <OffsetDays>3</OffsetDays>
  <BusinessDayConvention>Preceding</BusinessDayConvention>
  <IsAveraging>false</IsAveraging>
</CommodityFuture>
\end{minted}
\caption{NYMEX WTI \textnormal{\lstinline!CommodityFuture!} node}
\label{lst:ex_wti_comm_future_convention}
\end{listing}

The \lstinline!AveragingData! node referenced above has the structure shown in Listing \ref{lst:ave_data_comm_future_convention}. The meaning of each of the fields is as follows:

\begin{itemize}
\item \lstinline!CommodityName!: The name of the commodity being averaged.
\item \lstinline!Conventions!: The identifier for the conventions associated with the commodity being averaged.
\item \lstinline!Period!: This indicates the averaging period relative to the future expiry date. The allowable values are:
    \begin{itemize}
    \item \lstinline!PreviousMonth!: The calendar month prior to the month in which the (top level) future contract's expiry date falls is used as the averaging period.
    \item \lstinline!ExpiryToExpiry!: Given a (top level) future contract's expiry date, the averaging period is from and excluding the previous expiry date to and including the expiry date.
    \end{itemize}
\item \lstinline!PricingCalendar!: The pricing calendar(s) used to determine the pricing dates in the averaging period.
\item \lstinline!UseBusinessDays! [Optional]: A boolean flag that defaults to \lstinline!true! if omitted. When set to \lstinline!true!, the pricing dates in the averaging period are the set of \lstinline!PricingCalendar! good business days. When set to \lstinline!false!, the pricing dates in the averaging period are the complement of the set of \lstinline!PricingCalendar! good business days. This may be useful in certain situations. For example, the contract ICE PW2 with specifications \href{https://www.theice.com/products/71090520/PJM-Western-Hub-Real-Time-Peak-2x16-Fixed-Price-Future}{here} averages the PJM Western Hub locational marginal prices over each day in the averaging period that is a Saturday, Sunday or NERC holiday. So, in this case, \lstinline!UseBusinessDays! would be \lstinline!false! and \lstinline!PricingCalendar! would be \lstinline!US-NERC!.
\item \lstinline!DeliveryRollDays! [Optional]: This node allows any non-negative integer value. When averaging a commodity future contract price over the averaging period, the averaging period may include an underlying future contract expiry date. This node's value indicates when we should begin using the next future contract's price in the averaging. If the value is zero, we should include the future contract prices up to and including the contract expiry. If the value is one, we should include the contract prices up to and including the day that is one business day before the contract expiry and then switch to using the next future contract's price thereafter. Similarly for other non-negative integer values. If this node is omitted, it is set to zero.
\item \lstinline!FutureMonthOffset! [Optional]: This node allows any non-negative integer value. If this node is omitted, it is set to zero. This node indicates which future contract is being referenced on each \textit{Pricing Date} in the averaging period by acting as an offset from the next available expiry date. If \lstinline!FutureMonthOffset! is zero, the settlement price of the next available monthly contract that has not expired with respect to the \textit{Pricing Date} is used as the price on that \textit{Pricing Date}. If \lstinline!FutureMonthOffset! is one, the settlement price of the second available monthly contract that has not expired with respect to the \textit{Pricing Date} is used as the price on that \textit{Pricing Date}. Similarly for other positive values of \lstinline!FutureMonthOffset!.
\item \lstinline!DailyExpiryOffset! [Optional]: This node allows any non-negative integer value. It should only be used where the \lstinline!CommodityName! being averaged has a daily contract frequency. If this node is omitted, it is set to zero. This node indicates which future contract is being referenced on each \textit{Pricing Date} in the averaging period by acting as a business day offset, using the \lstinline!CommodityName!'s expiry calendar, from the \textit{Pricing Date}. It is useful in the base metals market where the future contract being averaged on each \textit{Pricing Date} is the cash contract on that \textit{Pricing Date} i.e.\ the contract with expiry date two business days after the \textit{Pricing Date}.
\end{itemize}

\begin{listing}[h!]
\begin{minted}[fontsize=\footnotesize]{xml}
<AveragingData>
  <CommodityName>...</CommodityName>
  <Conventions>...</Conventions>
  <Period>...</Period>
  <PricingCalendar>...</PricingCalendar>
  <UseBusinessDays>...</UseBusinessDays>
  <DeliveryRollDays>...</DeliveryRollDays>
  <FutureMonthOffset>...</FutureMonthOffset>
  <DailyExpiryOffset>...</DailyExpiryOffset>
</AveragingData>
\end{minted}
\caption{\lstinline!AveragingData! node structure}
\label{lst:ave_data_comm_future_convention}
\end{listing}

\subsubsection{Credit Default Swap Conventions}
\label{sss:cds_conventions}
A node with name \lstinline!CDS! is used to store conventions for credit default swaps. The structure of this node is shown in Listing \ref{lst:cds_conventions}.

\begin{listing}[H]
\begin{minted}[fontsize=\footnotesize]{xml}
<CDS>
  <Id>...</Id>
  <SettlementDays>...</SettlementDays>
  <Calendar>...</Calendar>
  <Frequency>...</Frequency>
  <PaymentConvention>...</PaymentConvention>
  <Rule>...</Rule>
  <DayCounter>...</DayCounter>
  <SettlesAccrual>...</SettlesAccrual>
  <PaysAtDefaultTime>...</PaysAtDefaultTime>
</CDS>
\end{minted}
\caption{CDS conventions}
\label{lst:cds_conventions}
\end{listing}

The meanings of the various elements in this node are as follows:
\begin{itemize}

\item \lstinline!Id!:
The identifier for the CDS convention.

\item \lstinline!SettlementDays!:
The number of days after the CDS trade date when protection starts i.e.\ the \textit{Protection effective date} or \textit{step-in date}. Any non-negative integer is allowed here. For standard CDS after, this is generally set to 1.

\item \lstinline!Calendar!:
The calendar associated with the CDS. For non-JPY currencies, this is generally \lstinline!WeekendsOnly! to agree with the ISDA standard. For JPY CDS, the ISDA standard calendar is \lstinline!TYO! documented at \url{https://www.cdsmodel.com/cdsmodel}. This could be set up as an additional calendar or \lstinline!JPN! could be used as a proxy. Allowable calendar values are given in Table \ref{tab:calendar}.

\item \lstinline!Frequency!:
The frequency of fee leg payments for the CDS. The ISDA standard is \lstinline!Quarterly! but any valid frequency is allowed.

\item \lstinline!PaymentConvention!:
The business day convention for payments on the CDS. The ISDA standard is \lstinline!Following! but any valid business day convention from Table \ref{tab:allow_stand_data} is allowed.

\item \lstinline!Rule!:
The date generation rule for the fee leg on the CDS. The ISDA standard is \lstinline!CDS2015! but any valid date generation rule is allowed.

\item \lstinline!DayCounter!:
The day counter for fee leg payments on the CDS. The ISDA standard is \lstinline!A360! but any valid day counter from Table \ref{tab:daycount} is allowed.

\item \lstinline!SettlesAccrual!:
A boolean value indicating if an accrued fee is due on the occurrence of a credit event. Allowable boolean values are given in the Table \ref{tab:boolean_allowable}. In general, this is set \lstinline!true!.

\item \lstinline!PaysAtDefaultTime!:
A boolean value indicating if the accrued fee, on the occurrence of a credit event, is payable at the credit event date or the end of the fee period. A value of \lstinline!true! indicates that the accrued is payable at the credit event date and a value of \lstinline!false! indicates that it is payable at the end of the fee period. In general, this is set \lstinline!true!.

\end{itemize}
