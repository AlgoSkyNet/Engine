%--------------------------------------------------------
\subsection{Conventions: {\tt conventions.xml}}
\label{sec:conventions}
%--------------------------------------------------------

The conventions to associate with a set market quotes in the construction of termstructures are specified in another xml
file which we will refer to as {\tt conventions.xml} in the following though the file name can be chosen by the user.
Each separate set of conventions is stored in an XML node. The type of conventions that a node holds is determined by
the node name. Every node has an \lstinline!Id! node that gives a unique identifier for the convention set. The
following sections describe the type of conventions that can be created and the allowed values.

%- - - - - - - - - - - - - - - - - - - - - - - - - - - - - - - - - - - - - - - -
\subsubsection{Zero Conventions}
%- - - - - - - - - - - - - - - - - - - - - - - - - - - - - - - - - - - - - - - -
A node with name \emph{Zero} is used to store conventions for direct zero rate quotes. Direct zero rate quotes can be
given with an explicit maturity date or with a tenor and a set of conventions from which the maturity date is
deduced. The node for a zero rate quote with an explicit maturity date is shown in Listing
\ref{lst:zero_conventions_date}. The node for a tenor based zero rate is shown in Listing
\ref{lst:zero_conventions_tenor}.

\begin{listing}[H]
%\hrule\medskip
\begin{minted}[fontsize=\footnotesize]{xml}
<Zero>
  <Id> </Id>
  <TenorBased>False</TenorBased>
  <DayCounter> </DayCounter>
  <CompoundingFrequency> </CompoundingFrequency>
  <Compounding> </Compounding>
</Zero>
\end{minted}
\caption{Zero conventions}
\label{lst:zero_conventions_date}
\end{listing}

\begin{listing}[H]
%\hrule\medskip
\begin{minted}[fontsize=\footnotesize]{xml}
<Zero>
  <Id> </Id>
  <TenorBased>True</TenorBased>
  <DayCounter> </DayCounter>
  <CompoundingFrequency> </CompoundingFrequency>
  <Compounding> </Compounding>
  <TenorCalendar> </TenorCalendar>
  <SpotLag> </SpotLag>
  <SpotCalendar> </SpotCalendar>
  <RollConvention> </RollConvention>
  <EOM> </EOM>
</Zero>
\end{minted}
\caption{Zero conventions, tenor based}
\label{lst:zero_conventions_tenor}
\end{listing}

The meanings of the various elements in this node are as follows:
\begin{itemize}
\item €TenorBased€: True if the conventions are for a tenor based zero quote and False if they are
for a zero quote with an explicit maturity date.
\item DayCounter: The day count basis associated with the zero rate quote (for choices see section
\ref{sec:allowable_values})
\item CompoundingFrequency: The frequency of compounding (Choices are {\em Once, Annual, Semiannual, Quarterly,
Bimonthly, Monthly, Weekly, Daily}).
\item Compounding: The type of compounding for the zero rate (Choices are {\em Simple, Compounded, Continuous,
SimpleThenCompounded}).
\item TenorCalendar: The calendar used to advance from the spot date to the maturity date by the zero rate tenor (for
choices see section \ref{sec:allowable_values}).
\item SpotLag [Optional]: The number of business days to advance from the valuation date before applying the zero rate
tenor. If not provided, this defaults to 0.
\item SpotCalendar [Optional]: The calendar to use for business days when applying the \lstinline!SpotLag!. If not
provided, it defaults to a calendar with no holidays.
\item RollConvention [Optional]: The roll convention to use when applying the zero rate tenor. If not provided, it
defaults to Following (Choices are {\em Backward, Forward, Zero, ThirdWednesday, Twentieth, TwentiethIMM, CDS}).
\item EOM [Optional]: Whether or not to use the end of month convention when applying the zero rate tenor. If not
provided, it defaults to false.
\end{itemize}

%- - - - - - - - - - - - - - - - - - - - - - - - - - - - - - - - - - - - - - - -
\subsubsection{Deposit Conventions}
%- - - - - - - - - - - - - - - - - - - - - - - - - - - - - - - - - - - - - - - -

A node with name \emph{Deposit} is used to store conventions for deposit or index fixing quotes. The conventions can be
index based, in which case all necessary conventions are deduced from a given index family. The structure of the index
based node is shown in Listing \ref{lst:deposit_conventions_index}. Alternatively, all the necessary conventions can be
given explicitly without reference to an index family. The structure of this node is shown in Listing
\ref{lst:deposit_conventions_explicit}.

\begin{listing}[H]
%\hrule\medskip
\begin{minted}[fontsize=\footnotesize]{xml}
<Deposit>
  <Id> </Id>
  <IndexBased>True</IndexBased>
  <Index> </Index>
</Deposit>
\end{minted}
\caption{Deposit conventions}
\label{lst:deposit_conventions_index}
\end{listing}

\begin{listing}[H]
%\hrule\medskip
\begin{minted}[fontsize=\footnotesize]{xml}
<Deposit>
  <Id> </Id>
  <IndexBased>False</IndexBased>
  <Calendar> </Calendar>
  <Convention> </Convention>
  <EOM> </EOM>
  <DayCounter> </DayCounter>
</Deposit>
\end{minted}
\caption{Deposit conventions}
\label{lst:deposit_conventions_explicit}
\end{listing}


The meanings of the various elements in this node are as follows:
\begin{itemize}
\item IndexBased: \emph{True} if the deposit conventions are index based and \emph{False} if the conventions are given
explicitly.
\item Index: The index family from which to imply the conventions for the deposit quote. For example, this could be
EUR-EURIBOR, USD-LIBOR etc.
\item Calendar: The business day calendar for the deposit quote.
\item Convention: The roll convention for the deposit quote.
\item EOM: \emph{True} if the end of month roll convention is to be used for the deposit quote and \emph{False} if not.
\item DayCounter: The day count basis associated with the deposit quote.
\end{itemize}

%- - - - - - - - - - - - - - - - - - - - - - - - - - - - - - - - - - - - - - - -
\subsubsection{Future Conventions}
%- - - - - - - - - - - - - - - - - - - - - - - - - - - - - - - - - - - - - - - -
A node with name \emph{Future} is used to store conventions for IMM Future quotes. The structure of this node is shown
in Listing \ref{lst:future_conventions}. The only piece of information needed is the underlying money market index name
and this is given in the \lstinline!Index! node. For example, this could be EUR-EURIBOR-3M, USD-LIBOR-3M etc.

\begin{listing}[H]
%\hrule\medskip
\begin{minted}[fontsize=\footnotesize]{xml}
<Future>
  <Id> </Id>
  <Index> </Index>
</Future>
\end{minted}
\caption{Future conventions}
\label{lst:future_conventions}
\end{listing}

%- - - - - - - - - - - - - - - - - - - - - - - - - - - - - - - - - - - - - - - -
\subsubsection{FRA Conventions}
%- - - - - - - - - - - - - - - - - - - - - - - - - - - - - - - - - - - - - - - -
A node with name \emph{FRA} is used to store conventions for FRA quotes. The structure of this node is shown in Listing 
\ref{lst:fra_conventions}. The only piece of information needed is the underlying index name and this is given in the 
\lstinline!Index! node. For example, this could be EUR-EURIBOR-6M, CHF-LIBOR-6M etc.

\begin{listing}[H]
%\hrule\medskip
\begin{minted}[fontsize=\footnotesize]{xml}
<FRA>
  <Id> </Id>
  <Index> </Index>
</FRA>
\end{minted}
\caption{FRA conventions}
\label{lst:fra_conventions}
\end{listing}

%- - - - - - - - - - - - - - - - - - - - - - - - - - - - - - - - - - - - - - - -
\subsubsection{OIS Conventions}
%- - - - - - - - - - - - - - - - - - - - - - - - - - - - - - - - - - - - - - - -

A node with name \emph{OIS} is used to store conventions for Overnight Indexed Swap (OIS) quotes. The structure of this
node is shown in Listing \ref{lst:ois_conventions}.

\begin{listing}[H]
%\hrule\medskip
\begin{minted}[fontsize=\footnotesize]{xml}
<OIS>
  <Id> </Id>
  <SpotLag> </SpotLag>
  <Index> </Index>
  <FixedDayCounter> </FixedDayCounter>
  <PaymentLag> </PaymentLag>
  <EOM> </EOM>
  <FixedFrequency> </FixedFrequency>
  <FixedConvention> </FixedConvention>
  <FixedPaymentConvention> </FixedPaymentConvention>
  <Rule> </Rule>
</OIS>
\end{minted}
\caption{OIS conventions}
\label{lst:ois_conventions}
\end{listing}

The meanings of the various elements in this node are as follows:
\begin{itemize}
\item SpotLag: The number of business days until the start of the OIS.
\item Index: The name of the overnight index. For example, this could be EUR-EONIA, USD-FedFunds etc.
\item FixedDayCounter: The day count basis on the fixed leg of the OIS.
\item PaymentLag [Optional]: The payment lag, as a number of business days, on both legs. If not provided, this defaults
to 0.
\item EOM [Optional]: \emph{True} if the end of month roll convention is to be used when generating the OIS schedule and
\emph{False} if not. If not provided, this defaults to \emph{False}.
\item FixedFrequency [Optional]: The frequency of payments on the fixed leg. If not provided, this defaults to
\emph{Annual}.
\item FixedConvention [Optional]: The roll convention for accruals on the fixed leg. If not provided, this defaults to
\emph{Following}.
\item FixedPaymentConvention [Optional]: The roll convention for payments on the fixed leg. If not provided, this
defaults to \emph{Following}.
\item Rule [Optional]: The rule used for generating the OIS dates schedule i.e.\ \emph{Backward} or \emph{Forward}. If
not provided, this defaults to \emph{Backward}.
\end{itemize}

%- - - - - - - - - - - - - - - - - - - - - - - - - - - - - - - - - - - - - - - -
\subsubsection{Swap Conventions}
%- - - - - - - - - - - - - - - - - - - - - - - - - - - - - - - - - - - - - - - -
A node with name \emph{Swap} is used to store conventions for vanilla interest rate swap (IRS) quotes. The structure of
this node is shown in Listing \ref{lst:swap_conventions}.

\begin{listing}[H]
%\hrule\medskip
\begin{minted}[fontsize=\footnotesize]{xml}
<Swap>
  <Id> </Id>
  <FixedCalendar> </FixedCalendar>
  <FixedFrequency> </FixedFrequency>
  <FixedConvention> </FixedConvention>
  <FixedDayCounter> </FixedDayCounter>
  <Index> </Index>
</Swap>
\end{minted}
\caption{Swap conventions}
\label{lst:swap_conventions}
\end{listing}

The meanings of the various elements in this node are as follows:
\begin{itemize}
\item FixedCalendar: The business day calendar on the fixed leg.
\item FixedFrequency: The frequency of payments on the fixed leg.
\item FixedConvention: The roll convention on the fixed leg.
\item FixedDayCounter: The day count basis on the fixed leg.
\item Index: The Ibor index on the floating leg.
\end{itemize}

%- - - - - - - - - - - - - - - - - - - - - - - - - - - - - - - - - - - - - - - -
\subsubsection{Average OIS Conventions}
%- - - - - - - - - - - - - - - - - - - - - - - - - - - - - - - - - - - - - - - -
A node with name \emph{AverageOIS} is used to store conventions for average OIS quotes. An average OIS is a swap where a
fixed rate is swapped against a daily averaged overnight index plus a spread. The structure of this node is shown in
Listing \ref{lst:average_ois_conventions}.

\begin{listing}[H]
%\hrule\medskip
\begin{minted}[fontsize=\footnotesize]{xml}
<AverageOIS>
  <Id> </Id>
  <SpotLag> </SpotLag>
  <FixedTenor> </FixedTenor>
  <FixedDayCounter> </FixedDayCounter>
  <FixedCalendar> </FixedCalendar>
  <FixedConvention> </FixedConvention>
  <FixedPaymentConvention> </FixedPaymentConvention>
  <Index> </Index>
  <OnTenor> </OnTenor>
  <RateCutoff> </RateCutoff>
</AverageOIS>
\end{minted}
\caption{Average OIS conventions}
\label{lst:average_ois_conventions}
\end{listing}


The meanings of the various elements in this node are as follows:
\begin{itemize}
\item SpotLag: Number of business days until the start of the average OIS.
\item FixedTenor: The frequency of payments on the fixed leg.
\item FixedDayCounter: The day count basis on the fixed leg.
\item FixedCalendar: The business day calendar on the fixed leg.
\item FixedFrequency: The frequency of payments on the fixed leg.
\item FixedConvention: The roll convention for accruals on the fixed leg.
\item FixedPaymentConvention: The roll convention for payments on the fixed leg.
\item Index: The name of the overnight index.
\item OnTenor: The frequency of payments on the overnight leg.
\item RateCutoff: The rate cut-off on the overnight leg. Generally, the overnight fixing is only observed up to a
certain number of days before the payment date and the last observed rate is applied for the remaining days in the
period. This rate cut-off gives the number of days e.g.\ 2 for Fed Funds average OIS.
\end{itemize}

%- - - - - - - - - - - - - - - - - - - - - - - - - - - - - - - - - - - - - - - -
\subsubsection{Tenor Basis Swap Conventions}
%- - - - - - - - - - - - - - - - - - - - - - - - - - - - - - - - - - - - - - - -
A node with name \emph{TenorBasisSwap} is used to store conventions for tenor basis swap quotes. The structure of this 
node is shown in Listing \ref{lst:tenor_basis_conventions}.

\begin{listing}[H]
%\hrule\medskip
\begin{minted}[fontsize=\footnotesize]{xml}
<TenorBasisSwap>
  <Id> </Id>
  <LongIndex> </LongIndex>
  <ShortIndex> </ShortIndex>
  <ShortPayTenor> </ShortPayTenor>
  <SpreadOnShort> </SpreadOnShort>
  <IncludeSpread> </IncludeSpread>
  <SubPeriodsCouponType> </SubPeriodsCouponType>
</TenorBasisSwap>
\end{minted}
\caption{Tenor basis swap conventions}
\label{lst:tenor_basis_conventions}
\end{listing}


The meanings of the various elements in this node are as follows:
\begin{itemize}
\item LongIndex: The name of the long tenor Ibor index.
\item ShortIndex: The name of the short tenor Ibor index.
\item ShortPayTenor [Optional]: The frequency of payments on the short tenor Ibor leg. This is usually the same as the
short tenor Ibor index's tenor. However, it can also be longer e.g.\ USD tenor basis swaps where the short tenor Ibor
index is compounded and paid on the same frequency as the long tenor Ibor index. If not provided, this defaults to the
short tenor Ibor index's tenor.
\item SpreadOnShort [Optional]: \emph{True} if the tenor basis swap quote has the spread on the short tenor Ibor index
leg and \emph{False} if not. If not provided, this defaults to \emph{True}.
\item IncludeSpread [Optional]: \emph{True} if the tenor basis swap spread is to be included when compounding is
performed on the short tenor Ibor index leg and \emph{False} if not. If not provided, this defaults to \emph{False}.
\item SubPeriodsCouponType [Optional]: This field can have the value \emph{Compounding} or \emph{Averaging} and it only
applies when the frequency of payments on the short tenor Ibor leg does not equal the short tenor Ibor index's tenor. If
\emph{Compounding} is specified, then the short tenor Ibor index is compounded and paid on the frequency specified in
the \lstinline!ShortPayTenor! node. If \emph{Averaging} is specified, then the short tenor Ibor index is averaged and
paid on the frequency specified in the \lstinline!ShortPayTenor! node. If not provided, this defaults to
\emph{Compounding}.
\end{itemize}

%- - - - - - - - - - - - - - - - - - - - - - - - - - - - - - - - - - - - - - - -
\subsubsection{Tenor Basis Two Swap Conventions}
%- - - - - - - - - - - - - - - - - - - - - - - - - - - - - - - - - - - - - - - -
A node with name \emph{TenorBasisTwoSwap} is used to store conventions for tenor basis swap quotes where the quote is
the spread between the fair fixed rate on two swaps against Ibor indices of different tenors. We call the swap against
the Ibor index of longer tenor the long swap and the remaining swap the short swap. The structure of the tenor basis two
swap conventions node is shown in Listing \ref{lst:tenor_basis_two_conventions}.

\begin{listing}[H]
%\hrule\medskip
\begin{minted}[fontsize=\footnotesize]{xml}
<TenorBasisTwoSwap>
  <Id> </Id>
  <Calendar> </Calendar>
  <LongFixedFrequency> </LongFixedFrequency>
  <LongFixedConvention> </LongFixedConvention>
  <LongFixedDayCounter> </LongFixedDayCounter>
  <LongIndex> </LongIndex>
  <ShortFixedFrequency> </ShortFixedFrequency>
  <ShortFixedConvention> </ShortFixedConvention>
  <ShortFixedDayCounter> </ShortFixedDayCounter>
  <ShortIndex> </ShortIndex>
  <LongMinusShort> </LongMinusShort>
</TenorBasisTwoSwap>
\end{minted}
\caption{Tenor basis two swap conventions}
\label{lst:tenor_basis_two_conventions}
\end{listing}

The meanings of the various elements in this node are as follows:
\begin{itemize}
\item Calendar: The business day calendar on both swaps.
\item LongFixedFrequency: The frequency of payments on the fixed leg of the long swap.
\item LongFixedConvention: The roll convention on the fixed leg of the long swap.
\item LongFixedDayCounter: The day count basis on the fixed leg of the long swap.
\item LongIndex: The Ibor index on the floating leg of the long swap.
\item ShortFixedFrequency: The frequency of payments on the fixed leg of the short swap.
\item ShortFixedConvention: The roll convention on the fixed leg of the short swap.
\item ShortFixedDayCounter: The day count basis on the fixed leg of the short swap.
\item ShortIndex: The Ibor index on the floating leg of the short swap.
\item LongMinusShort [Optional]: \emph{True} if the basis swap spread is to be interpreted as the fair rate on the long
swap minus the fair rate on the short swap and \emph{False} if the basis swap spread is to be interpreted as the fair
rate on the short swap minus the fair rate on the long swap. If not provided, it defaults to \emph{True}.
\end{itemize}

%- - - - - - - - - - - - - - - - - - - - - - - - - - - - - - - - - - - - - - - -
\subsubsection{FX Conventions}
%- - - - - - - - - - - - - - - - - - - - - - - - - - - - - - - - - - - - - - - -
A node with name \emph{FX} is used to store conventions for FX spot and forward quotes for a given currency pair. The
structure of this node is shown in Listing \ref{lst:fx_conventions}.

\begin{listing}[H]
%\hrule\medskip
\begin{minted}[fontsize=\footnotesize]{xml}
<FX>
  <Id> </Id>
  <SpotDays> </SpotDays>
  <SourceCurrency> </SourceCurrency>
  <TargetCurrency> </TargetCurrency>
  <PointsFactor> </PointsFactor>
  <AdvanceCalendar> </AdvanceCalendar>
  <SpotRelative> </SpotRelative>
  <AdditionalSettleCalendar> </AdditionalSettleCalendar>
</FX>
\end{minted}
\caption{FX conventions}
\label{lst:fx_conventions}
\end{listing}


The meanings of the various elements in this node are as follows:
\begin{itemize}
\item SpotDays: The number of business days to spot for the currency pair.
\item SourceCurrency: The source currency of the currency pair. The FX quote is assumed to give the number of units of
target currency per unit of source currency.
\item TargetCurrency: The target currency of the currency pair.
\item PointsFactor: The number by which a points quote for the currency pair should be divided before adding it to the
spot quote to obtain the forward rate.
\item AdvanceCalendar [Optional]: The business day calendar(s) used for advancing dates for both spot and forwards. If
not provided, it defaults to a calendar with no holidays.
\item SpotRelative [Optional]: \emph{True} if the forward tenor is to be interpreted as being relative to the spot date.
\emph{False} if the forward tenor is to be interpreted as being relative to the valuation date. If not provided, it
defaults to \emph{True}.
\item AdditionalSettleCalendar [Optional]: In some cases, when the spot date is calculated using the values in the
\lstinline!AdvanceCalendar! and \lstinline!SpotDays! nodes, it is checked against an additional settlement calendar(s)
and if it is not a good business day then it is moved forward until it is a good business day on both the additional
settlement calendar(s) and the AdvanceCalendar. This additional settlement calendar(s) can be specified here. If not
provided, it defaults to a calendar with no holidays.
\end{itemize}

%- - - - - - - - - - - - - - - - - - - - - - - - - - - - - - - - - - - - - - - -
\subsubsection{Cross Currency Basis Swap Conventions}
%- - - - - - - - - - - - - - - - - - - - - - - - - - - - - - - - - - - - - - - -
A node with name \emph{CrossCurrencyBasis} is used to store conventions for cross currency basis swap quotes. The
structure of this node is shown in Listing \ref{lst:xccy_basis_conventions}.

\begin{listing}[H]
%\hrule\medskip
\begin{minted}[fontsize=\footnotesize]{xml}
<CrossCurrencyBasis>
  <Id> </Id>
  <SettlementDays> </SettlementDays>
  <SettlementCalendar> </SettlementCalendar>
  <RollConvention> </RollConvention>
  <FlatIndex> </FlatIndex>
  <SpreadIndex> </SpreadIndex>
  <EOM> </EOM>
</CrossCurrencyBasis>
\end{minted}
\caption{Cross currency basis swap conventions}
\label{lst:xccy_basis_conventions}
\end{listing}


The meanings of the various elements in this node are as follows:
\begin{itemize}
\item SettlementDays: The number of business days to the start of the cross currency basis swap.
\item SettlementCalendar: The business day calendar(s) for both legs and to arrive at the settlement date using the
SettlementDays above.
\item RollConvention: The roll convention for both legs.
\item FlatIndex: The name of the index on the leg that does not have the cross currency basis spread.
\item SpreadIndex: The name of the index on the leg that has the cross currency basis spread.
\item EOM [Optional]: \emph{True} if the end of month convention is to be used when generating the schedule on both legs
and \emph{False} if not. If not provided, it defaults to \emph{False}.
\end{itemize}
