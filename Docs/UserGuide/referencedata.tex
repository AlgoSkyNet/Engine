%--------------------------------------------------------
\subsection{Reference Data {\tt referencedata.xml}}
\label{sec:referencedata}
%--------------------------------------------------------

Reference Data is used to ease the burden on portfolio xml representation, by taking common elements and storing them as
static data. Currently this can be used for \textit{Bond Derivatives} that require bond static information.

Bond reference data is also used to build yield curves fitted to liquid bond prices, see \ref{sec:fitted_bond_segment}.

The allowable types for ReferenceData is
\begin{enumerate}
\item \textbf{Bond} static data consists of the Leg data for a given bond.
\end{enumerate}

\begin{minted}{xml}
  <ReferenceData>
  <!-- Bond reference datum -->
  <ReferenceDatum id="SECURITY_1">
    <Type>Bond</Type>
    <BondReferenceData>
      <IssuerId>CPTY_C</IssuerId>
      <CreditCurveId>ZERO</CreditCurveId>
      <ReferenceCurveId>EURBENCHMARK-EUR-6M</ReferenceCurveId>
      <SettlementDays>2</SettlementDays>
      <Calendar>TARGET</Calendar>
      <IssueDate>20110215</IssueDate>
      <LegData>
        <LegType>Fixed</LegType>
        <Payer>false</Payer>
        <Currency>EUR</Currency>
        <Notionals>
          <Notional>1</Notional>
        </Notionals>
        <DayCounter>ActActISMA</DayCounter>
        <PaymentConvention>F</PaymentConvention>
        <FixedLegData>
          <Rates>
            <Rate>0.02</Rate>
          </Rates>
        </FixedLegData>
        <ScheduleData>
          <Rules>
            <StartDate>20190103</StartDate>
            <EndDate>20200103</EndDate>
            <Tenor>1Y</Tenor>
            <Calendar>TARGET</Calendar>
            <Convention>U</Convention>
            <TermConvention>U</TermConvention>
            <Rule>Forward</Rule>
            <EndOfMonth/>
            <FirstDate/>
            <LastDate/>
          </Rules>
        </ScheduleData>
      </LegData>
    </BondReferenceData>
  </ReferenceDatum>
</ReferenceData>
\end{minted}
