%- - - - - - - - - - - - - - - - - - - - - - - - - - - - - - - - - - - - - - - -
\subsection{Trade Components}
%- - - - - - - - - - - - - - - - - - - - - - - - - - - - - - - - - - - - - - - -

Trade components are XML sub-nodes used within the trade data containers to define sets of trade data that more than one
trade type can have in common, such as a leg or a schedule. A trade data container can include multiple trade components
such as a swap with multiple legs, and a trade component can itself contain further trade components in a nested way.

\vspace{1em}

An example of a \lstinline!SwapData! trade data container, including two \lstinline!LegData! trade components which in
turn include further trade components such as \lstinline!FixedLegData!, \lstinline!ScheduleData! and
\lstinline!FloatingLegData! is shown in Listing \ref{lst:trade_component}.

\begin{listing}[H]
%\hrule\medskip
\begin{minted}[fontsize=\footnotesize]{xml}
        <SwapData>
            <LegData>
                <Payer>true</Payer>
                <LegType>Fixed</LegType>
                <Currency>EUR</Currency>
                <PaymentConvention>Following</PaymentConvention>
                <DayCounter>30/360</DayCounter>
                <Notionals>
                    <Notional>1000000</Notional>
                </Notionals>
                <ScheduleData>
                ...
                </ScheduleData>
                <FixedLegData>
                    <Rates>
                        <Rate>0.035</Rate>
                    </Rates>
                </FixedLegData>
            </LegData>
            <LegData>
                ...
                <ScheduleData>
                    ...
                </ScheduleData>
                <FloatingLegData>
                    ...
                </FloatingLegData>
            </LegData>
        </SwapData>
\end{minted}
\caption{Trade Components Example}
\label{lst:trade_component}
\end{listing}

Descriptions of all trade components supported in ORE follow below.

\subsubsection{Option Data}
\label{ss:option_data} 
This trade component node is used within the \lstinline!SwaptionData! and \lstinline!FXOptionData! trade data
containers. It contains the \lstinline!ExerciseDates! sub-node which includes \lstinline!ExerciseDate! child
elements. An example structure of the \lstinline!OptionData! trade component node is shown in Listing
\ref{lst:option_data}.

\begin{listing}[H]
%\hrule\medskip
\begin{minted}[fontsize=\footnotesize]{xml}
            <OptionData>
                <LongShort>Long</LongShort>
                <OptionType>Call</OptionType>
                <Style>Bermudan</Style>
                <Settlement>Cash</Settlement> 
                <PayOffAtExpiry>true</PayOffAtExpiry>
                <ExerciseDates>
                    <ExerciseDate>2016-04-20</ExerciseDate>
                    <ExerciseDate>2017-04-20</ExerciseDate>
                </ExerciseDates>
            </OptionData>
\end{minted}
\caption{Option data}
\label{lst:option_data}
\end{listing}

The meanings and allowable values of the elements in the \lstinline!OptionData! node follow below.

\begin{itemize}
\item LongShort: Specifies whether the option position is long  or
  short.  

Allowable values: \emph{Long, LONG, long, L} or \emph{Short, SHORT,
  short, S}

\item OptionType: Specifies whether it is a call or a put option. 

Allowable values: \emph{Call} or \emph{Put} 

The meaning of Call and Put values depend on the asset class of the option, see Table \ref{tab:callput_specs}.

\begin{table}[H]
\centering
\begin{tabu} to 0.9\linewidth {| X[-1.5,l,m] | X[-5,l,m] |}
    \hline
      \bfseries{Asset Class}  & \bfseries{Call / Put Specifications} \\  \hline
Equity/ Commodity Option & \emph{Call}: The right to buy the underlying equity/commodity at the strike price.
\newline \emph{Put}: The right to sell the underlying equity/commodity at the strike price. \\  \hline
 IR Swaption &  \emph{Call/Put} values are ignored. Payer/Receiver swaption is determined in the Leg Data nodes of the underlying swap. \\ \hline
FX Option &  \emph{Call}: Bought and Sold currencies/amounts stay as determined in the trade data node. 
\newline \emph{Put}: Bought and Sold currencies/amounts are switched compared to the trade data node.  \\ \hline
  \end{tabu}
  \caption{Specification of Option Type Call / Put}
  \label{tab:callput_specs}
\end{table}


\item Style: The exercise style of the option. 

  Allowable values: \emph{European} or \emph{American} or \emph{Bermudan}. Note that trade type Swaption can have style
  \emph{European} or \emph{Bermudan}, but not \emph{American}. The FX Option trade type can have style \emph{European}
  or \emph{American}, but not \emph{Bermudan}.

\item Settlement: Delivery type. 

Allowable values: \emph{Cash} or \emph{Physical} 

\item PayOffAtExpiry [Optional]: Relevant for options with early
  exercise, i.e. the exercise occurs before expiry; \emph{true}
  indicates payoff at expiry, whereas \emph{false}  indicates payoff
  at exercise. Defaults to \emph{false}  if left blank or omitted. 

Allowable values: \emph{true}, \emph{false}
%TBC: Do not see payoffatexpiry used in either fxoption or swaption build() functions, to be logged as a bug.       

\item ExerciseDates: This node contains child elements of type
  \lstinline!ExerciseDate!.  Options of style \emph{European} or
  \emph{American} require a single exercise date expressed by one
  single \lstinline!ExerciseDate! child element.  \emph{Bermudan}
  style options must have two or more \lstinline!ExerciseDate! child
  elements.

\item PremiumAmount [Optional]: Option premium amount paid by the option buyer to the option seller.

Allowable values:  Any positive real number.

\item PremiumCurrency [Optional]: Currency of the option premium.

Allowable values:  See \lstinline!Currency! in Table \ref{tab:allow_stand_data}.

\item PremiumPayDate [Optional]: Date of the option premium payment.

Allowable values:  See \lstinline!Date! in Table \ref{tab:allow_stand_data}.

\end{itemize}



\subsubsection{Leg Data and Notionals}
\label{ss:leg_data}

The \lstinline!LegData! trade component node is used within the
\lstinline!CapFloorData!,  \lstinline!SwapData! and
\lstinline!SwaptionData! trade data containers. It contains a
\lstinline!ScheduleData! trade component sub-node, and a sub-node that depends on
the value of the \lstinline!LegType! element, e.g.:  \lstinline!FixedLegData! for \lstinline!LegType! \emph{Fixed} or \lstinline!FloatingLegData! for \lstinline!LegType! \emph{Floating}. The
\lstinline!LegData! node also includes a \lstinline!Notionals!
sub-node  with \lstinline!Notional! child elements described below. An
example structure of a \lstinline!LegData! node of \lstinline!LegType!
\emph{Floating} is shown in Listing \ref{lst:leg_data}.

\begin{listing}[H]
%\hrule\medskip
\begin{minted}[fontsize=\footnotesize]{xml}
            <LegData>
                <Payer>false</Payer>
                <LegType>Floating</LegType>
                <Currency>EUR</Currency>
                <PaymentConvention>Following</PaymentConvention>
                <DayCounter>30/360</DayCounter>
                <Notionals>
                    <Notional>1000000</Notional>
                </Notionals>
                <ScheduleData>
                    ...
                </ScheduleData>
                <FloatingLegData>
                    ...
                </FloatingLegData>
            </LegData>
\end{minted}
\caption{Leg data}
\label{lst:leg_data}
\end{listing}

The meanings and allowable values of the elements in the \lstinline!LegData! node follow below.

\begin{itemize}
\item LegType:  Determines which of the available sub-nodes must be
  used. 

Allowable values:  \emph{Fixed, Floating, Cashflow, CMS, CMSSpread, DigitalCMSSpread, Equity, YY, CPI}

\item Payer:  The flows of the leg are paid to the counterparty if
  \emph{true}, and received if \emph{false}.  

Allowable values:  \emph{true, false} 

\item Currency: The currency of the leg. 

Allowable values:  See \lstinline!Currency! in Table \ref{tab:allow_stand_data}.

\item PaymentConvention: The payment convention of the leg coupons. 

Allowable values: See \lstinline!Roll Convention! in Table \ref{tab:allow_stand_data}.

\item DayCounter: The day count convention of the leg coupons. 

Allowable values: See \lstinline!DayCount Convention! in Table \ref{tab:daycount}.

\item Notionals: This node contains child elements of type
  \lstinline!Notional!. If the notional is fixed over the life of the
  leg only one notional value should be entered. If the notional is
  amortising or accreting, this is represented by entering multiple
  notional values, each represented by a \lstinline!Notional! child
  element. The first notional value corresponds to the first coupon,
  the second notional value corresponds to the second coupon, etc. If
  the number of coupons exceeds the number of notional values, the
  notional will be kept flat at the value of last entered notional for
  the remaining coupons.  The number of entered notional values cannot
  exceed the number of coupons.

Allowable values: Each child element can take any positive real number.

\vspace{1em}

An example of a \lstinline!Notionals! element for an amortising leg with four coupons is shown in Listing \ref{lst:notionals}.
\begin{listing}[H]
%\hrule\medskip
\begin{minted}[fontsize=\footnotesize]{xml}
                <Notionals>
                    <Notional>65000000</Notional>
                    <Notional>65000000</Notional>
                    <Notional>55000000</Notional>
                    <Notional>45000000</Notional>
                </Notionals>
\end{minted}
\caption{Notional list}
\label{lst:notionals}
\end{listing}

Another allowable specification of the notional schedule is shown in Listing \ref{lst:notionals_dates}. 
\begin{listing}[H]
%\hrule\medskip
\begin{minted}[fontsize=\footnotesize]{xml}
                <Notionals>
                    <Notional>65000000</Notional>
                    <Notional startDate='2016-01-02'>65000000</Notional>
                    <Notional startDate='2017-01-02'>55000000</Notional>
                    <Notional startDate='2021-01-02'>45000000</Notional>
                </Notionals>
\end{minted}
\caption{Notional list with dates}
\label{lst:notionals_dates}
\end{listing}
The first notional must not have a start date, it will be associated
with the schedule's start, The subsequent notionals can have a start
date specified from which date onwards the new notional is applied. This allows
specifying notionals only for dates where the notional changes. 

\vspace{1em} 

In case of exchange of currencies an initial exchange, a final exchange
and an amortising exchange can be specified using an \lstinline!Exchanges! child element with \break
\lstinline!NotionalInitialExchange!, \lstinline!NotionalFinalExchange! and \break
\lstinline!NotionalAmortizingExchange! as subelements, see Listing
\ref{lst:notional_exchange}.

\begin{listing}[H]
%\hrule\medskip
\begin{minted}[fontsize=\footnotesize]{xml}
                <Notionals>
                    <Notional>65000000</Notional>
                    <Exchanges>
                      <NotionalInitialExchange>true</NotionalInitialExchange>
                      <NotionalFinalExchange>true</NotionalFinalExchange>
                      <NotionalAmortizingExchange>true</NotionalAmortizingExchange>
                    </Exchanges>
                </Notionals>
\end{minted}
\caption{Notional list with exchange}
\label{lst:notional_exchange}
\end{listing}

FX Resets,  used for Rebalancing Cross-currency swaps, can be specified using an \lstinline!FXReset! child element with the following subelements:  See Listing \ref{lst:notional_fxreset} for an example. \break

\item ForeignCurrency: The foreign currency the notional of the leg resets to.  

Allowable values:  See \lstinline!Currency!  in Table \ref{tab:allow_stand_data}.

\item ForeignAmount: The notional amount in the foreign currency that the notional of the leg resets to.  

Allowable values:  Any positive real number.

\item FXIndex: A reference to an FX Index source for the FX reset fixing. 

Allowable values:  A string on the form FX-SOURCE-CCY1-CCY2.

\item FixingDays: The FX fixing lag in business days

Allowable values:  Any integer.

\item FixingCalendar[Optional]: The calendar associated with the FX Index. 

Allowable values: See Table \ref{tab:calendar} Calendar. Defaults to the null calendar if left blank or omitted.



 \begin{listing}[H]
%\hrule\medskip
\begin{minted}[fontsize=\footnotesize]{xml}
                <Currency>USD</Currency>
                <Notionals>
                    <Notional>65000000</Notional> <!-- in USD -->
                    <FXReset>
                      <ForeignCurrency> EUR </ForeignCurrency>
                      <ForeignAmount> 60000000 </ForeignAmount>
                      <FXIndex> FX-ECB-USD-EUR </FXIndex>
                      <FixingDays> 2 </FixingDays>
                      <FixingCalendar>TARGET</FixingCalendar>
                    </FXReset>
                </Notionals>
\end{minted}
\caption{Notional list with fx reset}
\label{lst:notional_fxreset}
\end{listing}


After the \lstinline!Notional! sub-node the \lstinline!LegData! node includes a \lstinline!ScheduleData! sub-node, and a sub-node based on the choice of \lstinline!LegType! as per below:

\item ScheduleData: This is a trade component sub-node outlined in section \ref{ss:schedule_data} Schedule Data and
Dates.
\item FixedLegData: This trade component sub-node is required if \lstinline!LegType! is set to \emph{fixed} It is
outlined in section \ref{ss:fixedleg_data}.
\item FloatingLegData: This trade component sub-node is required if \lstinline!LegType! is set to \emph{floating} It is
outlined in section \ref{ss:floatingleg_data} Floating Leg Data and Spreads.
\item CMSLegData: This trade component sub-node is required if \lstinline!LegType! is set to \emph{CMS}. It is
  outlined in section \ref{ss:cmslegdata}.
\item CMSSpreadLegData: This trade component sub-node is required if \lstinline!LegType! is set to \emph{CMSSpread}. It is
  outlined in section \ref{ss:cmsspreadlegdata}.
\item DigitalCMSSpreadLegData: This trade component sub-node is required if \lstinline!LegType! is set to \emph{DigitalCMSSpread}. It is
  outlined in section \ref{ss:digitalcmsspreadlegdata}.
\item CPILegData: This trade component sub-node is required if \lstinline!LegType! is set to \emph{CPI}. It is
  outlined in section \ref{ss:cpilegdata}.
\item YYLegData: This trade component sub-node is required if \lstinline!LegType! is set to \emph{YY}. It is
  outlined in section \ref{ss:yylegdata}.
\end{itemize}

\subsubsection{Schedule Data (Rules and Dates)}\label{ss:schedule_data}

The \lstinline!ScheduleData! trade component node is used within the \lstinline!LegData! trade component. The Schedule can both be
rules based (at least one sub-node Rules exists) and dates based (at least one \lstinline!Dates! sub-node exists,
where the schedule is determined directly by \lstinline!Date! child elements). In rules based schedules, the schedule dates are generated from a
set of rules based on the entries of the sub-node Rules, having the elements StartDate, EndDate, Tenor, Calendar, Convention, TermConvention, and Rule.
Example structures of \lstinline!ScheduleData! nodes based on rules
respectively dates are shown in Listing \ref{lst:schedule_data_true} and Listing \ref{lst:schedule_data_false}, respectively.

\begin{listing}[H]
%\hrule\medskip
\begin{minted}[fontsize=\footnotesize]{xml}
              <ScheduleData>
                    <Rules>
                        <StartDate>2013-02-01</StartDate>
                        <EndDate>2030-02-01</EndDate>
                        <Tenor>1Y</Tenor>
                        <Calendar>UK</Calendar>
                        <Convention>MF</Convention>
                        <TermConvention>MF</TermConvention>
                        <Rule>Forward</Rule>
                    </Rules>
              </ScheduleData>
\end{minted}
\caption{Schedule data, rules based}
\label{lst:schedule_data_true}
\end{listing}

\begin{listing}[H]
%\hrule\medskip
\begin{minted}[fontsize=\footnotesize]{xml}
               <ScheduleData>
                    <Dates>
                        <Date>2012-01-06</Date>
                        <Date>2012-04-10</Date>
                        <Date>2012-07-06</Date>
                        <Date>2012-10-08</Date>
                        <Date>2013-01-07</Date>
                        <Date>2013-04-08</Date>
                    </Dates>
                </ScheduleData>
\end{minted}
\caption{Schedule data, date based}
\label{lst:schedule_data_false}
\end{listing}

The ScheduleData section can contain any number and combination of
{\tt <Dates>} and {\tt <Rules>} sections. The resulting schedule will
then be an ordered concatenation of individual schedules.
 
\medskip
The meanings and allowable values of the elements in the \lstinline!ScheduleData! node follow below.

\begin{itemize}
  % \item IsRulesBased: Determines whether the schedule is set by specifying dates directly, or by specifying rules that
  %   generate the schedule. If \emph{true}, the following entries are required: StartDate, EndDate, Tenor, Calendar,
  %   Convention, TermConvention, and Rule.  If false the Dates sub-node is required. \\ Allowable values: \emph{true,
  %   false}
\item Rules: a sub-node that determines whether the schedule is set by specifying rules that
generate the schedule. If existing, the following entries are required: StartDate, EndDate, Tenor, Calendar, 
Convention, TermConvention, and Rule. If not existing, the Dates
sub-node is required.
\item StartDate:  The schedule start date.  

Allowable values:  See \lstinline!Date! in Table \ref{tab:allow_stand_data}.

\item EndDate: The schedule end date.  

Allowable values:  See \lstinline!Date! in Table \ref{tab:allow_stand_data}.

\item Tenor: The tenor used to generate schedule dates. 

Allowable values: A string where the last character must be D or W or
M or Y.  The characters before that must be a positive integer. \\D
$=$ Day, W $=$ Week, M $=$ Month, Y $=$ Year

\item Calendar: The calendar used to generate schedule  dates. 

Allowable values: See Table \ref{tab:calendar} Calendar.

\item Convention: Determines the adjustment of the schedule dates with
  regards to the selected calendar. 

Allowable values: See \lstinline!Roll Convention! in Table
\ref{tab:allow_stand_data}.

\item TermConvention: Determines the adjustment of the final schedule
  date with regards to the selected calendar. 

Allowable values: See \lstinline!Roll Convention! in Table \ref{tab:allow_stand_data}.

\item Rule [Optional]: Rule for the generation of the schedule using given
  start and end dates, tenor, calendar and business day conventions. 

Allowable values and descriptions: See Table \ref{tab:rule} Rule. Defaults to \emph{Forward} if left blank or omitted.

\item EndOfMonth [Optional]: Specifies whether the date generation rule is different for end of month. 

Allowable values: \emph{True, False}. Defaults to \emph{False} if left blank or omitted

\item FirstDate [Optional]: Date for initial stub period.

Allowable values: See \lstinline!Date! in Table \ref{tab:allow_stand_data}.

\item LastDate [Optional]: Date for final stub period.

Allowable values: See \lstinline!Date! in Table \ref{tab:allow_stand_data}.

\item Dates: This is a sub-node and contains child elements of type
  \lstinline!Date!. In this case the schedule dates are determined
  directly by the \lstinline!Date! child elements.  At least two
  \lstinline!Date! child elements must be provided.     

  Allowable values: Each \lstinline!Date!  child element can take the allowable values listed in \lstinline!Date! in
  Table \ref{tab:allow_stand_data}.

\end{itemize}



\begin{table}[H]
\centering
\begin{tabular}{|l|p{6cm}|}
\hline
\multicolumn{2}{|l|}{\lstinline!Rule!}                    \\ \hline
\textbf{Allowable Values}                   & \textbf{Effect}                       \\ \hline
\emph{Backward}   &   Backward from termination date to effective date.   \\ \hline
\emph{Forward}   &   Forward from effective date to termination date.  \\ \hline
\emph{Zero}   &   No intermediate dates between effective date and termination date.  \\ \hline
\emph{ThirdWednesday}   &   All dates but effective date and
                          termination date are taken to be on the
                          third Wednesday of their month (with forward calculation.) \\ \hline
\emph{Twentieth}   &   All dates but the effective date are taken to be the twentieth of their month (used for CDS schedules in emerging markets.)  The termination date is also modified. \\ \hline
\emph{TwentiethIMM}   &   All dates but the effective date are  taken to be the twentieth of an IMM month (used for CDS schedules.)  The termination date is also modified. \\ \hline
\emph{OldCDS}   &   Same as TwentiethIMM with unrestricted date ends and log/short stub coupon period (old CDS convention).\\ \hline
CDS   &   Credit derivatives standard rule defined in 'Big Bang' changes in 2009.\\ \hline
CDS2015   &   Credit derivatives standard rule updated in 2015.\\ \hline
\end{tabular}
  \caption{Allowable Values for Rule}
  \label{tab:rule}
\end{table}

\subsubsection{Fixed Leg Data and Rates}
\label{ss:fixedleg_data}

The \lstinline!FixedLegData! trade component node is used within the \lstinline!LegData! trade component when the
\lstinline!LegType! element is set to \emph{Fixed}. The \lstinline!FixedLegData! node only includes the
\lstinline!Rates! sub-node which contains the rates of the fixed leg as child elements of type \lstinline!Rate!.

An example of a \lstinline!FixedLegData! node for a fixed leg with constant notional is shown in Listing \ref{lst:fixedleg_data}.
\begin{listing}[H]
%\hrule\medskip
\begin{minted}[fontsize=\footnotesize]{xml}
              <FixedLegData>
                    <Rates>
                        <Rate>0.05</Rate>
                    </Rates>
              </FixedLegData>
\end{minted}
\caption{Fixed leg data}
\label{lst:fixedleg_data}
\end{listing}

The meanings and allowable values of the elements in the \lstinline!FixedLegData! node follow below.

\begin{itemize}

\item Rates: This node contains child elements of type
  \lstinline!Rate!. If the rate is constant over the life of the fixed
  leg, only one rate value should be entered. If two or more coupons
  have different rates, multiple rate values are required, each
  represented by a \lstinline!Rate! child element. The first rate
  value corresponds to the first coupon, the second rate value
  corresponds to the second coupon, etc. If the number of coupons
  exceeds the number of rate values, the rate will be kept flat at the
  value of last entered rate for the remaining coupons.  The number of
  entered rate values cannot exceed the number of coupons. 

  Allowable values: Each child element can take any  real number. The rate is
  expressed in decimal form, e.g. 0.05 is a rate of 5\%.

As in the case of notionals, the rate schedule can be specified with
dates as shown in Listing \ref{lst:fixedleg_data_dates}.
\begin{listing}[H]
%\hrule\medskip
\begin{minted}[fontsize=\footnotesize]{xml}
              <FixedLegData>
                    <Rates>
                        <Rate>0.05</Rate>
                        <Rate startDate='2016-02-04'>0.05</Rate>
                        <Rate startDate='2019-02-05'>0.05</Rate>
                    </Rates>
              </FixedLegData>
\end{minted}
\caption{Fixed leg data with 'dated' rates}
\label{lst:fixedleg_data_dates}
\end{listing}

\end{itemize}

\subsubsection{Floating Leg Data, Spreads, Gearings, Caps and Floors}
\label{ss:floatingleg_data}

The \lstinline!FloatingLegData! trade component node is used within the \lstinline!LegData! trade component when the
\lstinline!LegType! element is set to \emph{Floating}. It is also used directly within the \lstinline!CapFloor! trade
data container.  The \lstinline!FloatingLegData! node includes elements specific to a floating leg as well as the
\lstinline!Spreads! sub-node which contains the spreads of the floating leg as child elements of type
\lstinline!Spread!.

An example of a \lstinline!FloatingLegData! node is shown in Listing \ref{lst:floatingleg_data}.
\begin{listing}[H]
%\hrule\medskip
\begin{minted}[fontsize=\footnotesize]{xml}
                <FloatingLegData>
                    <Index>USD-LIBOR-3M</Index>
                    <IsInArrears>false</IsInArrears>
                    <FixingDays>2</FixingDays>
                    <Spreads>
                        <Spread>0.005</Spread>
                    </Spreads>
                    <Gearings>
                        <Gearing>2.0</Gearing>
                    </Gearings>
                    <Caps>
                        <Cap>0.05</Cap>
                    </Caps>
                    <Floors>
                        <Floor>0.01</Floor>
                    </Floors>
                    <NakedOption>N</NakedOption>    
                </FloatingLegData>
\end{minted}
\caption{Floating leg data}
\label{lst:floatingleg_data}
\end{listing}

The meanings and allowable values of the elements in the \lstinline!FloatingLegData! node follow below.

\begin{itemize}
\item Index:  The combination of currency, index and term that
  identifies the relevant fixings and yield curve of the floating leg.  

  Allowable values: An alphanumeric string of the form CCY-INDEX-TERM. CCY, INDEX and TERM must be separated by dashes (-). CCY and INDEX must be among the supported currency and index combinations. TERM must be an integer followed by D, W,
  M or Y. See Table \ref{tab:indices}.

\item IsAveraged [Optional]:  For cases where there are multiple index fixings over a period \emph{true} indicates that the average of the fixings is used to calculate the coupon.  \emph{false} indicates that the coupon is calculated by compounding the fixings. 

Allowable values:  \emph{true, false}. Defaults to \emph{false} if left blank or omitted.

\item IsInArrears [Optional]:  \emph{true} indicates that  fixing is in arrears,
  i.e. the fixing gap is calculated in relation to the current period
  end date.\\ \emph{false} indicates that  fixing is in advance,
  i.e. the fixing gap is calculated in relation to the previous period
  end date.  

Allowable values:  \emph{true, false}. Defaults to \emph{false} if left blank or omitted.

\item FixingDays [Optional]: This is the fixing gap, i.e. the number of days
  before the period end date an index fixing is taken.   

Allowable values:  Positive integers.  Defaults to \emph{0} if left blank or omitted.

\item Spreads [Optional]: This node contains child elements of type
  \lstinline!Spread!. If the spread is constant over the life of the
  floating leg, only one spread value should be entered. If two or more
  coupons have different spreads, multiple spread values are required,
  each represented by a \lstinline!Spread! child element. The first
  spread value corresponds to the first coupon, the second spread
  value corresponds to the second coupon, etc. If the number of
  coupons exceeds the number of spread values, the spread will be kept
  flat at the value of last entered spread for the remaining coupons.
  The number of entered spread values cannot exceed the number of
  coupons. 

  Allowable values: Each child element can take any real number. The spread is expressed in decimal form, e.g. 0.005 is
  a spread of 0.5\% or 50 bp.

For the {\tt <Spreads>} section, the same applies as for notionals and
rates - a list of changing spreads can be specified without or with individual starte dates as shown
in Listing \ref{lst:spreads_dates}.
\begin{listing}[H]
%\hrule\medskip
\begin{minted}[fontsize=\footnotesize]{xml}
                    <Spreads>
                        <Spread>0.005</Spread>
                        <Spread startDate='2017-03-05'>0.007</Spread>
                        <Spread startDate='2019-03-05'>0.009</Spread>
                    </Spreads>
\end{minted}
\caption{'Dated' spreads}
\label{lst:spreads_dates}
\end{listing}

\item Gearings [Optional]: This node contains child elements of type \lstinline!Gearing! indicating that the coupon rate is
  multiplied by the given factors. The mode of specification is analogous to spreads, see above.

\item Caps [Optional]: This node contains child elements of type \lstinline!Cap! indicating that the coupon rate is capped at the
  given rate (after applying gearing and spread, if any). The mode of specification is analogous to spreads, see above.

\item Floors [Optional]: This node contains child elements of type \lstinline!Floor! indicating that the coupon rate is floored at
  the given rate (after applying gearing and spread, if any). The mode of specification is analogous to spreads, see
  above.

\item NakedOption [Optional]: Optional node (defaults to N), if Y the leg represents only the embedded floor, cap or collar. 
By convention these embedded options are considered long if the leg is a receiver leg, otherwise short. 

\end{itemize}

\subsubsection{Leg Data with Amortisation Structures}
\label{ss:amortisationdata}

Amortisation structures can (optionally) be added to a leg as
indicated in the following listing \ref{lst:amortisations}, within a
block of information enclosed by {\tt <Amortizations>} and {\tt
  </Amortizations>} tags.

\begin{listing}[H]
%\hrule\medskip
\begin{minted}[fontsize=\footnotesize]{xml}
      <LegData>
        <LegType> ... </LegType>
        <Payer> ... </Payer>
        <Currency> ... </Currency>
        <Notionals>
          <Notional>10000000</Notional>
        </Notionals>
        <Amortizations>
          <AmortizationData>
            <Type>FixedAmount</Type>
            <Value>1000000</Value>
            <StartDate>20170203</StartDate>
            <Frequency>1Y</Frequency>
            <Underflow>false</Underflow>
          </AmortizationData>
          <AmortizationData>
            ...
          </AmortizationData>
        </Amortizations>
        ...
      </LegData>
\end{minted}
\caption{Amortisation data}
\label{lst:amortisations}
\end{listing}

The user can specify a sequence of {\tt AmortizationData} items in
order to switch from one kind of amortisation to another etc.  
Within each {\tt AmortisationData} block the meaning of elements is

\begin{itemize}
\item Type: Amortisation type with allowable values {\em FixedAmount,
  RelativeToInitialNotional, RelativeToPreviousNotional, Annuity.}
\item Value: Interpreted depending on {\tt Type}, see below.
\item StartDate: Amortisation starts on first schedule date on or
  beyond StartDate.
\item Frequency, entered as a period: Frequency of amortisations.
\item Underflow:  Allow amortisation below zero notional if {\tt true},
  otherwise amortisation stops at zero notional.
\end{itemize}

The amortisation data block's {\tt Value} element  is interpreted
depending on the chosen {\tt Type}:
\begin{itemize}
\item FixedAmount: The value is interpreted as a notional amount to be
  subtracted from the current notional on each amortisation date.
\item RelativeToInitialNotional: The value is interpreted as a
  fraction of the {\bf initial} notional to be subtraced from the current
  notional on each amortisation date.
\item RelativeToPreviousNotional: The value is interpreted as a
  fraction of the {\bf previous} notional to be subtraced from the current
  notional on each amortisation date.
\item Annuity: The value is interpreted as annuity amount (redemption
  plus coupon).
\end{itemize}

Annuity type amortisation is supported for fixed rate legs as well as
floating (ibor) legs. 

Note:
\begin{itemize}
\item Floating annuities require at least one previous vanilla coupon
  in order to work out the first amortisation amount. 
\item Floating legs with annuity amortisation currently do not allow
  switching the amortisation type, i.e. only a  single block of {\tt
    AmortizationData}.
\end{itemize}

\subsubsection{CMS Leg Data}
\label{ss:cmslegdata}

Listing \ref{lst:cmslegdata} shows an example for a leg of type CMS. 

\begin{listing}[H]
%\hrule\medskip
\begin{minted}[fontsize=\footnotesize]{xml}
      <LegData>
        <LegType>CMS</LegType>
        <Payer>false</Payer>
        <Currency>GBP</Currency>
        <Notionals>
          <Notional>10000000</Notional>
        </Notionals>
        <DayCounter>ACT/ACT</DayCounter>
        <PaymentConvention>Following</PaymentConvention>
        <ScheduleData>
          ...
        </ScheduleData>
        <CMSLegData>
          <Index>EUR-CMS-10Y</Index>
          <Spreads>
            <Spread>0.0010</Spread>
          </Spreads>
          <Gearings>
            <Gearing>2.0</Gearing>
          </Gearings>
          <Caps>
            <Cap>0.05</Cap>
          </Caps>
          <Floors>
            <Floor>0.01</Floor>
          </Floors>
        </CMSLegData>
        <NakedOption>N</NakedOption>
      </LegData>
\end{minted}
\caption{CMS leg data}
\label{lst:cmslegdata}
\end{listing}
 
The CMSLegData block contains the following elements:

\begin{itemize}
\item Index: The underlying CMS index.
\item Spreads: The spreads applied to index fixings. As usual, this can be a single value, a vector of values or a dated vector of
  values.
\item IsInArrears:  \emph{true} indicates that  fixing is in arrears,
  i.e. the fixing gap is calculated in relation to the current period
  end date.\\ \emph{false} indicates that  fixing is in advance,
  i.e. the fixing gap is calculated in relation to the previous period
  end date.  
\item FixingDays: This is the fixing gap, i.e. the number of days
  before the period end date an index fixing is taken.   
\item Gearings: This node contains child elements of type \lstinline!Gearing! indicating that the coupon rate is
  multiplied by the given factors. The mode of specification is analogous to spreads, see above.
\item Caps: This node contains child elements of type \lstinline!Cap! indicating that the coupon rate is capped at the
  given rate (after applying gearing and spread, if any). The mode of specification is analogous to spreads, see above.
\item Floors: This node contains child elements of type \lstinline!Floor! indicating that the coupon rate is floored at
  the given rate (after applying gearing and spread, if any). The mode of specification is analogous to spreads, see
  above.
\item NakedOption: Optional node (defaults to N), if Y the leg represents only the embedded floor, cap or collar. 
By convention these embedded options are considered long if the leg is a receiver leg, otherwise short. 
\end{itemize}

\subsubsection{CMS Spread Leg Data}
\label{ss:cmsspreadlegdata}

Listing \ref{lst:cmsspreadlegdata} shows an example for a leg of type CMSSpread.

\begin{listing}[H]
%\hrule\medskip
\begin{minted}[fontsize=\footnotesize]{xml}
      <LegData>
        <LegType>CMSSpread</LegType>
        <Payer>false</Payer>
        <Currency>GBP</Currency>
        <Notionals>
          <Notional>10000000</Notional>
        </Notionals>
        <DayCounter>ACT/ACT</DayCounter>
        <PaymentConvention>Following</PaymentConvention>
        <ScheduleData>
          ...
        </ScheduleData>
        <CMSSpreadLegData>
          <Index1>EUR-CMS-10Y</Index1>
          <Index2>EUR-CMS-2Y</Index2>
          <Spreads>
            <Spread>0.0010</Spread>
          </Spreads>
          <Gearings>
            <Gearing>8.0</Gearing>
          </Gearings>
          <Caps>
            <Cap>0.05</Cap>
          </Caps>
          <Floors>
            <Floor>0.01</Floor>
          </Floors>
        </CMSSpreadLegData>
        <NakedOption>N</NakedOption>
      </LegData>
\end{minted}
\caption{CMS Spread leg data}
\label{lst:cmsspreadlegdata}
\end{listing}

The elements of the CMSSpreadLegData block are identical to those of the CMSLegData (see \ref{ss:cmslegdata}), except
for the index which is defined by two CMS indices as the difference between \verb+Index1+ and \verb+Index2+.

\subsubsection{Digital CMS Spread Leg Data}
\label{ss:cmsspreadlegdata}

Listing \ref{lst:digitalcmsspreadlegdata} shows an example for a leg of type DigitalCMSSpread.

\begin{listing}[H]
%\hrule\medskip
\begin{minted}[fontsize=\footnotesize]{xml}
      <LegData>
        <LegType>DigitalCMSSpread</LegType>
        <Payer>false</Payer>
        <Currency>GBP</Currency>
        <Notionals>
          <Notional>10000000</Notional>
        </Notionals>
        <DayCounter>ACT/ACT</DayCounter>
        <PaymentConvention>Following</PaymentConvention>
        <ScheduleData>
          ...
        </ScheduleData>
        <DigitalCMSSpreadLegData>
            <CMSSpreadLegData>
              <Index1>EUR-CMS-10Y</Index1>
              <Index2>EUR-CMS-2Y</Index2>
              <Spreads>
                <Spread>0.0010</Spread>
              </Spreads>
              <Gearings>
                <Gearing>8.0</Gearing>
              </Gearings>
              <NakedOption>N</NakedOption>
            </CMSSpreadLegData>
            <CallPosition>Long</CallPosition>
            <IsCallATMIncluded>false</IsCallATMIncluded>
            <CallStrikes>
                <Strike>0.0001</Strike>
            </CallStrikes>
            <CallPayoffs>
                <Payoff>0.0001</Payoff>
            </CallPayoffs>
            <PutPosition>Long</PutPosition>
            <IsPutATMIncluded>false</IsPutATMIncluded>
            <PutStrikes>
                <Strike>0.001</Strike>
            </PutStrikes>
            <PutPayoffs>
                <Payoff>0.001</Payoff>
            </PutPayoffs>
        </DigitalCMSSpreadLegData>
      </LegData>
\end{minted}
\caption{Digital CMS Spread leg data}
\label{lst:cmsspreadlegdata}
\end{listing}

The DigitalCMSLegData block contains the following elements:

\begin{itemize}
\item CMSLegData: a CMSSpreadLegData block describing the underlying cms spread leg (see \ref{ss:cmsspreadlegdata}.
\item CallPosition: Specifies whether the call option position is long  or
  short.
\item IsCallATMIncluded: inclusion flag on the call payoff if the call option ends at-the-money
\item CallStrikes: strike rate for the the call option
\item CallPayoffs: digital call option payoff rate. If included the option is cash-or-nothing, if excluded the option is asset-or-nothing
\item PutPosition: Specifies whether the put option position is long  or short.
\item IsPutATMIncluded: inclusion flag on the put payoff if the put option ends at-the-money
\item PutStrikes: strike rate for the the put option
\item PutPayoffs: digital put option payoff rate. If included the option is cash-or-nothing, if excluded the option is asset-or-nothing
\end{itemize}


\subsubsection{Equity Leg Data}
\label{ss:equitylegdata}

Listing \ref{lst:equitylegdata} shows an example of a leg of type Equity. The EquityLegData block contains the following
elements:

\begin{itemize}
\item ReturnType: \emph{Price} indicates that the coupons on the equity leg are determined by the price movement of the underlying equity, whereas  \emph{Total} indicates that coupons are determined by the total return of the underlying equity including dividends.

Allowable values:  \emph{Price} or  \emph{Total}

\item Name: The identifier of the underlying equity or equity index.
 
 Allowable values: Any string (provided it is an ID of an equity in the market configuration). Typically an ISIN-code with the \emph{ISIN:} prefix.
%\item Quantity: The number of units of the underlying equity.

%Allowable values:  Any positive real number.
\end{itemize} 

\begin{listing}[H]
%\hrule\medskip
\begin{minted}[fontsize=\footnotesize]{xml}
      <LegData>
        <LegType>Equity</LegType>
        <Payer>false</Payer>
        <Currency>EUR</Currency>
        <Notionals>
          <Notional>10000000</Notional>
        </Notionals>
        <DayCounter>ACT/ACT</DayCounter>
        <PaymentConvention>Following</PaymentConvention>
        <ScheduleData>
          <Rules>
            <StartDate>2016-03-01</StartDate>
            <EndDate>2036-03-01</EndDate>
            <Tenor>3M</Tenor>
            <Calendar>TARGET</Calendar>
            <Convention>ModifiedFollowing</Convention>
            <TermConvention>ModifiedFollowing</TermConvention>
            <Rule>Forward</Rule>
            <EndOfMonth/>
            <FirstDate/>
            <LastDate/>
          </Rules>
        </ScheduleData>
        <EquityLegData>
          <ReturnType>Price</ReturnType>
          <Name>ISIN:US78378X1072</Name>
        </EquityLegData>
      </LegData>
\end{minted}
\caption{Equity leg data}
\label{lst:equitylegdata}
\end{listing}





\subsubsection{CPI Leg Data}
\label{ss:cpilegdata}

Listing \ref{lst:cpilegdata} shows an example for a leg of type CPI. The CPILegData block contains the following
elements:

\begin{itemize}
\item Index: The underlying zero inflation index.

Allowable values:  Any string (provided it is the ID of an inflation index in the market configuration).
\item Rates: The fixed real rate(s) of the leg. As usual, this can be a single value, a vector of values or a dated vector of
  values.
 
 Allowable values: Each rate element can take any  real number. The rate is
  expressed in decimal form, e.g. 0.05 is a rate of 5\%.
\item BaseCPI: The base CPI used to determine the lifting factor for the fixed coupons.

Allowable values:  Any positive real number.
\item ObservationLag: The observation lag to be applied.

Allowable values: An integer followed by \emph{D}, \emph{W}, \emph{M} or \emph{Y}. Interpolation lags are typically expressed in \emph{M}, months.
\item Interpolated: A flag indicating whether interpolation should be applied to inflation fixings. 

Allowable values:  \emph{true, false} 
\item SubtractInflationNotional [Optional]: A flag indicating whether the non-inflation adjusted notional amount should be subtracted from the the final inflation-adjusted notional exchange at maturity.  Note that the final coupon payment is not affected by this flag. \\ Final notional payment if \emph{true}: $N \frac{CPI_T}{CPI_Base}-N$. \\ If  \emph{false}: $N \frac{CPI_T}{CPI_Base}$ 

Allowable values:  \emph{true, false} 
\\Defaults to \emph{false}  if left blank or omitted.
\end{itemize} 

\begin{listing}[H]
%\hrule\medskip
\begin{minted}[fontsize=\footnotesize]{xml}
      <LegData>
        <LegType>CPI</LegType>
        <Payer>false</Payer>
        <Currency>GBP</Currency>
        <Notionals>
          <Notional>10000000</Notional>
        </Notionals>
        <DayCounter>ACT/ACT</DayCounter>
        <PaymentConvention>Following</PaymentConvention>
        <ScheduleData>
          <Rules>
            <StartDate>2016-07-18</StartDate>
            <EndDate>2021-07-18</EndDate>
            <Tenor>1Y</Tenor>
            <Calendar>UK</Calendar>
            <Convention>ModifiedFollowing</Convention>
            <TermConvention>ModifiedFollowing</TermConvention>
            <Rule>Forward</Rule>
            <EndOfMonth/>
            <FirstDate/>
            <LastDate/>
          </Rules>
        </ScheduleData>
        <CPILegData>
          <Index>UKRPI</Index>
          <Rates>
            <Rate>0.02</Rate>
          </Rates>
          <BaseCPI>210</BaseCPI>
          <ObservationLag>2M</ObservationLag>
          <Interpolated>false</Interpolated>
        </CPILegData>
      </LegData>
\end{minted}
\caption{CPI leg data}
\label{lst:cpilegdata}
\end{listing}

\subsubsection{YY Leg Data}
\label{ss:yylegdata}

Listing \ref{lst:yylegdata} shows an example for a leg of type YY. The YYLegData block contains the following
elements:

\begin{itemize}
\item Index: The underlying zero inflation index.

Allowable values:  Any string (provided it is the ID of an inflation index in the market configuration).
\item FixingDays: The number of fixing days.

Allowable values: An integer followed by \emph{D},
\item ObservationLag: The observation lag to be applied.

Allowable values: An integer followed by \emph{D}, \emph{W}, \emph{M} or \emph{Y}. Interpolation lags are typically expressed in \emph{M}, months.
\item Interpolated: A flag indicating whether interpolation should be applied to inflation fixings.

Allowable values:  \emph{true, false} 
\end{itemize}

\begin{listing}[H]
%\hrule\medskip
\begin{minted}[fontsize=\footnotesize]{xml}
      <LegData>
        <LegType>YY</LegType>
        <Payer>false</Payer>
        <Currency>EUR</Currency>
        <Notionals>
          <Notional>10000000</Notional>
        </Notionals>
        <DayCounter>ACT/ACT</DayCounter>
        <PaymentConvention>Following</PaymentConvention>
        <ScheduleData>
          <Rules>
            <StartDate>2016-07-18</StartDate>
            <EndDate>2021-07-18</EndDate>
            <Tenor>1Y</Tenor>
            <Calendar>UK</Calendar>
            <Convention>ModifiedFollowing</Convention>
            <TermConvention>ModifiedFollowing</TermConvention>
            <Rule>Forward</Rule>
            <EndOfMonth/>
            <FirstDate/>
            <LastDate/>
          </Rules>
        </ScheduleData>
        <YYLegData>
          <Index>EUHICPXT</Index>
          <FixingDays>2</FixingDays>
          <ObservationLag>2M</ObservationLag>
          <Interpolated>true</Interpolated>
        </YYLegData>
      </LegData>
\end{minted}
\caption{YY leg data}
\label{lst:yylegdata}
\end{listing}

