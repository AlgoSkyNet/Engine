\documentclass[12pt, a4paper]{article}

\usepackage{supertabular}
\usepackage{hyperref}

\usepackage{minted}
\setminted{
   frame=lines,
   framesep=2mm
}

\addtolength{\textwidth}{0.8in}
\addtolength{\oddsidemargin}{-.4in}
\addtolength{\evensidemargin}{-.4in}
\addtolength{\textheight}{1.6in}
\addtolength{\topmargin}{-.8in}


\begin{document}

\title{Ester / Daily Spread Curve Setup in ORE}
\date{\today}
\maketitle

\newpage

%-------------------------------------------------------------------------------
\section*{Document History}

\begin{center} 
\begin{supertabular}{|l|l|l|p{7cm}|}
\hline
Version & Date & Author & Comment \\ 
\hline 
0.1 & 30 September & Peter Caspers & initial version\\
\hline
\end{supertabular}
\end{center}

\vspace{3cm}

\newpage
%\tableofcontents
%\newpage


%-------------------------------------------------------------------------------
\section{Introduction}

This document describes the setup of a rate curve in ORE with daily forward rates that are computed as a spread over the
daily forward rates of a reference yield curve. The main use case is to build an Ester curve from an Eonia curve using
the (constant) Ester-Eonia spread defined by the ECB (-8.5 bp), see \cite{ECB_EsterSpread}, for asof dates where Eonia
swaps are still liquid and no liquid Ester derivative market has developed yet.

\section{Derivation of an equivalent zero spread}

In the following derivation we will assume that the daily forward rates are compounding on calendar days which is an
approximation to reality where the comopounding is linear over holidays. The error introduced by this is analysed in
section \ref{approx_error}.

We write the discount factors of the reference curve in terms of simply compounded daily forwards $f_i$, i.e.

\begin{equation}
P(0,d) = \prod_{i=0}^{n-1} \frac{1}{1+\tau_i f_i}
\end{equation}

where $d$ is a future date, $i$ is the number of calendar days between the asof date and $d$ and $\tau_i$ is the year
fraction between $d$ and the next calendar day. A typical choice for the day count convnention is Actual/360 in which
case $\tau_i = \tau = 1 / 360$ is independent of $i$. The corresponding discount factor on the spreaded curve can be
written as

\begin{equation}
\label{spread_df}
P^s(0,d) = \prod_{i=0}^{n-1} \frac{1}{1+\tau_i (f_i + s)}
\end{equation}

where $s$ denotes the spread. For an efficient implementation is it desirable to avoid the explicit computation of
formula \ref{spread_df} and write $P^s(0,d)$ as a simple function of $P(0,d)$ and $s$. We have

\begin{eqnarray*}
  \frac{1}{1+\tau_i (f_i + s)} = \frac{1+\tau_i f_i}{1+\tau_i (f_i + s)} \frac{1}{1+\tau_i f_i}
\end{eqnarray*}

A first order approximation of the first factor around $f_i=0$ is given by

\begin{eqnarray*}
  \frac{1+\tau_i f_i}{1+\tau_i (f_i + s)} \approx \frac{1}{1+\tau_i s}
\end{eqnarray*}

so that

\begin{equation}
 \label{spread_df_approx}
  P^s(0,d) \approx \prod_{i=0}^{n-1} \frac{1}{1+\tau_i f_i} \frac{1}{1+\tau_i s} = P(0,d) \prod_{i=0}^{n-1} \frac{1}{1+\tau_i s}
\end{equation}

which reduces the calculation of the discount factor on the spreaded curve to the right factor of the RHS in formula
\ref{spread_df_approx} and the discount factor of the reference curve. The former factor can easily be computed in
closed form if $\tau_i = \tau$ independent of $i$, which is the case for the important case Actual/360, but also for
Actual/365 (Fixed).\footnote{For Actual/Actual (ISDA) the product can be split into products covering one year and
  within each year $\tau_i$ is constant. For other day counters the product can be evaluated explicitly or one can
  approximate $\tau_i \approx T / n$ where $T$ is the year fraction between the asof date and T.}

Assuming $\tau_i = \tau$ independent of $i$ we have

\begin{equation}
  P^s(0,d) \approx P(0,d) \left( \frac{1}{1+\tau s} \right)^n
\end{equation}

Now assume that the time on the reference curve is measured using a day count convention that yields daily year
fractions $\tau'_i = \tau'$ independent of $i$ but possibly different from the day count convention associated to the
daily forward rate above. A typical choice for rate curve day counter is Actual/365 (Fixed). The number $n$ of calendar
dates between the asof date and $d$ is then given by $n = t / \tau'$, so that

\begin{equation}
  P^s(0,t) \approx P(0,t) \left( \frac{1}{1+\tau s} \right)^{t / \tau'}
\end{equation}

From this we can easily back out a continuously compounded spread $s_c$ w.r.t. the curve day counter by equating

\begin{equation}
  P(0,t) \left( \frac{1}{1+\tau s} \right)^{t / \tau'} = P(0,t) e^{-s_c t} 
\end{equation}

which yields

\begin{equation}\label{cont_spread}
 s_c = \frac{\ln (1+\tau s)}{\tau'}
\end{equation}

\section{Ester Spread Curve Setup}

Assuming a rate curve day counter Actual/365 (Fixed) for the Eonia curve and a simply compounded Actual/360 spread
$s = -0.0085$ between Eonia and Ester we compute $s_c = -0.0008618$ using formula \ref{cont_spread}. Given the Eonia curve as

\begin{minted}[fontsize=\footnotesize]{xml}
<YieldCurve>
  <CurveId>EUR-EONIA</CurveId>
  <CurveDescription>EUR curve bootstrapped from OIS swap rates</CurveDescription>
  <Currency>EUR</Currency>
  ...
  <YieldCurveDayCounter>A365</YieldCurveDayCounter>
</YieldCurve>
\end{minted}

We can therefore setup the Ester curve in ORE as

\begin{minted}[fontsize=\footnotesize]{xml}
<YieldCurve>
  <CurveId>EUR-Ester</CurveId>
  <CurveDescription>Ester curve as spread over Eonia</CurveDescription>
  <Currency>EUR</Currency>
  <DiscountCurve/>
  <Segments>
    <ZeroSpread>
      <Type>Zero Spread</Type>
      <Quotes>
        <Quote>ZERO/YIELD_SPREAD/EUR/Ester_Eonia_Spread/A365/1Y</Quote>
        <Quote>ZERO/YIELD_SPREAD/EUR/Ester_Eonia_Spread/A365/2Y</Quote>
      </Quotes>
      <Conventions>EUR-ZERO-TENOR-BASED</Conventions>
      <ReferenceCurve>EUR-EONIA</ReferenceCurve>
    </ZeroSpread>
  </Segments>
</YieldCurve>
\end{minted}

with conventions

\begin{minted}[fontsize=\footnotesize]{xml}
<Zero>
  <Id>EUR-ZERO-TENOR-BASED</Id>
  <TenorBased>true</TenorBased>
  <DayCounter>A365</DayCounter>
  <Compounding>Continuous</Compounding>
  <CompoundingFrequency>Daily</CompoundingFrequency>
  <TenorCalendar>TARGET</TenorCalendar>
  <SpotLag>2</SpotLag>
  <SpotCalendar>TARGET</SpotCalendar>
  <RollConvention>Following</RollConvention>
  <EOM>false</EOM>
</Zero>
\end{minted}

and (fixed) market data

\begin{minted}[fontsize=\footnotesize]{xml}
YYYYMMDD ZERO/YIELD_SPREAD/EUR/Ester_Eonia_Spread/A365/1Y -0.0008618
YYYYMMDD ZERO/YIELD_SPREAD/EUR/Ester_Eonia_Spread/A365/2Y -0.0008618
\end{minted}

\section{Approximation error analysis}\label{approx_error}

We continue with an analysis of the approximation error in formula \ref{spread_df_approx}. For this we focus on the case
where $\tau_i$ is independent of $i$. Write

\begin{equation}
  g(x) = \frac{1+\tau x}{1+\tau (x + s)}
\end{equation}

Expanding $g$ around $x=0$ to the first order using a Langrange error term we get

\begin{equation}
  g(x) = g(0) + g'(\xi) x
\end{equation}

with $\xi$ between $0$ and $x$. We have

\begin{equation}
  g'(x) = \frac{\tau^2s}{(1+\tau(x+s))^2}
\end{equation}

To bound the approximation error $\alpha := g(x) - g(0)$ we assume that the the forward rate $x$ lies in $[f_0, f_1]$
and $s$ lies in $[s_0, s_1]$, such that $1+\tau (f_0 + s_0) > 0$. The latter restriction is of no practical relevance of
course.

The approximation error is then bounded by

\begin{equation}
  | \alpha | =| g(x) - g(0) | \leq \frac{\tau^2 \max(|s_0|, |s_1|)}{(1+\tau(f_0+s_0))^2} \max(|f_0|, |f_1|)
\end{equation}

If $r(d)$ is the continuous compounded zero rate for a date $d$ we have

\begin{equation}
 r(d) = - \frac{\ln P(0,d)}{t}
\end{equation}

with $t$ the yearfraction between the asof date and $d$ using the day count convention used to compute $\tau$. We have

\begin{equation}
P^s(0,d) = P(0,d) \prod_{i=0}^{n-1} (g(0) + \alpha) = P(0,d) \prod_{i=0}^{n-1} g(0) \left( 1 + \frac{\alpha}{g(0)} \right)
\end{equation}

Since $P(0,d) \prod_{i=0}^{n-1} g(0)$ is our approximation for $P^s(0,d)$, the approxaimation error in the zero rate
$r(d)$ is

\begin{equation}
-\frac{\ln \prod_{i=0}^{n-1}  \left( 1 + \frac{\alpha}{g(0)} \right)}{t}
\end{equation}

and since $t = n \tau$ and $g(0) = 1 / ( 1 + \tau s )$

\begin{equation}
 -\frac{ \ln ( 1 + \alpha ( 1+\tau s ))} { \tau }
\end{equation}

which is bounded by

\begin{equation}\label{error_bound}
  \frac{1}{\tau} \ln \left( 1 + \frac{\tau^2 \max(|s_0|, |s_1|)}{(1+\tau(f_0+s_0))^2} \max(|f_0|, |f_1|)  ( 1+\tau s_1 ) \right)
\end{equation}

The error bound for some choices of $f_0, f_1$ and $s_0, s_1$ is summarised in table \ref{sample_error_bounds}. In the
Ester use case the error is bounded by around $0.01$ basis points.

\begin{table}[!htbp]
  \centering
  \begin{tabular}{l | l | r}
    $[f_0, f_1]$ & $[s_0, s_1]$ & max error (bp) \\ \hline
    $[-0.1, 0.1]$ & $[-0.1, 0.1]$ &  0.278 \\
    $[-0.1, 0.1]$ & $[-0.01, 0.01]$ &  0.028 \\
    $[-0.05, 0.05]$ & $[-0.01, 0.01]$ &  0.014 \\
    $[-0.05, 0.05]$ & $[-0.0085, -0.0085]$ &  0.012 \\
  \end{tabular}
  \caption{Error bounds for the zero rate depending on prescribed ranges for the daily forwards and the spread.}
  \label{sample_error_bounds}
\end{table}

There is another source of approximation error introduced by the assumption that the daily compounding takes place on
each calendar day while in reality the compounding is linear over holidays, which are typically the 3 day periods over
weekends. The effective error introduced by this for the daily fixings on Fridays are displayed for some rate levels in
table \ref{sample_error_weekends}. Notice that they should be weighted by $1/5$ since they only occur on $1$ out of $5$
days. Also notice that since we use the {\em exact} discount factor in our approximation for $P^s$ in formula
\ref{spread_df_approx} the error only applies to the spread part of the forward rate. In any case the error lies in the
same order of magnitude as the error listed in table \ref{sample_error_bounds}.

\begin{table}[!htbp]
  \centering
  \begin{tabular}{l | r}
    $f$ & error (bp) \\ \hline
    -0.1 & 0.278 \\
    -0.05 & 0.069 \\
    -0.0085 & 0.002 \\
    0.05 & 0.069 \\
    0.1 & 0.278 \\
  \end{tabular}
  \caption{Error bounds for daily forwards originating from linear vs. daily compounding over weekends.}
  \label{sample_error_weekends}
\end{table}

\pagebreak

\begin{thebibliography}{*}
\bibitem{ECB_EsterSpread} \url{https://www.ecb.europa.eu/press/pr/date/2019/html/ecb.pr190531~a3788de8f8.en.html}
\end{thebibliography}
  
\end{document}
