%========================================================
\section{Netting Set Definitions}\label{sec:nettingsetinput}
%========================================================

The netting set definitions file - {\tt netting.xml} - 
contains a list of
definitions for various ISDA netting agreements. The file is written
in XML format. 

\vspace{1em}

Each netting set is defined within its own \lstinline!NettingSet!
node. All of these \lstinline!NettingSet! nodes are contained as
children of a \lstinline!NettingSetDefinitions! node.

\vspace{1em}

There are two distinct cases to consider:

\begin{itemize}
\item An ISDA agreement which does not contain a \emph{Credit Support
    Annex} (CSA).
\item An ISDA agreement which does contain a CSA.
\end{itemize}
%- - - - - - - - - - - - - - - - - - - - - - - - - - - - - - - - - - - - - - - -
\subsection{Uncollateralised Netting Set}
%- - - - - - - - - - - - - - - - - - - - - - - - - - - - - - - - - - - - - - - -
If an ISDA agreement does not contain a Credit Support Annex, the
portfolio exposures are not eligible for collateralisation. In such a
case the netting set can be defined within the following XML template:

\begin{listing}[H]
%\hrule\medskip
\begin{minted}[fontsize=\footnotesize]{xml}
    <NettingSet>
        <NettingSetId> </NettingSetId>
        <ActiveCSAFlag> </ActiveCSAFlag>
        <CSADetails></CSADetails>
    </NettingSet>
\end{minted}
\caption{Uncollateralised netting set definition}
\label{lst:nettingSetUncollat}
\end{listing}

The meanings of the various elements are as follows:
\begin{itemize}
\item NettingSetId: The unique identifier for the ISDA netting set.
\item ActiveCSAFlag: Boolean indicating whether the netting set is
  covered by a Credit Support Annex. For uncollateralised netting sets
  this flag should be \emph{False}.
\item CSADetails: Node containing as children details of the governing
  Credit Support Annex. For uncollateralised netting sets there is no
  need to store any information within this node.
\end{itemize}
%- - - - - - - - - - - - - - - - - - - - - - - - - - - - - - - - - - - - - - - -
\subsection{Collateralised Netting Set}
%- - - - - - - - - - - - - - - - - - - - - - - - - - - - - - - - - - - - - - - -
If an ISDA agreement contains a Credit Support Annex, the
portfolio exposures are eligible for collateralisation. In such a
case the netting set can be defined within the following XML template:

\begin{listing}[H]
%\hrule\medskip
\begin{minted}[fontsize=\footnotesize]{xml}
    <NettingSet>
        <NettingSetId> </NettingSetId>
        <ActiveCSAFlag> </ActiveCSAFlag>
        <CSADetails>
            <Bilateral> </Bilateral>
            <CSACurrency> </CSACurrency>
            <Index> </Index>
            <ThresholdPay> </ThresholdPay>
            <ThresholdReceive> </ThresholdReceive>
            <MinimumTransferAmountPay> </MinimumTransferAmountPay>
            <MinimumTransferAmountReceive> </MinimumTransferAmountReceive>
            <IndependentAmount>
                <IndependentAmountHeld> </IndependentAmountHeld>
                <IndependentAmountType> </IndependentAmountType>
            </IndependentAmount>
            <MarginingFrequency>
                <CallFrequency> </CallFrequency>
                <PostFrequency> </PostFrequency>
            </MarginingFrequency>
            <MarginPeriodOfRisk> </MarginPeriodOfRisk>
            <CollateralCompoundingSpreadReceive> 
            </CollateralCompoundingSpreadReceive>
            <CollateralCompoundingSpreadPay> </CollateralCompoundingSpreadPay>
            <EligibleCollaterals>
                <Currencies>
                    <Currency>USD</Currency>
                    <Currency>EUR</Currency>
                    <Currency>CHF</Currency>
                    <Currency>GBP</Currency>
                    <Currency>JPY</Currency>
                    <Currency>AUD</Currency>
                </Currencies>
            </EligibleCollaterals>
            <ApplyInitialMargin>Y</ApplyInitialMargin>
            <InitialMarginType>Bilateral</InitialMarginType>
            <CalculateIMAmount>true</CalculateIMAmount>
            <CalculateVMAmount>true</CalculateVMAmount>
        </CSADetails>
    </NettingSet>
\end{minted}
\caption{Collateralised netting set definition}
\label{lst:nettingSetCollat}
\end{listing}

The first few nodes are shared with the template for uncollateralised
netting sets:
\begin{itemize}
\item NettingSetId: The unique identifier for the ISDA netting set.
\item ActiveCSAFlag: Boolean indicating whether the netting set is
  covered by a Credit Support Annex. For collateralised netting sets
  this flag should be \emph{True}.
\item CSADetails: Node containing as children details of the governing
  Credit Support Annex. 
\end{itemize}

\subsubsection*{CSADetails}

The \lstinline!CSADetails! node contains details of the Credit Support
Annex which are relevant for the purposes of exposure calculation. The
meanings of the various elements are as follows:

\paragraph*{Bilateral} There are three possible values here:
\begin{itemize}
\item Bilateral: Both parties to the CSA are legally entitled to
  request collateral to cover their counterparty credit risk exposure
  on the underlying portfolio.
\item CallOnly: Only we are entitled to hold collateral; the
  counterparty has no such entitlement.
\item PostOnly: Only the counterparty is entitled to hold collateral;
  we have no such entitlement.
\end{itemize}

\paragraph*{CSACurrency} A three-letter ISO code specifying the master
currency of the CSA. All monetary values specified within the CSA are
assumed to be denominated in this currency.

\paragraph*{Index} The index is used to derive the fixing which is used
for compounding cash collateral in the master currency of the CSA.

\paragraph*{ThresholdPay} A threshold amount above which the
counterparty is entitled to request collateral to cover excess
exposure.

\paragraph*{ThresholdReceive} A threshold amount above which we are
entitled to request collateral from the counterparty to cover excess
exposure.

\paragraph*{MinimumTransferAmountPay} Any margin calls issued by the
counterparty must exceed this minimum transfer amount. If the
collateral shortfall is less than this amount, the counterparty is not
entitled to request margin.

\paragraph*{MinimumTransferAmountReceive} Any margin calls issued by us
to the counterparty must exceed this minimum transfer amount. If the
collateral shortfall is less than this amount, we are  not
entitled to request margin.

\paragraph*{IndependentAmount} This element contains two child nodes:
\begin{itemize}
\item IndependentAmountHeld: The netted sum of all independent amounts
  covered by this ISDA agreement/CSA. A negative number implies that
  the counterparty holds the independent amount.
\item IndependentAmountType: The nature of the independent amount as
  defined within the Credit Support Annex. The only supported value
  here is \emph{FIXED}. 
\end{itemize}
This covers only the case where only one party has to post an
independent amount. In a future release this will be extended to the
situation prescribed by the Basel/IOSCO regulation (initial margin to
be posted by both parties without netting).

\paragraph*{MarginingFrequency} This element contains two child nodes:
\begin{itemize}
\item CallFrequency: The frequency with which we are entitled to
  request additional margin from the counterparty (e.g. \emph{1D},
  \emph{2W}, \emph{1M}).
\item PostFrequency: The frequency with which the counterparty is entitled to
  request additional margin from us (e.g. \emph{1D},
  \emph{2W}, \emph{1M}).
\end{itemize}

\paragraph*{MarginPeriodOfRisk} The length of time assumed necessary
for closing out the portfolio position after a default event  (e.g. \emph{1D},
  \emph{2W}, \emph{1M}).

\paragraph*{CollateralCompoundingSpreadReceive} The spread over the O/N
interest accrual rate taken by the clearing house, when holding
collateral.

\paragraph*{CollateralCompoundingSpreadPay} The spread over the O/N
interest accrual rate taken by the clearing house, when collateral is
held by the counterparty.

\paragraph*{EligibleCollaterals} For now the only supported type of
collateral is cash. If the CSA specifies a set of currencies which
are eligible as collateral, these can be listed using
\lstinline!Currency! nodes.

\paragraph*{ApplyInitialMargin} Apply (dynamic) initial Margin in
addition to variation margin

\paragraph*{InitialMarginType} There are three possible values here:
\begin{itemize}
\item Bilateral: Both parties to the CSA are legally entitled to
  request collateral to cover their MPOR risk exposure
  on the underlying portfolio.
\item CallOnly: Only we are entitled to hold collateral; the
  counterparty has no such entitlement.
\item PostOnly: Only the counterparty is entitled to hold collateral;
  we have no such entitlement.

\paragraph*{CalculateIMAmount} Boolean indicating whether to calculate
  initial margin from SIMM. For uncollateralised netting sets
  this flag will be ignored. This only applies to the SA-CCR capital
  calculations.

\paragraph*{CalculateVMAmount} Boolean indicating whether to calculate
  variation margin from the netting set NPV. For uncollateralised netting
  sets this flag will be ignored. This only applies to the SA-CCR capital
  calculations.
\end{itemize}