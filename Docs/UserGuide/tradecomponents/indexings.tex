\subsubsection{Indexings}
\label{ss:indexings}

This trade component can be used as an optional node within the \lstinline!LegData! component to scale the notional of the coupons
of a leg by one or several index prices. This feature is typically used within equity swaps with notional reset to align
the notional of the funding leg with the one from the equity leg for all return periods. See \ref{ss:equity_swap} for
the specific usage in equity swaps. Notice that typically it is sufficient to set the \lstinline!FromAssetLeg! flag to \emph{true} in the
\verb+Indexings+ node definition, i.e.

\begin{minted}[fontsize=\footnotesize]{xml}
<LegData>
    <LegType>Floating</LegType>
    <!-- no notionals node required -->
    <ScheduleData> ... </ScheduleData>
    <Indexings>
        <FromAssetLeg>true</FromAssetLeg>
    </Indexings>
    <FloatingLegData> ... </FloatingLegData>
</LegData>
\end{minted}

which will cause the trade builder to pull all the indexing details from the asset leg (the equity leg in an equity
swap) and populate the indexing data on the funding leg accordingly. Notice that no definition of a \verb+Notionals+
node is required in this case, this will also generated automatically.

In what follows we will describe the full syntax of the Indexings node below for reference. The Indexing component can
be used in combination with the following leg types:

\begin{itemize}
\item Fixed
\item Floating
\item CMS
\item DigitalCMS
\item CMSSpread
\item DigitalCMSSpread
\end{itemize}

If specified the notional of the single coupons in the leg is scaled by one or several index prices and a quantity. The
indices can be equity or FX indices. Notice that if notional exchanges are enabled on a leg with \lstinline!Indexings! defined, the
notional exchanges are {\em not} influenced by the indexing definitions. In general we assume that notional exchanges
are not enabled in combination with indexings, but it is not forbidden technically. Listing \ref{lst:indexings} shows an
example of an Floating leg indexed by both an equity price and a FX rate.

\begin{listing}[H]
%\hrule\medskip
\begin{minted}[fontsize=\footnotesize]{xml}
<LegData>
    <LegType>Floating</LegType>
    <Notionals> ... </Notionals>
    <ScheduleData> ... </ScheduleData>
    <Indexings>
        <FromAssetLeg>false</FromAssetLeg>
        <Indexing>
            <Quantity>1000</Quantity>
            <Index>EQ-RIC:.STOXX50E</Index>
            <InitialFixing>2937.36</InitialFixing>
            <ValuationSchedule>
              <Dates>...</Dates>
              <Rules>...</Rules>
            </ValuationSchedule>
            <FixingDays>0</FixingDays>
            <FixingCalendar/>
            <FixingConvention>U</FixingConvention>
            <IsInArrears>false</IsInArrears>
        </Indexing>
        <Indexing>
            <Index>FX-ECB-EUR-USD</Index>
            <IndexFixingDays>2</IndexFixingDays>
            <IndexFixingCalendar>EUR,USD</IndexFixingCalendar>
            <InitialFixing>1.1469</InitialFixing>
            <ValuationSchedule> ... </ValuationSchedule>
            <FixingDays>0</FixingDays>
            <FixingCalendar/>
            <FixingConvention>U</FixingConvention>
            <IsInArrears>false</IsInArrears>
        </Indexing>
    </Indexings>
    <FloatingLegData> ... </FloatingLegData>
</LegData>
\end{minted}
\caption{Indexings node}
\label{lst:indexings}
\end{listing}

The Indexings node contains the following elements:

\begin{itemize}
\item \lstinline!FromAssetLeg! [Optional]: If \emph{true}, and the trade type supports this, the notionals on the funding leg (i.e. the leg with the \lstinline!FromAssetLeg! field) will be derived from the respective asset leg.  Internally, the trade builder will add \verb+Indexing+ blocks automatically reflecting the necessary indexings (equity price and FX in the  case of an equity swap) from the notional reset feature of the asset leg. Also, the Notionals node of the funding leg will internally be set to a single notional $1$.  The actual Notionals node in the XML on the funding leg is not required and can be omitted. 
  
\lstinline!FromAssetLeg! is supported for the following trade types:

\begin{itemize}
\item \lstinline!EquitySwap!: Setting \lstinline!FromAssetLeg! to \emph{true}, aligns the notionals for all return periods on the non-equity funding leg, to the equity leg by deriving equity price, quantity and FX from the equity leg. \\ Note that \lstinline!FromAssetLeg! is only supported if \lstinline!NotionalReset! is \emph{true} on the equity leg - \lstinline!FromAssetLeg! is ignored otherwise.
\item \lstinline!BondTRS!: Setting \lstinline!FromAssetLeg! to \emph{true}, aligns the notionals for all return periods on the funding leg (in the \lstinline!FundingData! block), to the total return leg (in the \lstinline!TotalReturnData! block) by deriving bond price, bond notional and FX from the total return leg, bond data and the reference bond. 
\end{itemize}

  
    Allowable values: \emph{true}, \emph{false}. Defaults to \emph{false} if left blank or omitted. 
  
\item \lstinline!Indexing! [Optional, an arbitrary number can be given]: Each Indexing node describes one indexing as follows:

\begin{itemize}

\item Quantity [Optional]: The quantity that applies. For equity that should be the number of shares, for
  FX it should be 1, i.e. for FX this field can be omitted. The notional of each coupon is
  in general determined as\\
  Original Coupon Notional x Quantity x Equity Price x FX Rate\\
  depending on which indexing types are given. Typically, the original coupon notional will be set to 1.

  Allowable values: Any number. Defaults to \emph{1} if left blank or omitted.

\item Index: The relevant index. This is either an equity or FX index. For an FX index, one of the currencies of the
  index must match the leg currency. It is then ensured that the FX conversion is applied using the correct direction,
  i.e. if the foreign currency of the index matches the leg currency, the reciprocals of the index fixings are used as a
  multiplier.

  Allowable values: This is ``FX-SOURCE-CCY1-CCY'' for FX, see \ref{ss:underlying} and \ref{tab:fxindex_data} for
  details, or ``EQ-NAME'' for Equity with ``Name'' being the general string representation for equity underlyings
  {IdentifierType}:{Name}:{Currency}:{Exchange}, see \ref{ss:underlying}.

\item IndexFixingDays [Optional]: fixing days of the index, this is only used for FX indices

  Allowable values: Non-negative whole numbers. Defaults to \emph{0} if left blank or omitted.

\item IndexFixingCalendar [Optional]: Fixing calendar of the index, this is only used for FX indices and Bond indices

  Allowable values: See Table \ref{tab:calendar} Calendar. Defaults to the \emph{NullCalendar} (no holidays) if left blank or omitted.

\item Dirty [Optional]: Only used for bond indices. Indicates whether to use dirty (\emph{true}) or clean
  (\emph{false}) prices.

  Allowable values: \emph{true}, \emph{false}. Defaults to \emph{true} if left blank or omitted. 

\item Relative [Optional]: Only used for bond indices. Indicates whether to use relative (\emph{true}) or
  absolute prices (\emph{false}). The absolute price is the dirty or clean npv as of the settlement date of the bond in
  absolute ``dollar'' terms using the bond details (in particular the notional) from the reference data. The relative
  price is the absolute price divided by the current notional as of the settlement date.

  Allowable values: \emph{true}, \emph{false}. Defaults to \emph{true} if left blank or omitted. 

\item ConditionalOnSurvival [Optional]: Only used for bond indices. Indicates whether to forecast bond
  prices conditional on survival (\emph{true}) or including the default probability from today until the fixing date (\emph{false}).

  Allowable values: \emph{true}, \emph{false}. Defaults to \emph{true} if left blank or omitted. 

\item InitialFixing [Optional]: If given the index fixing value to apply on the fixing date of the first coupon. If not
  given the value is read from the relevant fixing history.

  Allowable values: any number

\item ValuationSchedule [Optional]: If given the schedule from which the fixing dates are deduced. If not given, it
  defaults to the original leg's schedule.

  If the valuation schedule has the same size as the original leg's schedule,
  it is assumed that the periods correspond one to one, i.e. the $i$th fixing date is derived from the $i$th (inArrears
  = false) or $i+1$th (inArrears = true) date in the valuation schedule using the FixingDays, FixingCalendar and
  FixingConvention.

  If the valuation schedule has a different size than the original leg's schedule, the relevant valuation date for the
  $i$th original leg's coupon is determined as the latest valuation date that is less or equal to accrual start date
  (inArrears = false) resp. accrual end date (inArrears = true) of that coupon. The fixing date is derived from the
  relevant valuation date as above, i.e. using the FixingDays, FixingCalendar and FixingConvention.

  Allowable values: a valid schedule definition, see \ref{ss:schedule_data}

\item FixingDays [Optional]: If given defines the number of fixing days to apply when deriving the fixing
  dates from the valuation schedule (see above).

  Allowable values: Any non-negative whole number. Defaults to \emph{0} if left blank or omitted. 

\item FixingCalendar [Optional, defaults to NullCalendar (no holidays): If given defines the fixing calendar to use when
  deriving the fixing dates from the valuation schedule (see above).

  Allowable values:   Allowable values: See Table \ref{tab:calendar} Calendar. Defaults to the \emph{NullCalendar} (no holidays) if left blank or omitted.

\item FixingConvention [Optional]: If given defines the business day convention to use when
  deriving the fixing dates from the valuation schedule (see above). Defaults to  \emph{Preceding} if left blank or omitted.

  Allowable values: Any valid business day convention, e.g. (\emph{F, MF, P, MP, U}). See \lstinline!Roll Convention! in Table \ref{tab:allow_stand_data}.

\item IsInArrears [Optional]: If \emph{true}, the fixing dates are derived from the period end dates,
  otherwise from the period start dates as described for \lstinline!ValuationSchedule! above.

  Allowable values: \emph{true}, \emph{false}. Defaults to \emph{false} if left blank or omitted. 

\end{itemize}

\end{itemize}
