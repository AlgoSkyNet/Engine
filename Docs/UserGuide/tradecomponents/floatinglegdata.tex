\subsubsection{Floating Leg Data, Spreads, Gearings, Caps and Floors}
\label{ss:floatingleg_data}

The \lstinline!FloatingLegData! trade component node is used within the \lstinline!LegData! trade component when the
\lstinline!LegType! element is set to \emph{Floating}. It is also used directly within the \lstinline!CapFloor! trade
data container.  The \lstinline!FloatingLegData! node includes elements specific to a floating leg.

An example of a \lstinline!FloatingLegData! node is shown in Listing \ref{lst:floatingleg_data}.
\begin{listing}[H]
%\hrule\medskip
\begin{minted}[fontsize=\footnotesize]{xml}
                <FloatingLegData>
                    <Index>USD-LIBOR-3M</Index>
                    <IsInArrears>false</IsInArrears>
                    <IsAveraged>false</IsAveraged>
                    <HasSubPeriods>false</HasSubPeriods>
                    <IncludeSpread>false</IncludeSpread>
                    <FixingDays>2</FixingDays>
                    <Spreads>
                        <Spread>0.005</Spread>
                    </Spreads>
                    <Gearings>
                        <Gearing>2.0</Gearing>
                    </Gearings>
                    <Caps>
                        <Cap>0.05</Cap>
                    </Caps>
                    <Floors>
                        <Floor>0.01</Floor>
                    </Floors>
                    <NakedOption>false</NakedOption>
                    <LocalCapFloor>false</LocalCapFloor>
                </FloatingLegData>
\end{minted}
\caption{Floating leg data}
\label{lst:floatingleg_data}
\end{listing}

The meanings and allowable values of the elements in the \lstinline!FloatingLegData! node follow below.

\begin{itemize}
\item Index:  The combination of currency, index and term that
  identifies the relevant fixings and yield curve of the floating leg.  

  Allowable values: An alphanumeric string of the form CCY-INDEX-TERM. CCY, INDEX and TERM must be separated by dashes (-). CCY and INDEX must be among the supported currency and index combinations. TERM must be an integer followed by D, W,
  M or Y. See Table \ref{tab:indices}. TERM is not required for Overnight indices.

\item IsAveraged [Optional]:  For cases where there are multiple index fixings over a period \emph{true} indicates that
  the average of the fixings is used to calculate the coupon.  \emph{false} indicates that the coupon is calculated by
  compounding the fixings.  IsAveraged only applies to Overnight indices and Sub Periods Coupons.

Allowable values:  \emph{true, false}. Defaults to \emph{false} if left blank or omitted.

\item HasSubPeriods [Optional]: For cases where several Ibor fixings result in a single payment for a period, e.g. if
  the Ibor tenor is 3M and the schedule tenor is 6M, two fixings are used to compute the amount of the semiannual coupon
  payments. \emph{true} indicates that an average (IsAveraged = true) or a compounded (IsAveraged=false) value of the
  fixings is used to determine the payment rate.  \emph{false} indicates that the initial index period fixing determines the payment rate for the 
  full tenor, i.e. no further fixings, no averaging and no compounding. IsAveraged is ignored for Ibor legs when HasSubPeriods is set to  \emph{false}.
  HasSubPeriods does not apply to Overnight indices. 

Allowable values:  \emph{true, false}. Defaults to \emph{false} if left blank or omitted.

\item IncludeSpread [Optional]: Only applies to Sub Periods and (compounded) OIS Coupons. If \emph{true} the spread is
  included in the compounding, otherwise it is excluded.

Allowable values:  \emph{true, false}. Defaults to \emph{false} if left blank or omitted.

A Zero Coupon Floating leg with compounding that includes spread can be set up using a rules-based schedule as shown
in Listing \ref{lst:float_zero_coupon_leg_rules}. Note that the \lstinline!Tenor! in the rules-based schedule is not used when \lstinline!Rule! is set to \emph{Zero}.
\begin{listing}[H]
%\hrule\medskip
\begin{minted}[fontsize=\footnotesize]{xml}
            <LegData>
                <LegType>Floating</LegType>
                <Payer>false</Payer>
                <Currency>USD</Currency>
                <Notionals>
                    <Notional>200000.0000</Notional>
                </Notionals>
                <DayCounter>A360</DayCounter>
                <PaymentConvention>MF</PaymentConvention>
                <ScheduleData>
                    <Rules>
                        <StartDate>2020-01-14</StartDate>
                        <EndDate>2020-07-14</EndDate>
                        <Tenor>3M</Tenor>
                        <Calendar>USD</Calendar>
                        <Convention>MF</Convention>
                        <TermConvention>MF</TermConvention>
                        <Rule>Zero</Rule>
                    </Rules>
                </ScheduleData>
                <FloatingLegData>
                    <Index>USD-LIBOR-3M</Index>
                    <IsAveraged>false</IsAveraged>
                    <HasSubPeriods>true</HasSubPeriods>
                    <IncludeSpread>true</IncludeSpread>
                    <Spreads>
                        <Spread>0.006500</Spread>
                    </Spreads>
                    <IsInArrears>false</IsInArrears>
                    <FixingDays>2</FixingDays>
                </FloatingLegData>
            </LegData>
\end{minted}
\caption{Zero Coupon Floating Leg - Rules-based}
\label{lst:float_zero_coupon_leg_rules}
\end{listing}

A Zero Coupon Floating leg with compounding that includes spread can also be set up using a dates-based schedule with two dates (start and end) as shown
in Listing \ref{lst:float_zero_coupon_leg_dates}. \begin{listing}[H]
%\hrule\medskip
\begin{minted}[fontsize=\footnotesize]{xml}
            <LegData>
                <LegType>Floating</LegType>
                <Payer>false</Payer>
                <Currency>USD</Currency>
                <Notionals>
                    <Notional>200000.0000</Notional>
                </Notionals>
                <DayCounter>A360</DayCounter>
                <PaymentConvention>MF</PaymentConvention>
                <ScheduleData>
                    <Dates>
                        <Calendar>USD</Calendar>
                        <Convention>MF</Convention>
                        <Dates>
                            <Date>2020-01-14</Date>
                            <Date>2020-07-14</Date>
                        </Dates>
                    </Dates>
                </ScheduleData>
                <FloatingLegData>
                    <Index>USD-LIBOR-3M</Index>
                    <IsAveraged>false</IsAveraged>
                    <HasSubPeriods>true</HasSubPeriods>
                    <IncludeSpread>true</IncludeSpread>
                    <Spreads>
                        <Spread>0.006500</Spread>
                    </Spreads>
                    <IsInArrears>false</IsInArrears>
                    <FixingDays>2</FixingDays>
                </FloatingLegData>
            </LegData>
\end{minted}
\caption{Zero Coupon Floating Leg - Dates-based}
\label{lst:float_zero_coupon_leg_dates}
\end{listing}



\item IsInArrears [Optional]:  \emph{true} indicates that  fixing is in arrears,
  i.e. the fixing gap is calculated in relation to the current period
  end date.\\ \emph{false} indicates that  fixing is in advance,
  i.e. the fixing gap is calculated in relation to the previous period
  end date.  

Allowable values:  \emph{true, false}. Defaults to \emph{false} if left blank or omitted.

\item FixingDays [Optional]: The fixing gap. For Ibor coupons this is the number of business days before the accrual
  period's {\em reference} date to observe the index fixing. Here, the accrual period reference date is the accrual
  start date for an in advanced fixed coupon and the accrual end date for in arrears fixed coupon. \\
  For overnight coupons this is the number of business days by which the value dates are shifted into the past to get
  the fixing observation dates. In the context of the Ibor reform the FixingDays parameter is sometimes also called
  ``obervation lag''.

  Allowable values: A non-negative whole number.  Defaults to the index's fixing days if blank or omitted. See defaults per index in Table \ref{tab:fixingdaysdefaults}.

\item Lookback [Optional]: Only applicable to OIS legs. A period by which the value dates schedule of (averaged,
  compounded) OIS legs is shifted into the past. On top of this the gap defined by the FixingDays is applied to get the
  final fixing date for an original date in the OIS value dates schedule. In the context of the Ibor reform the Lookback
  parameter is sometimes also called ``shift''. With this terminology, first the shift and then the observation lag is
  applied to get the fixing date for an original value date of an overnight coupon.

  Allowable values: any valid period, e.g. 2D, 3M, 1Y

\item RateCutoff [Optional]: Only applicable to OIS legs. The number of fixing dates at the end of the fixing period for
  which the fixing value is held constant and set to the previous value. Defaults to $0$.

Allowable values: any non-negative whole number

\item Spreads [Optional]: This node contains child elements of type
  \lstinline!Spread!. If the spread is constant over the life of the
  floating leg, only one spread value should be entered. If two or more
  coupons have different spreads, multiple spread values are required,
  each represented by a \lstinline!Spread! child element. The first
  spread value corresponds to the first coupon, the second spread
  value corresponds to the second coupon, etc. If the number of
  coupons exceeds the number of spread values, the spread will be kept
  flat at the value of last entered spread for the remaining coupons.
  The number of entered spread values cannot exceed the number of
  coupons. 

  Allowable values: Each child element can take any real number. The spread is expressed in decimal form, e.g. 0.005 is
  a spread of 0.5\% or 50 bp. 

For the {\tt <Spreads>} section, the same applies as for notionals and
rates - a list of changing spreads can be specified without or with individual start dates as shown
in Listing \ref{lst:spreads_dates}.
\begin{listing}[H]
%\hrule\medskip
\begin{minted}[fontsize=\footnotesize]{xml}
                    <Spreads>
                        <Spread>0.005</Spread>
                        <Spread startDate='2017-03-05'>0.007</Spread>
                        <Spread startDate='2019-03-05'>0.009</Spread>
                    </Spreads>
\end{minted}
\caption{'Dated' spreads}
\label{lst:spreads_dates}
\end{listing}

If the entire {\tt <Spreads>} section is omitted, it defaults to a spread of \emph{0\%}.

\item Gearings [Optional]: This node contains child elements of type \lstinline!Gearing! indicating that the coupon rate is
  multiplied by the given factors. The mode of specification is analogous to spreads, see above.
  
If the entire {\tt <Gearings>} section is omitted, it defaults to a gearing of \emph{1}.

\item Caps [Optional]: This node contains child elements of type \lstinline!Cap! indicating that the coupon rate is
  capped at the given rate (after applying gearing and spread, if any). The mode of specification is analogous to
  spreads, see above. Caps / Floors are supported for Ibor and compounded OIS coupons, but not for coupons with
  subperiods or averaged OIS coupons.

  For compounded OIS coupons notice how the gearing $g$ and spread $s$ enter the calculation of the coupon amount $A$
  dependent on the IncludeSpread and LocalCapFloor flags and the cap rate $C$, floor rate $F$, daily rates $f_i$, daily
  accrual fractions $\tau_i$ and the coupon accrual fraction $\tau$:
  \begin{itemize}
  \item IncludeSpread = false, LocalCapFloor = false:\\
    $$ A = \min \left( \max \left( g \cdot \frac{\prod (1 + \tau_i f_i) - 1}{\tau} + s, F \right), C \right)$$
  \item IncludeSpread = true, LocalCapFloor = false:\\
    $$ A = \min \left( \max \left( g \cdot \frac{\prod (1 + \tau_i(f_i + s)) - 1}{\tau}, F \right), C \right)$$
  \item IncludeSpread = false, LocalCapFloor = true:\\
    $$ A = g \cdot \frac{\prod (1 + \tau_i \min ( \max ( f_i , F), C)) - 1}{\tau} + s $$
  \item IncludeSpread = true, LocalCapFloor = true:\\
    $$ A = g \cdot \frac{\prod (1 + \tau_i \min ( \max ( f_i + s , F), C)) - 1}{\tau} $$
  \end{itemize}

\item Floors [Optional]: This node contains child elements of type \lstinline!Floor! indicating that the coupon rate is floored at
  the given rate (after applying gearing and spread, if any). The mode of specification is analogous to spreads, see
  above.

\item NakedOption [Optional]: Optional node, if \emph{true} the leg represents only the embedded floor, cap or collar.
  By convention the embedded floor (or cap) are considered long if the leg is a receiver leg, otherwise short. For a
  collar the floor is long and the cap is short if the leg is a receiver leg. Notice that this is opposite to
  the definition of a collar in \ref{ss:capfloor}.

 Allowable values: \emph{true}, \emph{false} . Defaults to \emph{false} if left blank or omitted.

\item LocalCapFloor [Optional]: Optional node, if \emph{true} a cap (floor) will be applied to the daily rates of a
  compounded overnight coupon. If \emph{false} the effective period rate will be capped (floored). The flag is ignored
  for coupons other than compounded overnight coupons.

 Allowable values: \emph{true}, \emph{false} . Defaults to \emph{false} if left blank or omitted.

\end{itemize}
