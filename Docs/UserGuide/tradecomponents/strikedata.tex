\subsubsection{StrikeData}
\label{ss:strikedata}

This trade component that can be used to define the strike entity for an certain trade types. It can be used to define either a Price or Yield strike, with examples below in  \ref{lst:strikeprice} and \ref{lst:strikeyield} respectively.

\begin{listing}[H]
%\hrule\medskip
\begin{minted}[fontsize=\footnotesize]{xml}
        <StrikeData>
			<StrikePrice>
				<Value>1</Value>
				<Currency>EUR</Currency>
			</StrikePrice>
		</StrikeData>
\end{minted}
\caption{Strike Price}
\label{lst:strikeprice}
\end{listing}

The meanings and allowable values of the elements in the \lstinline!StrikePrice! node are as follows:

\begin{itemize}

\item \lstinline!Value!: The strike price.
Allowable values: Any positive real number.

\item \lstinline!Currency! [Optional]: The currency of the amount given in \lstinline!Value!.
Allowable values: See Table \ref{tab:currency} \lstinline!Currency!. Minor Currencies in Table \ref{tab:currency} are also allowable.

\end{itemize}

\begin{listing}[H]
%\hrule\medskip
\begin{minted}[fontsize=\footnotesize]{xml}
        <StrikeData>
			<StrikeYield>
				<Yield>0.055</Yield>
				<Compounding>SimpleThenCompounded</Compounding>
			</StrikeYield>
		</StrikeData>
\end{minted}
\caption{Strike Yield}
\label{lst:strikeyield}
\end{listing}

The meanings and allowable values of the elements in the \lstinline!StrikeYield! node are as follows:

\begin{itemize}

\item \lstinline!Yield!: A Yield quoted as a percentage, 10% should be entered as 0.1.
Allowable values: Any real number.

\item \lstinline!Compounding! [Optional]: The compounding or the yield given in  \lstinline!Yield!. Defaults to \lstinline!SimpleThenCompounded!.
Allowable values: {\em Simple, Compounded, Continuous, SimpleThenCompounded}.

\end{itemize}
