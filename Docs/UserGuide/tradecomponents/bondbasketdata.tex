\subsubsection{Bond Basket Data for Cashflow CDO}
\label{ss:bondbasketdata} 
This trade component node is used in a Cashflow CDO trade as explained in \ref{ss:CBOData}.
An example structure of the \lstinline!BondBasketData! trade component node is shown in Listing \ref{lst:bondbasketdata2}.

\begin{listing}[H]
%\hrule\medskip
\begin{minted}[fontsize=\footnotesize]{xml}
<BondBasketData>
    <Trade id="Bond_1">
      <TradeType>Bond</TradeType>
      <Envelope>
        ...
      </Envelope>
      <BondData>
        ...
      </BondData>
    </Trade>
    <Trade id="Bond_2">
      <TradeType>Bond</TradeType>
      <Envelope>
        ...
      </Envelope>
      <BondData>
        ...
      </BondData>
    </Trade>
</BondBasketData>
\end{minted}
\caption{Bond Basket Data for Cashflow CDO}
\label{lst:bondbasketdata2}
\end{listing}

The usage of the BondBasketData is akin to a portfolio of bond trades, 
but is embraced by the keyword \lstinline!BondBasketData! as opposed to \lstinline!Portfolio!.
Compare the vanilla bond section \ref{ss:bond} for usage and allowable values. 

