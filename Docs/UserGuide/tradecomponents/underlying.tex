\subsubsection{Underlying}
\label{ss:underlying}

This trade component can be used to define the underlying entity for an Equity, Commodity or FX trade. It can be used for a single underlying, or within a basket with associated weight.
For an equity underlying a string representation is used to match \lstinline!Underlying! node to requied configuration and reference data. The string representation is of the form {IdentifierType}:{Name}:{Currency}:{Exchange}, with all entries optional except for Name.


\begin{listing}[H]
%\hrule\medskip
\begin{minted}[fontsize=\footnotesize]{xml}
<Underlying>
  <Type>...</Type>
  <Name>...</Name>
  <Weight>...</Weight>
  <Currency>...</Currency>
  <IdentifierType>...</IdentifierType>
  <Exchange>...</Exchange>
  <PriceType>...</PriceType>
  <FutureMonthOffset>...</FutureMonthOffset>
  <DeliveryRollDays>...</DeliveryRollDays>
  <DeliveryRollCalendar>...</DeliveryRollCalendar>
</Underlying>
\end{minted}
\caption{Underlying node}
\label{lst:underlying}
\end{listing}

Example structures of the \lstinline!Underlying! trade component node are shown in Listings \ref{lst:equnderlyingric} and \ref{lst:equnderlyingisin} for
an equity underlying, in Listing \ref{lst:fxunderlying} for an fx underlying and in Listing \ref{lst:communderlying} for
a commodity underlying.

\begin{listing}[H]
%\hrule\medskip
\begin{minted}[fontsize=\footnotesize]{xml}
        <Underlying>
            <Type>Equity</Type>
            <Name>.SPX</Name>
            <Weight>1.0</Weight>
            <IdentifierType>RIC</IdentifierType>
        </Underlying>
\end{minted}
\caption{Equity Underlying - RIC}
\label{lst:equnderlyingric}
\end{listing}

\begin{listing}[H]
%\hrule\medskip
\begin{minted}[fontsize=\footnotesize]{xml}
        <Underlying>
            <Type>Equity</Type>
            <Name>GB00BH4HKS39</Name>
            <Weight>1.0</Weight>
            <IdentifierType>ISIN</IdentifierType>
            <Currency>GBP</Currency>
            <Exchange>XLON</Exchange>
        </Underlying>
\end{minted}
\caption{Equity Underlying - ISIN}
\label{lst:equnderlyingisin}
\end{listing}

\begin{listing}[H]
%\hrule\medskip
\begin{minted}[fontsize=\footnotesize]{xml}
        <Underlying>
          <Type>FX</Type>
          <Name>ECB-EUR-USD</Name>
          <Weight>1.0</Weight>
        </Underlying>
\end{minted}
\caption{FX Underlying}
\label{lst:fxunderlying}
\end{listing}

\begin{listing}[H]
%\hrule\medskip
\begin{minted}[fontsize=\footnotesize]{xml}
        <Underlying>
          <Type>Commodity</Type>
          <Name>NYMEX:CL</Name>
          <Weight>1.0</Weight>
          <PriceType>FutureSettlement</PriceType>
          <FutureMonthOffset>0</FutureMonthOffset>
          <DeliveryRollDays>0</DeliveryRollDays>
          <DeliveryRollCalendar>TARGET</DeliveryRollCalendar>
        </Underlying>
\end{minted}
\caption{Commodity Underlying}
\label{lst:communderlying}
\end{listing}


The meanings and allowable values of the elements in the \lstinline!Underlying! node are as follows:

\begin{itemize}

\item \lstinline!Type!: The type of the Underlying asset.

  Allowable values:  \emph{Equity}, \emph{FX}, \emph{Commodity}.

\item \lstinline!Name!:
  The name of the Underlying asset. 
  
  Allowable values:  

  \emph{Equity}: See \lstinline!Name! for equity trades in Table \ref{tab:equity_credit_data}

  \emph{FX}: A string on the form SOURCE-CCY1-CCY2, where SOURCE is \emph{ECB} (European Central Bank) for EUR pairs, \emph{TR20H} (Thomson Reuters 20H) for USD pairs and other fx-pairs,  \emph{PM} for precious metals, and \emph{OTHER} for fx-pairs not covered by \emph{ECB} or \emph{TR20H}. \\ For ECB CCY1 should always be EUR, and for \emph{TR20H} CCY1 should always be USD, except for
  \emph{TR20H-GBP-USD}, \emph{TR20H-EUR-USD}, \emph{TR20H-AUD-USD} and \emph{TR20H-NZD-USD}. For the USD-EUR pair, \emph{TR20H} or ECB (ECB-EUR-USD) can be used. \\ See Table
  \ref{tab:fxindex_data}, and note that the FX- prefix is not included in \lstinline!Name! as it is already included in \lstinline!Type!.

  \emph{Commodity}: An identifier specifying the commodity being referenced in the leg.
% The following needs to move into client-specific documentation of allowable values:
%The \lstinline!Name! is of the form \lstinline!Prefix:Identifier!. The \lstinline!Prefix! is either \lstinline!PM! for precious metal or a code representing the exchange on which the commodity is traded. For precious metals, the \lstinline!Identifier! is the precious metal code followed by the precious metal price currency. For future contracts, the \lstinline!Identifier! is the exchange code for the future contract.
Table \ref{tab:commodity_data} lists the allowable values for \lstinline!Name! and gives a description. \\
\item \lstinline!Weight! [Optional]:
The relative weight of the underlying if part of a basket. For a single underlying this can be omitted or set to 1. 

Allowable values: A number between 0 and 1. Defaults to 1 if left blank or omitted

\item \lstinline!IdentifierType! [Optional]:
Only valid when \lstinline!Type! is  \emph{Equity}. The type of the identifier being used.  

Allowable values:  \emph{RIC}, \emph{ISIN}. Defaults to \emph{RIC}, if left blank or omitted, and \lstinline!Type!: is  \emph{Equity}.

\item \lstinline!Currency! [Mandatory when \lstinline!IdentifierType! is  \emph{ISIN}]: Only valid when \lstinline!Type! is  \emph{Equity}. The currency the underlying equity is quoted in. Used when \lstinline!IdentifierType! is  \emph{ISIN}, to - together with the \lstinline!Exchange!  convert a given ISIN to a RIC code.  

Allowable values: See the \lstinline!Currency! section in Table \ref{tab:allow_stand_data}. Mandatory when \lstinline!IdentifierType! is  \emph{ISIN}, should not be used when  \lstinline!IdentifierType! is  \emph{RIC}, and ignored otherwise. When \lstinline!Type! is \emph{Equity}, \lstinline!Minor Currencies! in Table \ref{tab:allow_stand_data} are also allowable.

\item \lstinline!Exchange! [Mandatory when \lstinline!IdentifierType! is  \emph{ISIN}]:
Only valid when \lstinline!Type! is  \emph{Equity}. A string code representing the exchange the equity is traded on. Used when \lstinline!IdentifierType! is  \emph{ISIN}, to - together with the \lstinline!Currency!  convert a given ISIN to a RIC code.  

Allowable values:  The MIC code of the exchange, see Table \ref{tab:mic}. Mandatory when \lstinline!IdentifierType! is  \emph{ISIN}, should not be used when  \lstinline!IdentifierType! is \emph{RIC}, and ignored otherwise.

\item \lstinline!PriceType! [Optional]:
Only valid when  \lstinline!Type! is  \emph{Commodity}.  Whether the Spot or Future price is referenced. 

Allowable values:  \emph{Spot}, \emph{FutureSettlemment}. Mandatory when  \lstinline!Type! is  \emph{Commodity} .

\item \lstinline!FutureMonthOffset! [Optional]:
Only valid when  \lstinline!Type! is  \emph{Commodity}. Only relevant for the \emph{FutureSettlemment} price type, in which case the the $N+1$th future with
  expiry greater than ObservationDate for the given commodity underlying will be referenced.

Allowable values:  An integer. Mandatory for when  \lstinline!Type! is  \emph{Commodity} and \lstinline!PriceType! is \emph{FutureSettlemment}.

\item \lstinline!DeliveryRollDays! [Optional]:
Only valid when  \lstinline!Type! is  \emph{Commodity}.  The number of days the observation date is rolled forward before the
  next future expiry is looked up.
  
Allowable values: An integer. Defaults to 0 if left blank or omitted, and \lstinline!Type!: is  \emph{Commodity}.

\item \lstinline!DeliveryRollCalendar! [Optional]:
Only valid when  \lstinline!Type! is  \emph{Commodity}.  The calendar used to roll forward the observation date.

Allowable values: See Table \ref{tab:calendar}. Defaults to the null calendar if left blank or omitted, and \lstinline!Type!: is  \emph{Commodity}.

\end{itemize}
