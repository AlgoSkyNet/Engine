\subsubsection{CPI Leg Data}
\label{ss:cpilegdata}

A CPI leg contains a series of CPI-linked coupon payments $I(t)/I_0\,N\,c\,\delta$ and a final
inflation-linked redemption $I(t)/I_0\,N$. The final redemption can be
subtracting the (un-inflated) notional $N$, i.e. $(I(t)/I_0-1)\,N$,
see below.

Listing \ref{lst:cpilegdata} shows an example for a leg of type CPI. The CPILegData block contains the following
elements:

\begin{itemize}
\item Index: The underlying zero inflation index.

Allowable values:  See \lstinline!Inflation CPI Index! in Table \ref{tab:cpiindex_data}.
\item Rates: The fixed real rate(s) of the leg. As usual, this can be a single value, a vector of values or a dated vector of
  values.
 
 Allowable values: Each rate element can take any  real number. The rate is
  expressed in decimal form, e.g. 0.05 is a rate of 5\%.
\item BaseCPI: The base CPI value $I_0$ used to determine the lifting factor for the fixed coupons.

Allowable values:  Any positive real number.

\item StartDate: The start date needs to be provided here, because the payment
  schedule can be given by a sequence of dates, possibly only a single
  date, in which case the start date remains unspecified.
  
Allowable values:  See \lstinline!Date! in Table \ref{tab:allow_stand_data}. 

\item ObservationLag: The observation lag to be applied. It's the amount of time from the fixing at the start or end of the period, moving backward in time, to the inflation index observation date (the inflation fixing). 

Allowable values: An integer followed by \emph{D}, \emph{W}, \emph{M} or \emph{Y}. Interpolation lags are typically expressed in a positive number of  \emph{M}, months. Note that negative values are allowed, but mean that the inflation is observed forward in time from the period start/end date, which is unusual.  

\item Interpolation: The type of interpolation that is applied to inflation fixings.

Allowable values:  \emph{Linear, Flat, AsIndex} 

\medskip
The latter choice means that the underlying inflation index
interpolation is applied.

\item SubtractInflationNotional [Optional]: A flag indicating whether
  the non-inflation adjusted notional amount should be subtracted from
  the the final inflation-adjusted notional exchange at maturity.
  Note that the final coupon payment is not affected by this flag. \\ 
Final notional payment if \emph{true}: $N \,(I(T)/I_0-1)$. \\ 
Final notional payment if  \emph{false}: $N \,I(T)/I_0$ 

Allowable values: Boolean node, allowing \emph{Y, N, 1, 0, true, false} etc. The full set of allowable values is given in Table \ref{tab:boolean_allowable}.
\\Defaults to \emph{false}  if left blank or omitted.

\item Caps [Optional]: This node contains child elements of type
  \lstinline!Cap! indicating that the inflation indexed payment is
  capped; the cap is applied to the inflation index and expressed as
  an inflation rate, see CPI Cap.

\item Floors [Optional]: This node contains child elements of type \lstinline!Floor! indicating that the coupon rate is floored at
  the given rate (after applying gearing and spread, if any). The mode of specification is analogous to spreads, see
  above.

\item NakedOption [Optional]: Optional node (defaults to N), if Y the leg represents only the embedded floor, cap or collar. 
By convention these embedded options are considered long if the leg is a receiver leg, otherwise short. 
 
\end{itemize} 

Whether the leg cotains a final redemption flow at all or not depends on the
 notional exchange setting, see section \ref{ss:leg_data} and listing \ref{lst:notional_exchange}.

\begin{listing}[H]
%\hrule\medskip
\begin{minted}[fontsize=\footnotesize]{xml}
      <LegData>
        <LegType>CPI</LegType>
        <Payer>false</Payer>
        <Currency>GBP</Currency>
        <Notionals>
          <Notional>10000000</Notional>
        </Notionals>
        <DayCounter>ACT/ACT</DayCounter>
        <PaymentConvention>Following</PaymentConvention>
        <ScheduleData>
          <Rules>
            <StartDate>2016-07-18</StartDate>
            <EndDate>2021-07-18</EndDate>
            <Tenor>1Y</Tenor>
            <Calendar>UK</Calendar>
            <Convention>ModifiedFollowing</Convention>
            <TermConvention>ModifiedFollowing</TermConvention>
            <Rule>Forward</Rule>
            <EndOfMonth/>
            <FirstDate/>
            <LastDate/>
          </Rules>
        </ScheduleData>
        <CPILegData>
          <Index>UKRPI</Index>
          <Rates>
            <Rate>0.02</Rate>
          </Rates>
          <BaseCPI>210</BaseCPI>
          <StartDate>2016-07-18</StartDate>
          <ObservationLag>2M</ObservationLag>
          <Interpolation>Linear</Interpolation>
          <Caps>
             <Cap>0.03</Cap>
          </Caps>
          <Floors>
            <Floor>0.0</Floor>
          <Floors>
        </CPILegData>
        <NakedOption>N</NakedOption>
      </LegData>
\end{minted}
\caption{CPI leg data}
\label{lst:cpilegdata}
\end{listing}

