\subsubsection{CPI Leg Data}
\label{ss:cpilegdata}

A CPI leg contains a series of CPI-linked coupon payments $(I(t)/I_0)\,N\,c\,\delta$ and, if \lstinline!NotionalFinalExchange! is set to \emph{true}, a final
inflation-linked redemption $(I(t)/I_0)\,N$. The final redemption can be
subtracting the (un-inflated) notional $N$, i.e. $(I(t)/I_0-1)\,N$,
see below.

Note that CPI legs with just a final redemption and no coupons, can be set up with a dates-based Schedule containing just a single date - representing the date of the final redemption flow. In this case \lstinline!NotionalFinalExchange! must be set to \emph{true}, otherwise the whole leg is empty, and the Rate is not used and can be set to any value. 

Listing \ref{lst:cpilegdata} shows an example for a leg of type CPI with annual coupons, and \ref{lst:cpilegdatafinal} shows an example for a leg of type CPI with just the final redemption. 

The  \lstinline!CPILegData! block contains the following elements:

\begin{itemize}
\item Index: The underlying zero inflation index.

Allowable values:  See \lstinline!Inflation CPI Index! in Table \ref{tab:cpiindex_data}.
\item Rates: The fixed real rate(s) of the leg. As usual, this can be a single value, a vector of values or a dated vector of
  values.
 
 Allowable values: Each rate element can take any  real number. The rate is
  expressed in decimal form, e.g. 0.05 is a rate of 5\%.
\item BaseCPI: The base CPI value $I_0$ used to determine the lifting factor for the fixed coupons.

Allowable values:  Any positive real number.

\item StartDate [Optional]: The start date needs to be provided in case the schedule comprises only a single date. If
  the schedule has at least two dates and a start date is given at the same time, the first schedule date must be
  identical to the start date.
  
Allowable values:  See \lstinline!Date! in Table \ref{tab:allow_stand_data}. 

\item ObservationLag: The observation lag to be applied. It's the amount of time from the fixing at the start or end of the period, moving backward in time, to the inflation index observation date (the inflation fixing). 

Allowable values: An integer followed by \emph{D}, \emph{W}, \emph{M} or \emph{Y}. Interpolation lags are typically expressed in a positive number of  \emph{M}, months. Note that negative values are allowed, but mean that the inflation is observed forward in time from the period start/end date, which is unusual.  

\item Interpolation: The type of interpolation that is applied to inflation fixings. \emph{Linear} interpolation means that the inflation fixing for a given date is interpolated linearly between the surrounding - usually monthly - actual fixings, whereas with  \emph{Flat} interpolation the inflation fixings are constant for each day at the value of the previous/latest actual fixing (flat forward interpolation).  
%\emph{AsIndex} means that the underlying inflation index interpolation is applied. Note that if an inflation index have fixing interpolation switched off,  \emph{AsIndex}  is equivalent to \emph{Flat} interpolation.

Allowable values:  \emph{Linear, Flat} 

\item SubtractInflationNotional [Optional]: A flag indicating whether
  the non-inflation adjusted notional amount should be subtracted from
  the the final inflation-adjusted notional exchange at maturity.
  Note that the final coupon payment is not affected by this flag. \\ 
Final notional payment if \emph{true}: $N \,(I(T)/I_0-1)$. \\ 
Final notional payment if  \emph{false}: $N \,I(T)/I_0$ 

Allowable values: Boolean node, allowing \emph{Y, N, 1, 0, true, false} etc. The full set of allowable values is given in Table \ref{tab:boolean_allowable}.
\\Defaults to \emph{false}  if left blank or omitted.

\item Caps [Optional]: This node contains child elements of type
  \lstinline!Cap! indicating that the inflation indexed payment is
  capped; the cap is applied to the inflation index and expressed as
  an inflation rate, see CPI Cap/Floor in the Product Description. \\
  If the cap is constant over the life of the 
cpi leg, only one cap value should
be entered. If two or more coupons have different caps, multiple cap values
are required, each represented by a \lstinline!Cap! child element. The first cap value
corresponds to the first coupon, the second cap value corresponds to the
second coupon, etc. If the number of coupons exceeds the number of cap
values, the cap will be kept at at the value of last entered spread for the
remaining coupons. The number of entered cap values cannot exceed the
number of coupons.

Allowable values: Each child element can take any real number. The cap is
expressed in decimal form, e.g. 0.03 is a cap of 3\%.

\item Floors [Optional]: This node contains child elements of type
  \lstinline!Floor! indicating that the inflation indexed payment is
  floored; the floor is applied to the inflation index and expressed as
  an inflation rate. The mode of specification is analogous to caps, see
  above.

Allowable values: Each child element can take any real number. The floor is
expressed in decimal form, e.g. 0.01 is a cap of 1\%.

\item NakedOption [Optional]: Optional node, if \emph{true} the leg represents only the embedded floor, cap or collar. 
By convention these embedded options are considered long if the leg is a receiver leg, otherwise short. 
 
 Allowable values:  \emph{true}, \emph{false}. Defaults to \emph{false} if left blank or omitted.
 
\end{itemize} 

Whether the leg cotains a final redemption flow at all or not depends on the
 notional exchange setting, see section \ref{ss:leg_data} and listing \ref{lst:notional_exchange}.

\begin{listing}[H]
%\hrule\medskip
\begin{minted}[fontsize=\footnotesize]{xml}
      <LegData>
        <LegType>CPI</LegType>
        <Payer>false</Payer>
        <Currency>GBP</Currency>
        <Notionals>
          <Notional>10000000</Notional>
          <Exchanges>
            <NotionalInitialExchange>false</NotionalInitialExchange>
            <NotionalFinalExchange>true</NotionalFinalExchange>
          </Exchanges>          
        </Notionals>
        <DayCounter>ACT/ACT</DayCounter>
        <PaymentConvention>Following</PaymentConvention>
        <ScheduleData>
          <Rules>
            <StartDate>2016-07-18</StartDate>
            <EndDate>2021-07-18</EndDate>
            <Tenor>1Y</Tenor>
            <Calendar>UK</Calendar>
            <Convention>ModifiedFollowing</Convention>
            <TermConvention>ModifiedFollowing</TermConvention>
            <Rule>Forward</Rule>
            <EndOfMonth/>
            <FirstDate/>
            <LastDate/>
          </Rules>
        </ScheduleData>
        <CPILegData>
          <Index>UKRPI</Index>
          <Rates>
            <Rate>0.02</Rate>
          </Rates>
          <BaseCPI>210</BaseCPI>
          <StartDate>2016-07-18</StartDate>
          <ObservationLag>2M</ObservationLag>
          <Interpolation>Linear</Interpolation>
          <Caps>
             <Cap>0.03</Cap>
          </Caps>
          <Floors>
            <Floor>0.0</Floor>
          <Floors>
          <NakedOption>false</NakedOption>          
        </CPILegData>
      </LegData>
\end{minted}
\caption{CPI leg data with capped annual coupons}
\label{lst:cpilegdata}
\end{listing}

\begin{listing}[H]
%\hrule\medskip
\begin{minted}[fontsize=\footnotesize]{xml}
      <LegData>
        <Payer>false</Payer>
        <LegType>CPI</LegType>
        <Currency>GBP</Currency>
        <PaymentConvention>ModifiedFollowing</PaymentConvention>
        <DayCounter>ActActISDA</DayCounter>
        <Notionals>
          <Notional>25000000.0</Notional>
          <Exchanges>
            <NotionalInitialExchange>false</NotionalInitialExchange>
            <NotionalFinalExchange>true</NotionalFinalExchange>
          </Exchanges>
        </Notionals>
        <ScheduleData>
          <Dates>
            <Calendar>GBP</Calendar>
            <Dates>
              <Date>2020-08-17</Date>
            </Dates>
          </Dates>
        </ScheduleData>
        <CPILegData>
          <Index>UKRPI</Index>
          <Rates>
            <Rate>1.0</Rate>
          </Rates>
          <BaseCPI>280.64</BaseCPI>
          <StartDate>2018-08-19</StartDate>
          <ObservationLag>2M</ObservationLag>
          <Interpolated>false</Interpolated>
          <SubtractInflationNotional>true</SubtractInflationNotional>
        </CPILegData>
      </LegData>
\end{minted}
\caption{CPI leg data with just the final redemption}
\label{lst:cpilegdatafinal}
\end{listing}

