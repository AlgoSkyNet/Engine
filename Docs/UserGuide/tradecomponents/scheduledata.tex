\subsubsection{Schedule Data (Rules and Dates)}\label{ss:schedule_data}

The \lstinline!ScheduleData! trade component node is used within the \lstinline!LegData! trade component. The Schedule can either be
rules based (at least one sub-node Rules exists) or dates based (at least one \lstinline!Dates! sub-node exists,
where the schedule is determined directly by \lstinline!Date! child elements). In rules based schedules, the schedule dates are generated from a
set of rules based on the entries of the sub-node Rules, having the elements StartDate, EndDate, Tenor, Calendar, Convention, TermConvention, and Rule.
Example structures of \lstinline!ScheduleData! nodes based on rules
respectively dates are shown in Listing \ref{lst:schedule_data_true} and Listing \ref{lst:schedule_data_false}, respectively.

\begin{listing}[H]
%\hrule\medskip
\begin{minted}[fontsize=\footnotesize]{xml}
              <ScheduleData>
                    <Rules>
                        <StartDate>2013-02-01</StartDate>
                        <EndDate>2030-02-01</EndDate>
                        <Tenor>1Y</Tenor>
                        <Calendar>UK</Calendar>
                        <Convention>MF</Convention>
                        <TermConvention>MF</TermConvention>
                        <Rule>Forward</Rule>
                    </Rules>
              </ScheduleData>
\end{minted}
\caption{Schedule data, rules based}
\label{lst:schedule_data_true}
\end{listing}

\begin{listing}[H]
%\hrule\medskip
\begin{minted}[fontsize=\footnotesize]{xml}
               <ScheduleData>
                <Dates>
                 <Calendar>NYB</Calendar>
                 <Convention>Following</Convention>
                 <Tenor>3M</Tenor>
                 <EndOfMonth>false</EndOfMonth>
                    <Dates>
                        <Date>2012-01-06</Date>
                        <Date>2012-04-10</Date>
                        <Date>2012-07-06</Date>
                        <Date>2012-10-08</Date>
                        <Date>2013-01-07</Date>
                        <Date>2013-04-08</Date>
                    </Dates>
                 </Dates> 
                </ScheduleData>
\end{minted}
\caption{Schedule data, date based}
\label{lst:schedule_data_false}
\end{listing}

The ScheduleData section can contain any number and combination of
{\tt <Dates>} and {\tt <Rules>} sections. The resulting schedule will
then be an ordered concatenation of individual schedules.
 
\medskip
The meanings and allowable values of the elements in a {\tt <Rules>} based section of the  \lstinline!ScheduleData! node follow below.

\begin{itemize}
  % \item IsRulesBased: Determines whether the schedule is set by specifying dates directly, or by specifying rules that
  %   generate the schedule. If \emph{true}, the following entries are required: StartDate, EndDate, Tenor, Calendar,
  %   Convention, TermConvention, and Rule.  If false the Dates sub-node is required. \\ Allowable values: \emph{true,
  %   false}
\item Rules: a sub-node that determines whether the schedule is set by specifying rules that
generate the schedule. If existing, the following entries are required: StartDate, EndDate, Tenor, Calendar, 
Convention, TermConvention, and Rule. If not existing, the Dates
sub-node is required.
\item StartDate:  The schedule start date.  

Allowable values:  See \lstinline!Date! in Table \ref{tab:allow_stand_data}.

\item EndDate: The schedule end date.  

Allowable values:  See \lstinline!Date! in Table \ref{tab:allow_stand_data}.

\item Tenor: The tenor used to generate schedule dates. 

Allowable values: A string where the last character must be D or W or
M or Y.  The characters before that must be a positive integer. \\D
$=$ Day, W $=$ Week, M $=$ Month, Y $=$ Year

\item Calendar: The calendar used to generate schedule  dates. 

Allowable values: See Table \ref{tab:calendar} Calendar.

\item Convention: Determines the adjustment of the schedule dates with
  regards to the selected calendar. 

Allowable values: See \lstinline!Roll Convention! in Table
\ref{tab:allow_stand_data}.

\item TermConvention: Determines the adjustment of the final schedule
  date with regards to the selected calendar. 

Allowable values: See \lstinline!Roll Convention! in Table \ref{tab:allow_stand_data}.

\item Rule [Optional]: Rule for the generation of the schedule using given
  start and end dates, tenor, calendar and business day conventions. 

Allowable values and descriptions: See Table \ref{tab:rule} Rule. Defaults to \emph{Forward} if left blank or omitted.

\item EndOfMonth [Optional]: Specifies whether the date generation rule is different for end of month, so that the last date of each month is generated, regardless of number of days in the month.

Allowable values: Boolean node, allowing \emph{Y, N, 1, 0, true, false} etc. The full set of allowable values is given in Table \ref{tab:boolean_allowable}. Defaults to \emph{false} if left blank or omitted. Must be set to \emph{false} or omitted if the date generation Rule is set to \emph{CDS} or \emph{CDS2015}.

\item FirstDate [Optional]: Date for initial stub period. For date generation rules \emph{CDS} and \emph{CDS2015}, if given, this
  overwrites the first date of the schedule that is otherwise built from IMM dates.

Allowable values: See \lstinline!Date! in Table \ref{tab:allow_stand_data}.

\item LastDate [Optional]: Date for final stub period. For date generation rules \emph{CDS} and \emph{CDS2015}, if given, this
  overwrites the last date of the schedule that is otherwise built from IMM dates.

Allowable values: See \lstinline!Date! in Table \ref{tab:allow_stand_data}.
\end{itemize}

\medskip
The meanings and allowable values of the elements in a {\tt <Dates>} based section of the  \lstinline!ScheduleData! node follow below.

\begin{itemize}

\item Dates: a sub-node that determines that the schedule is set by specifying schedule dates explicitly. 

\item Calendar [Optional]: Calendar used to determine payment dates, and also to compute day count fractions for irregular periods when day count convention is ActActISMA and the schedule is dates based. 

Allowable values: See Table \ref{tab:calendar} Calendar. Defaults to \emph{NullCalendar} if omitted, i.e. no holidays at all, not even on weekends.

\item Convention [Optional]: Roll Convention to determine payment dates, and also used to compute day count fractions for irregular periods when day count convention is ActActISMA and the schedule is dates based.

Allowable values: See \lstinline!Roll Convention! in Table
\ref{tab:allow_stand_data}. Defaults to \emph{Unadjusted} if omitted.

\item Tenor [Optional]: Tenor used to compute day count fractions for irregular periods when day count convention is ActActISMA and the schedule is dates based.

Allowable values: A string where the last character must be D or W or
M or Y.  The characters before that must be a positive integer. \\D
$=$ Day, W $=$ Week, M $=$ Month, Y $=$ Year

Defaults to \emph{null} if omitted.

\item EndOfMonth [Optional]: Specifies whether the end of month convention is applied when calculating reference periods
  for irregular periods when the day count convention is ActActICMA and the schedule is dates based.

Allowable values: Boolean node, allowing \emph{Y, N, 1, 0, true, false} etc. The full set of allowable values is given in Table \ref{tab:boolean_allowable}. Defaults to \emph{false} if left blank or omitted.

\item Dates: This is a sub-sub-node and contains child elements of type
  \lstinline!Date!. In this case the schedule dates are determined
  directly by the \lstinline!Date! child elements.  At least two
  \lstinline!Date! child elements must be provided. Note that if no calendar and roll convention is given, the specified dates must be given as adjusted dates.    

  Allowable values: Each \lstinline!Date!  child element can take the allowable values listed in \lstinline!Date! in
  Table \ref{tab:allow_stand_data}.

\end{itemize}



\begin{table}[H]
\centering
\begin{tabular}{|l|p{6cm}|}
\hline
\multicolumn{2}{|l|}{\lstinline!Rule!}                    \\ \hline
\textbf{Allowable Values}                   & \textbf{Effect}                       \\ \hline
\emph{Backward}   &   Backward from termination date to effective date.   \\ \hline
\emph{Forward}   &   Forward from effective date to termination date.  \\ \hline
\emph{Zero}   &   No intermediate dates between effective date and termination date.  \\ \hline
\emph{ThirdWednesday}   &   All dates but effective date and
                          termination date are taken to be on the
                          third Wednesday of their month (with forward calculation.) \\ \hline
\emph{Twentieth}   &   All dates but the effective date are taken to be the twentieth of their month (used for CDS schedules in emerging markets.)  The termination date is also modified. \\ \hline
\emph{TwentiethIMM}   &   All dates but the effective date are  taken to be the twentieth of an IMM month (used for CDS schedules.)  The termination date is also modified. \\ \hline
\emph{OldCDS}   &   Same as TwentiethIMM with unrestricted date ends and long/short stub coupon period (old CDS convention).\\ \hline
CDS   &   Credit derivatives standard rule defined in 'Big Bang' changes in 2009. \\ \hline
CDS2015   &   Credit derivatives standard rule updated in 2015. Same as  \emph{CDS} but with termination dates adjusted to 20th June and 20th December. \\ \hline
\end{tabular}
  \caption{Allowable Values for Rule}
  \label{tab:rule}
\end{table}
