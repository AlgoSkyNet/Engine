\subsubsection{Constant Maturity Bond Leg Data}
\label{ss:cmblegdata}

In close analogy to the CMS leg one can create a leg that is linked to a
{\em Constant Maturity Bond} yield index. The associated leg type is {\em CMB}.  

Listing \ref{lst:cmblegdata} shows an example for a leg of type CMB. 

\begin{listing}[H]
%\hrule\medskip
\begin{minted}[fontsize=\footnotesize]{xml}
      <LegData>
        <LegType>CMB</LegType>
        <Payer>false</Payer>
        <Currency>EUR</Currency>
        <Notionals>
          <Notional>10000000</Notional>
        </Notionals>
        <DayCounter>ACT/ACT</DayCounter>
        <PaymentConvention>Following</PaymentConvention>
        <ScheduleData>
          ...
        </ScheduleData>
        <CMBLegData>
          <Index>CMB-US-TBILL-HD-13W</Index>
          <FixingDays>2</FixingDays>
          <Spreads>
            <Spread>0.0010</Spread>
          </Spreads>
          <Gearings>
            <Gearing>2.0</Gearing>
          </Gearings>
        </CMBLegData>
      </LegData>
\end{minted}
\caption{CMB leg data}
\label{lst:cmblegdata}
\end{listing}

The CMBLegData block contains the following elements:

\begin{itemize}
\item Index: The underlying CMB index.

Allowable values: A string of the form CMB-FAMILY-TENOR, where FAMILY might consist of several tokens separated by ``-'' as e.g. in CMB-US-TBILL-HD or CMB-DE-BUND, and TENOR is a valid period.

\item Spreads [Optional]: The spreads applied to index fixings. As usual, this can be a single value, a vector of values or a dated vector of
  values.
  
  Allowable values: Each child spread element can take any real number. The spread is expressed in decimal form, e.g. 0.005 is
  a spread of 0.5\% or 50 bp.  Defaults to zero if left blank or omitted.
    
\item FixingDays: This is the fixing gap, i.e. the number of days
  before the period end date an index fixing is taken. Defaults to the index's fixing gap.
  
    Allowable values: A non-negative whole number.  Defaults to the fixing days of the Ibor index the swap references if blank or omitted. See defaults per index in Table \ref{tab:fixingdaysdefaults}.

\item IsInArrears [Optional]:  \emph{true} indicates that  fixing is in arrears, 
  i.e. the fixing gap is calculated in relation to the current period
  end date.\\ \emph{false} indicates that  fixing is in advance,
  i.e. the fixing gap is calculated in relation to the previous period
  end date.
  
Allowable values:  \emph{true, false}. Defaults to \emph{false} if left blank or omitted.
    
\item Gearings [Optional]: This node contains child elements of type \lstinline!Gearing! indicating that the coupon rate is
  multiplied by the given factors. The mode of specification is analogous to spreads, see above.
  
  If the entire {\tt <Gearings>} section is omitted, it defaults to a gearing of \emph{1}.

\end{itemize}

Note:
\begin{itemize}
\item For each CMB index name one needs to maintain bond reference data with ID equal to the index name.
  This reference data is used to project the CMB index fixings as follows: For a future index period (from future date to future date + index tenor), a forward starting bond is constructed using the schedule frequency defined in the reference data and with constant maturity (future date + tenor).
  The forward bond is priced using the reference yield curve and credit curve defined in the reference data.
  And the bond price is then converted into a bond yield using the bond yield conventions (compounding, frequncy, price type) maintained for that same ID.
  If the conventions are not set up, then default values are used (compounded, annual, clean).
\item For periods with start dates in the past, historical index fixings need to be provided, as for interest rate indices.
\item No convexity adjustment is applied here yet, in contrast to CMS index projections.  
\item The CMB leg does not support Caps or Floors yet, in contrast to the CMS leg. 
\end{itemize}
