\subsubsection{Leg Data and Notionals}
\label{ss:leg_data}

The \lstinline!LegData! trade component node is used within the
\lstinline!CapFloorData!,  \lstinline!SwapData! ,
\lstinline!SwaptionData! and \lstinline!EquitySwapData! trade data containers. It contains a
\lstinline!ScheduleData! trade component sub-node, and a sub-node that depends on
the value of the \lstinline!LegType! element, e.g.:  \lstinline!FixedLegData! for \lstinline!LegType! \emph{Fixed} or \lstinline!FloatingLegData! for \lstinline!LegType! \emph{Floating}. The
\lstinline!LegData! node also includes a \lstinline!Notionals!
sub-node  with \lstinline!Notional! child elements described below. An
example structure of a \lstinline!LegData! node of \lstinline!LegType!
\emph{Floating} is shown in Listing \ref{lst:leg_data}.

\begin{listing}[H]
%\hrule\medskip
\begin{minted}[fontsize=\footnotesize]{xml}
            <LegData>
                <Payer>false</Payer>
                <LegType>Floating</LegType>
                <Currency>EUR</Currency>
                <PaymentConvention>Following</PaymentConvention>
                <DayCounter>30/360</DayCounter>
                <Notionals>
                    <Notional>1000000</Notional>
                </Notionals>
                <ScheduleData>
                    ...
                </ScheduleData>
                <FloatingLegData>
                    ...
                </FloatingLegData>
            </LegData>
\end{minted}
\caption{Leg data}
\label{lst:leg_data}
\end{listing}

The meanings and allowable values of the elements in the \lstinline!LegData! node follow below.

\begin{itemize}
\item LegType:  Determines which of the available sub-nodes must be
  used. 

Allowable values:  \emph{Fixed, Floating, Cashflow, CMS, CMSSpread, DigitalCMSSpread, Equity, YY, CPI, ZeroCouponFixed}
%YY,

\item Payer:  The flows of the leg are paid to the counterparty if
  \emph{true}, and received if \emph{false}.  

Allowable values:  \emph{true, false} 

\item Currency: The currency of the leg. 

Allowable values:  See \lstinline!Currency! in Table \ref{tab:allow_stand_data}.

\item PaymentConvention: The payment convention of the leg coupons. 

Allowable values: See \lstinline!Roll Convention! in Table \ref{tab:allow_stand_data}.

\item PaymentLag [optional]: The payment lag given as as number business days, applies to Fixed legs, and Floating legs with Ibor and OIS indices (but not to BMA/SIFMA indices). PaymentLag is also not supported for  CapFloor Floating legs that have Ibor coupons with sub periods (hasSubPeriods = \emph{true}), nor for CapFloor Floating legs with averaged ON coupons (isAveraged = \emph{true}).

Allowable values: A non-negative whole number, if not given it defaults to zero.

\item DayCounter: The day count convention of the leg coupons. 

Allowable values: See \lstinline!DayCount Convention! in Table \ref{tab:daycount}.

\item Notionals: This node contains child elements of type
  \lstinline!Notional!. If the notional is fixed over the life of the
  leg only one notional value should be entered. If the notional is
  amortising or accreting, this is represented by entering multiple
  notional values, each represented by a \lstinline!Notional! child
  element. The first notional value corresponds to the first coupon,
  the second notional value corresponds to the second coupon, etc. If
  the number of coupons exceeds the number of notional values, the
  notional will be kept flat at the value of last entered notional for
  the remaining coupons.  The number of entered notional values cannot
  exceed the number of coupons.

Allowable values: Each child element can take any positive real number.

\vspace{1em}

An example of a \lstinline!Notionals! element for an amortising leg with four coupons is shown in Listing \ref{lst:notionals}.
\begin{listing}[H]
%\hrule\medskip
\begin{minted}[fontsize=\footnotesize]{xml}
                <Notionals>
                    <Notional>65000000</Notional>
                    <Notional>65000000</Notional>
                    <Notional>55000000</Notional>
                    <Notional>45000000</Notional>
                </Notionals>
\end{minted}
\caption{Notional list}
\label{lst:notionals}
\end{listing}

Another allowable specification of the notional schedule is shown in Listing \ref{lst:notionals_dates}. 
\begin{listing}[H]
%\hrule\medskip
\begin{minted}[fontsize=\footnotesize]{xml}
                <Notionals>
                    <Notional>65000000</Notional>
                    <Notional startDate='2016-01-02'>65000000</Notional>
                    <Notional startDate='2017-01-02'>55000000</Notional>
                    <Notional startDate='2021-01-02'>45000000</Notional>
                </Notionals>
\end{minted}
\caption{Notional list with dates}
\label{lst:notionals_dates}
\end{listing}
The first notional must not have a start date, it will be associated
with the schedule's start, The subsequent notionals can have a start
date specified from which date onwards the new notional is applied. This allows
specifying notionals only for dates where the notional changes. 

\vspace{1em} 

In case of exchange of currencies an initial exchange, a final exchange
and an amortising exchange can be specified using an \lstinline!Exchanges! child element with \break
\lstinline!NotionalInitialExchange!, \lstinline!NotionalFinalExchange! and \break
\lstinline!NotionalAmortizingExchange! as subelements, see Listing
\ref{lst:notional_exchange}.

Allowable values for \lstinline!NotionalInitialExchange!,  \lstinline!NotionalFinalExchange! and \lstinline!NotionalAmortizingExchange!:   \emph{true, false}. Defaults to  \emph{false} if omitted, or if the entire \lstinline!Exchanges! block is omitted.

\begin{listing}[H]
%\hrule\medskip
\begin{minted}[fontsize=\footnotesize]{xml}
                <Notionals>
                    <Notional>65000000</Notional>
                    <Exchanges>
                      <NotionalInitialExchange>true</NotionalInitialExchange>
                      <NotionalFinalExchange>true</NotionalFinalExchange>
                      <NotionalAmortizingExchange>true</NotionalAmortizingExchange>
                    </Exchanges>
                </Notionals>
\end{minted}
\caption{Notional list with exchange}
\label{lst:notional_exchange}
\end{listing}

FX Resets,  used for Rebalancing Cross-currency swaps, can be specified using an \lstinline!FXReset! child element with the following subelements:  See Listing \ref{lst:notional_fxreset} for an example. \break

\item ForeignCurrency: The foreign currency the notional of the leg resets to.  

Allowable values:  See \lstinline!Currency!  in Table \ref{tab:allow_stand_data}.

\item ForeignAmount: The notional amount in the foreign currency that the notional of the leg resets to.  

Allowable values:  Any positive real number.

\item FXIndex: A reference to an FX Index source for the FX reset fixing. 

Allowable values:  A string on the form FX-SOURCE-CCY1-CCY2.

\item FixingDays: The FX fixing lag in business days

Allowable values:  Any integer.

\item FixingCalendar[Optional]: The calendar associated with the FX Index. 

Allowable values: See Table \ref{tab:calendar} Calendar. Defaults to the null calendar if left blank or omitted.



 \begin{listing}[H]
%\hrule\medskip
\begin{minted}[fontsize=\footnotesize]{xml}
                <Currency>USD</Currency>
                <Notionals>
                    <Notional>65000000</Notional> <!-- in USD -->
                    <FXReset>
                      <ForeignCurrency> EUR </ForeignCurrency>
                      <ForeignAmount> 60000000 </ForeignAmount>
                      <FXIndex> FX-ECB-USD-EUR </FXIndex>
                      <FixingDays> 2 </FixingDays>
                      <FixingCalendar>TARGET</FixingCalendar>
                    </FXReset>
                </Notionals>
\end{minted}
\caption{Notional list with fx reset}
\label{lst:notional_fxreset}
\end{listing}


After the \lstinline!Notional! sub-node the \lstinline!LegData! node includes a \lstinline!ScheduleData! sub-node, and a sub-node based on the choice of \lstinline!LegType! as per below:

\item ScheduleData: This is a trade component sub-node outlined in section \ref{ss:schedule_data} Schedule Data and
Dates.
\item \lstinline!PaymentDates! [Optional]: This node allows for the specification of a list of explicit payment dates that overrides the payment dates generated by the payment conventions. This is an optional node and is currently only used in commodity trades.
See Listing \ref{lst:paymentdates} for an example. \break

\begin{listing}[H]
%\hrule\medskip
\begin{minted}[fontsize=\footnotesize]{xml}
                <PaymentDates>
                      <PaymentDate>2020-01-15</PaymentDate>
                      <PaymentDate>2021-01-15</PaymentDate>
                      <PaymentDate>2022-01-17</PaymentDate>
                </PaymentDates>
\end{minted}
\caption{Payment dates}
\label{lst:paymentdates}
\end{listing}

\item FixedLegData: This trade component sub-node is required if \lstinline!LegType! is set to \emph{fixed} It is
outlined in section \ref{ss:fixedleg_data}.
\item FloatingLegData: This trade component sub-node is required if \lstinline!LegType! is set to \emph{floating} It is
outlined in section \ref{ss:floatingleg_data} Floating Leg Data and Spreads.
\item CMSLegData: This trade component sub-node is required if \lstinline!LegType! is set to \emph{CMS}. It is
  outlined in section \ref{ss:cmslegdata}.
\item CMSSpreadLegData: This trade component sub-node is required if \lstinline!LegType! is set to \emph{CMSSpread}. It is
  outlined in section \ref{ss:cmsspreadlegdata}.
\item DigitalCMSSpreadLegData: This trade component sub-node is required if \lstinline!LegType! is set to \emph{DigitalCMSSpread}. It is
  outlined in section \ref{ss:digitalcmsspreadlegdata}.
\item EquityLegData: This trade component sub-node is required if \lstinline!LegType! is set to \emph{Equity}. It is
  outlined in section \ref{ss:equitylegdata}.    
\item CPILegData: This trade component sub-node is required if \lstinline!LegType! is set to \emph{CPI}. It is
  outlined in section \ref{ss:cpilegdata}.
\item YYLegData: This trade component sub-node is required if \lstinline!LegType! is set to \emph{YY}. It is
  outlined in section \ref{ss:yylegdata}.
\item ZeroCouponFixedLegData: This trade component sub-node is required if \lstinline!LegType! is set to \emph{ZeroCouponFixed}. It is
  outlined in section \ref{ss:zerolegdata}.
\end{itemize}
