\subsubsection{Leg Data with Amortisation Structures}
\label{ss:amortisationdata}

Amortisation structures can (optionally) be added to a leg as
indicated in the following listing \ref{lst:amortisations}, within a
block of information enclosed by {\tt <Amortizations>} and {\tt
  </Amortizations>} tags.

\begin{listing}[H]
%\hrule\medskip
\begin{minted}[fontsize=\footnotesize]{xml}
      <LegData>
        <LegType> ... </LegType>
        <Payer> ... </Payer>
        <Currency> ... </Currency>
        <Notionals>
          <Notional>10000000</Notional>
        </Notionals>
        <Amortizations>
          <AmortizationData>
            <Type>FixedAmount</Type>
            <Value>1000000</Value>
            <StartDate>20170203</StartDate>
            <Frequency>1Y</Frequency>
            <Underflow>false</Underflow>
          </AmortizationData>
          <AmortizationData>
            ...
          </AmortizationData>
        </Amortizations>
        ...
      </LegData>
\end{minted}
\caption{Amortisation data}
\label{lst:amortisations}
\end{listing}

The user can specify a sequence of {\tt AmortizationData} items in
order to switch from one kind of amortisation to another etc.  
Within each {\tt AmortisationData} block the meaning of elements is

\begin{itemize}
\item Type: Amortisation type with allowable values {\em FixedAmount,
  RelativeToInitialNotional, RelativeToPreviousNotional, Annuity.}
\item Value: Interpreted depending on {\tt Type}, see below.
\item StartDate: Amortisation starts on first schedule date on or
  beyond StartDate.
\item Frequency, entered as a period: Frequency of amortisations.
\item Underflow:  Allow amortisation below zero notional if {\tt true},
  otherwise amortisation stops at zero notional.
\end{itemize}

The amortisation data block's {\tt Value} element  is interpreted
depending on the chosen {\tt Type}:
\begin{itemize}
\item FixedAmount: The value is interpreted as a notional amount to be
  subtracted from the current notional on each amortisation date.
\item RelativeToInitialNotional: The value is interpreted as a
  fraction of the {\bf initial} notional to be subtraced from the current
  notional on each amortisation date.
\item RelativeToPreviousNotional: The value is interpreted as a
  fraction of the {\bf previous} notional to be subtraced from the current
  notional on each amortisation date.
\item Annuity: The value is interpreted as annuity amount (redemption
  plus coupon).
\end{itemize}

Annuity type amortisation is supported for fixed rate legs as well as
floating (ibor) legs. 

Note:
\begin{itemize}
\item Floating annuities require at least one previous vanilla coupon
  in order to work out the first amortisation amount. 
\item Floating legs with annuity amortisation currently do not allow
  switching the amortisation type, i.e. only a  single block of {\tt
    AmortizationData}.
\end{itemize}
