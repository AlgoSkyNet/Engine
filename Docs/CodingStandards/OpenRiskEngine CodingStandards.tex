\documentclass[12pt, a4paper]{article}

\usepackage{supertabular}
\usepackage{hyperref}

\addtolength{\textwidth}{0.8in}
\addtolength{\oddsidemargin}{-.4in}
\addtolength{\evensidemargin}{-.4in}
\addtolength{\textheight}{1.6in}
\addtolength{\topmargin}{-.8in}


\begin{document}

\title{OpenRiskEngine Coding Standards}
\date{\today}
\maketitle

\newpage

%-------------------------------------------------------------------------------
\section*{Document History}

\begin{center} 
\begin{supertabular}{|l|l|l|p{7cm}|}
\hline
Version & Date & Author & Comment \\ 
\hline 
0.1 & 27 December 2015 & Niall O'Sullivan& initial version\\
\hline
0.2 & 3 June 2016 & Niall O'Sullivan& pull requests\\
\hline
\end{supertabular}
\end{center}

\vspace{3cm}

\newpage
%\tableofcontents
%\newpage


%-------------------------------------------------------------------------------
\section*{Introduction}

This document outlines the coding standards to be employed in all OpenRiskEngine development.

\subsection*{C++ Version}
OpenRiskEngine is written using C++11, C++14 features are not permitted. The most commonly used feature in C++11 is auto boxing and foreach loops, e.g.
\begin{verbatim}
vector<int> v = { 1, 2, 3 };
for (const auto& a : v)
  cout << a << endl;
\end{verbatim}
and also the ability to define nested templates without conflicting with the right shift operator.
\begin{verbatim}
vector<vector<int>> v;
\end{verbatim}
Note that the use of QuantLib means we must use  \textbf{boost::shared\_ptr} for the majority of objects. For simplicity we will use
\textbf{boost::shared\_ptr} everywhere and there will be no \textbf{\#define PTR or Ptr} macros.
OpenRiskEngine
QuantExt development must be comparable with the QuantLib coding standards and so must be C++98 compliant.

\subsection*{Platform Support}
OpenRiskEngine should be able to build on any platform that QuantLib can be built on, development files will be maintained for MSVC 2013 on Windows and a QL style automake for Unix and MacOS.

All development will be done on 64-bit architecture.

\subsection*{Third Party Libraries}
OpenRiskEngine is written using the following libraries
\begin{enumerate}
\item Boost - Any version of Boost greater than 1.54 (?), this environment variable \$BOOST must be defined and should point to the root Boost folder.
\item QuantLib - strictly 1.7, source code is stored in the OpenRiskEngine git repository and cannot be modified.
\item RapidXML - 1.13, source code is stored in the OpenRiskEngine git repository and cannot be modified.
\end{enumerate}
Other libraries may be included in the future, including:
\begin{enumerate}
\item zlib
\item muparser
\item hdf5
\item log4cxx
\end{enumerate}

\subsection*{Environment variables}
OpenRiskEngine will not depend or use any environment variables or windows registry files. The only environment variable ever used will be \$BOOST as defined above for development purposes.

\subsection*{Threading and singletons}
OpenRiskEngine is a single threaded application, the use of singletons is to be avoided where possible. If a developer feels that a singleton must be employed than this must be justified in detail.

\subsection*{Code Layout and indention}
All code is to be indented with four spaces, tabs are not to be included in any source file. Redundant whitespace is not to be inserted into files. 

\noindent
A full set of rules and/or a configuration for a source code formatting tool (e.g. astyle or clang-format) will be defined.

\subsection*{Namespace}
With the exception of QuantExt, all code will be in the namespace openxva, nested namespaces should be used for logical modules (like boost), all namespaces should be in lower case, e.g. \textbf{openxva::data, openxva::trade, openxva::engine}.

\subsection*{Comments}
All code should be commented, all comments are to be in (British) English, no swear words are to be included.

\subsection*{Doxygen}
All header files should include Doxygen style comments and description, the building of doxygen html reference documentation will be included into both build systems and should be run along side unit tests before code is committed.

\subsection*{Include Guards}
All header files should have a \textbf{\#pragma once} statement instead of \textbf{\#define} statements

\subsection*{Copyright}
All source files should be copyright \textbf{Quaternion Risk Management Ltd.}, exact licence to be confirmed.

%-------------------------------------------------------------------------------
\section*{git usage - pull requests}
Code will be integrated by using git pull requests.

Pull requests are branch specific and should be done on new branches that each developer creates. Once these have been merged the branch will be deleted. This is slightly different on codebasehq.com and github.com, on codebase we are only able to issue pull requests on the same repo, thus every developer must push their new branches and pull requests to origin.

The git workflow for a developer (bob) is as follows.
\begin{enumerate}
\item Developer must maintain a local, clean copy of origin/master, this can be their own master branch or some other label (official - which we will use herein).
\item To create a new official branch \\
 \texttt{\% git checkout -b official origin/master} \\
 If you are worried that your official has gotten out of sync, simple delete it and repeat the above.\\
 \texttt{\% git branch -D official} \\
 \texttt{\% git checkout -b official origin/master}
\item Alternatively, if official already exists, the developer must merge origin master to ensure that they are up to date\\
 \texttt{\% git checkout offical} \\
 \texttt{\% git fetch origin} \\
 \texttt{\% git merge origin/master} \\
\item Developer creates a new branch, the branch name should be the developers name or initials, an underscore and then a description of the branch. This is purely to keep the branch names unique and not create conflict.\\
 \texttt{\% git checkout -b bob\_widget}
\item Developer codes as normal, pushing to their public repo (bob) if they so wish.\\
\textbf{Optional:} \texttt{\% git push bob bob\_widget}
\item Once the coding task in complete, developer pushes the branch to the origin repo. Care should be taken not to push to origin/master.\\
 \texttt{\% git push origin bob\_widget}
 \item Then go to codebasehq.com \url{https://qrm.codebasehq.com/projects/openxva/repositories/openxva/tree/master} and click on "Merge Request" and then "New Merge Request" with the following details:
 \begin{itemize}
 \item \textbf{Subject} A short description
 \item \textbf{User} The merge manager (currently Niall O'Sullivan)
 \item \textbf{Source Branch} the branch to be merged (bob\_widget)
 \item \textbf{Target Branch} master
 \end{itemize}
\item The Developer may add comments to the request.
\end{enumerate}

At this point the developers work is complete, the appropriate ticket can be closed.

All developers are free to comment on pull requests on codebasehq. The majority will be integrated quickly, some of them may generate a debate or discussion which is normal. There is no time limit on merging.


The git workflow for the merge manager is as follows
\begin{enumerate}
\item Note that everyone who has a codebase account can preform these tasks, currently there is one designated manager who may delegate.
\item go to merge requests in codebase
\item if there are no conflicts, click on "merge request"
\item if there are conflicts, send it back to the developer
\item Once the request has been merged, the temporary branch should be deleted. This is not strictly necessary but means origin is kept cleaner and avoids conflicts.\\
 \texttt{\% git push --delete origin bob\_widget}
\end{enumerate}




\end{document}
