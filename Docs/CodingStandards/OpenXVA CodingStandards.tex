\documentclass[12pt, a4paper]{article}

\usepackage{supertabular}
\usepackage{hyperref}

\addtolength{\textwidth}{0.8in}
\addtolength{\oddsidemargin}{-.4in}
\addtolength{\evensidemargin}{-.4in}
\addtolength{\textheight}{1.6in}
\addtolength{\topmargin}{-.8in}


\begin{document}

\title{OpenXVA Coding Standards}
\date{\today}
\maketitle

\newpage

%-------------------------------------------------------------------------------
\section*{Document History}

\begin{center} 
\begin{supertabular}{|l|l|l|p{7cm}|}
\hline
Version & Date & Author & Comment \\ 
\hline 
0.1 & 27 December 2015 & Niall O'Sullivan& initial version\\
\hline
\end{supertabular}
\end{center}

\vspace{3cm}

\newpage
%\tableofcontents
%\newpage


%-------------------------------------------------------------------------------
\section*{Introduction}

This document outlines the coding standards to be employed in all OpenXVA development.

\subsection*{C++ Version}
OpenXVA is written using C++11, C++14 features are not permitted. The most commonly used feature in C++11 is auto boxing and foreach loops, e.g.
\begin{verbatim}
vector<int> v = { 1, 2, 3 };
for (const auto& a : v)
  cout << a << endl;
\end{verbatim}
and also the ability to define nested templates without conflicting with the right shift operator.
\begin{verbatim}
vector<vector<int>> v;
\end{verbatim}
Note that the use of QuantLib means we must use  \textbf{boost::shared\_ptr} for the majority of objects. For simplicity we will use
\textbf{boost::shared\_ptr} everywhere and there will be no \textbf{\#define PTR or Ptr} macros.

QuantExt development must be comparable with the QuantLib coding standards and so must be C++98 compliant.

\subsection*{Platform Support}
OpenXVA should be able to build on any platform that QuantLib can be built on, development files will be maintained for MSVC 2013 on Windows and a QL style automake for Unix and MacOS.

All development will be done on 64-bit architecture.

\subsection*{Third Party Libraries}
OpenXVA is written using the following libraries
\begin{enumerate}
\item Boost - Any version of Boost greater than 1.54 (?), this environment variable \$BOOST must be defined and should point to the root Boost folder.
\item QuantLib - strictly 1.7, source code is stored in the OpenXVA git repository and cannot be modified.
\item RapidXML - 1.13, source code is stored in the OpenXVA git repository and cannot be modified.
\end{enumerate}
Other libraries may be included in the future, including:
\begin{enumerate}
\item zlib
\item muparser
\item hdf5
\item log4cxx
\end{enumerate}

\subsection*{Environment variables}
OpenXVA will not depend or use any environment variables or windows registry files. The only environment variable ever used will be \$BOOST as defined above for development purposes.

\subsection*{Threading and singletons}
OpenXVA is a single threaded application, the use of singletons is to be avoided where possible. If a developer feels that a singleton must be employed than this must be justified in detail.

\subsection*{Code Layout and indention}
All code is to be indented with four spaces, tabs are not to be included in any source file. Redundant whitespace is not to be inserted into files. 

\noindent
A full set of rules and/or a configuration for a source code formatting tool (e.g. astyle or clang-format) will be defined.

\subsection*{Namespace}
With the exception of QuantExt, all code will be in the namespace openxva, nested namespaces should be used for logical modules (like boost), all namespaces should be in lower case, e.g. \textbf{openxva::data, openxva::trade, openxva::engine}.

\subsection*{Comments}
All code should be commented, all comments are to be in (British) English, no swear words are to be included.

\subsection*{Doxygen}
All header files should include Doxygen style comments and description, the building of doxygen html reference documentation will be included into both build systems and should be run along side unit tests before code is committed.

\subsection*{Include Guards}
All header files should have a \textbf{\#pragma once} statement instead of \textbf{\#define} statements

\subsection*{Copyright}
All source files should be copyright \textbf{Quaternion Risk Management Ltd.}, exact licence to be confirmed.

\end{document}
